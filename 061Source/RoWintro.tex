\section{Forward: A Brief History of Fighting Men}

In its origins, D\&D was a wargame like Warmachine or Warhammer. You had a field filled with tiny men, and they fought each other with swords and bows. Eventually, someone got really lazy, and wanted to replace a large number of fighting men with heroic fighting men who would be easier to paint because there were much less of them. And that, right there, is the origins of D\&D. The smaller number of better Fighting Men would be your ``army" and eventually people started playing magical teaparty with their fighting men, and it turned into a roleplaying game. So it isn't surprising that at first you ``roleplayed" a small group of heroic fighting men.

When the new classes (such as ``Magic User" and eventually ``Thief" and ``Cleric") were introduced, they were intended to be better than the Fighting Men. And, well, they totally were. Indeed, players still controlled lots of characters, and it was deemed impractical for more than one or two of those characters to be any good or in any fashion important. So you rolled up stats for each guy, and if you rolled well enough on a guy he could be something other than a Fighting Man, and the rest of your guys were basically just speed bumps whose lot in life was to stand between the monsters and the Magic Users so that the real characters could survive to another day.

Well, that isn't how things work anymore. Now every character is supposed to be individually important and have some background and so on and so forth. No longer are we allowing our Fighting Men to go without a last name unless and until they get to fourth level without being eaten by an owlbear. And so we really need Fighting Men to be a lot more interesting and effective than they are in the rules. The basic setup of the game has changed a lot, but Fighters have changed only a little. In a very real way, the Player's Handbook hands us Fighting Men who would be better suited to appear in groups of 3 per player than to stand alone. And really, that has got to stop.

\section{War in D\&D}
\vspace*{-10pt}
\quot{``War is not about who is right, but who is left."}

D\&D is a game about stabbing people in the face, rifling through their pockets and/or home, and then going back to your own home where the beer is cold and the woman are warm and waiting for the next foolio to present himself for stabbing (and rifling). That being said, war is the same thing, but writ large.

War in the D\&D universe is very nasty, very brutal, and very short. It all comes down to the question ``who's got the bigger heroes?" Peasant uprisings of plucky farmers just don't happen in a world where a 1st level mage with a Wand of Fireballs and a decent Hide check can set an army of thousands on fire, and the bravest and best trained units of knights just aren't going to conquer the land/government that has a guy chain-binding vrocks to serve as elite terror squads to kill every peasant in a hundred mile radius of your capital.

If you have the bigger heroes, they knock down any smaller heroes, then walk up to the Kingdom of Good King Draxall  \ldots\  yada yada yada \ldots\ and hear the lamentation of his womenfolk. It doesn't really matter if King Draxall's castle is now full of lava because the attackers opened a gate to a volcano in his throne room or if they went all Die Hard on the King's personal guard and gutted the bunch \ldots\ the truly important troops (i.e. heroes) traveled at least as fast as griffonback and smashed the Kingdom while the King was still training his peasants on which end of a spear to poke people with.

That doesn't mean that armies don't have a place in D\&D. Once the important business of nailing enemy heroes to a tree is done, someone has to pacify the new populous, enslave them to work the salt mines, collect taxes, and generally put down any rebellions or resistance movements of local yahoos (which might be gnoll bandits, a wandering ankheg, or other unimportant challenge for our heros). Heroes are generally more concerned with bigger and more rewarding problems like the undead pouring out of the newly discovered (i.e. unlooted) ruins in Moil than the fact that the peasants of the former King Draxall are up in arms over the latest taxes on grain.

But occasionally, someone does attempt a military victory. It might be an aristocrat with more gold than sense or a necromancer with an animation fixation, but troops will be secretly trained, mercenaries will be hired, and cadres of spies will pour into the prospective target land. Sometimes this crap works, as the relevant heroes who might defend the land might be bribed to stand aside, assassinated with extreme prejudice, or just be on another plane at the time, and then it's the Wytch King's skeletal footman vs. King Draxall's Knights of the Holy Relic for real old-timey war on respectable battlefields.

The problem is that this kind of thing is that it generally doesn't last. Once the local hero population replenishes itself, those guys will become the local rulers by default, even if they only pay lip service to King Draxall in public. Empires lasting thousands of years are not products of military might, but a good PR department with an eye for finding up-and-coming heroes who are smart enough to maintain the fiction of a stable society rather than upset the peasants by reminding them that they live and die by the whims of guys who think that summoning angels from heaven to set off dungeon traps is an acceptable practice.

\subsection{Fighting with Honor}
\vspace*{-8pt}
\quot{``There is only one ethical system and it is pragmatism. Only goals change."}

The concept of honorable combat is pretty fishy when you look at it carefully. Your goal is to painfully kill another sapient being with a deadly weapon, and the other guy is attempting to do the same to you. Why then, would any rational person take time to consider the ``honor" of whatever horribly painful and potentially lethal act they were intent upon inflicting on another?

The answer is: The Long Term. The concept of honor in War is incredibly ancient, and the ideas of what is and is not an honorable act have varied unrecognizably over that period. But one thing has remained the same throughout: the idea of what is honorable in warfare has always been inextricably linked to the needs of the powerful. In olden days, the powerful had superior nutrition, superior training, superior equipment and came in really small numbers. So naturally of course, the rule was that you didn't gang up on people or use poison. In modern days, bullets go through pretty much anything, but powerful people have more troops and helicopters, so the rule is that you don't assassinate people in honorable combat. The penalties for being dishonorable have remained pretty static over the generations -- you get kicked out of the rosters of the powerful and other power blocs attempt to band together to crush you.

That's all fine and dandy, but what does that mean for characters in the D\&D world? The risks of using poison gas in terms of collateral damage really aren't there (cloudkill goes pretty much exactly where you tell it to), and the ranks of the powerful really do include high level Rogues and Assassins. Most of the stuff you think of as being dishonorable in historical chivalric codes are perfectly fine in D\&D chivalric codes. Like all chivalric codes, the one found in the D\&D universes is there to keep people in their place -- in this case powerful adventurers on top, and little people and monsters on the bottom. Here's how it works:

\listone
    \item Getting a lot of help on any project is dishonorable. A 9th level wizard can wave his hands and make a dungeon, and two rogues can stab a frost giant in the back of the head and the face in synchrony. But peasants can't do jack without the help of like 20 guys. Therefore, working in groups larger than about 10 on any single project is dishonorable in the extreme. The end result is that decent goods can really only be produced by the master artisans and the little people are trapped in obscurity.
    \item Poisoning Food is without honor. Druids can spit poison and Assassins can shoot poison darts, but pretty much anyone can put warfarin into an enchilada. So while injected poisons aren't considered dishonorable, ingested poisons are.
    \item Being Gargantuan or Larger is dishonorable. It may seem downright bizarre that people in the D\&D world endeavor to look down on things which stand tall. But when you think about the locations that the truly tremendous live in, it makes sense. When gargantuan creatures rear themselves, it is expected practice for all groups to drop what they are doing and attack. And that is why Titans and Dragons live on remote mountaintops instead of owning the world. It isn't that taking them down isn't a lot of effort, it's that the small creatures made a gentleman's agreement to actually put that effort in a long time ago.
    \item Honorable people do not create Spawn.: This is one that bones the monsters and certain kinds of spellcasters like necromancers, and its designed so that people don't take Steve the Crap-Covered Farmer and turn him into a hero-level threat like a vampire spawn. We know how this works for the people that do it: they tip the balance in favor of the monsters and the heroes and society loses. Even if every Shadow only makes one other Shadow each day, in three weeks your kingdom is full of Shadows \ldots\ people in the D\&D universe know how this is going to end and it makes them very unhappy.
    \item Impersonating specific people with magic is a dishonorable act. Heroes live and die by their reputation, and part and parcel of being a hero is that people know who you are and where to find you so that they can shower you with job offers and money. That actually works for society, because this is a pre-Internet universe and we don't have Craigslist to make sure that people get the right jobs.
    \item Destroying Magic Items is something no honorable person would do. Magic is in many ways, a finite resource. The people in power, need it to stay in power. Artifacts are essentially irreplaceable, but they are corruptible. Maybe not by you, but by someone. If you destroy a great artifact of Evil, you've actually hurt Good some too. You've reduced the total amount of power available to anyone. And that doesn't fly for people who have all the power.
    \item Changing Alignment is dishonorable. Every power group wants people to pretty much stay on whatever side they are on, because otherwise how do you know who is on what side? It's very pragmatic, those who switch sides are never afforded the same trust in their new side as they were given from their old side lest they change back. That isn't to say that Good and Evil aren't proselytizing
    \item Honorable people take credit for their kills. Not only is it just good form to advertise your abilities so that people know who in the kingdom actually can kill an Ettin in single combat, but its actually safer for everyone if society in general know why powerful monsters keep dropping out of the sky. When people find an Old Red Dragon dead in a random field, they are going to want to know what killed it and if it has plans on their favorite tavern. Not claiming your kills means that actual hero-hours are going to be spent finding out the nature of this threat when they could be better spent curbing the excesses of the Wytch King's Empire. That pisses people off, and leads to occasional hero-on-hero violence that only serves Team Monster.
\end{list}

\vspace{8pt}
So you want to be honorable, right? Maybe give your coat to handsome members of the opposite sex, keep your word, and make sure your taxes are paid on time? Yeah, that has to do with your alignment probably (depending upon what you think Law, Chaos, Good, and Evil actually represent), not with you overall honor. Honor really is about whether society in general is going to attempt to ostracize you. So you can be Evil and Chaotic and still fit into society, still be considered honorable. In fact, D\&D has entire Chaotic Evil societies where that sort of thing is expected.
