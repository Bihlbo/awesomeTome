\section{Multiclassing Characters}
\vspace{-10pt}
\quot{''What do I do? I stab things in the face\ldots\ Fine, I'm a Fighter/Ranger/Barbarian/Master of Black Fire/Cloud Jumper."}

It's time to face the music, Warriors are the only people who multiclass, and thus the proper place to discuss multiclassing characters is right here in the book about War. The reasons for this are extremely simple: the only level appropriate abilities that Warriors get are based on skill ranks and BAB, and those things stack up between classes just fine. A Warrior can take three different classes and still be getting abilities that are as appropriate to his level as if he had taken one all the way through. Spellcasting, on the other hand, grants its primary level appropriate abilities based on the class spell level chart, and that does not stack between classes at all. In fact, if a Spellcaster takes prestige class that simply does not advance spellcasting every level, he has permanently sacrificed he ability to ride the level appropriate ability train forever, and D\&D does not have the possibility of a quick fix for this. Maybe 4th edition is going to have a universal ability slots system in which characters have level appropriate daily slots and different classes could allow you to use them for different things (such as casting powerful spells or performing amazing non-magical stunts) -- certainly it would if I were writing it. But for now that kind of overhaul is simply outside the scope of this document.

What we can do is eliminate some of the rough edges that occur with multiclassing non-spellcasting classes. That's pretty simple, and most of it is common sense:

\subsection{XP Penalties}
I don't know why these ever seemed like a good idea to anyone, but they weren't. The best spellcasters are single classed and classically warrior builds have rarely taken more than 2 levels of anything. So really multiclass XP penalties only happen to organic and concept characters. And those are the characters we don't want to jack over. So poof! No multiclass XP penalties. That was easy, wasn't it?

\subsection{Favored Classes}
So we're getting rid of Multiclassing XP Penalties because they are dumb\ldots\  what then is the Favored Class supposed to do? Well, it's supposed to be a minor advantage for races to play those classes (instead of being a complete waste of our time like it is in the base book), so let's make it a real minor advantage. If you are taking your race's favored class, you can take racial substitution levels, if you want. And yes, that means that a Human can take any racial substitution levels that they want -- merry christmas.

And while we're on the subject, it has probably come to your attention that when the idea of Favored Classes was first thrown down there were only 11 base classes. Now there are\ldots\  many classes. That's why all the races listed in this document have two favored classes. We suggest that you do the same for any other races you allow (for example: Gnomes have favored classes of Bard and Wizard).

\subsection{Saving Throws}
As we all know, characters multiclassing get saving throws that are crazy-go-nuts. The simple fact that the good save is restarted every time you start a new class (or prestige class into a variant class) means that the maximum save bonus at 20th level is 40, and the minimum is zero. That means that two characters are different in their base saves by more than two entire random number generators. And while you can come up with fractional schemes to fix this problem, experience has shown that players actually can't keep track of those. What's needed is something simple that works within the existing D\&D rules framework. Our suggestion is to throw down the caveat that if you start a progression with a good save and you already have at least one level with a good save in that category that you gain +1 instead of +2. It's simple, easy to understand, and pulls in the crazy just enough that you can overlook the mathematical inadequacies of the system and play the game.

\subsection{Skill Points}

It goes without saying that the entire idea of paying skill points at the cross-class rate breaks D\&D. No one should ever buy things at the cross-class rate. Ever. Cross class skill maximums are fine, but the Cross Class skill rate exists only as a method to perform repeatable actions to permanently increase or reduce your total skill points.
