

%%%%%%%%%%%%%%%%%%%%%%%%%%%%%%%%%%%%%%%%%%%%%%%%%%
\classentry{Monk}
%%%%%%%%%%%%%%%%%%%%%%%%%%%%%%%%%%%%%%%%%%%%%%%%%%

\textbf{Alignment:} Any lawful.

\textbf{Hit Die:} d8.

\textbf{Class Skills}

The monk's class skills (and the key ability for each skill) are \linkskill{Balance} (Dex), 
\linkskill{Climb} (Str), \linkskill{Concentration} (Con), \linkskill{Craft} (Int), \linkskill{Diplomacy} (Cha), \linkskill{Escape Artist} (Dex), 
\linkskill{Hide} (Dex), \linkskill{Jump} (Str), \linkskill{Knowledge} (arcana) (Int), \linkskill{Knowledge} (religion) (Int), \linkskill{Listen} 
(Wis), \linkskill{Move Silently} (Dex), \linkskill{Perform} (Cha), \linkskill{Profession} (Wis), \linkskill{Sense Motive} (Wis), 
\linkskill{Spot} (Wis), \linkskill{Swim} (Str), and \linkskill{Tumble} (Dex).

\textbf{Skill Points at 1st Level:} (4 + Int modifier) x4.

\textbf{Skill Points at Each Additional Level:} 4 + Int modifier.

\begin{table}[htb]
\rowcolors{1}{white}{offyellow}
\caption{The Monk}
\centering
\begin{tabular}{*{7}{l}}
\textbf{Level} & \textbf{BAB} & \textbf{Fort} & \textbf{Reflex} & \textbf{Will} & \textbf{Special} & \textbf{Speed Bonus}\\
1st & +0 & +2 & +2 & +2 & Bonus Feat, Flurry Of Blows, Unarmed Strike & +0ft\\
2nd & +1 & +3 & +3 & +3 & Bonus Feat, Evasion & +0ft\\
3rd & +2 & +3 & +3 & +3 & Still Mind & +10ft\\
4th & +3 & +4 & +4 & +4 & Ki Strike (Magic), Slow Fall 20ft & +10ft\\
5th & +3 & +4 & +4 & +4 & Purity Of Body & +10ft\\
6th & +4 & +5 & +5 & +5 & Bonus Feat, Slow Fall 30ft & +20ft\\
7th & +5 & +5 & +5 & +5 & Wholeness of Body & +20ft\\
8th & +6 & +6 & +6 & +6 & Slow Fall 40ft & +20ft\\
9th & +6 & +6 & +6 & +6 & Improved Evasion & +30ft\\
10th & +7 & +7 & +7 & +7 & Ki Strike (Lawful), Slow Fall 50ft & +30ft\\
11th & +8 & +7 & +7 & +7 & Diamond Body, Greater Flurry & +30ft\\
12th & +9 & +8 & +8 & +8 & Abundant Step, Slow Fall 60ft & +40ft\\
13th & +9 & +8 & +8 & +8 & Diamond Soul & +40ft\\
14th & +10 & +9 & +9 & +9 & Slow Fall 70ft & +40ft\\
15th & +11 & +9 & +9 & +9 & Quivering Palm & +50ft\\
16th & +12 & +10 & +10 & +10 & Ki Strike (Adamantine), Slow Fall 80ft & +50ft\\
17th & +12 & +10 & +10 & +10 & Timeless Body, Tongue of The Sun And Moon & +50ft\\
18th & +13 & +11 & +11 & +11 & Slow Fall 90ft & +60ft\\
19th & +14 & +11 & +11 & +11 & Empty Body & +60ft\\
20th & +15 & +12 & +12 & +12 & Perfect Self, Slow Fall Any Distance & +60ft\\
\end{tabular}
\end{table}

%%%%%%%%%%%%%%%%%%%%%%%%%
\ClassFeatures
%%%%%%%%%%%%%%%%%%%%%%%%%

All of the following are class features of the monk.

\textbf{Weapon and Armor Proficiency:} Monks are proficient with club, crossbow 
(light or heavy), dagger, handaxe, javelin, kama, nunchaku, quarterstaff, sai, 
shuriken, siangham, and sling.

Monks are not proficient with any armor or shields.

When wearing armor, using a shield, or carrying a medium or heavy load, a monk 
loses her AC bonus, as well as her fast movement and flurry of blows abilities.

\textbf{AC Bonus (Ex):} When unarmored and unencumbered, the monk adds her Wisdom 
bonus (if any) to her AC. In addition, a monk gains a +1 bonus to AC at 5th level. 
This bonus increases by 1 for every five monk levels thereafter (+2 at 10th, +3 
at 15th, and +4 at 20th level).

These bonuses to AC apply even against touch attacks or when the monk is flat-footed. 
She loses these bonuses when she is immobilized or helpless, when she wears any 
armor, when she carries a shield, or when she carries a medium or heavy load.

\textbf{Flurry of Blows (Ex):} When unarmored, a monk may strike with a flurry 
of blows at the expense of accuracy. When doing so, she may make one extra attack 
in a round at her highest base attack bonus, but this attack takes a -2 penalty, 
as does each other attack made that round. The resulting modified base attack bonuses 
are shown in the Flurry of Blows Attack Bonus column on Table: The Monk. This penalty 
applies for 1 round, so it also affects attacks of opportunity the monk might make 
before her next action. When a monk reaches 5th level, the penalty lessens to -1, 
and at 9th level it disappears. A monk must use a full attack action to strike 
with a flurry of blows.

When using flurry of blows, a monk may attack only with unarmed strikes or with 
special monk weapons (kama, nunchaku, quarterstaff, sai, shuriken, and siangham). 
She may attack with unarmed strikes and special monk weapons interchangeably as 
desired. When using weapons as part of a flurry of blows, a monk applies her Strength 
bonus (not Str bonus x1-1/2 or x1/2) to her damage rolls for all successful 
attacks, whether she wields a weapon in one or both hands. The monk can't use any 
weapon other than a special monk weapon as part of a flurry of blows.

In the case of the quarterstaff, each end counts as a separate weapon for the purpose 
of using the flurry of blows ability. Even though the quarterstaff requires two 
hands to use, a monk may still intersperse unarmed strikes with quarterstaff strikes, 
assuming that she has enough attacks in her flurry of blows routine to do so. 

When a monk reaches 11th level, her flurry of blows ability improves. In addition 
to the standard single extra attack she gets from flurry of blows, she gets a second 
extra attack at her full base attack bonus.

\textbf{Unarmed Strike:} At 1st level, a monk gains Improved Unarmed Strike as 
a bonus feat. A monk's attacks may be with either fist interchangeably or even 
from elbows, knees, and feet. This means that a monk may even make unarmed strikes 
with her hands full. There is no such thing as an off-hand attack for a monk striking 
unarmed. A monk may thus apply her full Strength bonus on damage rolls for all 
her unarmed strikes.

Usually a monk's unarmed strikes deal lethal damage, but she can choose to deal 
nonlethal damage instead with no penalty on her attack roll. She has the same choice 
to deal lethal or nonlethal damage while grappling.

A monk's unarmed strike is treated both as a manufactured weapon and a natural 
weapon for the purpose of spells and effects that enhance or improve either manufactured 
weapons or natural weapons.

A monk also deals more damage with her unarmed strikes than a normal person of their size would, 
as shown on Table: Monk Unarmed Damage.

\begin{table}[htb]
\rowcolors{1}{white}{offyellow}
\caption{Monk Unarmed Damage}
\centering
\begin{tabular}{l c c c}
\textbf{Level} & \textbf{Small} & \textbf{Medium} & \textbf{Large}\\
1st-3rd & 1d4 & 1d6 & 1d8 \\
4th-7th & 1d6 & 1d8 & 2d6 \\
8th-11th & 1d8 & 1d10 & 2d8 \\
12th-15th & 1d10 & 2d6 & 3d6 \\
16th-19th & 2d6 & 2d8 & 3d8 \\
20th & 2d8 & 2d10 & 4d8 \\
\end{tabular}
\end{table}

\textbf{Bonus Feat:} At 1st level, a monk may select either \linkfeat{Improved Grapple} or 
\linkfeat{Stunning Fist} as a bonus feat. At 2nd level, she may select either \linkfeat{Combat Reflexes} 
or \linkfeat{Deflect Arrows} as a bonus feat. At 6th level, she may select either \linkfeat{Improved Disarm}
or \linkfeat{Improved Trip} as a bonus feat. A monk need not have any of the prerequisites 
normally required for these feats to select them.

\textbf{Evasion (Ex):} At 2nd level or higher if a monk makes a successful Reflex 
saving throw against an attack that normally deals half damage on a successful 
save, she instead takes no damage. Evasion can be used only if a monk is wearing 
light armor or no armor. A helpless monk does not gain the benefit of evasion.

\textbf{Fast Movement (Ex):} At 3rd level, a monk gains an enhancement bonus to 
her speed, as shown on Table: The Monk. A monk in armor or carrying a medium or 
heavy load loses this extra speed.

\textbf{Still Mind (Ex):} A monk of 3rd level or higher gains a +2 bonus on saving 
throws against spells and effects from the school of enchantment.

\textbf{Ki Strike (Su):} At 4th level, a monk's unarmed attacks 
are empowered with \textit{ki}. Her unarmed attacks are treated as magic weapons 
for the purpose of dealing damage to creatures with damage reduction. Ki 
strike improves with the character's monk level. At 10th level, her unarmed attacks 
are also treated as lawful weapons for the purpose of dealing damage to creatures 
with damage reduction. At 16th level, her unarmed attacks are treated as adamantine 
weapons for the purpose of dealing damage to creatures with damage reduction and 
bypassing hardness.

\textbf{Slow Fall (Ex):} At 4th level or higher, a monk within arm's reach of a 
wall can use it to slow her descent. When first using this ability, she takes damage 
as if the fall were 20 feet shorter than it actually is. The monk's ability to 
slow her fall (that is, to reduce the effective distance of the fall when next 
to a wall) improves with her monk level until at 20th level she can use a nearby 
wall to slow her descent and fall any distance without harm.

\textbf{Purity of Body (Ex):} At 5th level, a monk gains immunity to all diseases 
except for supernatural and magical diseases.

\textbf{Wholeness of Body (Su):} At 7th level or higher, a monk can heal her own 
wounds. She can heal a number of hit points of damage equal to twice her current 
monk level each day, and she can spread this healing out among several uses.

\textbf{Improved Evasion (Ex):} At 9th level, a monk's evasion ability improves. 
She still takes no damage on a successful Reflex saving throw against attacks, 
but henceforth she takes only half damage on a failed save. A helpless monk does 
not gain the benefit of improved evasion.

\textbf{Diamond Body (Su):} At 11th level, a monk gains immunity to poisons of 
all kinds.

\textbf{Abundant Step (Su):} At 12th level or higher, a monk can slip magically 
between spaces, as if using the spell \linkspell{Dimension Door}, once per day. Her 
caster level for this effect is one-half her monk level (rounded down).

\textbf{Diamond Soul (Ex):} At 13th level, a monk gains spell resistance equal 
to her current monk level + 10. In order to affect the monk with a spell, a spellcaster 
must get a result on a caster level check (1d20 + caster level) that equals or 
exceeds the monk's spell resistance.

\textbf{Quivering Palm (Su):} Starting at 15th level, a monk can set up vibrations 
within the body of another creature that can thereafter be fatal if the monk so 
desires. She can use this quivering palm attack once a week, and she must announce 
her intent before making her attack roll. Constructs, oozes, plants, undead, incorporeal 
creatures, and creatures immune to critical hits cannot be affected. Otherwise, 
if the monk strikes successfully and the target takes damage from the blow, the 
quivering palm attack succeeds. Thereafter the monk can try to slay the victim 
at any later time, as long as the attempt is made within a number of days equal 
to her monk level. To make such an attempt, the monk merely wills the target to 
die (a free action), and unless the target makes a Fortitude saving throw (DC 10 
+ 1/2 the monk's level + the monk's Wis modifier), it dies. If the saving throw 
is successful, the target is no longer in danger from that particular quivering 
palm attack, but it may still be affected by another one at a later time.

\textbf{Timeless Body (Ex):} Upon attaining 17th level, a monk no longer takes 
penalties to her ability scores for aging and cannot be magically aged. Any such 
penalties that she has already taken, however, remain in place. Bonuses still accrue, 
and the monk still dies of old age when her time is up.

\textbf{Tongue of the Sun and Moon (Ex):} A monk of 17th level or higher can speak 
with any living creature.

\textbf{Empty Body (Su):} At 19th level, a monk gains the ability to assume an 
ethereal state for 1 round per monk level per day, as though using the spell \linkspell{Etherealness}. 
She may go ethereal on a number of different occasions during any single day, as 
long as the total number of rounds spent in an ethereal state does not exceed her 
monk level.

\textbf{Perfect Self:} At 20th level, a monk becomes a magical creature. She is 
forevermore treated as an outsider rather than as a humanoid (or whatever the monk's 
creature type was) for the purpose of spells and magical effects. Additionally, 
the monk gains damage reduction 10/magic, which allows her to ignore the first 
10 points of damage from any attack made by a nonmagical weapon or by any natural 
attack made by a creature that doesn't have similar damage reduction. Unlike other 
outsiders, the monk can still be brought back from the dead as if she were a member 
of her previous creature type.

%%%%%%%%%%%%%%%%%%%%%%%%%
\subsection{Ex-Monks}
%%%%%%%%%%%%%%%%%%%%%%%%%

A monk who becomes nonlawful cannot gain new levels as a monk but retains all monk 
abilities.

Like a member of any other class, a monk may be a multiclass character, but multiclass 
monks face a special restriction. A monk who gains a new class or (if already multiclass) 
raises another class by a level may never again raise her monk level, though she 
retains all her monk abilities.