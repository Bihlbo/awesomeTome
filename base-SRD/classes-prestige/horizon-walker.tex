%%%%%%%%%%%%%%%%%%%%%%%%%%%%%%%%%%%%%%%%%%%%%%%%%%
\classentry{Horizon Walker}
%%%%%%%%%%%%%%%%%%%%%%%%%%%%%%%%%%%%%%%%%%%%%%%%%%

%%%%%%%%%%%%%%%%%%%%%%%%%
\Requirements
%%%%%%%%%%%%%%%%%%%%%%%%%

To qualify to become a horizon walker, a character must fulfill all the following 
criteria.

\textbf{Skills:} \linkskill{Knowledge} (geography) 8 ranks.

\textbf{Feats:} \linkfeat{Endurance}.

%%%%%%%%%%%%%%%%%%%%%%%%%
\Basics
%%%%%%%%%%%%%%%%%%%%%%%%%

\textbf{Hit Die:} d8.

\textbf{Class Skills}

The horizon walker's class skills (and the key ability for each skill) are \linkskill{Balance} 
(Dex), \linkskill{Climb} (Str), \linkskill{Diplomacy} (Cha), \linkskill{Handle Animal} (Cha), \linkskill{Hide} (Dex), \linkskill{Knowledge} 
(geography) (Int), \linkskill{Listen} (Wis), \linkskill{Move Silently} (Dex), \linkskill{Profession} (Wis), \linkskill{Ride} (Dex), 
\linkskill{Speak Language} (none), \linkskill{Spot} (Wis), and \linkskill{Survival} (Wis). 

\textbf{Skill Points at Each Level:} 4 + Int modifier.

\begin{table}[htb]
\rowcolors{1}{white}{offyellow}
\caption{The Horizon Walker}
\centering
\begin{tabular}{*{6}{l}}
\textbf{Level} & \textbf{BAB} & \textbf{Fort} & \textbf{Reflex} & \textbf{Will} & \textbf{Special}\\
1st & +1 & +2 & +0 & +0 & Terrain Mastery\\
2nd & +2 & +3 & +0 & +0 & Terrain Mastery\\
3rd & +3 & +3 & +1 & +1 & Terrain Mastery \\
4th & +4 & +4 & +1 & +1 & Terrain Mastery \\
5th & +5 & +4 & +1 & +1 & Terrain Mastery \\
6th & +6 & +5 & +2 & +2 & Planar Terrain Mastery \\
7th & +7 & +5 & +2 & +2 & Planar Terrain Mastery \\
8th & +8 & +6 & +2 & +2 & Planar Terrain Mastery \\
9th & +9 & +6 & +3 & +3 & Planar Terrain Mastery \\
10th & +10 & +7 & +3 & +3 & Planar Terrain Mastery \\
\end{tabular}
\end{table}

%%%%%%%%%%%%%%%%%%%%%%%%%
\ClassFeatures
%%%%%%%%%%%%%%%%%%%%%%%%%

All of the following are features of the horizon walker prestige class.

\textbf{Weapon and Armor Proficiency:} Horizon walkers gain no proficiency with 
any weapon or armor.

\textbf{Terrain Mastery:} At each level, the Horizon Walker adds a new terrain 
environment to their repertoire from those given below. Terrain mastery gives a 
horizon walker a bonus on checks involving a skill useful in that terrain, or some 
other appropriate benefit. A horizon walker also knows how to fight dangerous creatures 
typically found in that terrain, gaining a +1 insight bonus on attack rolls and 
damage rolls against creatures with that terrain mentioned in the Environment entry 
of their descriptions. The horizon walker only gains the bonus if the creature 
description specifically lists the terrain type.

Horizon walkers take their terrain mastery with them wherever they go. They retain 
their terrain mastery bonuses on skill checks, attack rolls, and damage rolls whether 
they're actually in the relevant terrain or not.

\textbf{Planar Terrain Mastery:} Planar terrain mastery functions just like terrain 
mastery, except that the horizon walker can choose one of the planar categories 
at each level. The horizon walker can take a non-planar terrain type instead, if 
she wishes.

%%%
\subsubsection{Terrain Mastery Benefits}
%%%

\textbf{Aquatic:} You gain a +4 competence bonus on Swim checks, or a +10-foot 
bonus to your swim speed if you have one. You gain a +1 insight bonus on attack 
and damage rolls against aquatic creatures.

\textbf{Desert: }You resist effects that tire you. You are immune to fatigue, and 
anything that would cause you to become exhausted makes you fatigued instead. You 
gain a +1 insight bonus on attack and damage rolls against desert creatures.

\textbf{Forest:} You have a +4 competence bonus on Hide checks. You gain a +1 insight 
bonus on attack and damage rolls against forest creatures.

\textbf{Hills:} You gain a +4 competence bonus on Listen checks. You gain a +1 
insight bonus on attack and damage rolls against hills creatures.

\textbf{Marsh:} You have a +4 competence bonus on Move Silently checks. You gain 
a +1 insight bonus on attack and damage rolls against marsh creatures.

\textbf{Mountains:} You gain a +4 competence bonus on Climb checks, or a +10- foot 
bonus to your climb speed if you have one. You gain a +1 insight bonus on attack 
and damage rolls against mountain creatures.

\textbf{Plains}: You have a +4 competence bonus on Spot checks. You gain a +1 insight 
bonus on attack and damage rolls against plains creatures.

\textbf{Underground: }You have 60-foot darkvision, or 120-foot darkvision if you 
already had darkvision from another source. You gain a +1 insight bonus on attack 
and damage rolls against underground creatures.

\textbf{Fiery (Planar):} This kind of planar terrain mastery provides you with 
resistance to fire 20. You gain a +1 insight bonus on attack and damage rolls against 
outsiders and elementals with the fire subtype.

\textbf{Weightless (Planar):} You gain a +30-foot bonus to your fly speed on planes 
with no gravity or subjective gravity. You gain a +1 insight on attack and damage 
rolls against creatures native to the Astral Plane, the Elemental Plane of Air, 
and the Ethereal Plane.

\textbf{Cold (Planar):} This kind of planar terrain mastery provides you with resistance 
to cold 20. You gain a +1 insight bonus on attack and damage rolls against outsiders 
and elementals with the cold subtype.

\textbf{Shifting (Planar):} You instinctively anticipate shifts in the reality 
of the plane that bring you closer to your destination, giving you the spell-like 
ability to use \textit{dimension door }(as the spell cast at your character level) 
once every 1d4 rounds. You gain a +1 insight bonus on attack and damage rolls against 
outsiders and elementals native to a shifting plane.

\textbf{Aligned (Planar):} You have the instinctive ability to mimic the dominant 
alignment of the plane. You incur none of the penalties for having an alignment 
at odds with that of the plane, and spells and abilities that harm those of the 
opposite alignment don't affect you. You have the dominant alignment of the plane 
with regard to magic, but your behavior and any alignment-related Class Features 
you have are unaffected.

\textbf{Cavernous (Planar):} You gain tremorsense with a 30-foot range.

\textbf{Other (Planar):} If other planes are in use additional Planar Terrains 
can be created.
