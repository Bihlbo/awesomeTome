%%%%%%%%%%%%%%%%%%%%%%%%%
\skillentry{Use Magic Device}{(Cha; Trained Only)}
%%%%%%%%%%%%%%%%%%%%%%%%%

Use this skill to activate magic

\textbf{Check:} You can use this skill to read a spell or to activate a magic item. 
Use Magic Device lets you use a magic item as if you had the spell ability or class 
features of another class, as if you were a different race, or as if you were of 
a different alignment.

You make a Use Magic Device check each time you activate a device such as a wand. 
If you are using the check to emulate an alignment or some other quality in an 
ongoing manner, you need to make the relevant Use Magic Device check once per hour.

You must consciously choose which requirement to emulate. That is, you must know 
what you are trying to emulate when you make a Use Magic Device check for that 
purpose. The DCs for various tasks involving Use Magic Device checks are summarized 
on the table below.

\begin{table}[htb]
\rowcolors{1}{white}{offyellow}
\caption{Use Magic Device DCs}
\centering
\begin{tabular}{l c}
\textbf{Task} & \textbf{Use Magic Device DC}\\
Activate Blindly & 25\\
Decipher a written spell & 25 + spell level\\
Use a scroll & 20 + caster level\\
Use a wand & 20\\
Emulate a class feature & 20\\
Emulate an ability score & See Text\\
Emulate a race & 25\\
Emulate an alignment & 30\\
\end{tabular}
\end{table}

\textit{Activate Blindly:} Some magic items are activated by special words, thoughts, 
or actions. You can activate such an item as if you were using the activation word, 
thought, or action, even when you're not and even if you don't know it. You do 
have to perform some equivalent activity in order to make the check. That is, you 
must speak, wave the item around, or otherwise attempt to get it to activate. You 
get a special +2 bonus on your Use Magic Device check if you've activated the item 
in question at least once before. If you fail by 9 or less, you can't activate 
the device. If you fail by 10 or more, you suffer a mishap. A mishap means that 
magical energy gets released but it doesn't do what you wanted it to do. The default 
mishaps are that the item affects the wrong target or that uncontrolled magical 
energy is released, dealing 2d6 points of damage to you. This mishap is in addition 
to the chance for a mishap that you normally run when you cast a spell from a scroll 
that you could not otherwise cast yourself.

\textit{Decipher a Written Spell:} This usage works just like deciphering a written 
spell with the Spellcraft skill, except that the DC is 5 points higher. Deciphering 
a written spell requires 1 minute of concentration.

\textit{Emulate an Ability Score:} To cast a spell from a scroll, you need a high 
score in the appropriate ability (Intelligence for wizard spells, Wisdom for divine 
spells, or Charisma for sorcerer or bard spells). Your effective ability score 
(appropriate to the class you're emulating when you try to cast the spell from 
the scroll) is your Use Magic Device check result minus 15. If you already have 
a high enough score in the appropriate ability, you don't need to make this check.

\textit{Emulate an Alignment:} Some magic items have positive or negative effects 
based on the user's alignment. Use Magic Device lets you use these items as if 
you were of an alignment of your choice. You can emulate only one alignment at 
a time.

\textit{Emulate a Class Feature:} Sometimes you need to use a class feature to 
activate a magic item. In this case, your effective level in the emulated class 
equals your Use Magic Device check result minus 20.  This skill does not let you 
actually use the class feature of another class. It just lets you activate items 
as if you had that class feature. If the class whose feature you are emulating 
has an alignment requirement, you must meet it, either honestly or by emulating 
an appropriate alignment with a separate Use Magic Device check (see above).

\textit{Emulate a Race:} Some magic items work only for members of certain races, 
or work better for members of those races. You can use such an item as if you were 
a race of your choice. You can emulate only one race at a time.

\textit{Use a Scroll:} If you are casting a spell from a scroll, you have to decipher 
it first. Normally, to cast a spell from a scroll, you must have the scroll's spell 
on your class spell list. Use Magic Device allows you to use a scroll as if you 
had a particular spell on your class spell list. The DC is equal to 20 + the caster 
level of the spell you are trying to cast from the scroll. In addition, casting 
a spell from a scroll requires a minimum score (10 + spell level) in the appropriate 
ability. If you don't have a sufficient score in that ability, you must emulate 
the ability score with a separate Use Magic Device check (see above).

This use of the skill also applies to other spell completion magic items.

\textit{Use a Wand:} Normally, to use a wand, you must have the wand's spell on 
your class spell list. This use of the skill allows you to use a wand as if you 
had a particular spell on your class spell list. This use of the skill also applies 
to other spell trigger magic items, such as staffs.

\textbf{Action:} None. The Use Magic Device check is made as part of the action 
(if any) required to activate the magic item.

\textbf{Try Again:} Yes, but if you ever roll a natural 1 while attempting to activate 
an item and you fail, then you can't try to activate that item again for 24 hours.

\textbf{Special:} You cannot take 10 with this skill.

You can't aid another on Use Magic Device checks. Only the user of the item may 
attempt such a check.

If you have the \linkfeat{Magical Aptitude} feat, you get a +2 bonus on Use Magic Device checks.

\textbf{Synergy:} If you have 5 or more ranks in \linkskill{Spellcraft}, you get a +2 bonus 
on Use Magic Device checks related to scrolls.

If you have 5 or more ranks in \linkskill{Decipher Script}, you get a +2 bonus on Use Magic 
Device checks related to scrolls.

If you have 5 or more ranks in Use Magic Device, you get a +2 bonus to Spellcraft 
checks made to decipher spells on scrolls.
