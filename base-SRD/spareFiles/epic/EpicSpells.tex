%&LaTeX
% !TEX encoding = UTF-8 Unicode
\documentclass{article}
\usepackage[utf8x]{inputenc}
\usepackage[T1]{fontenc}
\usepackage{textcomp}

\usepackage{longtable}
\usepackage{multirow}

\newcommand{\tab}{\hspace{5mm}}

nt, and is licensed for public use under the terms 
of the Open Game License v1.0a.

{\LARGE EPIC SPELLS}

\vspace{12pt}
WHAT ARE EPIC SPELLS? 

Epic spells are spells developed from the ground up using a list of magical ingredients 
called seeds. Despite their power, epic spells still follow the basic rules for 
casting spells, except as specifically noted otherwise. Epic casters can manipulate 
the seeds of true magic, but knowing the seeds and how to manipulate them does 
not instantly grant ultimate power. Each epic spell must be laboriously developed 
before it can be used. 

\vspace{12pt}
ACQUIRING EPIC SPELLS 

A character with the Epic Spellcasting feat may start acquiring epic spells immediately. 
Using epic spells is a two-step procedure: development and spellcasting. 

\vspace{12pt}
EPIC SPELL DEVELOPMENT 

Before it can be cast, an epic spell must be developed. The process of development 
can be a time-consuming and expensive process. It is during development that a 
caster determines whether a given epic spell lies within his or her abilities or 
beyond them. The basis of that determination lies in an epic spell's Spellcraft 
DC. 

The easiest way to develop an epic spell is to use one already given. The description 
of each of these unique spells gives the amount of gold, time, and experience points 
required to develop the spell. If a character pays a spell's development cost, 
he or she develops (and thus knows) that spell. 

For information on developing an epic spell completely from scratch, see Developing 
Unique Epic Spells, below. 

\vspace{12pt}
EPIC SPELLCASTING 

Once an epic spell is developed, the caster knows the spell. A developed epic spell 
becomes an indelible part of the caster and may be prepared without a spellbook 
(if a wizard is the caster). Characters who cast spells spontaneously, such as 
sorcerers, can cast a developed epic spell by using any open epic spell slot. Druids, 
clerics, and similar spellcasters can likewise prepare epic spells using epic spell 
slots. 

A spellcaster can prepare or cast any epic spell he or she knows as many times 
per day as he or she has available epic spell slots. A spellcaster who can cast 
epic spells has a number of open epic spell slots per day equal to one-tenth his 
or her ranks in the Knowledge skill appropriate to the spell and the caster's class. 
Knowledge (arcana) is appropriate for arcane casters, and Knowledge (religion) 
or Knowledge (nature) is appropriate for divine casters. The rules for rest between 
casting a day's allotment of epic spells are the same as for rest required to prepare 
standard spells. If the caster doesn't use up a day's allotment of epic spell slots, 
the unused slots remain available whether or not the spellcaster receives appropriate 
rest. 

Even if the epic spell has been developed and an epic spell slot is available, 
successfully casting an epic spell isn't assured. The caster's Spellcraft skill 
modifier is vital for casting an epic spell. To cast an epic spell, a spellcaster 
makes a Spellcraft check against the epic spell's Spellcraft DC. If the check succeeds, 
the spell is cast. If the caster fails the check, the epic spell fizzles and the 
epic spell slot is used for the day. 

Because epic spells require Spellcraft checks, a spell is beyond the caster's ability 
if the final Spellcraft DC is greater than 20 + the spellcaster's Spellcraft modifier. 
Epic spells with DCs higher than 10 + the spellcaster's Spellcraft modifier are 
risky; a caster can take 10 when casting an epic spell, but he or she can't take 
20. When routinely casting epic spells, most spellcasters take 10 on their Spellcraft 
checks. 

\textbf{Epic Spell Levels:} Epic spells have no fixed level. However, for purposes 
of Concentration checks, spell resistance, and other possible situations where 
spell level is important, epic spells are all treated as if they were 10th-level 
spells. 

\textbf{Metamagic, Items, and Epic Spells:} Metamagic feats and other epic feats 
that manipulate normal spells cannot be used with epic spells. 

A character can't craft a magic item that casts an epic spell, regardless of whether 
the item is activated with spell completion, a spell trigger, a command word, or 
simple use. Only major artifacts, which are beyond the means of even epic characters 
to create, can possibly contain magic of this power. 

The saving throw against a character's epic spell has a DC of 20 + the character's 
relevant ability score modifier. It's possible to develop epic spells that have 
even higher DCs, however, by applying the appropriate factor. 

\vspace{12pt}
EPIC SPELL TERMS

\textbf{Epic Spell: }Spells that are different from common spells. Epic spells 
are usually custom-made. Epic spells do not take up normal spell slots, but instead 
are gained and used under a completely separate progression. 

\textbf{Epic Spell Slots: }A character must have an available epic spell slot to 
prepare or cast an epic spell, just as he or she needs a normal spell slot for 
a nonepic spell. A character doesn't gain epic spell slots by virtue of his or 
her level and class, however. A character gets one epic spell slot for every 10 
ranks he or she has in the relevant Knowledge skill. 

\textbf{Factor: }When creating an epic spell, a character can modify the basic 
use of a seed. Each modification is called a factor, and most factors increase 
the difficulty of casting the spell. 

\textbf{Mitigating Factor: }The opposite of a normal factor, a mitigating factor 
modifies the spell but makes it easier to cast. 

\textbf{Seed: }Every custom epic spell created by spellcasters begins with a base 
effect called a seed. Seeds are the fundamental building blocks of epic spells. 

\textbf{Spellcraft DC: }For epic spells, the Spellcraft DC is a measure of how 
difficult the spell is for a spellcaster to cast. It also measures how powerful 
an epic spell is. 

\vspace{12pt}
\subsection*{EPIC SPELL FORMULAS AND CALCULATIONS }

The following formulas are important to epic spellcasters. 

\textbf{Epic Spells Per Day: }Knowledge (arcana), Knowledge (religion), or Knowledge 
(nature) ranks ÷ 10 (round down). 

\textbf{To Cast an Epic Spell: }Spellcraft check (DC = epic spell's Spellcraft 
DC). 

\textbf{Level of an Epic Spell: }Epic spells are considered 10th level for the 
purpose of Concentration checks, spell resistance, and other determinations. 

\textbf{Saving Throw for an Epic Spell: }DC = 20 + key ability modifier. 

\vspace{12pt}
DISPELLING,EPIC SPELLS,AND ANTIMAGIC FIELD

A lucky nonepic spellcaster casting \textit{greater dispel magic }might be able 
to dispel an epic spell. The game mechanics do not change, and epic spells do not 
occupy any privileged position allowing them to resist being dispelled other than 
their presumably high caster level. Likewise, epic spells using the \textit{dispel 
}seed can dispel nonepic spells. Such epic spells use the same game mechanic: The 
check to dispel is 1d20 + a specified number (usually dispeller's level), and the 
DC is 11 + the spellcaster's level. 

\textit{Antimagic field }does not automatically suppress epic spells as it does 
standard spells. Instead, each time an epic spell is subject to an \textit{antimagic 
fiel}d, make a dispel check as a 20th-level caster (1d20 + 20). The epic spell 
has a DC of 11 + the epic spell's spellcaster level. If the suppression check is 
successful, the epic spell is suppressed like any other spell. If the dispel check 
is unsuccessful, the epic spell functions normally.

\vspace{12pt}
EPIC SPELL DESCRIPTIONS

Each epic spell description follows the same format used for 0- to 9th-level spells\textit{. 
}There are two additional entries for epic spells: Spellcraft DC and To Develop. 

\textbf{Spellcraft DC:} This is the DC of the Spellcraft check required to cast 
the epic spell. When casting an epic spell, the character gains a +5 bonus on his 
or her Spellcraft check if the base seed of the epic spell is from the character's 
arcane school specialty or primary psionic discipline. The character takes a -15 
penalty if the epic spell seed is from his or her prohibited arcane school.

\textbf{To Develop:} The first part of this entry shows the resources in gold, 
time, and experience points a character must expend to develop the spell shown. 
If the character expends the resources, he or she develops the spell if he or she 
has access to all the seeds. Spells containing the \textit{life }or \textit{heal 
}seed are typically only available to those with 24 or more ranks in Knowledge 
(religion) or Knowledge (nature). The rest of the development entry details the 
seeds and factors used to create the epic spell. This information is provided as 
an example for characters when they attempt to create and develop their own unique 
epic spells. 

\vspace{12pt}
\subsection*{Animus Blast }

Evocation [Cold] 

\textbf{Spellcraft DC:} 50 

\textbf{Components:} V, S 

\textbf{Casting Time:} 1 standard action 

\textbf{Range:} 300 ft. 

\textbf{Area:} 20-ft.-radius hemisphere burst 

\textbf{Duration:} Instantaneous 

\textbf{Saving Throw:} Reflex half 

\textbf{Spell Resistance:} Yes 

\textbf{To Develop:} 450,000 gp; 9 days; 18,000 XP. Seeds: \textit{energy }(DC 
19), \textit{animate dead }(DC 23). Factors: set undead type to skeleton (-12 DC), 
1-action casting time (+20 DC). 

When this spell is cast, enemies within range are dealt 10d6 points of cold damage. 
However, up to twenty of those victims that perish as a result of this blast are 
then instantly animated as Medium skeletons. These skeletons serve the character 
indefinitely. The character cannot exceed the normal limit for controlling undead 
through use of this spell, but other means that allow the character to exceed the 
normal limit for controlled undead work just as well with undead created with \textit{animus 
blas}t. 

\vspace{12pt}
\subsection*{Animus Blizzard }

Evocation [Cold] 

\textbf{Spellcraft DC:} 78 

\textbf{Components:} V, S 

\textbf{Casting Time:} 1 minute 

\textbf{Range:} 300 ft. 

\textbf{Area:} 20-ft.-radius hemisphere burst

\textbf{Duration:} Instantaneous 

\textbf{Saving Throw:} Reflex half 

\textbf{Spell Resistance:} Yes 

\textbf{To Develop:} 702,000 gp; 15 days; 28,080 XP. Seeds: \textit{energy }(DC 
19), \textit{animate dead }(DC 23). Factors: increase damage to 30d6 (+40 DC), 
set undead type to wight (-4 DC). 

When this spell is cast, enemies within range are dealt 30d6 points of cold damage. 
However, up to five victims that perish as a result of this blast are then instantly 
animated as wights. These five wights serve the character indefinitely. The character 
cannot exceed the normal limit for controlling undead through use of this spell, 
but other means that allow the character to exceed the normal limit for controlled 
undead work just as well with undead created with \textit{animus blizzar}d. 

\vspace{12pt}
\subsection*{Contingent Resurrection }

Conjuration (Healing) 

\textbf{Spellcraft DC:} 52 

\textbf{Components:} V, S , D F

\textbf{Casting Time:} 1 minute 

\textbf{Range:} Touch 

\textbf{Target:} You or creature touched 

\textbf{Duration:} Contingent until expended, then instantaneous 

\textbf{Saving Throw:} None (see text) 

\textbf{Spell Resistance:} Yes (harmless) 

\textbf{To Develop:} 468,000 gp; 10 days; 18,720 XP. Seed: \textit{life }(DC 27). 
Factor: activates when subject is slain (+25 DC). 

\textit{Contingent resurrection }returns the subject to life if he or she is slain. 
Once cast, the spell remains quiescent and does not activate until the trigger 
conditions have been met (but each day it remains untriggered, it uses up an epic 
spell slot, even if the character cast it on another creature). Once triggered, 
the spell is expended. If the subject is killed (the trigger), he or she is restored 
to life and complete health 1 minute later, so long as even a tiny bit of dust 
remains for \textit{contingent resurrection }to act upon. A shaft of light shines 
down from the heavens, illuminating the subject and everything within 20 feet. 
The creature is restored to full hit points, vigor, and health, with no loss of 
prepared spells. However, the subject loses one level (or 1 point of Constitution 
if the subject was 1st level). \textit{Contingent resurrection }does not work on 
a creature that has died of old age. 

\vspace{12pt}
\subsection*{Create Living Vault (Ritual)}

Conjuration (Creation) 

\textbf{Spellcraft DC:} 58 

\textbf{Components:} V, S, XP 

\textbf{Casting Time:} 100 days, 11 minutes 

\textbf{Range:} 0 ft. 

\textbf{Effect:} One living vault, 50 ft. by 50 ft. by 10 ft. 

\textbf{Duration:} Instantaneous 

\textbf{Saving Throw:} None 

\textbf{Spell Resistance:} None 

\textbf{To Develop:} 540,000 gp; 11 days; 21,600 XP. Seeds: \textit{animate }(DC 
25) large chunk of stone, \textit{fortify }(DC 27). Factors: allow vault to ``grow'' 
to proper size after created in 4d4 days (ad hoc +20 DC), increase HD of object 
by 92 (+184 DC), grant magical immunity (ad hoc +105 DC), increase damage reduction 
to 15 (+28 DC) and to /epic (+15 DC), make permanent (x5 DC). Mitigating factors: 
increase casting time by 10 minutes (-20 DC), increase casting time by 100 days 
(-200 DC), 9d6 backlash (-9 DC), seven additional casters contributing one epic 
spell slot (\-{}133 DC), burn 20,000 XP per epic caster (\-{}1,600 DC). 

The character creates a construct known as a living vault to protect and hide his 
or her treasures. Upon completion, the vault initially measures only 5 feet on 
a side, but it gradually increases to its proper size over the following 4d4 days. 
The vault is attuned to the character, allowing him or her alone entrance and egress 
in a manner similar to a \textit{dimension door }spell. When the character desires 
the vault to hide itself, he or she gives it a simple command. To summon the vault, 
the character may cast a \textit{sending }spell or arrange some other manner to 
contact it. 

\textit{XP Cost: }20,000 XP. 

\vspace{12pt}
\subsection*{Crown of Vermin }

Conjuration (Summoning) 

\textbf{Spellcraft DC:} 56 

\textbf{Components:} V, S 

\textbf{Casting Time:} 1 minute 

\textbf{Range:} Personal 

\textbf{Effect:} Aura of one thousand insects that surrounds you in a 10-ft.-radius 
spread 

\textbf{Duration:} 20 rounds (D) 

\textbf{Saving Throw:} None (see text) 

\textbf{Spell Resistance:} No 

\textbf{To Develop:} 504,000 gp; 11 days; 20,160 XP. Seeds: \textit{summon }(DC 
14), \textit{fortify }(DC 17). Factors: summon vermin mass instead of one creature 
(ad hoc +8 DC), grant damage reduction 1/epic (+15 DC), allow mass to move at your 
speed (ad hoc +2 DC), perfect control of vermin (ad hoc +2 DC). Mitigating factor: 
change range to personal (-2 DC). 

After casting \textit{crown of vermin, }one thousand venomous, biting and stinging 
spiders, scorpions, beetles, and centipedes erupt from the very air around the 
character. This swarm forms a living aura around the character to a radius of 10 
feet. The character is immune to his or her own \textit{crown of vermi}n. The swarm 
goes where the character goes at his or her speed, even if the character takes 
to the air or water (though water drowns the vermin after 1 full round of immersion, 
unless the spell is cast underwater, in which case aquatic or marine vermin answer 
the call and cannot leave the water). Each vermin in the \textit{crown of vermin 
}bites a creature who enters the area occupied by the effect (or the character 
forces the effect into an area occupied by another creature) for 1 point of damage, 
and then dies. Each victim takes enough points of damage to kill it, destroying 
that number of vermin in the process. Victims get a Reflex saving throw each round 
to avoid the full press, and if successful, take only 10d10 bites (and 10d10 points 
of damage). A total of 1,000 points of damage can be dealt to those who fall prey 
to the \textit{crown of vermi}n. The vermin have damage reduction 1/epic, so the 
vermin's natural weapons are treated as epic for the purpose of overcoming damage 
reduction. If there aren't enough vermin to kill all the creatures in the spell's 
effect, the creature with the fewest hit points is affected first, then the creature 
with the second fewest hit points, and so on. After all creatures that can be killed 
have been killed, any remaining damage is distributed among the survivors equally. 

The character has utter control over the vermin in his or her aura, and can force 
them into areas that would normally deter common vermin. The character can completely 
suppress his or her vermin aura as a free action so that no vermin are visible 
at all. The time that vermin are suppressed does not count toward the spell's duration. 
Alternatively, the character can roughly shape and move the vermin in any fashion 
he or she desires within the limits of the 10-foot-radius spread as a move-equivalent 
action. The vermin cannot be wrested from the character's control through any means. 
The vermin make all saving throws to avoid damaging effects using the character's 
base saving throw bonuses. They gain the character's spell resistance, if any, 
and they get saving throws against spells that would otherwise automatically slay 
vermin. A character can see through his or her \textit{crown of vermin }without 
difficulty, but gains one-half concealment against enemy attacks launched both 
outside and within the character's \textit{crown of vermi}n. 

\vspace{12pt}
\subsection*{Damnation }

Enchantment (Compulsion) [Teleportation] [Mind-Affecting] 

\textbf{Spellcraft DC:} 97 

\textbf{Components:} V, S, XP 

\textbf{Casting Time:} 1 standard action 

\textbf{Target:} Creature touched 

\textbf{Duration:} Instantaneous (20 hours for compulsion) 

\textbf{Saving Throw:} Will negates (see text) 

\textbf{Spell Resistance:} Yes 

\textbf{To Develop:} 873,000 gp; 18 days; 34,920 XP. Seeds: \textit{foresee }(to 
preview likely hellscape) (DC 17), \textit{transport }(DC 27), \textit{compel }(to 
keep target in hell) (DC 19). Factors: interplanar travel (+4 DC), unwilling target 
(+4 DC), 1-action casting time (+20 DC), +15 to DC of subject's save (+30 DC). 
Mitigating factor: burn 2,400 XP (-24 DC). 

The character sends his or her foe to hell. If the character succeeds at a melee 
touch attack, the target must succeed at a Will saving throw (DC = the standard 
epic spell DC + 15). If he or she fails this saving throw, he or she is sent straight 
to a layer of a lawful evil plane (or a chaotic evil plane, at the character's 
option) swarming with fiends. The subject will not willingly leave the plane for 
20 hours, believing that his or her predicament is a just reward for an ill-spent 
life. Even after the compulsion fades, he or she must devise his or her own escape 
from the plane. Unless the GM devises a specific location and scenario in the Nine 
Hells, the subject encounters a group of 1d4 pit fiends (or balors, if in a chaotic 
evil plane) every hour he or she spends in hell. 

\textit{XP Cost: }2,000 XP. 

\vspace{12pt}
\subsection*{Demise Unseen }

Necromancy (Death, Evil), Illusion (Figment) 

\textbf{Spellcraft DC:} 80 

\textbf{Components:} V, S 

\textbf{Casting Time:} 1 standard action 

\textbf{Range:} 300 ft. 

\textbf{Target:} One creature of up to 80 HD 

\textbf{Duration:} Instantaneous 

\textbf{Saving Throw:} Fort negates 

\textbf{Spell Resistance:} Yes 

\textbf{To Develop:} 738,000 gp; 15 days; 29,520 XP. Seeds: \textit{slay }(DC 25), 
\textit{animate dead }(DC 23), \textit{delude }(DC 14). Factors: change undead 
type to ghoul (-10 DC), apply to all five senses (+8 DC), 1-action casting time 
(+20 DC). 

The character instantly slays a single target and at the same moment animate the 
body so that it appears that nothing has happened to the creature. The target's 
companions (if any) do not immediately realize what has transpired. The target 
receives a Fortitude saving throw to survive the attack. If the save fails, the 
target remains in its exact position with no apparent ill effects. In reality, 
it is now a ghoul under the character's control. The target's companions notice 
nothing unusual about the state of the target until they interact with it, at which 
time each companion receives a Will saving throw to notice discrepancies. The ghoul 
serves the character indefinitely. The character cannot exceed the normal limit 
for controlling undead through use of this spell, but other means that allow the 
character to exceed the normal limit for controlled undead work just as well with 
undead created with \textit{demise unsee}n. 

\vspace{12pt}
\subsection*{Dire Winter }

Evocation [Cold] 

\textbf{Spellcraft DC:} 319 

\textbf{Components:} V, S , X P 

\textbf{Casting Time:} 1 minute 

\textbf{Range:} 1,000 ft. 

\textbf{Area:} 1,000-ft.-radius emanation 

\textbf{Duration:} 20 hours 

\textbf{Saving Throw:} None 

\textbf{Spell Resistance:} None 

\textbf{To Develop:} 2,871,000 gp; 58 days; 114,840 XP. Seed: \textit{energy }(emanate 
2d6 cold in 10-ft. radius) (DC 19). Factor: 100 times increase in base area (+400 
DC). Mitigating factor: burn 10,000 XP (-100 DC). 

The creature or object targeted emanates bitter cold to a radius of 1,000 feet 
for 20 hours. The emanated cold deals 2d6 points of damage per round against unprotected 
creatures (the target is susceptible if not magically protected or otherwise resistant 
to the energy). The intense cold freezes water out of the air, causing constant 
snowfall and wind. The snow and wind produce a blizzard effect within the area. 

\textit{XP Cost: }10,000 XP. 

\vspace{12pt}
\subsection*{Dragon Knight (Ritual)}

Conjuration (Summoning) [Fire] 

\textbf{Spellcraft DC:} 38 

\textbf{Components:} V, S, Ritual 

\textbf{Casting Time:} 1 standard action 

\textbf{Range:} 75 ft. 

\textbf{Effect:} One summoned adult red dragon 

\textbf{Duration:} 20 rounds (D) 

\textbf{Saving Throw:} None (see text) 

\textbf{Spell Resistance:} No 

\textbf{To Develop:} 342,000 gp; 7 days; 13,680 XP. Seed: \textit{summon }(DC 14). 
Factors: summon creature other than outsider (+10 DC), summon CR 14 creature (+24 
DC), 1-action casting time (+20 DC). Mitigating factor: two additional casters 
contributing 8th-level spell slots (-30 DC). 

This spell summons an adult red dragon. It appears where the character designates 
and acts immediately. It attacks the character's opponents to the best of its abilities 
(on the first round, it prefers to breathe fire on an enemy, if possible). The 
character can direct the dragon not to attack, to attack particular enemies, or 
to perform other actions. This is a ritual spell requiring two other spellcasters, 
each of which must contribute an unused 8th-level spell slot to the casting. 

\vspace{12pt}
\subsection*{Dragon Strike (Ritual)}

Conjuration (Summoning) [Fire] 

\textbf{Spellcraft DC:} 50 

\textbf{Components:} V, S, Ritual, XP 

\textbf{Casting Time:} 1 standard action 

\textbf{Range:} 75 ft. 

\textbf{Effect:} Ten summoned adult red dragons 

\textbf{Duration:} 20 rounds (D) 

\textbf{Saving Throw:} None (see text) 

\textbf{Spell Resistance:} No 

\textbf{To Develop:} 450,000 gp; 9 days; 18,000 XP. Seed: \textit{summon }(DC 14). 
Factors: summon creature other than outsider (+10 DC), summon CR 14 creature (+24 
DC), summon ten creatures (x10 DC), 1-action casting time (+20 DC). Mitigating 
factors: eleven additional casters contributing 9th-level spell slots (-187 DC), 
burn 2,000 XP per caster (-240 DC), 3d6 backlash (\-{}3 DC). 

This spell summons ten adult red dragons. They appear where the character designates 
and act immediately. They attack the character's opponents to the best of their 
abilities (on the first round, they all prefer to simultaneously breathe fire on 
an enemy, if possible). The character can direct the dragons not to attack, to 
attack particular enemies, or to perform other actions. 

\textit{XP Cost: }2,000 XP (per caster). 

\vspace{12pt}
\subsection*{Dreamscape }

Conjuration [Teleportation] 

\textbf{Spellcraft DC:} 29 

\textbf{Components:} V, S 

\textbf{Casting Time:} 1 minute 

\textbf{Range:} Touch 

\textbf{Target:} You and other touched willing creatures weighing up to 1,000 lb. 

\textbf{Duration:} Instantaneous (D) 

\textbf{Saving Throw:} Yes (harmless) (see text) S

pell Resistance: Yes (harmless) 

\textbf{To Develop:} 261,000 gp; 6 days; 10,400 XP. Seed: \textit{transport }(DC 
27). Factor: transport to region of dreams (+2 DC). 

The character and any creatures he or she touches are drawn into the region of 
dreams. The character can take more than one creature along (subject to the character's 
weight limit), but all must be touching each other. The character physically enters 
the land of dreams, leaving nothing behind. For every minute the character moves 
through the dream landscape, he or she can ``wake'' to find him or her self five 
miles displaced in the waking world. The character does not know precisely where 
he or she will come out in the waking world, nor the conditions of the waking world 
through which the character travels. The character knows approximately where he 
or she will end up based on time spent traveling in dream. \textit{Dreamscape }can 
also be used to travel to other planes that contain creatures that dream, but doing 
this requires crossing into the dreams of outsiders, where the character is subject 
to the dangers of alien dream realities. This is a potentially perilous proposition. 
Transferring to another plane of existence requires 1d4 hours of uninterrupted 
journey. Any creatures touched by the character when \textit{dreamscape }is cast 
also make the transition to the borders of unconscious thought. They may opt to 
follow the character, wander off into the dreams of others, or stumble back into 
the waking world (50\% chance for either of the latter results if they are lost 
or abandoned by the character). Creatures unwilling to accompany the character 
into the region of dreams receive a Will save, negating the effect if successful. 

\vspace{12pt}
\subsection*{Eclipse }

Conjuration (Creation) [Transportation]

\textbf{Spellcraft DC:} 42 

\textbf{Components:} V, S , X P 

\textbf{Casting Time:} 10 minutes 

\textbf{Range:} 200 miles 

\textbf{Area:} 5-mile radius, centered on you 

\textbf{Duration:} Up to 8 hours (D) 

\textbf{Saving Throw:} None 

\textbf{Spell Resistance:} No 

\textbf{To Develop:} 378,000 gp; 8 days; 15,1200 XP. Seeds: \textit{conjure }(DC 
21), \textit{transport }(to move disk into position 100 miles up) (DC 27). Factors: 
increase mass by 1,000\% (+40 DC), spread mass into paper-thin disk (ad hoc +2 
DC), keep disk in place for 8 hours (ad hoc +10 DC). Mitigating factors: increase 
casting time by 9 minutes (-18 DC), burn 4,000 XP (-40 DC). 

With this spell, the character can create a limited eclipse, as though a heavenly 
body moves between the sun and the earth. The landscape within a five-mile radius 
of the character's location experiences the dimming of the sun as a disk the character 
creates passes in front of it, culminating in a complete blackout and accompanying 
coronal ring. The eclipse follows the character across the landscape for up to 
8 hours, or until the sun goes down, or until the character dismisses the eclipse. 
The character does not need to concentrate on the eclipse while it lasts. 

\textit{XP Cost: }4,000 XP. 

\vspace{12pt}
\subsection*{Eidolon }

Conjuration (Creation) [Transportation]

\textbf{Spellcraft DC:} 79 

\textbf{Components:} V, S , X P 

\textbf{Casting Time:} 1 minute 

\textbf{Range:} 5 ft. 

\textbf{Effect:} One duplicate of caster 

\textbf{Duration:} 8 hours 

\textbf{Saving Throw:} None 

\textbf{Spell Resistance:} No 

\textbf{To Develop:} 711,000 gp; 15 days; 28,440 XP. Seed: \textit{conjure }(to 
make base substance) (DC 21), \textit{transform }(DC 21), \textit{transport }(to 
move part of caster's soul into duplicate) (DC 27). Factors: nonliving substance 
to humanoid (+10 DC), transform into specific individual (+25 DC). Mitigating factor: 
burn 2,500 XP (-25 DC). 

Upon casting \textit{eidolon, }the character creates a duplicate version of him 
or her self as the character was when he or she was a 21st-level character, and 
the character gains one negative level while the duplicate persists. For each additional 
negative level the character bestow upon him or her self at the time of casting, 
the eidolon has one additional character level. No matter how many negative levels 
the character bestows on him or her self, the eidolon can never have more character 
levels than the character has (taking the negative levels into account). Treat 
the duplicate as the character with a number of negative levels conferred that 
would lower him or her to the character level of the eidolon. The eidolon is considered 
fresh and rested when created. It may cast any spell the character has access to, 
including an epic spell. Use the eidolon's Spellcraft modifier as the basis for 
the number of epic spells it can cast in a day, and its effective character level 
as a basis for its skills, feats, and other abilities. The eidolon is effectively 
lower level than the character and probably can't cast all the spells he or she 
knows. A powerful enough eidolon might conceivably cast the \textit{eidolon }spell 
itself. The eidolon appears in whatever mundane clothing the character desires 
when initially conjured, but it has no other possessions. It shares part of the 
character's soul, so it is the character for all intents and purposes. The character 
and his or her \textit{eidolon }communicate with each other normally. Usually, 
the eidolon does not begrudge its brief existence, because it is still part of 
the character. If the eidolon is killed prior to the expiration of the spell's 
duration, the character immediately regains the lost levels. Normally, the eidolon 
does not last long enough to threaten the character with permanent level drain. 

\textit{XP Cost: }2,500 XP. 

\vspace{12pt}
\subsection*{Enslave (Ritual)}

Enchantment (Compulsion) [Mind-Affecting] 

\textbf{Spellcraft DC:} 80

\textbf{Components:} V, S, XP 

\textbf{Casting Time:} 1 standard action 

\textbf{Range:} 75 ft. 

\textbf{Target:} One living creature 

\textbf{Duration:} Permanent

\textbf{Saving Throw:} Will negates 

\textbf{Spell Resistance:} Yes 

\textbf{To Develop:} 720,000 gp; 15 days; 28,800 XP. Seed: compel (DC 19). Factors: 
stricter compulsion of any creature (ad hoc +11 DC), 1-action casting time (+20 
DC), permanent (x5 DC). Mitigating factors: 2d6 backlash (-2 DC), four additional 
casters contributing one 9th-level spell slot (-68 DC), burn 2,000 XP per caster 
(-100 DC).

The character makes a permanent thrall of any living creature. The character establishes 
a telepathic link with the subject's mind. If the creature has a language, the 
character can generally force the subject to perform as he or she desires, within 
the limits of its abilities. If the creature has no language, the character can 
communicate only basic commands. The character knows what the subject is experiencing, 
but does not receive direct sensory input from it. A subject forced to take an 
action against its nature receives a saving throw with a penalty of -10 to resist 
taking that particular action, but if it succeeds, it still remains the character's 
thrall despite its minor mutiny. Once a subject makes a successful saving throw 
to resist a specific order, it makes all future saving throws to resist taking 
that specific action without a penalty. \textit{Protection from evil }or a similar 
spell can prevent the character from exercising control or using the telepathic 
link while the subject is so protected, but it does not prevent the establishment 
of \textit{enslave }or dispel it. 

\textit{XP Cost: }2,000 XP. 

\vspace{12pt}
\subsection*{Epic Counterspell }

Abjuration 

\textbf{Spellcraft DC:} 69 

\textbf{Components:} V, S 

\textbf{Casting Time:} 1 standard action 

\textbf{Range:} 300 ft. 

\textbf{Target:} One creature or object 

\textbf{Duration:} Instantaneous 

\textbf{Saving Throw:} None 

\textbf{Spell Resistance:} No 

\textbf{To Develop:} 621,000 gp; 13 days; 24,840 XP. Seed: \textit{dispel }(DC 
19). Factors: +30 to dispel check (+30 DC), 1-action casting time (+20 DC). 

To use \textit{epic counterspel}l, select an opponent as the target. The character 
does this by readying an action, electing to wait to complete his or her action 
until the opponent tries to cast a spell (the character may still move his or her 
speed, because readying a counterspell is a standard action). If the target tries 
to cast a spell, make a dispel check: Roll d20+40 against a DC of 11 + the foe's 
caster level. If the check is successful, the character's spell negates the foe's 
spell. 

\vspace{12pt}
\subsection*{Epic Mage Armor }

Conjuration (Creation) [Force] 

\textbf{Spellcraft DC:} 46 

\textbf{Components:} V, S 

\textbf{Casting Time:} 1 minute 

\textbf{Range:} Touch 

\textbf{Target:} Creature touched 

\textbf{Duration:} 24 hours (D) 

\textbf{Saving Throw:} Will negates (harmless) 

\textbf{Spell Resistance:} Yes (harmless) 

\textbf{To Develop:} 414,000 gp; 9 days; 16,560 XP. Seed: \textit{armor }(DC 14). 
Factor: +16 additional armor bonus (+32 DC). 

An invisible but tangible field of force surrounds the subject of \textit{epic 
mage armor, }providing a +20 armor bonus to Armor Class. Unlike mundane armor, 
\textit{epic mage armor }entails no armor check penalty, arcane spell failure chance, 
or speed reduction. Because \textit{epic mage armor }is made of force, incorporeal 
creatures can't bypass it the way they do normal armor. 

\vspace{12pt}
\subsection*{Epic Repulsion }

Abjuration 

\textbf{Spellcraft DC:} 52 

\textbf{Components:} V, S 

\textbf{Casting Time:} 10 minutes 

\textbf{Range:} Touch 

\textbf{Target:} Object or creature touched 

\textbf{Duration:} Permanent 

\textbf{Saving Throw:} None 

\textbf{Spell Resistance:} Yes 

\textbf{To Develop:} 468,000 gp; 10 days; 18,720 XP. Seed: \textit{ward }(DC 14). 
Factor: permanent $($5 DC). Mitigating factor: increase casting time by 9 minutes 
(-18 DC). 

The character can create a permanent ward against a specific creature type. Any 
creature of the specified type cannot attack or touch the warded creature or object. 
The protection ends if the warded creature makes an attack against or intentionally 
moves to within 5 feet of a specified creature. Spell resistance can allow a creature 
to overcome this protection and touch the warded creature. 

\vspace{12pt}
\subsection*{Epic Spell Reflection }

Abjuration

\textbf{Spellcraft DC:} 68 

\textbf{Components:} V, S , X P 

\textbf{Casting Time:} 41 days, 11 minutes 

\textbf{Range:} Touch 

\textbf{Target:} Object or creature touched 

\textbf{Duration:} Permanent 

\textbf{Saving Throw:} None 

\textbf{Spell Resistance:} Yes 

\textbf{To Develop:} 630,000 gp; 13 days; 25,200 XP. Seed: reflect (DC 27). Factors: 
reflect up to 9th-level spells (+160 DC), change range to touch (+2 DC), permanent 
(x5 DC). Mitigating factors: increase casting time by 10 minutes (-20 DC), increase 
casting time by 41 days (-82 DC), 20d6 backlash (-20 DC), six additional casters 
contributing one 8th-level spell slot (-90 DC), burn 9,500 XP per caster (-665 
DC).

The character can create a permanent ward against all spells of 1st through 9th 
level that target the subject. These spells are reflected back on the caster. Spells 
that affect an area are not affected by this spell. 

\textit{XP Cost: }9,500 XP. 

\vspace{12pt}
\subsection*{Eternal Freedom }

Abjuration 

\textbf{Spellcraft DC:} 150 

\textbf{Components:} V, S, Ritual, XP 

\textbf{Casting Time:} 1 minute 

\textbf{Range:} Touch 

\textbf{Target:} Touched creature or object of 2,000 lb. or less 

\textbf{Duration:} Permanent 

\textbf{Saving Throw:} Will negates 

\textbf{Spell Resistance:} Yes 

\textbf{To Develop:} 1,350,000 gp; 27 days; 54,000 XP. Seed: \textit{ward }(DC 
14). Factors: specific protections \textit{entangle }(+0 DC), \textit{hold monster 
}(+8 DC), \textit{hold person }(+4 DC), \textit{imprisonment }(+16 DC), paralysis 
(ad hoc +6 DC), petrification (ad hoc +6 DC), \textit{sleep }(+0 DC), \textit{slow 
}(+4 DC), stunning (ad hoc +6 DC), \textit{temporal stasis }(+16 DC), and \textit{web 
}(+4 DC); permanent $($5 DC). Mitigating factors: ten additional casters contributing 
9th-level spell slots (-170 DC), burn 10,000 XP (-100 DC). 

The subject becomes permanently immune to the following specific spells, effects, 
and spell-like abilities: \textit{entangl}e, \textit{hol}d, \textit{imprisonmen}t, 
paralysis, petrification, \textit{slee}p, \textit{slo}w, stunning, \textit{temporal 
stasi}s, and \textit{we}b. This is a ritual spell requiring ten other spellcasters, 
each of whom must contribute an unused 9th-level spell slot to the casting. 

\textit{XP Cost: }10,000 XP. 

\vspace{12pt}
\subsection*{Greater Spell Resistance }

Transmutation 

\textbf{Spellcraft DC:} 45 

\textbf{Components:} V, S, Ritual 

\textbf{Casting Time:} 1 minute 

\textbf{Range:} Touch 

\textbf{Target:} Creature touched 

\textbf{Duration:} 20 hours 

\textbf{Saving Throw:} Will negates (harmless) 

\textbf{Spell Resistance:} Yes (harmless) 

\textbf{To Develop:} 405,000 gp; 9 days; 16,200 XP. Seed: \textit{fortify }(DC 
27). Factor: +10 additional SR (+40 DC). Mitigating factor: two additional casters 
contributing 6th-level spell slots (-22 DC). 

The character grants the subject touched spell resistance 35 until the duration 
expires. The spell resistance granted does not stack, but overlaps with, any previous 
spell resistance. This is a ritual spell, requiring two other spellcasters, each 
of whom must contribute an unused 6th-level spell slot to the casting. 

\vspace{12pt}
\subsection*{Greater Ruin }

Transmutation 

\textbf{Spellcraft DC:} 59 

\textbf{Components:} V, S , X P 

\textbf{Casting Time:} 1 standard action

\textbf{Range:} 12,000 ft. 

\textbf{Target:} One creature, or up to a 10-foot cube of nonliving matter 

\textbf{Duration:} Instantaneous 

\textbf{Saving Throw:} Fortitude half 

\textbf{Spell Resistance:} Yes 

\textbf{To Develop:} 531,000 gp; 11 days; 21,240 XP. Seed: \textit{destroy }(DC 
29). Factors: increase damage to 35d6 (+30 DC), 1-action casting time (+20 DC).Mitigating 
factor: burn 2,000 XP (-20 DC). 

The character deals 35d6 points of damage to a single target within range and line 
of sight. If the target is reduced to -10 hit points or less (or a construct, object, 
or undead is reduced to 0 hit points), it is utterly destroyed as if disintegrated. 
Only a trace of fine dust remains. \textit{XP Cost: }2,000 XP.

\vspace{12pt}
\subsection*{Hellball }

Evocation [Acid, Fire, Electricity, Sonic] 

\textbf{Spellcraft DC:} 90 

\textbf{Components:} V, S , X P 

\textbf{Casting Time:} 1 standard action 

\textbf{Range:} 300 ft. 

\textbf{Area:} 40-ft.-radius spread 

\textbf{Duration:} Instantaneous 

\textbf{Saving Throw:} Reflex half 

\textbf{Spell Resistance:} Yes 

\textbf{To Develop:} 810,000 gp; 17 days; 32,400 XP. Seed: \textit{energy }(deals 
10d6 each of acid, fire, electricity, sonic) (DC 76). Factors: double base area 
(+6 DC), 1-action casting time (+20 DC). Mitigating factors: 10d6 backlash (-10 
DC), burn 200 XP (-2 DC).

A \textit{hellball} deals 10d6 points of acid damage, 10d6 points of fire damage, 
10d6 points of electricity damage, and 10d6 points of sonic damage to all creatures 
within the area. Unattended objects also take this damage. The character takes 
10d6 points of damage upon casting (in addition to burning 200 XP). The character 
points his or her finger and determine the range (distance and height) at which 
the \textit{hellball }is to detonate. A sun-bright, fist-sized globe of energy 
streaks forth and, unless it impacts a material body or solid barrier prior to 
attaining the indicated range, expands into its full area.

\textit{XP Cost: 2}00 XP. 

\vspace{12pt}
\subsection*{Kinetic Control }

Abjuration 

\textbf{Spellcraft DC:} 100 

\textbf{Components:} V, S 

\textbf{Casting Time:} 1 minute 

\textbf{Range:} Personal; touch 

\textbf{Target:} You; creature or object touched 

\textbf{Duration:} 12 hours or until discharged 

\textbf{To Develop:} 927,000 gp; 19 days; 37,080 XP. Seeds: \textit{ward }(5 points 
against bludgeoning and piercing) (DC 14), \textit{reflect }(DC 27). Factors: also 
against slashing (+4 DC), additional 15 points of protection (+30 DC), contingent 
reflection of damage on creature touched (+25 DC). 

Once a character has cast this spell, he or she can absorb, store, and redirect 
the energy contained in any physical (melee or ranged) attack. The character absorbs 
20 points of each separate slashing, bludgeoning, and piercing attack made against 
him or her, saving it for later. A character can absorb up to 150 points of damage 
in this fashion; however, if the stored damage is not discharged prior to reaching 
the 150-point limit, the spell automatically discharges, dealing the 150 points 
of damage to the character. The character keeps track of the number of points of 
damage he or she has absorbed (the character doesn't have to keep track of the 
type of damage). At any time during the spell's duration, the character can make 
a touch attack against another creature or object. If successful, the character 
deals the target some or all (character's choice) of the points of damage he or 
she has stored. The damage delivered is considered bludgeoning damage. A character 
can absorb and discharge damage any number of times during the spell's duration, 
so long as the character doesn't absorb more than 150 points at a time. When the 
spell expires, any stored damage the character has not redirected is discharged 
into the character. 

\vspace{12pt}
\subsection*{Let Go of Me }

Transmutation 

\textbf{Spellcraft DC:} 43 

\textbf{Components:} None 

\textbf{Casting Time:} 1 free action 

\textbf{Range:} Touch (see text) 

\textbf{Target:} One creature or force grappling you 

\textbf{Duration:} Instantaneous 

\textbf{Saving Throw:} Fortitude half 

\textbf{Spell Resistance:} Yes 

\textbf{To Develop:} 387,000 gp; 8 days; 15,480 XP. Seed: \textit{destroy }(DC 
29). Factors: quickened (+28 DC), no verbal or somatic components (+4 DC). Mitigating 
factors: limited circumstance (ad hoc -8 DC), 10d6 backlash (-10 DC). 

The character deals 20d6 points of damage to any creature grappling him or her. 
The damage dealt is of no particular type or energy---it is a purely destructive 
impulse. If grappled by a magical force the force is automatically destroyed.

\vspace{12pt}
\subsection*{Living Lightning }

Evocation [Electricity] 

\textbf{Spellcraft DC:} 140 

\textbf{Components:} None 

\textbf{Casting Time:} 1 standard action 

\textbf{Range:} 300 ft. or 150 ft. 

\textbf{Area:} A bolt 5 ft. wide by 300 ft. long, or 10 ft. wide by 150 ft. long 

\textbf{Duration:} Instantaneous 

\textbf{Saving Throw:} Reflex half 

\textbf{Spell Resistance:} Yes 

\textbf{To Develop:} 1,260,000 gp; 26 days; 50,400 XP. Seeds: \textit{life }(DC 
27), \textit{energy }(DC 19). Factors: 1-action casting time (+20 DC), no verbal 
or somatic components (+4 DC), giving life to a spell (ad hoc x2 DC). 

The character develops a spell that he or she can cast; thereafter, the spell can 
effectively ``cast itself.'' When the character casts the spell, a stroke of energy 
deals 10d6 points of electricity damage to each creature within the spell's area. 
\textit{Living lightning }follows all the standard rules for epic spell casting 
when the character casts it. \textit{Living lightning }is sentient and generally 
friendly toward the character. It has the character's mental ability scores, but 
it has no physical ability scores. It senses the world through the character's 
senses and communicates with him or her by thought. As a self-triggering spell, 
it isn't truly alive but is a fragment of the character's personality. It cares 
little for the world around it, but at the character's urging (and sometimes at 
its own discretion) it casts itself at his or her foes. Casters who prepare spells 
before casting must prepare \textit{living lightning }normally in order for it 
to cast itself. When the spell casts itself, it acts on the character's initiative 
but does not count against his or her own actions in the round. The character cannot 
simultaneously cast \textit{living lightning }while it is casting its own effect, 
even if it has been prepared more than once. \textit{Living lightning }uses up 
one of the character's epic spell slots for the day whenever it casts itself. When 
the character has used up all his or her epic spell slots for the day (or has cast 
all his or her prepared \textit{living lightning }spells, if a caster who must 
prepare spells), \textit{living lightning }becomes quiescent. It remains so until 
the character has rested to regain his or her epic spell slots for the next day. 

\vspace{12pt}
\subsection*{Lord of Nightmares }

Conjuration (Summoning) 

\textbf{Spellcraft DC:} 50 

\textbf{Components:} V, S, X P 

\textbf{Casting Time:} 1 standard action 

\textbf{Range:} 75 ft. 

\textbf{Effect:} One summoned creature 

\textbf{Duration:} 20 rounds 

\textbf{Saving Throw:} None 

\textbf{Spell Resistance:} No 

\textbf{To Develop:} 450,000 gp; 9 days; 18,000 XP. Seed: \textit{summon }(DC 14). 
Factors: summon CR 31 creature (+58 DC), allow creature to possess body and act 
at its own discretion (ad hoc -20 DC), 1-action casting time (+20 DC). Mitigating 
factors: 12d6 backlash (-12 DC), burn 1,000 XP (-10 DC). 

The character is possessed by a dream larva. For 20 rounds, the dream larva's body 
physically replaces the character's, though the dream larva has the character's 
equipment. The dream larva is free to call on all its own powers and abilities, 
or use the character's equipment. The character's consciousness and physical form 
are suppressed for the duration of the possession. The character has no way to 
dismiss the spell, communicate, or otherwise maintain awareness once possession 
has commenced. The dream larva, temporarily freed from its imprisonment in some 
distant nightmare, will attempt to slay and incapacitate any creature it can see 
or find, whether it is a friend or foe of the caster. Casting \textit{lord of nightmares 
}entails some risk for the caster, since it's unknown what a dream larva might 
do over the course of 20 rounds. The larva will dispatch all enemies it can find 
before turning to its own concerns. Sometimes a dream larva will attempt to place 
itself in a dangerous or precarious situation prior to the end of the spell, leaving 
the caster to extricate him or herself. If the dream larva is slain during the 
duration of the spell, the character's consciousness is instantly restored to aware-ness 
within his or her own body. The character's condition remains what it was when 
he or she completed casting \textit{lord of nightmare}s, regardless of what damage 
the dream larva received. However, magic item charges used, potions consumed, and 
other physical resources used up by the dream larva are permanent. 

\textit{XP Cost: }1,000 XP. 

\vspace{12pt}
\subsection*{Mass Frog }

Transmutation 

\textbf{Spellcraft DC:} 55 

\textbf{Components:} V, S 

\textbf{Casting Time:} 1 standard action 

\textbf{Range:} 300 ft. 

\textbf{Area:} 40-ft.-radius hemisphere 

\textbf{Duration:} Permanent 

\textbf{Saving Throw:} Fortitude negates 

\textbf{Spell Resistance:} Yes 

\textbf{To Develop:} 495,000 gp; 10 days; 19,800 XP. Seed: \textit{transform }(DC 
21). Factors: change target to area of 20-ft. hemisphere (+10 DC), increase area 
by 100\% (+4 DC), 1 action casting time (+20 DC). 

This epic spell turns all Medium or smaller creatures in the area into frogs. The 
transformed creatures retain their mental faculties, including personality, Intelligence, 
Wisdom, and Charisma scores, level and class, hit points (despite any change in 
Consitution score), alignment, base attack bonus, base save bonuses, extraordinary 
abilities, spells, and spell-like abilities, but not supernatural abilities. They 
assume the physical characteristics of frogs, including natural size and Strength, 
Dexterity, and Constitution scores. (Use the statistics for the toad.) All the 
creatures' equipment drops to the ground upon transformation. 

\vspace{12pt}
\subsection*{Momento Mori }

Necromancy [Death] 

\textbf{Spellcraft DC:} 86 

\textbf{Components:} None 

\textbf{Casting Time:} 1 quickened action 

\textbf{Range:} 300 ft. 

\textbf{Target:} One living creature 

\textbf{Duration:} Instantaneous 

\textbf{Saving Throw:} Fortitude partial (see text) 

\textbf{Spell Resistance:} Yes 

\textbf{To Develop:} 774,000 gp; 16 days; 30,960 XP. Seed: \textit{slay }(DC 25). 
Factor: increase to 160 HD (+8 DC), quickened (+28 DC), no verbal or somatic components 
(+4 DC), +10 to DC of subject's save (+20 DC). 

As a free action that counts as a quickened spell, the character wills the target 
dead without a word or gesture. The character's thought snuffs out the life force 
of a living creature of 160 or fewer HD, killing it instantly. The subject is entitled 
to a Fortitude saving throw (DC 30 + relevant ability modifier) to have a chance 
of surviving the attack. If the save is successful, the target instead takes 3d6+20 
points of damage. 

\vspace{12pt}
\subsection*{Mummy Dust }

Necromancy [Evil] 

\textbf{Spellcraft DC:} 35 

\textbf{Components:} V, S ,M, XP 

\textbf{Casting Time:} 1 standard action 

\textbf{Range:} Touch 

\textbf{Effect:} Two 18-HD mummies 

\textbf{Duration:} Instantaneous 

\textbf{Saving Throw:} None 

\textbf{Spell Resistance:} No 

\textbf{To Develop:} 315,000 gp; 7 days; 12,600 XP. Seed: \textit{animate dead 
}(DC 23). Factors: 1-action casting time (+20 DC). Mitigating factors: burn 400 
XP (-4 DC), expensive material component (ad hoc -4 DC). 

When the character sprinkles the dust of ground mummies in conjunction with casting 
\textit{mummy dus}t, two Large 18-HD mummies (see below) spring up from the dust 
in an area adjacent to the character. The mummies follow the character's every 
command according to their abilities, until they are destroyed or the character 
loses control of them by attempting to control more Hit Dice of undead than he 
or she has caster levels. 

\textit{Material Component: }Specially prepared mummy dust (10,000 gp). 

\textit{XP Cost: }2,000 XP. 

Mummy, Advanced: CR 8; Large undead; HD 18d12+3; hp 120; Init -1; Spd 20 ft.; AC 
20, touch 8, flat-footed 20; Base Atk +9; Grp +24; Atk +20 melee (1d8+16 plus mummy 
rot); Full Atk +20 melee (1d8+16 plus mummy rot); Space/Reach 10 ft./10 ft.; SA 
Despair, mummy rot; SQ Damage reduction 5/-, darkvision 60 ft., undead traits, 
vulnerability to fire; AL LE; SV Fort +8, Ref +7, Will +13; Str 32, Dex 8, Con 
--, Int 6, Wis 14, Cha 15. \textit{Skills and Feats:} Hide -5, Listen +9, Move 
Silently +10, Spot +9; Alertness, Blind-Fight, Great Fortitude, Lightning Reflexes, 
Power Attack, Toughness, Weapon Focus (slam).

Despair (Su): At the sight of a mummy, the viewer must succeed at a Will save (DC 
21), or be paralyzed with fear for 1d4 rounds. Whether or not the save is successful, 
that creature cannot be affected again by that mummy's despair ability for one 
day. Mummy Rot (Su): Supernatural disease---slam, Fortitude save (DC 21), incubation 
period 1 minute; damage 1d6 Con and 1d6 Cha. The save DC is Charisma-based. Unlike 
normal diseases, mummy rot continues until the victim reaches Constitution 0 (and 
dies) or is cured as described below.  Mummy rot is a powerful curse, not a natural 
disease. A character attempting to cast any conjuration (healing) spell on a creature 
afflicted with mummy rot must succeed on a DC 20 caster level check, or the spell 
has no effect on the afflicted character. To eliminate mummy rot, the curse must 
first be broken with \textit{break enchantment }or \textit{remove curse }(requiring 
a DC 20 caster level check for either spell), after which a caster level check 
is no longer necessary to cast healing spells on the victim, and the mummy rot 
can be magically cured as any normal disease.

An afflicted creature who dies of mummy rot shrivels away into sand and dust that 
blow away into nothing at the first wind.

\vspace{12pt}
\subsection*{Nailed to the Sky }

Conjuration [Teleportation] 

\textbf{Spellcraft DC:} 62 

\textbf{Components:} V, S ,XP 

\textbf{Casting Time:} 1 standard action 

\textbf{Range:} 300 ft. 

\textbf{Target:} Creature or object weighing up to 1,000 lb. 

\textbf{Duration:} Instantaneous 

\textbf{Saving Throw:} Will negates 

\textbf{Spell Resistance:} Yes 

\textbf{To Develop:} 558,000 gp; 12 days; 22,320 XP. Seeds: \textit{foresee }(to 
preview endpoint of teleportation) (DC 17), \textit{transport }(DC 27). Factors: 
unwilling target (+4 DC), increase range from touch (+4 DC), 1-action casting time 
(+20 DC). Mitigating factor: burn 1,000 XP (-10 DC). 

\textit{Nailed to the sky }actually places the target so far from the surface of 
the world and at such a speed that it keeps missing the surface as it falls back, 
so it enters an eternal orbit. Unless the target can magically fly or has some 
other form of non-physical propulsion available, the target is stuck until someone 
else rescues it. Even if the target can fly, the surface is 2 to 4 hours away, 
assuming a \textit{fly }spell, which allows a maximum speed of 720 feet per round 
while descending. The target may not survive that long. Depending on the world 
where \textit{nailed to the sky }is cast, conditions so far from its surface may 
be deadly. Deleterious effects include scorching heat, cold, and vacuum. Targets 
subject to these conditions take 2d6 points of damage each from heat or cold and 
1d4 points of damage from the vacuum each round. The target immediately begins 
to suffocate. 

\textit{XP Cost: }1,000 XP. 

\vspace{12pt}
\subsection*{Origin of Species: Achaierai }

Conjuration (Creation, Healing)

\textbf{Spellcraft DC:} 38 

\textbf{Components:} V, S, DF, XP 

\textbf{Casting Time:} 100 days, 11 minutes 

\textbf{Range:} 0 ft. 

\textbf{Effect:} One constructed creature up to Medium (20 cu. ft.) 

\textbf{Duration:} Permanent 

\textbf{Saving Throw:} None 

\textbf{Spell Resistance:} No 

\textbf{To Develop:} 360,000 gp; 8 days; 14,400 XP. Seeds: conjure (DC 21), life 
(DC 27), fortify (DC 17). Factors: +4 HD (5 hp per HD) (+20 DC), +6 to natural 
AC (+12 DC), add three more natural attacks (ad hoc +6 DC), add black cloud spell-like 
ability (+33 DC), add SR 19 (+15 DC), permanent (x5 DC). Mitigating factors: 50d6 
backlash (-50 DC), increase casting time by 10 minutes (-20 DC), increase casting 
time by 100 days (-200 DC), burn 10,000 XP (-100 DC), eleven additional casters 
contributing 9th-level spell slots (-187 DC), ten additional casters contributing 
8th-level spell slots (-150 DC), ten additional casters contributing 1st-level 
spell slots (-10 DC).

This spell creates a new creature: an achaierai. When first created, the achaierai 
is Medium, but it grows to Large size in 1d4 days. A created achaierai does not 
possess the treasure, culture, or specific knowledge of a normal achaierai. If 
released to be among its own kind, it quickly picks up achaierai traits and alignment. 

\textit{XP Cost: }10,000 XP. 

\vspace{12pt}
\subsection*{Peripety }

Abjuration 

\textbf{Spellcraft DC:} 27 

\textbf{Components:} V, S 

\textbf{Casting Time:} 1 minute 

\textbf{Range:} Personal 

\textbf{Target:} You 

\textbf{Duration:} 12 hours 

\textbf{To Develop:} 243,000 gp; 5 days; 9,720 XP. Seed: \textit{reflect }(DC 27). 

Ranged attacks targeted against the character rebound on the original attacker. 
Any time during the duration, five attacks are automatically reflected back on 
the original attacker; the character decides which attacks before damage is rolled. 
The reflected attack rebounds on the attacker using the same attack roll. Once 
five attacks are so reflected, the spell ends. 

\vspace{12pt}
\subsection*{Pestilence }

Conjuration, Necromancy 

\textbf{Spellcraft DC:} 104 

\textbf{Components:} V, S, Ritual, XP 

\textbf{Casting Time:} 10 minutes 

\textbf{Range:} 0 ft. 

\textbf{Area:} 1,000-ft.-radius hemisphere 

\textbf{Duration:} Instantaneous 

\textbf{Saving Throw:} Fortitude negates 

\textbf{Spell Resistance:} Yes 

\textbf{To Develop:} 936,000 gp; 19 days; 37,440 XP. Seed: afflict (DC 19). Factors: 
additional target type (plants) (+10 DC). change target to area (+10 DC), change 
20-ft. radius to 1,000-ft. radius (+200 DC), disease effects (as per contagion 
spell) (ad hoc +21 DC). Mitigating factors: casting time increased by 9 minutes 
(-18 DC), two additional casters contributing epic spell slots (-38 DC), burn 10,000 
XP (-100 DC).

When \textit{pestilence }is successfully cast, a wave of illness radites outward 
from the site of the ritual, instantly infecting every living thing in the area 
with the debilitating disease known as slimy doom. Within 24 hours, everything 
in the area begins to show signs of rot and decay.

Each day that a victim fails a Fortitude save, it takes 1d4 points of temporary 
Constitution damage. If the victim then fails a second save, 1 point of that damage 
is permanent drain. If the victim succeeds at the first saving throw of the day 
on consecutive days, he or she has recovered from the disease. This magical form 
of the disease is not contagious and will not spread beyond those initially infected. 
Fruits and vegetables infected with slimy doom are unfit for consumption, as are 
disease-ridden livestock. This is a ritual spell requiring two other spellcasters, 
each of whom must expend an unused epic spell slot for the casting. The primary 
caster must also burn 10,000 XP. 

\textit{XP Cost: }10,000 XP. 

\vspace{12pt}
\subsection*{Rain of Fire }

Evocation [Fire] 

\textbf{Spellcraft DC:} 50 

\textbf{Components:} V, S 

\textbf{Casting Time:} 1 minute 

\textbf{Range:} 0 ft. 

\textbf{Area:} 2-mile-radius emanation 

\textbf{Duration:} 20 hours 

\textbf{Saving Throw:} Reflex negates (see text) 

\textbf{Spell Resistance:} Yes 

\textbf{To Develop:} 450,000 gp; 9 days; 18,000 XP. Seeds: \textit{energy (fire) 
}(DC 19), \textit{energy (weather) }(DC 19). Factor: change rain to wisps of flame 
(ad hoc +12 DC). 

This spell summons a swirling thunderstorm that rains fire rather than raindrops 
down on the character and everything within a two-mile radius of him or her. Everything 
caught unprotected or unsheltered in the flaming deluge takes 1 point of fire damage 
each round. A successful Reflex save results in no damage, but the save must be 
repeated each round. Unless the ground is exceedingly damp, all vegetation is eventually 
blackened and destroyed, leaving behind a barren wasteland similar to the aftermath 
of a grass or forest fire. The fiery storm is stationary and persists even if the 
caster leaves. 

\vspace{12pt}
\subsection*{Raise Island }

Conjuration (Creation) 

\textbf{Spellcraft DC:} 38 

\textbf{Components:} V, S, XP, Ritual

\textbf{Casting Time:} 65 days, 11 minutes 

\textbf{Range:} 0 ft. 

\textbf{Area:} 100-ft.-radius hemispherical island 

\textbf{Duration:} Permanent

\textbf{Saving Throw:} None 

\textbf{Spell Resistance:} No 

\textbf{To Develop:} 360,000 gp; 8 days; 14,400 XP. Seed: conjure (DC 21). Factors: 
change area to 10-ft. radius, 30-ft. high cylinder (+2 DC), change radius to 100 
ft. (+40 DC), change height to 1,000 feet (+133 DC), permanent (x5 DC). Mitigating 
factors: increase casting time by 10 minutes (-20 DC), increase casting time by 
65 days (-130 DC), nineteen additional casters contributing epic spell slots (-361 
DC), one additional caster contributing one 6th-level spell slot (-11 DC), burn 
2,000 XP per epic caster (-400 DC), spell only works on liquid (ad hoc -20 DC).

The character can literally raise a new island from out of the sea, bringing to 
the surface a sandy or rocky but otherwise barren protrusion that is solid, stable, 
and permanently established. The island is roughly circular and about 200 feet 
in diameter. \textit{Raise island }only works if the ocean is less than 1,000 feet 
deep where the spell is cast.

\textit{XP Cost: }2,000 XP. 

\vspace{12pt}
\subsection*{Ruin }

Transmutation 

\textbf{Spellcraft DC:} 27 

\textbf{Components:} V, S, X P 

\textbf{Casting Time:} 1 full round 

\textbf{Range:} 12,000 ft. 

\textbf{Target:} One creature, or up to a 10-foot cube of nonliving matter 

\textbf{Duration:} Instantaneous 

\textbf{Saving Throw:} Fortitude half 

\textbf{Spell Resistance:} Yes 

\textbf{To Develop:} 243,000 gp; 5 days; 9,720 XP. Seed: \textit{destroy }(DC 29). 
Factor: reduce casting time by 9 rounds (+18 DC). Mitigating factor: burn 2,000 
XP (-20 DC). 

The character deals 20d6 points of damage to a single target within range and line 
of sight. If the target is reduced to -10 hit points or less (or a construct, object, 
or undead is reduced to 0 hit points), it is utterly destroyed as if disintegrated. 
Only a trace of fine dust remains. 

\textit{XP Cost: }2,000 XP. 

\vspace{12pt}
\subsection*{Safe Time }

Conjuration [Teleportation] 

\textbf{Spellcraft DC:} 64 

\textbf{Components:} V, S 

\textbf{Casting Time:} 1 minute 

\textbf{Range:} Touch 

\textbf{Target:} You or creature touched 

\textbf{Duration:} Contingent until expended, then 1 round of safe time 

\textbf{Saving Throw:} None 

\textbf{Spell Resistance:} No 

\textbf{To Develop:} 576,000 gp; 12 days; 23,040 XP. Seed: \textit{transport }(DC 
27). Factors: move to time stream (+8 DC), reduce static time to 1 round (ad hoc 
+4 DC), activates when you would otherwise take 50 or more points of damage (+25 
DC). 

\textit{Safe time }can move the character (or the target) out of harm's way by 
shunting him or her into a static time stream. Once cast, the spell remains quiescent 
and does not activate until the trigger conditions have been met. Each day it remains 
untriggered, it uses up an epic spell slot, even if you cast it on another creature. 
Once triggered, the spell is expended normally. When the character would otherwise 
be subject to any instantaneous effect that would deal him or her 50 or more points 
of damage, he or she is instead transported to a static time stream where time 
ceases to flow. The character's condition becomes fixed---no force or effect can 
harm him or her until 1 round of real time has passed. Thus, the character avoids 
the damage he or she would otherwise receive, but the character also misses out 
on one round of activity. To the character, no time passes at all, but to onlookers 
who are part of real time, the character stands frozen and fixed in space for 1 
full round. 

\vspace{12pt}
\subsection*{Soul Dominion }

Divination, Enchantment (Compulsion) [Mind-Affecting] 

\textbf{Spellcraft DC:} 72 

\textbf{Components:} V, S 

\textbf{Casting Time:} 10 minutes 

\textbf{Range:} See text 

\textbf{Target:} One other living creature 

\textbf{Duration:} 20 minutes (D) 

\textbf{Saving Throw:} Will negates (see text) 

\textbf{Spell Resistance:} No 

\textbf{To Develop:} 648,000 gp; 13 days; 25,920 XP. Seeds: \textit{contact }(DC 
23), \textit{reveal }(DC 19), \textit{compel }(DC 19). Factors: apply to all five 
senses (+8 DC), total compulsory control (+10 DC), stricter compulsion of any creature 
(ad hoc +11 DC). Mitigating factor: increase casting time by 9 minutes (-18 DC). 

When a character casts this spell, he or she is temporarily able to take control 
of another sentient creature with whom the character is familiar (by meeting, observing, 
or successfully scrying the subject). The target receives a Will save, and if successful, 
prevents the character from making the telepathic connection. The target is aware 
of the attempted takeover as a strange, momentary tingling. If the Will save fails, 
the character is able to control the subject's body as if it were his or her own, 
hearing, seeing, feeling, smelling, and tasting everything the target senses. Once 
the character dismisses the spell or its duration ends, the target resumes control 
of its body, fully aware of all events that occurred, having been a helpless witness 
trapped inside its own body. The target knows the name and general nature of its 
possessor if it succeeds at an additional Will saving throw. A character cannot 
control undead or incorporeal creatures with \textit{soul dominio}n.

\vspace{12pt}
\subsection*{Soul Scry }

Divination 

\textbf{Spellcraft DC:} 55 

\textbf{Components:} V, S 

\textbf{Casting Time:} 10 minutes 

\textbf{Range:} See text 

\textbf{Target:} One other living creature 

\textbf{Duration:} 20 minutes (D) 

\textbf{Saving Throw:} Will negates 

\textbf{Spell Resistance:} No 

\textbf{To Develop:} 495,000 gp; 10 days; 19,800 XP. Seeds: \textit{contact }(DC 
23), \textit{reveal }(DC 19), \textit{conceal }(DC 17). Factors: apply to all five 
senses (+8 DC), conceal detection (ad hoc +6 DC). Mitigating factor: increase casting 
time by 9 minutes (-18 DC). 

When a character casts this spell, he or she is temporarily able to tap the consciousness 
of another sentient creature with whom the character is familiar (by meeting, observing, 
or successfully scrying the subject), experiencing everything he or she does with 
all five senses. The target receives a Will save, and if successful, prevents the 
character from making the telepathic connection. Whether the saving throw is successful 
or not, the target is unaware of the attempted intrusion. Once the subject is tapped, 
the character is able to hear, see, feel, smell, and taste everything the subject 
senses. The character cannot control the subject, however. The character can only 
see what the subject chooses to look at, and the character tastes something only 
if the subject eats or drinks it during the spell's duration. During this time, 
the character's own body remains in a trance-like state. If the subject takes damage, 
the character senses the injuries, although his or her own body does not actually 
suffer any ill effects. If the subject is knocked unconscious or killed, the spell 
immediately ends. 

\vspace{12pt}
\subsection*{Spell Worm }

Enchantment (Compulsion) [Mind-Affecting] 

\textbf{Spellcraft DC:} 45 

\textbf{Components:} V, S 

\textbf{Casting Time:} 1 standard action 

\textbf{Range:} 75 ft. 

\textbf{Target:} One living creature 

\textbf{Duration:} 20 hours or until completed 

\textbf{Saving Throw:} Will negates 

\textbf{Spell Resistance:} Yes 

\textbf{To Develop:} 405,000 gp; 9 days; 16,200 XP. Seed: \textit{compel }(DC 19). 
Factors: unobtrusive (ad hoc +6 DC), 1-action casting time (+20 DC). 

On a failed save, the subject must spend a standard action each round abandoning 
his or her highest-level spell (or losing his or her highest-level unused spell 
slot). Each round, the subject eliminates another spell or spell slot, moving to 
lower-level spells once all the higher-level spells are gone. In the case of prepared 
spells, the subject decides which spells to abandon at each level. If the subject 
has more than one standard action allowed in the round, he or she may spend those 
actions as he or she desires. The subject doesn't realize the spells or spell slots 
are gone until he or she tries to cast a spell and finds it unavailable. Abandoning 
a spell slot or losing a spell is standard action, but it does not draw an attack 
of opportunity. It is a purely mental exercise not obvious to observers. 

\vspace{12pt}
\subsection*{Summon Behemoth }

Conjuration (Summoning) 

\textbf{Spellcraft DC:} 72 

\textbf{Components:} V, S 

\textbf{Casting Time:} 1 standard action 

\textbf{Range:} 75 ft. 

\textbf{Effect:} Summoned creature 

\textbf{Duration:} 20 rounds (D) 

\textbf{Saving Throw:} None 

\textbf{Spell Resistance:} No 

\textbf{To Develop:} 648,000 gp; 13 days; 25,920 XP. Seed: \textit{summon }(DC 
14). Factors: summon CR 21 creature (DC +38), 1-action casting time (+20 DC). 

The character can summon a behemoth to attack his or her enemies. It appears where 
the character designates and acts immediately, on the character's turn. It attacks 
the character's opponents to the best of its ability. If the character can communicate 
with the creature, he or she can direct it not to attack, to attack particular 
enemies, or to perform other actions. Summoned creatures act normally on the last 
round of the spell and disappear at the end of their turn. 

\vspace{12pt}
\subsection*{Superb Dispelling }

Abjuration 

\textbf{Spellcraft DC:} 59 

\textbf{Components:} V, S 

\textbf{Casting Time:} 1 standard action 

\textbf{Range:} 300 ft. 

\textbf{Target:} One creature or object 

\textbf{Duration:} Instantaneous 

\textbf{Saving Throw:} None 

\textbf{Spell Resistance:} No 

\textbf{To Develop:} 531,000 gp; 11 days; 21,240 XP. Seed: \textit{dispel }(DC 
19). Factors: additional +30 to dispel check (+30 DC), 1-action casting time (+20 
DC). Mitigating factor: 10d6 backlash (-10 DC). 

As \textit{greater dispel magic}, except that the maximum bonus on the dispel check 
is +40, and the character takes 10d6 points of backlash damage. 

\vspace{12pt}
\subsection*{Time Duplicate }

Conjuration [Teleportation] 

\textbf{Spellcraft DC:} 71 

\textbf{Components:} V, S 

\textbf{Casting Time:} 1 free action 

\textbf{Effect:} You 

\textbf{Duration:} 1 round (see text) 

\textbf{Saving Throw:} None (harmless) 

\textbf{Spell Resistance:} None (harmless) 

\textbf{To Develop:} 639,000 gp; 13 days; 25,560 XP. Seed: \textit{transport }(to 
move future you back in time 1 round) (DC 27). Factors: move to time stream (+8 
DC), stretch the base temporal effect (ad hoc +8 DC), quickened (+28 DC). 

The character snatches him or her self from 1 round in the future, depositing this 
future self in an adjacent space as a free action that counts as a quickened spell. 
The character's future self is technically only a possible future self (the time 
stream is a maelstrom of multiple probabilities), but snatching that future self 
from 1 round in the future collapses probability, and the possible future becomes 
the definite future. The character and his or her future self are both free to 
act normally this round (the character has already used up the limit of one quickened 
spell per round, but his or her duplicate hasn't). The future self has all the 
resources the character has at the moment he or she finishes casting \textit{time 
duplicat}e. Because the future self was previously only a possibility, his or her 
resources are not depleted as a result of whatever might occur this round (even 
if the character dies this round). Likewise, he or she doesn't have any special 
knowledge of what might occur during this round. Because the future self is still 
part of the time stream, the round it spends with the character is a round it misses 
in its own future. Because the chracter's future duplicate is also the character, 
the character misses the next round as well. He or she simply isn't there. Tampering 
with the time stream is a tricky business. Here is a round-by-round summary. 

\textit{Round One: }The character casts \textit{time duplicat}e, the future self 
from round two arrives, and both act normally. 

\textit{Round Two: }The future self---the character---gets snatched back in time 
to help the past self. During this round, there are no versions of the character 
present. 

\textit{Round Three: }The character rejoins the time stream. The character arrives 
in the same location and condition that the future self ended with at the end of 
the first round. Any resources (spells, damage, staff charges) the future self 
used up in round one are gone for real. Record them now. Using this spell to snatch 
a single future self stretches time and probability to its limit; more powerful 
versions of \textit{time duplicate }are not possible. A character cannot bring 
more than a single future version of him or her self back at one time, nor can 
a character snatch a version of him or her from farther in the future. 

\vspace{12pt}
\subsection*{Vengeful Gaze of God }

Transmutation 

\textbf{Spellcraft DC:} 419 

\textbf{Components:} V, S 

\textbf{Casting Time:} 1 standard action 

\textbf{Range:} 12,000 ft. 

\textbf{Target:} One creature, or up to a 10-foot cube of nonliving matter in line 
of sight 

\textbf{Duration:} Instantaneous 

\textbf{Saving Throw:} Fortitude half 

\textbf{Spell Resistance:} Yes 

\textbf{To Develop:} 3,771,000 gp; 76 days; 150,840 XP. Seed: \textit{destroy }(DC 
29). Factor: increase damage to 305d6 (+570 DC), 1-action casting time (+20 DC). 
Mitigating factor: 200d6 backlash (-200 DC). 

The target of this spell is subject to 305d6 points of damage (or half of that 
if a Fortitude save succeeds). If the target is reduced to -10 hit points or less 
(or a construct, object, or undead is reduced to 0 hit points), it is utterly destroyed 
as if disintegrated, leaving behind only a trace of fine dust. The caster is likewise 
dealt 200d6 points of damage

\vspace{12pt}
\subsection*{Verdigris }

Conjuration (Creation) 

\textbf{Spellcraft DC:} 58 

\textbf{Components:} V, S 

\textbf{Casting Time:} 1 minute 

\textbf{Range:} 300 ft. 

\textbf{Area:} 100-ft.-radius hemisphere 

\textbf{Duration:} 24 hours 

\textbf{Saving Throw:} Reflex half 

\textbf{Spell Resistance:} No 

\textbf{To Develop:} 522,000 gp; 11 days; 20,880 XP. Seed: \textit{conjure }(DC 
21). Factors: change area to 20-ft. radius hemisphere (+2 DC), increase radius 
to 100 ft. (+16 DC), deal 10d6 damage during growth (ad hoc +19 DC). 

This spell creates a tsunami of grass, shrubs, and trees that overgrows the area 
like a tidal wave. The plant growth creeps and curls across every-thing in the 
area, ensnaring it and coiling around it as if it had been growing there for a 
century or more. Creatures in the area must make a Reflex saving throw to avoid 
the fast-moving growth, which otherwise deals 10d6 points of damage from the crushing 
press. Buildings are engulfed and they likewise take 10d6 points of damage. Those 
destroyed by the damage have their foundations uprooted and walls crumbled. The 
plant growth remains for 24 hours, after which it vanishes. 

\vspace{12pt}
\subsection*{Verdigris Tsunami }

Conjuration (Creation) 

\textbf{Spellcraft DC:} 170

\textbf{Components:} V, S, Ritual, XP 

\textbf{Casting Time:} 10 minutes 

\textbf{Range:} 1,500 ft. 

\textbf{Area:} 1,000-ft.-radius hemisphere 

\textbf{Duration:} Permanent 

\textbf{Saving Throw:} Reflex half 

\textbf{Spell Resistance:} No 

\textbf{To Develop:} 1,530,000 gp; 31 days; 61,200 XP. Seed: conjure (DC 21). Factor: 
change area to 20-ft.-radius hemisphere (+2 DC), increase radius to 1,000 ft. (+196 
DC), increase range to 1,500 ft. (+8 DC), deal 10d6 damage during growth (ad hoc 
+19 DC), increase damage to 40d6 (+60 DC), permanent (x5 DC). Mitigating factors: 
increase casting time by 9 minutes (-18 DC), eleven additional casters contributing 
6th-level spell slots (-121 DC), three additional casters contributing 4th-level 
spell slots (-21 DC), burn 10,000 XP per 6th-level spell contributor plus caster 
(-1,200 DC).

This spell creates a tsunami of grass, shrubs, and trees that overgrows the area 
like a tidal wave. The plant growth creeps and curls across every-thing in the 
area, ensnaring it and coiling around it as if it had been growing there for a 
century or more. Creatures in the area must make a Reflex saving throw to avoid 
the fast-moving growth, which otherwise deals 40d6 points of damage from the crushing 
press. Buildings are engulfed and they likewise take 40d6 points of damage. Those 
destroyed by the damage have their foundations uprooted and walls crumbled. The 
plant growth is permanent. This is a ritual spell requiring fourteen other spellcasters, 
each of whom must contribute an unused 6th-level spell slot to the casting. 

\textit{XP Cost: }10,000 XP. 

\vspace{12pt}
{\LARGE DEVELOPING EPIC SPELLS }

An epic spell is developed from smaller pieces called seeds and connecting pieces 
called factors. Every epic seed has a base Spellcraft DC, and every factor has 
a Spell-craft DC adjustment. When a desired spell is developed, the spellcaster 
spends resources and time to assemble the pieces that make up the epic spell. The 
base Spellcraft DCs of each seed are added together; then the DC adjustments of 
the factors are added to that total. The sum equals the final Spellcraft DC for 
the epic spell. 

The final Spellcraft DC is the most significant gauge of the epic spell's power. 
A spellcaster attempts to cast an epic spell by making a Spellcraft check against 
the epic spell's Spellcraft DC. Thus, a spellcaster knows immediately, based on 
his or her own Spellcraft bonus, what epic spells are within his or her capability 
to cast, which are risky, and which are beyond him or her. Epic casters don't commit 
time and money to develop epic spells until they are powerful enough to cast them. 

An epic spell developed by an arcane spellcaster is arcane, and an epic spell developed 
by a divine spellcaster is divine. A character who can cast both divine and arcane 
epic spells chooses whether a particular spell he or she develops will be arcane 
or divine. If that same caster uses the \textit{heal }or \textit{life }seed in 
an epic spell, that spell is always considered divine. All the epic spells described 
here can be developed independently by a character who spends the necessary time, 
money, and experience points. Alternatively, a character can use those spells as 
a starting point when creating customized versions of the spells. 

\vspace{12pt}
\subsection*{\textbf{Table: Epic Seeds }}

\subsection*{\begin{longtable}{llllllll}
\hline
% ROW 1
\multicolumn{1}{|p{0.951in}|}{\begin{minipage}[t]{0.951in}\raggedright
\textbf{Seed}\end{minipage}} & \multicolumn{1}{p{0.934in}|}{\begin{minipage}[t]{0.934in}\centering
\textbf{Base Spellcraft DC}\end{minipage}} & \multicolumn{1}{p{1.098in}|}{\begin{minipage}[t]{1.098in}\centering
\textbf{Seed}\end{minipage}} & \multicolumn{1}{p{0.777in}|}{\begin{minipage}[t]{0.777in}\centering
\textbf{Base Spellcraft DC}\end{minipage}}\\
\hline
% ROW 2
\multicolumn{1}{p{0.069in}|}{\begin{minipage}[t]{0.069in}\centering
\textit{Afflict}\end{minipage}} & \multicolumn{1}{p{0.069in}|}{\begin{minipage}[t]{0.069in}\centering
14\end{minipage}} & \multicolumn{1}{p{0.069in}|}{\begin{minipage}[t]{0.069in}\centering
\textit{Energy }\end{minipage}} & \multicolumn{1}{p{0.069in}|}{\begin{minipage}[t]{0.069in}\centering
19\end{minipage}}\\
\hline
% ROW 3
\multicolumn{1}{|p{0.951in}|}{\begin{minipage}[t]{0.951in}\centering
\textit{Animate }\end{minipage}} & \multicolumn{1}{p{0.934in}|}{\begin{minipage}[t]{0.934in}\centering
25\end{minipage}} & \multicolumn{1}{p{1.098in}|}{\begin{minipage}[t]{1.098in}\centering
\textit{Foresee }\end{minipage}} & \multicolumn{1}{p{0.777in}|}{\begin{minipage}[t]{0.777in}\centering
17\end{minipage}}\\
\hline
% ROW 4
\multicolumn{1}{p{0.069in}|}{\begin{minipage}[t]{0.069in}\centering
\textit{Animate dead}\end{minipage}} & \multicolumn{1}{p{0.069in}|}{\begin{minipage}[t]{0.069in}\centering
23\end{minipage}} & \multicolumn{1}{p{0.069in}|}{\begin{minipage}[t]{0.069in}\centering
\textit{Fortify }\end{minipage}} & \multicolumn{1}{p{0.069in}|}{\begin{minipage}[t]{0.069in}\centering
17\end{minipage}}\\
\hline
% ROW 5
\multicolumn{1}{|p{0.951in}|}{\begin{minipage}[t]{0.951in}\centering
\textit{Armor }\end{minipage}} & \multicolumn{1}{p{0.934in}|}{\begin{minipage}[t]{0.934in}\centering
14\end{minipage}} & \multicolumn{1}{p{1.098in}|}{\begin{minipage}[t]{1.098in}\centering
\textit{Heal*}\end{minipage}} & \multicolumn{5}{p{1.055in}|}{\begin{minipage}[t]{1.055in}\centering
25\end{minipage}}\\
\hline
% ROW 6
\multicolumn{1}{p{0.069in}|}{\begin{minipage}[t]{0.069in}\centering
\textit{Banish }\end{minipage}} & \multicolumn{1}{p{0.069in}|}{\begin{minipage}[t]{0.069in}\centering
27\end{minipage}} & \multicolumn{1}{p{0.069in}|}{\begin{minipage}[t]{0.069in}\centering
\textit{Life*}\end{minipage}} & \multicolumn{1}{p{0.069in}|}{\begin{minipage}[t]{0.069in}\centering
27\end{minipage}}\\
\hline
% ROW 7
\multicolumn{1}{|p{0.951in}|}{\begin{minipage}[t]{0.951in}\centering
\textit{Compel }\end{minipage}} & \multicolumn{1}{p{0.934in}|}{\begin{minipage}[t]{0.934in}\centering
19\end{minipage}} & \multicolumn{1}{p{1.098in}|}{\begin{minipage}[t]{1.098in}\centering
\textit{Reflect }\end{minipage}} & \multicolumn{5}{p{1.055in}|}{\begin{minipage}[t]{1.055in}\centering
27\end{minipage}}\\
\hline
% ROW 8
\multicolumn{1}{|p{0.951in}|}{\begin{minipage}[t]{0.951in}\centering
\textit{Conceal }\end{minipage}} & \multicolumn{1}{p{0.934in}|}{\begin{minipage}[t]{0.934in}\centering
17\end{minipage}} & \multicolumn{1}{p{1.098in}|}{\begin{minipage}[t]{1.098in}\centering
\textit{Reveal }\end{minipage}} & \multicolumn{5}{p{1.055in}|}{\begin{minipage}[t]{1.055in}\centering
19\end{minipage}}\\
\hline
% ROW 9
\multicolumn{1}{|p{0.951in}|}{\begin{minipage}[t]{0.951in}\centering
\textit{Conjure }\end{minipage}} & \multicolumn{1}{p{0.934in}|}{\begin{minipage}[t]{0.934in}\centering
21\end{minipage}} & \multicolumn{1}{p{1.098in}|}{\begin{minipage}[t]{1.098in}\centering
\textit{Slay }\end{minipage}} & \multicolumn{5}{p{1.055in}|}{\begin{minipage}[t]{1.055in}\centering
25\end{minipage}}\\
\hline
% ROW 10
\multicolumn{1}{|p{0.951in}|}{\begin{minipage}[t]{0.951in}\centering
\textit{Contact }\end{minipage}} & \multicolumn{1}{p{0.934in}|}{\begin{minipage}[t]{0.934in}\centering
23\end{minipage}} & \multicolumn{1}{p{1.098in}|}{\begin{minipage}[t]{1.098in}\centering
\textit{Summon }\end{minipage}} & \multicolumn{5}{p{1.055in}|}{\begin{minipage}[t]{1.055in}\centering
14\end{minipage}}\\
\hline
% ROW 11
\multicolumn{1}{|p{0.951in}|}{\begin{minipage}[t]{0.951in}\centering
\textit{Delude }\end{minipage}} & \multicolumn{1}{p{0.934in}|}{\begin{minipage}[t]{0.934in}\centering
14\end{minipage}} & \multicolumn{1}{p{1.098in}|}{\begin{minipage}[t]{1.098in}\centering
\textit{Transform }\end{minipage}} & \multicolumn{5}{p{1.055in}|}{\begin{minipage}[t]{1.055in}\centering
21\end{minipage}}\\
\hline
% ROW 12
\multicolumn{1}{|p{0.951in}|}{\begin{minipage}[t]{0.951in}\centering
\textit{Destroy }\end{minipage}} & \multicolumn{1}{p{0.934in}|}{\begin{minipage}[t]{0.934in}\centering
29\end{minipage}} & \multicolumn{1}{p{1.098in}|}{\begin{minipage}[t]{1.098in}\centering
\textit{Transport }\end{minipage}} & \multicolumn{5}{p{1.055in}|}{\begin{minipage}[t]{1.055in}\centering
27\end{minipage}}\\
\hline
% ROW 13
\multicolumn{1}{|p{0.951in}|}{\begin{minipage}[t]{0.951in}\centering
\textit{Dispel }\end{minipage}} & \multicolumn{1}{p{0.934in}|}{\begin{minipage}[t]{0.934in}\centering
19\end{minipage}} & \multicolumn{1}{p{1.098in}|}{\begin{minipage}[t]{1.098in}\centering
\textit{Ward }\end{minipage}} & \multicolumn{5}{p{1.055in}|}{\begin{minipage}[t]{1.055in}\centering
14\end{minipage}}\\
\hline
\end{longtable}
}

*Spellcasters without at least 24 ranks in Knowledge (religion) or Knowledge (nature) 
may not use \textit{heal }or \textit{life }spell seeds.

\vspace{12pt}
\textbf{Resource Cost:} The development of an epic spell uses up raw materials 
costing a number of gold pieces equal to 9,000 xthe final Spellcraft DC of the 
epic spell being developed. 

\textbf{Development Time:} Developing an epic spell takes one day for each 50,000 
gp in resources required to develop the spell, rounded up to whole days. 

\textbf{XP Cost:} To develop an epic spell, a character must spend 1/25 of its 
resource price in experience points. 

\textbf{Adding Seed DCs: }When two or more epic seeds are combined in an epic spell, 
their base Spellcraft DCs are added together. Both contribute toward the spell's 
final Spellcraft DC. 

\textbf{Determining School:} When combining two or more seeds to develop an epic 
spell, the school of the finished spell is decided by the caster from among the 
seeds that make up the epic spell. 

\textbf{Combining Descriptors: }When two or more epic seeds are combined in an 
epic spell, all the descriptors from each seed apply to the finished spell. 

\textbf{Combining Components and Casting Times: }Almost every epic spell has verbal 
and somatic components and a 1-minute casting time, regardless of the number of 
epic seeds combined. The only exceptions are epic spells with the \textit{heal 
}and \textit{life }seeds, which have divine focus components. 

\textbf{Combining Range, Targets, Area, and Effect:} One seed might have a range 
of 12,000 feet, another seed might have a range of 400 feet, and a third seed might 
not have a range at all. Likewise, some seeds have targets, while others have an 
effect or an area. To determine which seed takes precedence in the finished epic 
spell, the character must decide which seed is the base seed. The seed most important 
to the spell's overall purpose is the base seed, and it determines the casting 
time, range, target, and so on. The other seeds apply only their specific effects 
to the finished spell. It is occasionally difficult to determine a base seed by 
examining the spell's effects. If no one seed is most important, simply pick one 
seed for the purposes of making this determination. 

\textbf{Combining Durations:} When combining two or more seeds to develop an epic 
spell, the seed with the shortest duration determines the duration of the finished 
epic spell. If any seed of an epic spell is dismissible by the caster, the epic 
spell is dismissible. 

\textbf{Saving Throws:} Even if more than one seed has an associated saving throw, 
the final spell will have only a single saving throw. If two or more seeds have 
the same kind of saving throw (Fortitude, Reflex, or Will), then obviously that 
will be used for the spell's saving throw. If the seeds have different kinds of 
saving throws, simply choose the saving throw that seems most appropriate for the 
final spell. 

\textbf{Spell Resistance:} When combining two or more seeds to develop an epic 
spell, if even one seed is subject to spell resistance, the finished epic spell 
is subject to it as well. 

\textbf{Factors: }Factors are not part of epic seeds, but they are the tools used 
to modify specific parameters of any given seed. Applying factors to the seeds 
of an epic spell can increase or decrease the final Spellcraft DC, increase the 
duration, change the area of a spell, and affect many other aspects of the spell. 

There are three kinds of factors: 

1. Those that can affect a number of seeds. 

2. Those that can only be used with specific seeds. 

3. Those that reduce the Spellcraft DC rather than increasing it. These are referred 
to as mitigating factors. To calculate the final Spellcraft DC of an epic spell 
correctly, it's important to determine the mitigating factors last, after all the 
factors that increase the DC have been accounted for. 

\textbf{Development Is an Art:} Many times developing a completely new epic spell 
requires some guesswork and rule stretching. As with making and pricing magic items, 
a sort of balancing act is required. Often the description of a seed will need 
to be stretched for a particular spell. If necessary, assess an ``ad hoc'' Spellcraft 
DC adjustment for any effect that cannot be extrapolated from the seeds and factors 
presented here---the example spells use ad hoc factors frequently. In all cases, 
the GM determines the actual Spellcraft DC of the new spell

\textbf{Approval:} This is the final step, and it's critically important. The epic 
spell development work and reasoning must be shown to the GM and receive his or 
her approval. If the GM doesn't approve, then the epic spell cannot be developed. 
However, the GM should explain why the epic spell wasn't approved and possibly 
offer suggestions on how to create an epic spell that will be acceptable. 

\vspace{12pt}
\subsection*{\textbf{Table: Epic Spell Factors }}

\subsection*{\begin{longtable}{llll}
\hline
% ROW 1
\multicolumn{1}{|p{3.504in}|}{\begin{minipage}[t]{3.504in}\raggedright
\end{minipage}} & \multicolumn{1}{p{0.908in}|}{\begin{minipage}[t]{0.908in}\raggedleft
\textbf{Spellcraft DC Modifier}\end{minipage}}\\
\hline
% ROW 2
\multicolumn{1}{p{0.044in}|}{\begin{minipage}[t]{0.044in}\raggedleft
\textit{\textbf{Casting Time}}\end{minipage}} & \multicolumn{1}{p{0.044in}|}{\begin{minipage}[t]{0.044in}\raggedleft
\end{minipage}}\\
\hline
% ROW 3
\multicolumn{1}{|p{3.504in}|}{\begin{minipage}[t]{3.504in}\raggedleft
Reduce casting time by 1 round (minimum 1 round)\end{minipage}} & \multicolumn{1}{p{0.908in}|}{\begin{minipage}[t]{0.908in}\raggedleft
+2\end{minipage}}\\
\hline
% ROW 4
\multicolumn{1}{p{0.044in}|}{\begin{minipage}[t]{0.044in}\raggedleft
1-action casting time\end{minipage}} & \multicolumn{1}{p{0.044in}|}{\begin{minipage}[t]{0.044in}\raggedleft
+20\end{minipage}}\\
\hline
% ROW 5
\multicolumn{1}{|p{3.504in}|}{\begin{minipage}[t]{3.504in}\raggedleft
Quickened spell (limit one quickened action/round)\end{minipage}} & \multicolumn{3}{p{0.996in}|}{\begin{minipage}[t]{0.996in}\raggedleft
+28\end{minipage}}\\
\hline
% ROW 6
\multicolumn{1}{p{0.044in}|}{\begin{minipage}[t]{0.044in}\raggedleft
Contingent on specific trigger\textsuperscript{1}\end{minipage}} & \multicolumn{1}{p{0.044in}|}{\begin{minipage}[t]{0.044in}\raggedleft
+25\end{minipage}}\\
\hline
% ROW 7
\multicolumn{1}{|p{3.504in}|}{\begin{minipage}[t]{3.504in}\raggedleft
Components No verbal component\end{minipage}} & \multicolumn{3}{p{0.996in}|}{\begin{minipage}[t]{0.996in}\raggedleft
+2\end{minipage}}\\
\hline
% ROW 8
\multicolumn{1}{|p{3.504in}|}{\begin{minipage}[t]{3.504in}\raggedleft
No somatic component\end{minipage}} & \multicolumn{3}{p{0.996in}|}{\begin{minipage}[t]{0.996in}\raggedleft
+2\end{minipage}}\\
\hline
% ROW 9
\multicolumn{1}{|p{3.504in}|}{\begin{minipage}[t]{3.504in}\raggedleft
\textit{\textbf{Duration}}\textsuperscript{\textbf{2}}\end{minipage}} & \multicolumn{3}{p{0.996in}|}{\begin{minipage}[t]{0.996in}\raggedleft
\end{minipage}}\\
\hline
% ROW 10
\multicolumn{1}{|p{3.504in}|}{\begin{minipage}[t]{3.504in}\raggedleft
Increase duration by 100\%\end{minipage}} & \multicolumn{3}{p{0.996in}|}{\begin{minipage}[t]{0.996in}\raggedleft
+2\end{minipage}}\\
\hline
% ROW 11
\multicolumn{1}{|p{3.504in}|}{\begin{minipage}[t]{3.504in}\raggedleft
Permanent duration (apply this factor after all other epic spell factors but before 
mitigating factors)\end{minipage}} & \multicolumn{3}{p{0.996in}|}{\begin{minipage}[t]{0.996in}\raggedleft
x5\end{minipage}}\\
\hline
% ROW 12
\multicolumn{1}{|p{3.504in}|}{\begin{minipage}[t]{3.504in}\raggedleft
Dismissible by caster (if not already)\end{minipage}} & \multicolumn{3}{p{0.996in}|}{\begin{minipage}[t]{0.996in}\raggedleft
+2\end{minipage}}\\
\hline
% ROW 13
\multicolumn{1}{|p{3.504in}|}{\begin{minipage}[t]{3.504in}\raggedleft
Range Increase range by 100\%\end{minipage}} & \multicolumn{3}{p{0.996in}|}{\begin{minipage}[t]{0.996in}\raggedleft
+2\end{minipage}}\\
\hline
% ROW 14
\multicolumn{1}{|p{3.504in}|}{\begin{minipage}[t]{3.504in}\raggedleft
\textit{\textbf{Target}}\textsuperscript{\textbf{3}}\textbf{ }\end{minipage}} & \multicolumn{3}{p{0.996in}|}{\begin{minipage}[t]{0.996in}\raggedleft
\end{minipage}}\\
\hline
% ROW 15
\multicolumn{1}{|p{3.504in}|}{\begin{minipage}[t]{3.504in}\raggedleft
Add extra target within 300 ft.\end{minipage}} & \multicolumn{3}{p{0.996in}|}{\begin{minipage}[t]{0.996in}\raggedleft
+10\end{minipage}}\\
\hline
% ROW 16
\multicolumn{1}{|p{3.504in}|}{\begin{minipage}[t]{3.504in}\raggedleft
Change from target to area (pick area option below)\end{minipage}} & \multicolumn{3}{p{0.996in}|}{\begin{minipage}[t]{0.996in}\raggedleft
+10\end{minipage}}\\
\hline
% ROW 17
\multicolumn{1}{|p{3.504in}|}{\begin{minipage}[t]{3.504in}\raggedleft
Change from personal to area (pick area option below)\end{minipage}} & \multicolumn{3}{p{0.996in}|}{\begin{minipage}[t]{0.996in}\raggedleft
+15\end{minipage}}\\
\hline
% ROW 18
\multicolumn{1}{|p{3.504in}|}{\begin{minipage}[t]{3.504in}\raggedleft
Change from target to touch or ray (300-ft. range)\end{minipage}} & \multicolumn{3}{p{0.996in}|}{\begin{minipage}[t]{0.996in}\raggedleft
+4\end{minipage}}\\
\hline
% ROW 19
\multicolumn{1}{|p{3.504in}|}{\begin{minipage}[t]{3.504in}\raggedleft
Change from touch or ranged touch attack to target\end{minipage}} & \multicolumn{3}{p{0.996in}|}{\begin{minipage}[t]{0.996in}\raggedleft
+4\end{minipage}}\\
\hline
% ROW 20
\multicolumn{1}{|p{3.504in}|}{\begin{minipage}[t]{3.504in}\raggedleft
\textit{\textbf{Area}}\textsuperscript{\textbf{4}}\end{minipage}} & \multicolumn{3}{p{0.996in}|}{\begin{minipage}[t]{0.996in}\raggedleft
\end{minipage}}\\
\hline
% ROW 21
\multicolumn{1}{|p{3.504in}|}{\begin{minipage}[t]{3.504in}\raggedleft
Change area to bolt (5 ft. x300 ft. or 10 ft. x150 ft.)\end{minipage}} & \multicolumn{3}{p{0.996in}|}{\begin{minipage}[t]{0.996in}\raggedleft
+2\end{minipage}}\\
\hline
% ROW 22
\multicolumn{1}{|p{3.504in}|}{\begin{minipage}[t]{3.504in}\raggedleft
Change area to cylinder (10-ft. radius,  30 ft. high)\end{minipage}} & \multicolumn{3}{p{0.996in}|}{\begin{minipage}[t]{0.996in}\raggedleft
+2\end{minipage}}\\
\hline
% ROW 23
\multicolumn{1}{|p{3.504in}|}{\begin{minipage}[t]{3.504in}\raggedleft
Change area to 40-ft. cone\end{minipage}} & \multicolumn{3}{p{0.996in}|}{\begin{minipage}[t]{0.996in}\raggedleft
+2\end{minipage}}\\
\hline
% ROW 24
\multicolumn{1}{|p{3.504in}|}{\begin{minipage}[t]{3.504in}\raggedleft
Change area to four 10-ft. cubes\end{minipage}} & \multicolumn{3}{p{0.996in}|}{\begin{minipage}[t]{0.996in}\raggedleft
+2\end{minipage}}\\
\hline
% ROW 25
\multicolumn{1}{|p{3.504in}|}{\begin{minipage}[t]{3.504in}\raggedleft
Change area to 20-ft. radius\end{minipage}} & \multicolumn{3}{p{0.996in}|}{\begin{minipage}[t]{0.996in}\raggedleft
+2\end{minipage}}\\
\hline
% ROW 26
\multicolumn{1}{|p{3.504in}|}{\begin{minipage}[t]{3.504in}\raggedleft
Change area to target\end{minipage}} & \multicolumn{3}{p{0.996in}|}{\begin{minipage}[t]{0.996in}\raggedleft
+4\end{minipage}}\\
\hline
% ROW 27
\multicolumn{1}{|p{3.504in}|}{\begin{minipage}[t]{3.504in}\raggedleft
Change area to touch or ray (close range)\end{minipage}} & \multicolumn{3}{p{0.996in}|}{\begin{minipage}[t]{0.996in}\raggedleft
+4\end{minipage}}\\
\hline
% ROW 28
\multicolumn{1}{|p{3.504in}|}{\begin{minipage}[t]{3.504in}\raggedleft
Increase area by 100\%\end{minipage}} & \multicolumn{3}{p{0.996in}|}{\begin{minipage}[t]{0.996in}\raggedleft
+4\end{minipage}}\\
\hline
% ROW 29
\multicolumn{1}{|p{3.504in}|}{\begin{minipage}[t]{3.504in}\raggedleft
\textit{\textbf{Saving Throw}}\end{minipage}} & \multicolumn{3}{p{0.996in}|}{\begin{minipage}[t]{0.996in}\raggedleft
\end{minipage}}\\
\hline
% ROW 30
\multicolumn{1}{|p{3.504in}|}{\begin{minipage}[t]{3.504in}\raggedleft
Increase spell's saving throw DC by +1\end{minipage}} & \multicolumn{3}{p{0.996in}|}{\begin{minipage}[t]{0.996in}\raggedleft
+2\end{minipage}}\\
\hline
% ROW 31
\multicolumn{1}{|p{3.504in}|}{\begin{minipage}[t]{3.504in}\raggedleft
\textit{\textbf{Spell Resistance}}\end{minipage}} & \multicolumn{3}{p{0.996in}|}{\begin{minipage}[t]{0.996in}\raggedleft
\end{minipage}}\\
\hline
% ROW 32
\multicolumn{1}{|p{3.504in}|}{\begin{minipage}[t]{3.504in}\raggedleft
Gain +1 bonus on caster level check to overcome target's spell resistance\end{minipage}} & \multicolumn{3}{p{0.996in}|}{\begin{minipage}[t]{0.996in}\raggedleft
+2\end{minipage}}\\
\hline
% ROW 33
\multicolumn{1}{|p{3.504in}|}{\begin{minipage}[t]{3.504in}\raggedleft
Gain +1 on caster level check to beat foe's dispel effect\end{minipage}} & \multicolumn{3}{p{0.996in}|}{\begin{minipage}[t]{0.996in}\raggedleft
+2\end{minipage}}\\
\hline
% ROW 34
\multicolumn{1}{|p{3.504in}|}{\begin{minipage}[t]{3.504in}\raggedleft
\textit{\textbf{Other}}\end{minipage}} & \multicolumn{3}{p{0.996in}|}{\begin{minipage}[t]{0.996in}\raggedleft
\end{minipage}}\\
\hline
% ROW 35
\multicolumn{1}{|p{3.504in}|}{\begin{minipage}[t]{3.504in}\raggedleft
Recorded onto stone tablet\textsuperscript{5}\end{minipage}} & \multicolumn{3}{p{0.996in}|}{\begin{minipage}[t]{0.996in}\raggedleft
x2\end{minipage}}\\
\hline
% ROW 36
\multicolumn{1}{|p{3.504in}|}{\begin{minipage}[t]{3.504in}\raggedleft
Increase damage die by one step (d20 maximum)\end{minipage}} & \multicolumn{3}{p{0.996in}|}{\begin{minipage}[t]{0.996in}\raggedleft
+10\end{minipage}}\\
\hline
\end{longtable}
}

Unless stated otherwise, the same factor can be applied more than once. 

1 Each contingent spell in use counts as a slot used from the caster's daily epic 
spell slots. 

2 Seeds that already have an instantaneous or permanent duration cannot be increased. 

3 When changing a targeted or area seed to a touch or ranged attack, the seed no 
longer requires a save if it deals damage, instead requiring a successful attack 
roll. Seeds with a nondamaging effect still allow the target a save. Area spells 
changed to touch or ranged attacks now affect only the creature successfully attacked. 

4 When changing a touch or ranged attack seed to a targeted seed, the seed no longer 
requires an attack roll if it deals damage, instead requiring a saving throw from 
the target. On a failed saving throw, the target takes half damage. Area seeds 
changed to targeted seeds now only affect the target. The GM determines the most 
appropriate kind of saving throw for the epic spell. 

5 Epic spells may only be inscribed on stone tablets or other substances of equal 
or greater hardness. Once a spell is so inscribed, another epic spellcaster can 
learn it without going through the process of development. Once an inscribed epic 
spell is learned by another epic spellcaster in this fashion, the tablet upon which 
it is inscribed is destroyed and cannot be mended. 

\vspace{12pt}
\subsection*{\textbf{Table: Epic Spell Mitigating Factors }}

\subsection*{\begin{longtable}{llll}
\hline
% ROW 1
\multicolumn{1}{|p{3.246in}|}{\begin{minipage}[t]{3.246in}\raggedright
\end{minipage}} & \multicolumn{1}{p{1.151in}|}{\begin{minipage}[t]{1.151in}\centering
\textbf{Spellcraft DC Modifier}\end{minipage}}\\
\hline
% ROW 2
\multicolumn{1}{p{0.051in}|}{\begin{minipage}[t]{0.051in}\centering
Backlash 1d6 points of damage (max d6 = caster's HD x2)\textsuperscript{\textbf{1}}\end{minipage}} & \multicolumn{1}{p{0.051in}|}{\begin{minipage}[t]{0.051in}\centering
-1\end{minipage}}\\
\hline
% ROW 3
\multicolumn{1}{|p{3.246in}|}{\begin{minipage}[t]{3.246in}\centering
Burn 100 XP during casting (max 20,000 XP)\end{minipage}} & \multicolumn{1}{p{1.151in}|}{\begin{minipage}[t]{1.151in}\centering
-1\end{minipage}}\\
\hline
% ROW 4
\multicolumn{1}{p{0.051in}|}{\begin{minipage}[t]{0.051in}\centering
Increase casting time by 1 minute (max 10 minutes)\textsuperscript{\textbf{2}}\end{minipage}} & \multicolumn{1}{p{0.051in}|}{\begin{minipage}[t]{0.051in}\centering
-2\end{minipage}}\\
\hline
% ROW 5
\multicolumn{1}{|p{3.246in}|}{\begin{minipage}[t]{3.246in}\centering
Increase casting time by 1 day (max 100 days)\textsuperscript{\textbf{2}}\end{minipage}} & \multicolumn{3}{p{1.254in}|}{\begin{minipage}[t]{1.254in}\centering
-2\end{minipage}}\\
\hline
% ROW 6
\multicolumn{1}{p{0.051in}|}{\begin{minipage}[t]{0.051in}\centering
Change from target, touch, or area to personal\end{minipage}} & \multicolumn{1}{p{0.051in}|}{\begin{minipage}[t]{0.051in}\centering
-2\end{minipage}}\\
\hline
% ROW 7
\multicolumn{1}{|p{3.246in}|}{\begin{minipage}[t]{3.246in}\centering
Additional participants (ritual)\end{minipage}} & \multicolumn{3}{p{1.254in}|}{\begin{minipage}[t]{1.254in}\centering
see Table: Additional Participants in Rituals\end{minipage}}\\
\hline
% ROW 8
\multicolumn{1}{|p{3.246in}|}{\begin{minipage}[t]{3.246in}\centering
Decrease damage die by one step (d4 minimum)\end{minipage}} & \multicolumn{3}{p{1.254in}|}{\begin{minipage}[t]{1.254in}\centering
-5\end{minipage}}\\
\hline
\end{longtable}
}

Note: Mitigating factors are always applied after all epic spell factors (see above) 
are accounted for in the development of an epic spell.

1 The caster cannot somehow avoid or make him or her self immune to backlash damage. 
For spells with durations longer than instantaneous, the backlash damage is per 
round. If backlash damage kills a caster, no spell or method exists that will return 
life to the caster's body without costing the caster a level---not even \textit{wish, 
true resurrection, miracle, }or epic spells that return life to the deceased\textit{. 
}Spells that normally penalize the recipient one level when they return him or 
her to life penalize a caster killed by backlash two levels. 

2 When increasing the casting time of a spell in order to reduce the Spellcraft 
DC, a character must first ``use up'' the maximum of 10 minutes (for a total DC 
modifier of -20). After that, a character can continue to add days to the casting 
time, with a further modifier of -2 per day, up to the maximum of 100 days.

\vspace{12pt}
\textbf{Additional Participants:} Epic spells can be developed that specifically 
require additional participants. These spells are called rituals. An epic spell 
developed as a ritual requires a specific number of additional participants, who 
each must use up one spell slot of a specified level for the day. During an epic 
spell's development, the spell's creator determines the number of additional participants 
and the level of the spell slots to be contributed. If the exact number of spellcasters 
does not partake in the casting, or if the casters do not each contribute the proper 
spell slot, the epic spell automatically fails. To participate, each participant 
readies an action to contribute his or her raw spell energy when the primary caster 
begins the epic spell. Additional participants in a ritual spell reduce the Spellcraft 
DC, as shown on Table: Additional Participants in Rituals. Each additional participant 
may only contribute one spell slot. It doesn't matter whether the additional participants 
are arcane or divine spellcasters; only the level of the spell slot contributed 
matters. A contributed spell slot is treated as if normally cast. A wizard may 
contribute either a prepared, uncast spell slot, or an open, unprepared slot. The 
Spellcraft DC adjustments for each additional participant stack. 

\textit{Special: }A ritual epic spell that takes longer than 1 standard action 
to cast requires all extra participants to stand as if casting for the same amount 
of time. If an extra participant is attacked while contributing a spell slot, the 
participant must make a Concentration check as if casting a spell of the same level 
as the slot contributed. If the attack disrupts the participant in the ritual, 
the epic spell is not necessarily ruined. However, the Spellcraft DC reduction 
that would have been provided by that additional participant cannot be applied 
to the final Spellcraft DC of the epic spell. Thus the ritual epic spell will be 
harder for the primary spellcaster to cast. 

\vspace{12pt}
\subsection*{\textbf{Table: Additional Participants in Rituals}}

\subsection*{\begin{longtable}{llllllll}
\hline
% ROW 1
\multicolumn{1}{|p{1.010in}|}{\begin{minipage}[t]{1.010in}\raggedright
\textbf{Spell Slot Level Contributed}\end{minipage}} & \multicolumn{1}{p{0.875in}|}{\begin{minipage}[t]{0.875in}\raggedright
\textbf{Spellcraft DC Reduction }\end{minipage}} & \multicolumn{1}{p{1.000in}|}{\begin{minipage}[t]{1.000in}\raggedright
\textbf{Spell Slot Level Contributed}\end{minipage}} & \multicolumn{1}{p{0.875in}|}{\begin{minipage}[t]{0.875in}\raggedright
\textbf{Spellcraft DC Reduction }\end{minipage}}\\
\hline
% ROW 2
\multicolumn{1}{p{0.069in}|}{\begin{minipage}[t]{0.069in}\raggedright
1st\end{minipage}} & \multicolumn{1}{p{0.069in}|}{\begin{minipage}[t]{0.069in}\raggedright
-1 \end{minipage}} & \multicolumn{1}{p{0.069in}|}{\begin{minipage}[t]{0.069in}\raggedright
6th\end{minipage}} & \multicolumn{1}{p{0.069in}|}{\begin{minipage}[t]{0.069in}\raggedright
-11 \end{minipage}}\\
\hline
% ROW 3
\multicolumn{1}{|p{1.010in}|}{\begin{minipage}[t]{1.010in}\raggedright
2nd\end{minipage}} & \multicolumn{1}{p{0.875in}|}{\begin{minipage}[t]{0.875in}\raggedright
-3 \end{minipage}} & \multicolumn{1}{p{1.000in}|}{\begin{minipage}[t]{1.000in}\raggedright
7th\end{minipage}} & \multicolumn{1}{p{0.875in}|}{\begin{minipage}[t]{0.875in}\raggedright
-13 \end{minipage}}\\
\hline
% ROW 4
\multicolumn{1}{p{0.069in}|}{\begin{minipage}[t]{0.069in}\raggedright
3rd\end{minipage}} & \multicolumn{1}{p{0.069in}|}{\begin{minipage}[t]{0.069in}\raggedright
-5 \end{minipage}} & \multicolumn{1}{p{0.069in}|}{\begin{minipage}[t]{0.069in}\raggedright
8th\end{minipage}} & \multicolumn{1}{p{0.069in}|}{\begin{minipage}[t]{0.069in}\raggedright
-15 \end{minipage}}\\
\hline
% ROW 5
\multicolumn{1}{|p{1.010in}|}{\begin{minipage}[t]{1.010in}\raggedright
4th\end{minipage}} & \multicolumn{1}{p{0.875in}|}{\begin{minipage}[t]{0.875in}\raggedright
-7\end{minipage}} & \multicolumn{1}{p{1.000in}|}{\begin{minipage}[t]{1.000in}\raggedright
9th\end{minipage}} & \multicolumn{5}{p{1.153in}|}{\begin{minipage}[t]{1.153in}\raggedright
-17 \end{minipage}}\\
\hline
% ROW 6
\multicolumn{1}{p{0.069in}|}{\begin{minipage}[t]{0.069in}\raggedright
5th\end{minipage}} & \multicolumn{1}{p{0.069in}|}{\begin{minipage}[t]{0.069in}\raggedright
-9 \end{minipage}} & \multicolumn{1}{p{0.069in}|}{\begin{minipage}[t]{0.069in}\raggedright
Epic slot\end{minipage}} & \multicolumn{1}{p{0.069in}|}{\begin{minipage}[t]{0.069in}\raggedright
-19 \end{minipage}}\\
\hline
\end{longtable}
}

\vspace{24pt}
\subsection*{{\LARGE SEED DESCRIPTIONS }}

Each seed description hereafter follows the same format used for 0- to 9th-level 
spells\textit{. }An additional line, Spellcraft DC, indicates the base DC of the 
Spellcraft check required to cast an epic spell with this seed. 

\vspace{12pt}
\subsection*{SEED:AFFLICT }

Enchantment (Compulsion) [Fear, Mind-Affecting] 

\textbf{Spellcraft DC:} 14 

\textbf{Components:} V, S 

\textbf{Casting Time:} 1 standard action 

\textbf{Range:} 300 ft. 

\textbf{Target:} One living creature 

\textbf{Duration:} 20 minutes 

\textbf{Saving Throw:} Will negates 

\textbf{Spell Resistance:} Yes 

Afflicts the target with a -2 morale penalty on attack rolls, checks, and saving 
throws. For each additional -1 penalty assessed on either the target's attack rolls, 
checks, or saving throws, increase the Spellcraft DC by +2. A character may also 
develop a spell with this seed that afflicts the target with a -1 penalty on caster 
level checks, a -1 penalty to an ability score, a -1 penalty to spell resistance, 
or a -1 penalty to some other aspect of the target. For each additional -1 penalty 
assessed in one of the above categories, increase the Spell-craft DC by +4. This 
seed can afflict a character's ability scores to the point where they reach 0, 
except for Constitution where 1 is the minimum. If a factor is applied to increase 
the duration of this seed, ability score penalties instead become temporary ability 
damage. If a factor is applied to make the duration permanent, any ability score 
penalties become permanent ability drain. Finally, by increasing the Spellcraft 
DC by +2, one of the target's senses can be afflicted: sight, smell, hearing, taste, 
touch, or a special sense the target possesses. If the target fails its saving 
throw, the sense selected doesn't function for the spell's duration, with all attendant 
penalties that apply for losing the specified sense. 

\vspace{12pt}
\subsection*{SEED:ANIMATE }

Transmutation 

\textbf{Spellcraft DC:} 25 

\textbf{Components:} V, S 

\textbf{Casting Time:} 1 minute 

\textbf{Range:} 300 ft. 

\textbf{Target:} Object or 20 cu. ft. of matter 

\textbf{Duration:} 20 rounds 

\textbf{Saving Throw:} None 

\textbf{Spell Resistance:} No 

This seed can imbue inanimate objects with mobility and a semblance of life (not 
actual life). The animated object attacks whomever or whatever the caster initially 
designates. The animated object can be of any nonmagical material. The caster can 
also animate part of a larger mass of raw matter, such as a volume of water in 
the ocean, part of a stony wall, or the earth itself, as long as the volume of 
material does not exceed 20 cubic feet. For each additional 10 cubic feet of matter 
animated, increase the Spellcraft DC by +1, up to 1,000 cubic feet. For each additional 
100 cubic feet of matter animated after the first 1,000 cubic feet, increase the 
spellcraft DC by +1. For each additional Hit Die granted to an animated object 
of a given size, increase the Spellcraft DC by +2. To animate attended objects 
(objects carried or worn by another creature), increase the Spellcraft DC by +10. 

\vspace{12pt}
\subsection*{SEED:ANIMATE DEAD }

Necromancy [Evil] 

\textbf{Spellcraft DC:} 23 

\textbf{Components:} V, S 

\textbf{Casting Time:} 1 minute 

\textbf{Range:} Touch 

\textbf{Target:} One or more corpses touched 

\textbf{Duration:} Instantaneous 

\textbf{Saving Throw:} None 

\textbf{Spell Resistance:} No 

The caster can turn the bones or bodies of dead creatures into undead that follow 
his or her spoken commands. The undead can follow the caster, or they can remain 
in an area and attack any creature (or a specific type of creature) entering the 
place. The undead remain animated until they are destroyed. (A destroyed undead 
can't be animated again.) Intelligent undead can follow more sophisticated commands. 
The \textit{animate dead }seed allows a character to create 20 HD of undead. For 
each additional 1 HD of undead created, increase the Spellcraft DC by +1. The undead 
created remain under the caster's control indefinitely. A caster can naturally 
control 1 HD per caster level of undead creatures he or she has personally created, 
regardless of the method used. If the caster exceeds this number, newly created 
creatures fall under his or her control, and excess undead from previous castings 
become uncontrolled (the caster chooses which creatures are released). If the caster 
is a cleric, any undead he or she commands through his or her ability to command 
or rebuke undead do not count toward the limit. For each additional 2 HD of undead 
to be controlled, increase the Spellcraft DC by +1. Only undead in excess of 20 
HD created with this seed can be controlled using this DC adjustment. To both create 
and control more than 20 HD of undead, increase the Spellcraft DC by +3 per additional 
2 HD of undead. 

Type of Undead: All types of undead can be created with the \textit{animate dead 
}seed, although creating more powerful undead increases the Spellcraft DC of the 
epic spell, according to the table below. The GM must set the Spellcraft DC for 
undead not included on the table, using similar undead as a basis for comparison. 

\begin{longtable}{llllllll}
\hline
% ROW 1
\multicolumn{1}{|p{0.760in}|}{\begin{minipage}[t]{0.760in}\centering
\textbf{Undead}\end{minipage}} & \multicolumn{1}{p{1.141in}|}{\begin{minipage}[t]{1.141in}\centering
\textbf{Spellcraft DC Modifier}\end{minipage}} & \multicolumn{1}{p{0.734in}|}{\begin{minipage}[t]{0.734in}\centering
\textbf{Undead}\end{minipage}} & \multicolumn{1}{p{1.000in}|}{\begin{minipage}[t]{1.000in}\centering
\textbf{Spellcraft DC Modifier}\end{minipage}}\\
\hline
% ROW 2
\multicolumn{1}{p{0.069in}|}{\begin{minipage}[t]{0.069in}\centering
Skeleton\end{minipage}} & \multicolumn{1}{p{0.069in}|}{\begin{minipage}[t]{0.069in}\centering
-12\end{minipage}} & \multicolumn{1}{p{0.069in}|}{\begin{minipage}[t]{0.069in}\centering
Wraith\end{minipage}} & \multicolumn{1}{p{0.069in}|}{\begin{minipage}[t]{0.069in}\centering
-2\end{minipage}}\\
\hline
% ROW 3
\multicolumn{1}{|p{0.760in}|}{\begin{minipage}[t]{0.760in}\centering
Zombie\end{minipage}} & \multicolumn{1}{p{1.141in}|}{\begin{minipage}[t]{1.141in}\centering
-12\end{minipage}} & \multicolumn{1}{p{0.734in}|}{\begin{minipage}[t]{0.734in}\centering
Mummy\end{minipage}} & \multicolumn{1}{p{1.000in}|}{\begin{minipage}[t]{1.000in}\centering
+0\end{minipage}}\\
\hline
% ROW 4
\multicolumn{1}{p{0.069in}|}{\begin{minipage}[t]{0.069in}\centering
Ghoul\end{minipage}} & \multicolumn{1}{p{0.069in}|}{\begin{minipage}[t]{0.069in}\centering
-10\end{minipage}} & \multicolumn{1}{p{0.069in}|}{\begin{minipage}[t]{0.069in}\centering
Spectre\end{minipage}} & \multicolumn{1}{p{0.069in}|}{\begin{minipage}[t]{0.069in}\centering
+2\end{minipage}}\\
\hline
% ROW 5
\multicolumn{1}{|p{0.760in}|}{\begin{minipage}[t]{0.760in}\centering
Shadow\end{minipage}} & \multicolumn{1}{p{1.141in}|}{\begin{minipage}[t]{1.141in}\centering
-8\end{minipage}} & \multicolumn{1}{p{0.734in}|}{\begin{minipage}[t]{0.734in}\centering
Morhg\end{minipage}} & \multicolumn{5}{p{1.278in}|}{\begin{minipage}[t]{1.278in}\centering
+4\end{minipage}}\\
\hline
% ROW 6
\multicolumn{1}{p{0.069in}|}{\begin{minipage}[t]{0.069in}\centering
Ghast\end{minipage}} & \multicolumn{1}{p{0.069in}|}{\begin{minipage}[t]{0.069in}\centering
-6\end{minipage}} & \multicolumn{1}{p{0.069in}|}{\begin{minipage}[t]{0.069in}\centering
Vampire\end{minipage}} & \multicolumn{1}{p{0.069in}|}{\begin{minipage}[t]{0.069in}\centering
+6\end{minipage}}\\
\hline
% ROW 7
\multicolumn{1}{|p{0.760in}|}{\begin{minipage}[t]{0.760in}\centering
Wight\end{minipage}} & \multicolumn{1}{p{1.141in}|}{\begin{minipage}[t]{1.141in}\centering
-4\end{minipage}} & \multicolumn{1}{p{0.734in}|}{\begin{minipage}[t]{0.734in}\centering
Ghost\end{minipage}} & \multicolumn{5}{p{1.278in}|}{\begin{minipage}[t]{1.278in}\centering
+8\end{minipage}}\\
\hline
\end{longtable}

\vspace{12pt}
\subsection*{SEED: ARMOR }

Conjuration (Creation) [Force] 

\textbf{Spellcraft DC:} 14 

\textbf{Components:} V, S 

\textbf{Casting Time:} 1 minute 

\textbf{Range:} Touch 

\textbf{Target:} Creature touched

\textbf{Duration:} 24 hours (D) 

\textbf{Saving Throw:} Will negates (harmless) 

\textbf{Spell Resistance:} Yes (harmless) 

This seed grants a creature additional armor, providing a +4 bonus to Armor Class. 
The bonus is either an armor bonus or a natural armor bonus, whichever the caster 
selects. Unlike mundane armor, the \textit{armor }seed provides an intangible protection 
that entails no armor check penalty, arcane spell failure chance, or speed reduction. 
Incorporeal creatures can't bypass the \textit{armor }seed the way they can ignore 
normal armor. For each additional point of Armor Class bonus, increase the Spellcraft 
DC by +2. The caster can also grant a creature a +1 bonus to Armor Class using 
a different bonus type, such as deflection, divine, or insight. For each additional 
point of bonus to Armor Class of one of these types, increase the Spellcraft DC 
by +10. 

\vspace{12pt}
\subsection*{SEED: BANISH }

Abjuration 

\textbf{Spellcraft DC:} 27 

\textbf{Components:} V, S 

\textbf{Casting Time:} 1 minute 

\textbf{Range:} 75 ft. 

\textbf{Target:} One or more extraplanar creatures, no two of which can be more 
than 30 ft. apart 

\textbf{Duration:} Instantaneous 

\textbf{Saving Throw:} Will negates 

\textbf{Spell Resistance:} Yes 

This seed forces extraplanar creatures out of the caster's home plane. The caster 
can banish up to 14 HD of extraplanar creatures. For each additional 2 HD of extraplanar 
creatures banished, increase the Spellcraft DC by +1. To specify a type or sub-type 
of creature other than outsider to be banished, increase the Spellcraft DC by +20. 

\vspace{12pt}
\subsection*{SEED: COMPEL }

Enchantment (Compulsion) [Mind-Affecting] 

\textbf{Spellcraft DC:} 19 

\textbf{Components:} V, M 

\textbf{Casting Time:} 1 minute 

\textbf{Range:} 75 ft. 

\textbf{Target:} One living creature 

\textbf{Duration:} 20 hours or until completed 

\textbf{Saving Throw:} Will negates 

\textbf{Spell Resistance:} Yes 

This seed compels a target to follow a course of activity. At the basic level of 
effect, a spell using the \textit{compel }seed must be worded in such a manner 
as to make the activity sound reasonable. Asking the creature to do an obviously 
harmful act automatically negates the effect (unless the Spellcraft DC has been 
increased to avoid this limitation; see below). To compel a creature to follow 
an outright unreasonable course of action, increase the Spellcraft DC by +10. The 
compelled course of activity can continue for the entire duration. If the compelled 
activity can be completed in a shorter time, the spell ends when the subject finishes 
what he or she was asked to do. The caster can instead specify conditions that 
will trigger a special activity during the duration. If the condition is not met 
before the spell using this seed expires, the activity is not performed. 

\vspace{12pt}
\subsection*{SEED: CONCEAL }

Illusion (Glamer) 

\textbf{Spellcraft DC:} 17 

\textbf{Components:} V, S 

\textbf{Casting Time:} 1 minute 

\textbf{Range:} Personal or touch 

\textbf{Target:} You or a creature or object of up to 2,000 lb. 

\textbf{Duration:} 200 minutes or until expended (D) 

\textbf{Saving Throw:} None or Will negates (harmless, object) 

\textbf{Spell Resistance:} No or Yes (harmless, object) 

This seed can conceal a creature or object touched from sight, even from darkvision. 
If the subject is a creature carrying gear, the gear vanishes too, rendering the 
creature invisible. A spell using the \textit{conceal }seed ends if the subject 
attacks any creature. Actions directed at unattended objects do not break the spell, 
and causing harm indirectly is not an attack. To create invisibility that lasts 
regardless of the actions of the subject, increase the Spellcraft DC by +4. Alternatively, 
this seed can conceal the exact location of the subject so that it appears to be 
about 2 feet away from its true location; this increases the Spellcraft DC by +2. 
The subject benefits from a 50\% miss chance as if it had total concealment. However, 
unlike actual total concealment, this displacement effect does not prevent enemies 
from targeting him or her normally. The \textit{conceal }seed can also be used 
to block divination spells, spell-like effects, and epic spells developed using 
the \textit{reveal }seed; this increases the Spellcraft DC by +6. In all cases 
where divination magic of any level, including epic level, is employed against 
the subject of a spell using the \textit{conceal }seed for this purpose, an opposed 
caster level check determines which spell works. 

\vspace{12pt}
\subsection*{SEED: CONJURE }

Conjuration (Creation) 

\textbf{Spellcraft DC:} 21 

\textbf{Components:} V, S 

\textbf{Casting Time:} 1 minute 

\textbf{Range:} 0 ft.

\textbf{Effect:} Unattended, nonmagical object of nonliving matter up to 20 cu. 
ft. 

\textbf{Duration:} 8 hours 

\textbf{Saving Throw:} None 

\textbf{Spell Resistance:} No 

This seed creates a nonmagical, unattended object of nonliving matter of up to 
20 cubic feet in volume. The caster must succeed at an appropriate skill check 
to make a complex item. The seed can create matter ranging in hardness and rarity 
from vegetable matter all the way up to mithral and even adamantine. Simple objects 
have a natural duration of 24 hours. For each additional cubic foot of matter created, 
increase the Spellcraft DC by +2. Attempting to use any created object as a material 
component or a resource during epic spell development causes the spell to fail 
and the object to disappear. 

The \textit{conjure }seed can be used in conjunction with the \textit{life }and 
\textit{fortify }seeds for an epic spell that creates an entirely new creature, 
if made permanent. To give a creature spell-like abilities, apply other epic seeds 
to the epic spell that replicate the desired ability. To give the creature a supernatural 
or extraordinary ability rather than a spell-like ability, double the cost of the 
relevant seed. Remember that two doublings equals a tripling, and so forth. To 
give a creature Hit Dice, use the \textit{fortify }seed. Each 5 hit points granted 
to the creature gives it an additional 1 HD. Once successfully created, the new 
creature will breed true. 

\vspace{12pt}
\subsection*{SEED: CONTACT }

Divination 

\textbf{Spellcraft DC:} 23 

\textbf{Components:} V, S 

\textbf{Casting Time:} 1 minute 

\textbf{Range:} See text 

\textbf{Target:} One creature 

\textbf{Duration:} 200 minutes 

\textbf{Saving Throw:} None 

\textbf{Spell Resistance:} No 

This seed forges a telepathic bond with a particular creature with which the caster 
is familiar (or one that the caster can currently see directly or through magical 
means) and can converse back and forth. The subject recognizes the caster if it 
knows him or her. It can answer in like manner immediately, though it does not 
have to. The caster can forge a communal bond among more than two creatures. For 
each additional creature contacted, increase the Spellcraft DC by +1. The bond 
can be established only among willing subjects, which therefore receive no saving 
throw or spell resistance. For telepathic communication through the bond regardless 
of language, increase the Spellcraft DC by +4. No special influence is established 
as a result of the bond, only the power to communicate at a distance. 

At the base Spellcraft DC of 20, a caster can also use the \textit{contact }seed 
to imbue an object (or creature) with a message he or she prepares that appears 
as written text for the spell's duration or is spoken aloud in a language the caster 
knows. The spoken message can be of any length, but the length of written text 
is limited to what can be contained (as text of a readable size) on the surface 
of the target. The message is delivered when specific conditions are fulfilled 
according to the caster's desire when the spell is cast.

\vspace{12pt}
\subsection*{SEED: DELUDE }

Illusion (Figment) 

\textbf{Spellcraft DC:} 14 

\textbf{Components:} V, S 

\textbf{Casting Time:} 1 minute 

\textbf{Range:} 12,000 ft. 

\textbf{Effect:} Visual figment that can extend for up to twenty 30-ft. cubes (S) 

\textbf{Duration:} Concentration plus 20 hours 

\textbf{Saving Throw:} Will disbelief (if interacted with) 

\textbf{Spell Resistance:} No 

A spell developed with the \textit{delude }seed creates the visual illusion of 
an object, creature, or force, as visualized by the caster. The caster can move 
the image within the limits of the size of the effect by concentrating (the image 
is otherwise stationary). The image disappears when struck by an opponent unless 
the caster causes the illusion to react appropriately. For an illusion that includes 
audible, olfactory, tactile, taste, and thermal aspects, increase the Spellcraft 
DC by +2 per extra aspect. Even realistic tactile and thermal illusions can't deal 
damage, however. For each additional image to be created, increase the Spellcraft 
DC by +1. For an illusion that follows a script determined by the caster, increase 
the Spellcraft DC by +9. The figment follows the script without the caster having 
to concentrate on it. The illusion can include intelligible speech if desired. 
For an illusion that makes any area appear to be something other than it is, increase 
the Spellcraft DC by +4. Additional components, such as sounds, can be added as 
noted above. Concealing creatures requires additional spell development using this 
or other seeds. 

\vspace{12pt}
\subsection*{SEED: DESTROY }

Transmutation 

\textbf{Spellcraft DC:} 29 

\textbf{Components:} V, S 

\textbf{Casting Time:} 1 minute 

\textbf{Range:} 12,000 ft. 

\textbf{Target:} One creature, or up to a 10-foot cube of nonliving matter 

\textbf{Duration:} Instantaneous 

\textbf{Saving Throw:} Fortitude half 

\textbf{Spell Resistance:} Yes 

This seed deals 20d6 points of damage to the target. The damage is of no particular 
type or energy. For each additional 1d6 points of damage dealt, increase the Spellcraft 
DC by +2. If the target is reduced to -10 hit points or less (or a construct, object, 
or undead is reduced to 0 hit points), it is utterly destroyed as if disintegrated, 
leaving behind only a trace of fine dust. Up to a 10-foot cube of nonliving matter 
is affected, so a spell using the \textit{destroy }seed destroys only part of any 
very large object or structure targeted. The \textit{destroy }seed affects even 
magical matter, energy fields, and force effects that are normally only affected 
by the \textit{disintegrate }spell\textit{. }Such effects are automatically destroyed. 
Epic spells using the \textit{ward }seed may also be destroyed, though the caster 
must succeed at an opposed caster level check against the other spellcaster to 
bring down a \textit{ward }spell. 

\vspace{12pt}
\subsection*{SEED: DISPEL }

Abjuration 

\textbf{Spellcraft DC:} 19 

\textbf{Components:} V, S 

\textbf{Casting Time:} 1 minute 

\textbf{Range:} 300 ft. 

\textbf{Target:} One creature, object, or spell 

\textbf{Duration:} Instantaneous 

\textbf{Saving Throw:} None 

\textbf{Spell Resistance:} No 

This seed can end ongoing spells that have been cast on a creature or object, temporarily 
suppress the magical abilities of a magic item, or end ongoing spells (or at least 
their effects) within an area. A dispelled spell ends as if its duration had expired. 
The \textit{dispel }seed can defeat all spells, even those not normally subject 
to \textit{dispel magi}c. The \textit{dispel }seed can dispel (but not counter) 
the ongoing effects of supernatural abilities as well as spells, and it affects 
spell-like effects just as it affects spells. One creature, object, or spell is 
the target of the \textit{dispel }seed. The caster makes a dispel check against 
the spell or against each ongoing spell currently in effect on the object or creature. 
A dispel check is 1d20 + 10 against a DC of 11 + the target spell's caster level. 
For each additional +1 on the dispel check, increase the Spellcraft DC by +1. If 
targeting an object or creature that is the effect of an ongoing spell, make a 
dispel check to end the spell that affects the object or creature. If the object 
targeted is a magic item, make a dispel check against the item's caster level. 
If succeessful, all the item's magical properties are suppressed for 1d4 rounds, 
after which the item recovers on its own. A suppressed item becomes nonmagical 
for the duration of the effect. An interdimensional interface is temporarily closed. 
A magic item's physical properties are unchanged. Any creature, object, or spell 
is potentially subject to the \textit{dispel }seed, even the spells of gods and 
the abilities of artifacts. A character automatically succeeds on the dispel check 
against any spell that he or she cast him or her self. 

\vspace{12pt}
\subsection*{SEED: ENERGY }

Evocation [Acid, Fire, Electricity, Cold, or Sonic] 

\textbf{Spellcraft DC:} 19 

\textbf{Components:} V, S 

\textbf{Casting Time:} 1 minute 

\textbf{Range:} 300 ft. or touched creature or object of 2,000 lb. or less 

\textbf{Area:} A bolt 5 ft. wide to 300 ft. long; or a 10-ft.-radius emanation; 
or a wall whose area is up to one 200-ft. square; or a sphere or hemi-sphere with 
a radius of up to 20 ft. 

\textbf{Duration:} Instantaneous or 20 hours (see text)

\textbf{Saving Throw:} Reflex half 

\textbf{Spell Resistance:} Yes 

This seed uses whichever one of five energy types the caster chooses: acid, cold, 
electricity, fire, or sonic. The caster can cast the energy forth as a bolt, imbue 
an object with the energy, or create a freestanding manifestation of the energy. 
If the spell developed using the \textit{energy }seed releases a bolt, that bolt 
instantaneously deals 10d6 points of damage of the appropriate energy type, and 
all in the bolt's area must make a Reflex save for half damage. For each additional 
1d6 points of damage dealt, increase the Spellcraft DC by +2. The bolt begins at 
the caster's fingertips. To imbue another creature with the ability to use an energy 
bolt as a spell-like ability at its option or when a particular condition is met, 
increase the Spellcraft DC by +25. The caster can also cause a creature or object 
to emanate the specific energy type out to a radius of 10 feet for 20 hours. The 
emanated energy deals 2d6 points of energy damage per round against unprotected 
creatures (the target creature is susceptible if not separately warded or otherwise 
resistant to the energy). For each additional 1d6 points of damage emanated, increase 
the Spellcraft DC by +2. The caster may also create a wall, half-circle, circle, 
dome, or sphere of the desired energy that emanates the energy for up to 20 hours. 
One side of the wall, selected by the caster, sends forth waves of energy, dealing 
2d4 points of energy damage to creatures within 10 feet and 1d4 points of energy 
damage to those past 10 feet but within 20 feet. The wall deals this damage when 
it appears and in each round that a creature enters or remains in the area. In 
addition, the wall deals 2d6+20 points of energy damage to any creature passing 
through it. The wall deals double damage to undead creatures. For each additional 
1d4 points of damage, increase the Spellcraft DC by +2. 

The caster can also use the \textit{energy }seed to create a spell that carefully 
releases and balances the emanation of cold, electricity, and fire, creating specific 
weather effects for a period of 20 hours. Using the \textit{energy }seed this way 
has a base Spellcraft DC of 25. The area extends to a two-mile-radius centered 
on the caster. Once the spell is cast, the weather takes 10 minutes to manifest. 
Ordinarily, a caster can't directly target a creature or object, though indirect 
effects are possible. This seed can create cold snaps, heat waves, thunderstorms, 
fogs, blizzards---even a tornado that moves randomly in the affected area. Creating 
targeted damaging effects requires an additional use of the \textit{energy }seed. 

\vspace{12pt}
\subsection*{SEED: FORESEE }

Divination 

\textbf{Spellcraft DC:} 17 

\textbf{Components:} V, S 

\textbf{Casting Time:} 1 minute 

\textbf{Range:} Personal 

\textbf{Target:} You 

\textbf{Duration:} Instantaneous or concentration (see text) 

The caster can foretell the immediate future, or gain information about specific 
questions. He or she is 90\% likely to receive a meaningful reading of the future 
of the next 30 minutes. If successful, the caster knows if a particular action 
will bring good results, bad results, or no result. For each additional 30 minutes 
into the future, multiply the Spellcraft DC by x2. For better results, the caster 
can pose up to ten specific questions (one per round while he or she concentrates) 
to unknown powers of other planes, but the base Spellcraft DC for such an attempt 
is 23.  The answers return in a language the caster understands, but use only one-word 
replies: ``yes,'' ``no,'' ``maybe,'' ``never,'' ``irrelevant,'' or some other one-word 
answer. Unlike 0- to 9th-level spells of similar type, all questions answered are 
90\% likely to be answered truthfully. However, a specific spell using the \textit{foresee 
}seed can only be cast once every five weeks. The \textit{foresee }seed is also 
useful for epic spells requiring specific information before functioning, such 
as spells using the \textit{reveal }and \textit{transport }seeds. The \textit{foresee 
}seed can also be used to gain one basic piece of information about a living target: 
level, class, alignment, or some special ability (or one of an object's magical 
abilities, if any). For each additional piece of information revealed, increase 
the Spellcraft DC by +2. 

\vspace{12pt}
\subsection*{SEED: FORTIFY }

Transmutation 

\textbf{Spellcraft DC:} 17 (see text) 

\textbf{Components:} V, S 

\textbf{Casting Time:} 1 minute 

\textbf{Range:} Touch 

\textbf{Target:} Creature touched

\textbf{Duration:} 20 hours; permanent for age adjustment 

\textbf{Saving Throw:} Will negates (harmless) 

\textbf{Spell Resistance:} Yes (harmless) 

Spells using the \textit{fortify }seed grant a +1 enhancement bonus to whichever 
one of the following the caster chooses: 

-Any one ability score. 

-Any one kind of saving throw. 

-Spell resistance. 

-Natural armor. 

The \textit{fortify }seed can also grant energy resistance 1 for one energy type 
or 1 temporary hit point. For each additional +1 bonus, point of energy resistance, 
or hit point, increase the Spellcraft DC by +2. 

The \textit{fortify }seed has a base Spellcraft DC of 23 if it grants a +1 bonus 
of a type other than enhancement. For each additional +1 bonus of a type other 
than enhancement, increase the Spellcraft DC by +6. If the caster applies a factor 
to make the duration permanent, the bonus must be an inherent bonus, and the maximum 
inherent bonus allowed is +5. 

The \textit{fortify }seed has a base Spellcraft DC of 27 if it grants a creature 
a +1 bonus to an ability score or other statistic it does not possess. For each 
additional +1 bonus, increase the Spellcraft DC by +4. If a spell with the \textit{fortify 
}seed grants an inanimate object an ability score it would not normally possess 
(such as Intelligence), the spell must also incorporate the \textit{life }seed. 

Granting spell resistance to a creature that doesn't already have it is a special 
case; the base Spellcraft DC of 27 grants spell resistance 25, and each additional 
point of spell resistance increases the Spellcraft DC by +4 (each -1 to spell resistance 
reduces the Spellcraft DC by -2).

The \textit{fortify }seed can also grant damage reduction 1/magic. For each additional 
point of damage reduction, increase the Spellcraft DC by +2. To increase the damage 
reduction value to epic, increase the Spellcraft DC by +15.

A special use of the \textit{fortify }seed grants the target a permanent +1 year 
to its current age category. For each additional +1 year added to the creature's 
current age category, increase the Spellcraft DC by +2. Incremental adjustments 
to a creature's maximum age do not stack; they overlap. When a spell increases 
a creature's current age category, all higher age categories are also adjusted 
accordingly.

\vspace{12pt}
\subsection*{SEED: HEAL }

Conjuration (Healing) 

\textbf{Spellcraft DC:} 25 

\textbf{Components:} V, S, DF 

\textbf{Casting Time:} 1 minute 

\textbf{Range:} Touch 

\textbf{Target:} Creature touched 

\textbf{Duration:} Instantaneous 

\textbf{Saving Throw:} Yes (harmless; see text) 

\textbf{Spell Resistance:} Yes (harmless) 

Spells developed with the \textit{heal }seed channel positive energy into a creature 
to wipe away disease and injury. Such a spell completely cures all diseases, blindness, 
deafness, hit point damage, and temporary ability damage. To restore permanently 
drained ability score points, increase the Spellcraft DC by +6. The \textit{heal 
}seed neutralizes poisons in the subject's system so that no additional damage 
or effects are suffered. It offsets feeblemindedness and cures mental disorders 
caused by spells or injury to the brain. It dispels all magical effects penalizing 
the character's abilities, including effects caused by spells, even epic spells 
developed with the \textit{afflict }seed. Only a single application of the spell 
is needed to simultaneously achieve all these effects. This seed does not restore 
levels or Constitution points lost due to death. To dispel all negative levels 
afflicting the target, increase the Spellcraft DC by +2. This reverses level drains 
by a force or creature. The drained levels are restored only if the creature lost 
the levels within the last 20 weeks. For each additional week since the levels 
were drained, increase the Spellcraft DC by +2. 

Against undead, the influx of positive energy causes the loss of all but 1d4 hit 
points if the undead fails a Fortitude saving throw. 

An epic caster with 24 ranks in Knowledge (arcana), Knowledge (nature), or Knowledge 
(religion) can cast a spell developed with a special version of the heal seed that 
flushes negative energy into the subject, healing undead completely but causing 
the loss of all but 1d4 hit points in living creatures if they fail a Fortitude 
saving throw. Alternatively, a living target that fails its Fortitude saving throw 
could gain four negative levels for the next 8 hours. For each additional negative 
level bestowed, increase the Spellcraft DC by +4, and for each extra hour the negative 
levels persist, increase the Spellcraft DC by +2. If the subject has at least as 
many negative levels as Hit Dice, it dies. If the subject survives and the negative 
levels persist for 24 hours or longer, the subject must make another Fortitude 
saving throw, or the negative levels are converted to actual level loss.

\vspace{12pt}
\subsection*{SEED: LIFE }

Conjuration (Healing) 

\textbf{Spellcraft DC:} 27 

\textbf{Components:} V, S, DF 

\textbf{Casting Time:} 1 minute 

\textbf{Range:} Touch 

\textbf{Target:} Dead creature touched 

\textbf{Duration:} Instantaneous 

\textbf{Saving Throw:} None (see text) 

\textbf{Spell Resistance:} Yes (harmless) 

A spell developed with the \textit{life }seed will restore life and complete vigor 
to any deceased creature. The condition of the remains is not a factor. So long 
as some small portion of the creature's body still exists, it can be returned to 
life, but the portion receiving the spell must have been part of the creature's 
body at the time of death. (The remains of a creature hit by a \textit{disintegrate 
}spell count as a small portion of its body.) The creature can have been dead for 
no longer than two hundred years. For each additional ten years, increase the Spellcraft 
DC by +1. The creature is immediately restored to full hit points, vigor, and health, 
with no loss of prepared spells. However, the subject loses one level (or 1 point 
of Constitution if the subject was 1st level). The life seed cannot revive someone 
who has died of old age. 

An epic caster with 24 ranks in Knowledge (arcana), Knowledge (nature), or Knowledge 
(religion) can cast a spell developed with a special version of the life seed that 
gives actual life to normally inanimate objects. The caster can give inanimate 
plants and animals a soul, personality, and humanlike sentience. To succeed, the 
caster must make a Will save (DC 10 + the target's Hit Dice, or the Hit Dice a 
plant will have once it comes to life).

The newly living object, intelligent animal, or sentient plant is friendly toward 
the caster. An object or plant has characteristics as if it were an animated object, 
except that its Intelligence, Wisdom, and Charisma scores are all 3d6. Animated 
objects and plants gain the ability to move their limbs, projections, roots, carved 
legs and arms, or other appendages, and have senses similar to a human's. A newly 
intelligent animal gets 3d6 Intelligence, +1d3 Charisma, and +2 HD. Objects, animals, 
and plants speak one language that the caster knows, plus one additional language 
that he or she knows per point of Intelligence bonus (if any). 

\vspace{12pt}
\subsection*{SEED: REFLECT }

Abjuration 

\textbf{Spellcraft DC:} 27 

\textbf{Components:} V, S 

\textbf{Casting Time:} 1 minute 

\textbf{Range:} Personal 

\textbf{Target:} You 

\textbf{Duration:} Until expended or 12 hours 

Attacks targeted against the caster rebound on the original attacker. Each use 
of the \textit{reflect }seed in an epic spell is effective against one type of 
attack only: spells (and spell-like effects), ranged attacks, or melee attacks. 
To reflect an area spell, where the caster is not the target but are caught in 
the vicinity, increase the Spellcraft DC by +20. A single successful use of \textit{reflect 
}expends its protection. Spells developed with the \textit{reflect }seed against 
spells and spell-like effects return all spell effects of up to 1st level. For 
each additional level of spells to be reflected, increase the Spellcraft DC by 
+20. Epic spells are treated as 10th-level spells for this purpose. 

The desired effect is automatically reflected if the spell in question is 9th level 
or lower. An opposed caster level check is required when the reflect seed is used 
against another epic spell. If the enemy spellcaster gets his or her spell through 
by winning the caster level check, the epic spell using the reflect seed is not 
expended, just momentarily suppressed.

If the \textit{reflect }seed is used against a melee attack or ranged attack, five 
such attacks are automatically reflected back on the original attacker. For each 
additional attack reflected, increase the Spellcraft DC by +4. The reflected attack 
rebounds on the attacker using the same attack roll. Once the allotted attacks 
are reflected, the spell using the \textit{reflect }seed is expended.

\vspace{12pt}
\subsection*{SEED: REVEAL }

Divination 

\textbf{Spellcraft DC:} 19 (see text) 

\textbf{Components:} V, S 

\textbf{Casting Time:} 1 minute 

\textbf{Range:} See text 

\textbf{Effect:} Magical sensor 

\textbf{Duration:} 20 minutes (D) 

\textbf{Saving Throw:} None 

\textbf{Spell Resistance:} No 

The caster of this seed can see some distant location or hear the sounds at some 
distant location almost as if he or she was there. To both hear and see, increase 
the Spellcraft DC by +2. Distance is not a factor, but the locale must be known---a 
place familiar to the caster or an obvious one. The spell creates an invisible 
sensor that can be dispelled. Lead sheeting or magical protection blocks the spell, 
and the caster senses that the spell is so blocked. If the caster prefers to create 
a mobile sensor (speed 30 feet) that he or she controls, increase the Spellcraft 
DC by +2. To use the \textit{reveal }seed to reach one specific different plane 
of existence, increase the Spellcraft DC by +8. To allow magically enhanced senses 
to work through a spell built with the \textit{reveal }seed, increase the Spellcraft 
DC by +4. To cast any spell from the sensor whose range is touch or greater, increase 
the Spellcraft DC by +6; however, the caster must maintain line of effect to the 
sensor at all times. If the line of effect is obstructed, the spell ends. To free 
the caster of the line of effect restriction for casting spells through the sensor, 
multiply the Spellcraft DC by x10. 

The \textit{reveal }seed has a base Spellcraft DC of 25 if used to pierce illusions 
and see things as they really are. The caster can see through normal and magical 
darkness, notice secret doors hidden by magic, see the exact locations of creatures 
or objects under \textit{blur }or \textit{displacement }effects, see invisible 
creatures or objects normally, see through illusions, see onto the Ethereal Plane 
(but not into extradimensional spaces), and see the true form of polymorphed, changed, 
or transmuted things. The range of such sight is 120 feet. 

The \textit{reveal }seed can also be used to develop spells that will do any one 
of the following: duplicate the \textit{read magic }spell, comprehend the written 
and verbal language of another, or speak in the written or verbal language of another. 
To both comprehend and speak a language, increase the Spellcraft DC by +4. 

\vspace{12pt}
\subsection*{SEED: SLAY }

Necromancy [Death] 

\textbf{Spellcraft DC:} 25 

\textbf{Components:} V, S 

\textbf{Casting Time:} 1 minute 

\textbf{Range:} 300 ft.

\textbf{Target:} One living creature 

\textbf{Duration:} Instantaneous 

\textbf{Saving Throw:} Fortitude partial or half (see text) 

\textbf{Spell Resistance:} Yes 

A spell developed using the \textit{slay }seed snuffs out the life force of a living 
creature, killing it instantly. The \textit{slay }seed kills a creature of up to 
80 HD. The subject is entitled to a Fortitude saving throw to survive the attack. 
If the save is successful, it instead takes 3d6+20 points of damage. For each additional 
80 HD affected (or each additional creature affected), increase the Spellcraft 
DC by +8. Alternatively, a caster can use the \textit{slay }seed in an epic spell 
to suppress the life force of the target by bestowing 2d4 negative levels on the 
target (or half as many negative levels on a successful Fortitude save). For each 
additional 1d4 negative levels bestowed, increase the Spellcraft DC by +4. If the 
subject has at least as many negative levels as Hit Dice, it dies. If the subject 
survives and the negative levels persist for 24 hours or longer, the subject must 
make another Fortitude saving throw, or the negative levels are converted to actual 
level loss. 

\vspace{12pt}
\subsection*{SEED: SUMMON }

Conjuration (Summoning) 

\textbf{Spellcraft DC:} 14 

\textbf{Components:} V, S 

\textbf{Casting Time:} 1 minute 

\textbf{Range:} 75 ft. 

\textbf{Effect:} One summoned creature 

\textbf{Duration:} 20 rounds (D) 

\textbf{Saving Throw:} Will negates (see text) 

\textbf{Spell Resistance:} Yes (see text) 

This seed can summon an outsider. It appears where the caster designates and acts 
immediately, on his or her turn, if its spell resistance is overcome and it fails 
a Will saving throw. It attacks the caster's opponents to the best of its ability. 
If the caster can communicate with the outsider, he or she can direct it not to 
attack, to attack particular enemies, or to perform other actions. The spell conjures 
an outsider the caster selects of CR 2 or less. For each +1 CR of the summoned 
outsider, increase the Spellcraft DC by +2. For each additional outsider of the 
same Challenge Rating summoned, multiply the Spellcraft DC by x2. When a caster 
develops a spell with the \textit{summon }seed that summons an air, chaotic, earth, 
evil, fire, good, lawful, or water creature, the completed spell is also of that 
type. 

If the caster increases the Spellcraft DC by +10, he or she can summon a creature 
of CR 2 or less from another monster type or subtype. The summoned creature is 
assumed to have been plucked from some other plane (or somewhere on the same plane). 
The summoned creature attacks the caster's opponents to the best of its ability; 
or, if the caster can communicate with it, it will perform other actions. However, 
the summoning ends if the creature is asked to perform a task inimical to its nature. 
For each +1 CR of the summoned creature, increase the Spellcraft DC by +2. 

Finally, by increasing the Spellcraft DC by +60, the caster can summon a unique 
individual he or she specifies from anywhere in the multiverse. The caster must 
know the target's name and some facts about its life, defeat any magical protection 
against discovery or other protection possessed by the target, and overcome the 
target's spell resistance, and it must fail a Will saving throw. The target is 
under no special compulsion to serve the caster. 

\vspace{12pt}
\subsection*{SEED: TRANSFORM }

Transmutation 

\textbf{Spellcraft DC:} 21 

\textbf{Components:} V, S 

\textbf{Casting Time:} 1 minute 

\textbf{Range:} 300 ft. 

\textbf{Target:} One creature or inanimate, nonmagical object 

\textbf{Duration:} Permanent 

\textbf{Saving Throw:} Fortitude negates (see text) 

\textbf{Spell Resistance:} Yes 

Spells using the \textit{transform }seed change the subject into another form of 
creature or object. The new form can range in size from Diminutive to one size 
larger than the subject's normal form. For each additional increment of size change, 
increase the Spellcraft DC by +6. If the caster wants to transform a nonmagical, 
inanimate object into a creature of his or her type or transform a creature into 
a nonmagical, inanimate object, increase the Spellcraft DC by +10. To change a 
creature of one type into another type increase the Spellcraft DC by +5. 

Transformations involving nonmagical, inanimate substances with hardness are more 
difficult; for each 2 points of hardness, increase the Spellcraft DC by +1. 

To transform a creature into an incorporeal or gaseous form, increase the Spellcraft 
DC by +10. Conversely, to overcome the natural immunity of a gaseous or incorporeal 
creature to transformation, increase the Spellcraft DC by +10. 

The \textit{transform }seed can also change its target into someone specific. To 
transform an object or creature into the specific likeness of another individual 
(including memories and mental abilities), increase the Spellcraft DC by +25. If 
the transformed creature doesn't have the level or Hit Dice of its new likeness, 
it can only use the abilities of the creature at its own level or Hit Dice. If 
slain or destroyed, the transformed creature or object reverts to its original 
form. The subject's equipment, if any, remains untransformed or melds into the 
new form's body, at the caster's option. The transformed creature or object acquires 
the physical and natural abilities of the creature or object it has been changed 
into while retaining its own memories and mental ability scores. Mental abilities 
include personality, Intelligence, Wisdom, and Charisma scores, level and class, 
hit points (despite any change in its Constitution score), alignment, base attack 
bonus, base saves, extraordinary abilities, spells, and spell-like abilities, but 
not its supernatural abilities. Physical abilities include natural size and Strength, 
Dexterity, and Constitution scores. Natural abilities include armor, natural weapons, 
and similar gross physical qualities (presence or absence of wings, number of extremities, 
and so forth), and possibly hardness. Creatures transformed into inanimate objects 
do not gain the benefit of their untransformed physical abilities, and may well 
be blind, deaf, dumb, and unfeeling. Objects transformed into creatures gain that 
creature's average physical ability scores, but are considered to have mental ability 
scores of 0 (the \textit{fortify }seed can add points to each mental ability, if 
desired). For each normal extraordinary ability or supernatural ability granted 
to the transformed creature, increase the Spellcraft DC by +10. The transformed 
subject can have no more Hit Dice than the caster has or than the subject has (whichever 
is greater). In any case, for each Hit Die the assumed form has above 15, increase 
the Spellcraft DC by +2. 

\vspace{12pt}
\subsection*{SEED: TRANSPORT }

Conjuration [Teleportation] 

\textbf{Spellcraft DC:} 27 

\textbf{Components:} V, S 

\textbf{Casting Time:} 1 minute 

\textbf{Range:} Touch 

\textbf{Target:} You and touched objects or other touched willing creatures weighing 
up to 1,000 lb. 

\textbf{Duration:} Instantaneous, or 5 rounds for temporal transport 

\textbf{Saving Throw:} None or Will negates (see text) 

\textbf{Spell Resistance:} No or Yes (see text) 

Spells using the \textit{transport }seed instantly take the caster to a designated 
destination, regardless of distance. For interplanar travel, increase the Spellcraft 
DC by +4. For each additional 50 pounds in objects and willing creatures beyond 
the base 1,000 pounds, increase the Spellcraft DC by +2. The base use of the \textit{transport 
}seed provides instantaneous travel through the Astral Plane. To shift the transportation 
medium to another medium increase the Spellcraft DC by +2. The caster does not 
need to make a saving throw, nor is spell resistance applicable to him or her. 
Only objects worn or carried (attended) by another person receive saving throws 
and spell resistance. For a spell intended to transport unwilling creatures, increase 
the Spellcraft DC by +4. The caster must have at least a reliable description of 
the place to which he or she is transporting. If the caster attempts to use the 
\textit{transport }seed with insufficient or misleading information, the character 
disappears and simply reappear in his or her original location. 

As a special use of the \textit{transport }seed, a caster can develop a spell that 
temporarily transports him or her into a different time stream (leaving the caster 
in the same physical location); this increases the Spellcraft DC by +8. If the 
caster moves him or herself, or the subject, into a slower time stream for 5 rounds, 
time ceases to flow for the subject, and its condition becomes fixed---no force 
or effect can harm it until the duration expires. If the caster moves him or her 
self into a faster time stream, the caster speeds up so greatly that all other 
creatures seem frozen, though they are actually still moving at their normal speeds. 
The caster is free to act for 5 rounds of apparent time. Fire, cold, poison gas, 
and similar effects can still harm the caster. While the caster is in the fast 
time stream, other creatures are invulnerable to his or her attacks and spells; 
however, the caster can create spell effects and leave them to take effect when 
he or she reenters normal time. Because of the branching nature of time, epic spells 
used to transport a subject into a faster time stream cannot be made permanent, 
nor can the duration of 5 rounds be extended. More simply, the seed can \textit{haste 
}or \textit{slow }a subject for 20 rounds by transporting it to the appropriate 
time stream. This decreases the Spellcraft DC by -4. 

\vspace{12pt}
\subsection*{SEED: WARD }

Abjuration 

\textbf{Spellcraft DC:} 14 

\textbf{Components:} V, S 

\textbf{Casting Time:} 1 minute 

\textbf{Range:} Touch 

\textbf{Target or Effect:} Touched creature or object of 2,000 lb. or less; or 
10-ft.-radius spherical emanation, centered on you 

\textbf{Duration:} 24 hours 

\textbf{Saving Throw:} None 

\textbf{Spell Resistance:} Yes 

This seed can grant a creature protection from damage of a specified type. The 
caster can protect a creature from standard damage or from energy damage. The caster 
can protect a creature or area from magic. Alternatively, he or she can hedge out 
a type of creature from a specified area. A ward against standard damage protects 
a creature from whichever two the caster selects of the three damage types: bludgeoning, 
piercing, and slashing. For a ward against all three types, increase the Spellcraft 
DC by +4. Each round, the spell created with the \textit{ward }seed absorbs the 
first 5 points of damage the creature would otherwise take, regardless of whether 
the source of the damage is natural or magical. For each additional point of protection, 
increase the Spellcraft DC by +2. 

A ward against energy grants a creature protection from whichever one the caster 
selects of the five energy types: acid, cold, electricity, fire, or sonic. Each 
round, the spell absorbs the first 5 points of damage the creature would otherwise 
take from the specified energy type, regardless of whether the source of damage 
is natural or magical. The spell protects the recipient's equipment as well. For 
each additional point of protection, increase the Spellcraft DC by +1. 

A ward against a specific type of creature prevents bodily contact from whichever 
one of several monster types the caster selects. This causes the natural weapon 
attacks of such creatures to fail and the creatures to recoil if such attacks require 
touching the warded creature. The protection ends if the warded creature makes 
an attack against or intentionally moves within 5 feet of the blocked creature. 
Spell resistance can allow a creature to overcome this protection and touch the 
warded creature. 

A ward against magic creates an immobile, faintly shimmering magical sphere (with 
radius 10 feet) that surrounds the caster and excludes all spell effects of up 
to 1st level. Alternatively, the caster can ward just the target and not create 
the radius effect. For each additional level of spells to be excluded, increase 
the Spellcraft DC by +20 (but see below). The area or effect of any such spells 
does not include the area of the ward, and such spells fail to affect any target 
within the ward. This includes spell-like abilities and spells or spell-like effects 
from magic items. However, any type of spell can be cast through or out of the 
ward. The caster can leave and return to the protected area without penalty (unless 
the spell specifically targets a creature and does not provide a radius effect). 
The ward could be brought down by a targeted \textit{dispel magic }spell. Epic 
spells using the \textit{dispel }seed may bring down a ward if the enemy spellcaster 
succeeds at a caster level check. The ward may also be brought down with a targeted 
epic spell using the \textit{destroy }seed if the enemy spellcaster succeeds at 
a caster level check. 

Instead of creating an epic spell that uses the \textit{ward }seed to nullify all 
spells of a given level and lower, the caster can create a ward that nullifies 
a specific spell (or specific set of spells). For each specific spell so nullified, 
increase the Spellcraft DC by +2 per spell level above 1st. 

\vspace{12pt}
\subsection*{{\LARGE EPIC PSIONIC POWERS }}

The following adjustments should be made if taking psionic characters to epic levels.

\vspace{12pt}
\subsection*{Epic Psionic Seeds }

Psionic characters can acquire epic powers.  Generally, all the epic spell rules 
work for epic powers as well, except as noted below for displays.

Psionic characters take the Epic Manifestation feat, which works just like the 
Epic Spellcasting feat. The prerequisites for this feat are 24 ranks of Psicraft, 
24 ranks of Knowledge (psionics), and the ability to manifest 9th-level psionic 
powers. 

Just as spellcasters use no spell slots to cast epic spells, psionic characters 
use no power points to manifest epic powers. Instead, they freely manifest their 
known epic powers a number of times per day equal to their Knowledge (psionics) 
skill divided by 10 (round down). 

\vspace{12pt}
\textbf{Table: Psionic Seeds }

\begin{longtable}{llllllll}
\hline
% ROW 1
\multicolumn{1}{|p{1.271in}|}{\begin{minipage}[t]{1.271in}\centering
\end{minipage}} & \multicolumn{1}{p{0.865in}|}{\begin{minipage}[t]{0.865in}\centering
\textbf{Base Psicraft DC}\end{minipage}} & \multicolumn{1}{p{0.875in}|}{\begin{minipage}[t]{0.875in}\centering
\end{minipage}} & \multicolumn{1}{p{0.875in}|}{\begin{minipage}[t]{0.875in}\centering
\textbf{Base Psicraft DC}\end{minipage}}\\
\hline
% ROW 2
\multicolumn{1}{p{0.069in}|}{\begin{minipage}[t]{0.069in}\centering
\textit{Psychometabolism }\end{minipage}} & \multicolumn{1}{p{0.069in}|}{\begin{minipage}[t]{0.069in}\centering
\end{minipage}} & \multicolumn{1}{p{0.069in}|}{\begin{minipage}[t]{0.069in}\centering
\textit{Telepathy }\end{minipage}} & \multicolumn{1}{p{0.069in}|}{\begin{minipage}[t]{0.069in}\centering
\end{minipage}}\\
\hline
% ROW 3
\multicolumn{1}{|p{1.271in}|}{\begin{minipage}[t]{1.271in}\centering
Fortify\end{minipage}} & \multicolumn{1}{p{0.865in}|}{\begin{minipage}[t]{0.865in}\centering
17\end{minipage}} & \multicolumn{1}{p{0.875in}|}{\begin{minipage}[t]{0.875in}\centering
Compel\end{minipage}} & \multicolumn{1}{p{0.875in}|}{\begin{minipage}[t]{0.875in}\centering
19\end{minipage}}\\
\hline
% ROW 4
\multicolumn{1}{p{0.069in}|}{\begin{minipage}[t]{0.069in}\centering
Slay\end{minipage}} & \multicolumn{1}{p{0.069in}|}{\begin{minipage}[t]{0.069in}\centering
25\end{minipage}} & \multicolumn{1}{p{0.069in}|}{\begin{minipage}[t]{0.069in}\centering
Contact\end{minipage}} & \multicolumn{1}{p{0.069in}|}{\begin{minipage}[t]{0.069in}\centering
23\end{minipage}}\\
\hline
% ROW 5
\multicolumn{1}{|p{1.271in}|}{\begin{minipage}[t]{1.271in}\centering
Transform\end{minipage}} & \multicolumn{1}{p{0.865in}|}{\begin{minipage}[t]{0.865in}\centering
21\end{minipage}} & \multicolumn{1}{p{0.875in}|}{\begin{minipage}[t]{0.875in}\centering
Delude\end{minipage}} & \multicolumn{5}{p{1.153in}|}{\begin{minipage}[t]{1.153in}\centering
14\end{minipage}}\\
\hline
% ROW 6
\multicolumn{1}{p{0.069in}|}{\begin{minipage}[t]{0.069in}\centering
Heal\end{minipage}} & \multicolumn{1}{p{0.069in}|}{\begin{minipage}[t]{0.069in}\centering
50\end{minipage}} & \multicolumn{1}{p{0.069in}|}{\begin{minipage}[t]{0.069in}\centering
\textit{Psychokinesis }\end{minipage}} & \multicolumn{1}{p{0.069in}|}{\begin{minipage}[t]{0.069in}\centering
\end{minipage}}\\
\hline
% ROW 7
\multicolumn{1}{|p{1.271in}|}{\begin{minipage}[t]{1.271in}\centering
\textit{Psychoportation }\end{minipage}} & \multicolumn{1}{p{0.865in}|}{\begin{minipage}[t]{0.865in}\centering
\end{minipage}} & \multicolumn{1}{p{0.875in}|}{\begin{minipage}[t]{0.875in}\centering
Dispel\end{minipage}} & \multicolumn{5}{p{1.153in}|}{\begin{minipage}[t]{1.153in}\centering
19\end{minipage}}\\
\hline
% ROW 8
\multicolumn{1}{|p{1.271in}|}{\begin{minipage}[t]{1.271in}\centering
Banish\end{minipage}} & \multicolumn{1}{p{0.865in}|}{\begin{minipage}[t]{0.865in}\centering
27\end{minipage}} & \multicolumn{1}{p{0.875in}|}{\begin{minipage}[t]{0.875in}\centering
Energy\end{minipage}} & \multicolumn{5}{p{1.153in}|}{\begin{minipage}[t]{1.153in}\centering
19\end{minipage}}\\
\hline
% ROW 9
\multicolumn{1}{|p{1.271in}|}{\begin{minipage}[t]{1.271in}\centering
Summon\end{minipage}} & \multicolumn{1}{p{0.865in}|}{\begin{minipage}[t]{0.865in}\centering
14\end{minipage}} & \multicolumn{1}{p{0.875in}|}{\begin{minipage}[t]{0.875in}\centering
Reflect\end{minipage}} & \multicolumn{5}{p{1.153in}|}{\begin{minipage}[t]{1.153in}\centering
27\end{minipage}}\\
\hline
% ROW 10
\multicolumn{1}{|p{1.271in}|}{\begin{minipage}[t]{1.271in}\centering
Transport\end{minipage}} & \multicolumn{1}{p{0.865in}|}{\begin{minipage}[t]{0.865in}\centering
27\end{minipage}} & \multicolumn{1}{p{0.875in}|}{\begin{minipage}[t]{0.875in}\centering
Destroy\end{minipage}} & \multicolumn{5}{p{1.153in}|}{\begin{minipage}[t]{1.153in}\centering
29\end{minipage}}\\
\hline
% ROW 11
\multicolumn{1}{|p{1.271in}|}{\begin{minipage}[t]{1.271in}\centering
\textit{Clairsentience }\end{minipage}} & \multicolumn{1}{p{0.865in}|}{\begin{minipage}[t]{0.865in}\centering
\end{minipage}} & \multicolumn{1}{p{0.875in}|}{\begin{minipage}[t]{0.875in}\centering
Ward\end{minipage}} & \multicolumn{5}{p{1.153in}|}{\begin{minipage}[t]{1.153in}\centering
14\end{minipage}}\\
\hline
% ROW 12
\multicolumn{1}{|p{1.271in}|}{\begin{minipage}[t]{1.271in}\centering
Afflict\end{minipage}} & \multicolumn{1}{p{0.865in}|}{\begin{minipage}[t]{0.865in}\centering
14\end{minipage}} & \multicolumn{1}{p{0.875in}|}{\begin{minipage}[t]{0.875in}\centering
\textit{Metacreativity }\end{minipage}} & \multicolumn{5}{p{1.153in}|}{\begin{minipage}[t]{1.153in}\centering
\end{minipage}}\\
\hline
% ROW 13
\multicolumn{1}{|p{1.271in}|}{\begin{minipage}[t]{1.271in}\centering
Foresee\end{minipage}} & \multicolumn{1}{p{0.865in}|}{\begin{minipage}[t]{0.865in}\centering
17\end{minipage}} & \multicolumn{1}{p{0.875in}|}{\begin{minipage}[t]{0.875in}\centering
Armor\end{minipage}} & \multicolumn{5}{p{1.153in}|}{\begin{minipage}[t]{1.153in}\centering
14\end{minipage}}\\
\hline
% ROW 14
\multicolumn{1}{|p{1.271in}|}{\begin{minipage}[t]{1.271in}\centering
Reveal\end{minipage}} & \multicolumn{1}{p{0.865in}|}{\begin{minipage}[t]{0.865in}\centering
19\end{minipage}} & \multicolumn{1}{p{0.875in}|}{\begin{minipage}[t]{0.875in}\centering
Conjure\end{minipage}} & \multicolumn{5}{p{1.153in}|}{\begin{minipage}[t]{1.153in}\centering
21\end{minipage}}\\
\hline
% ROW 15
\multicolumn{1}{|p{1.271in}|}{\begin{minipage}[t]{1.271in}\centering
Conceal\end{minipage}} & \multicolumn{1}{p{0.865in}|}{\begin{minipage}[t]{0.865in}\centering
17\end{minipage}} & \multicolumn{1}{p{0.875in}|}{\begin{minipage}[t]{0.875in}\centering
Animate dead\end{minipage}} & \multicolumn{5}{p{1.153in}|}{\begin{minipage}[t]{1.153in}\centering
23\end{minipage}}\\
\hline
% ROW 16
\multicolumn{1}{|p{1.271in}|}{\begin{minipage}[t]{1.271in}\centering
\end{minipage}} & \multicolumn{1}{p{0.865in}|}{\begin{minipage}[t]{0.865in}\centering
Animate\end{minipage}} & \multicolumn{1}{p{0.875in}|}{\begin{minipage}[t]{0.875in}\centering
25\end{minipage}}\\
\hline
% ROW 17
\multicolumn{5}{p{1.153in}|}{\begin{minipage}[t]{1.153in}\centering
\end{minipage}} & \multicolumn{1}{|p{1.271in}|}{\begin{minipage}[t]{1.271in}\centering
Life\end{minipage}} & \multicolumn{1}{p{0.865in}|}{\begin{minipage}[t]{0.865in}\centering
55\end{minipage}}\\
\hline
\end{longtable}

\vspace{12pt}
\subsection*{\textbf{Table: Psionic Factors}}

\subsection*{\begin{longtable}{|p{2.760in}|p{1.443in}|p{0.069in}|p{0.069in}|}
\hline
% ROW 1
\begin{minipage}[t]{2.760in}\raggedright \end{minipage} & \begin{minipage}[t]{1.443in}\centering \textbf{Psicraft 
DC Modifier}\end{minipage}\\
\hline
% ROW 2
\begin{minipage}[t]{0.069in}\centering \textit{Discipline }\end{minipage} & \begin{minipage}[t]{0.069in}\centering -5\end{minipage}\\
\hline
% ROW 3
\begin{minipage}[t]{2.760in}\centering Seed within primary discipline\end{minipage} & \begin{minipage}[t]{1.443in} \multirow{2}{1.443in}{\centering }\end{minipage}\\
\cline{1-1}\cline{2-2}\cline{3-3}\cline{4-4}
% ROW 4
\begin{minipage}[t]{0.069in}\centering \textit{Display }\end{minipage} & \begin{minipage}[t]{0.069in} \multirow{2}{0.069in}{\centering }\end{minipage}\\
\cline{1-1}\cline{2-2}\cline{3-3}\cline{4-4}
% ROW 5
\begin{minipage}[t]{2.760in}\centering Hide visual display (epic psionic seeds 
substitute one Vi display for V and S components)\end{minipage} & \begin{minipage}[t]{1.443in}\centering +4\end{minipage}\\
\hline
\end{longtable}
}



\end{document}
