%&LaTeX
% !TEX encoding = UTF-8 Unicode
\documentclass{article}
\usepackage[utf8x]{inputenc}
\usepackage[T1]{fontenc}
\usepackage{textcomp}

\usepackage{longtable}

\newcommand{\tab}{\hspace{5mm}}

pen Game Content, and is licensed for public use under the terms 
of the Open Game License v1.0a.

\subsubsection*{{\LARGE TYPES, SUBTYPES, \& SPECIAL ABILITIES}}

\vspace{12pt}
\textbf{Aberration Type:} An aberration has a bizarre anatomy, strange abilities, 
an alien mindset, or any combination of the three.

\textit{Features: }An aberration has the following features.---

d8 Hit Dice.---

Base attack bonus equal to 3/4 total Hit Dice (as cleric).---

Good Will saves.---

Skill points equal to (2 + Int modifier, minimum 1) per Hit Die, with quadruple 
skill points for the first Hit Die.

\textit{Traits: }An aberration possesses the following traits (unless otherwise 
noted in a creature's entry).---

Darkvision out to 60 feet.---

Proficient with its natural weapons. If generally humanoid in form, proficient 
with all simple weapons and any weapon it is described as using.---

Proficient with whatever type of armor (light, medium, or heavy) it is described 
as wearing, as well as all lighter types. Aberrations not indicated as wearing 
armor are not proficient with armor. Aberrations are proficient with shields if 
they are proficient with any form of armor.---

Aberrations eat, sleep, and breathe.

\vspace{12pt}
\textbf{Ability Score Loss (Su):} Some attacks reduce the opponent's score in one 
or more abilities. This loss can be temporary (ability damage) or permanent (ability 
drain).

\textit{Ability Damage: }This attack damages an opponent's ability score. The creature's 
descriptive text gives the ability and the amount of damage. If an attack that 
causes ability damage scores a critical hit, it deals twice the indicated amount 
of damage (if the damage is expressed as a die range, roll two dice). Ability damage 
returns at the rate of 1 point per day for each affected ability.

\textit{Ability Drain: }This effect permanently reduces a living opponent's ability 
score when the creature hits with a melee attack. The creature's descriptive text 
gives the ability and the amount drained. If an attack that causes ability drain 
scores a critical hit, it drains twice the indicated amount (if the damage is expressed 
as a die range, roll two dice). Unless otherwise specified in the creature's description, 
a draining creature gains 5 temporary hit points (10 on a critical hit) whenever 
it drains an ability score no matter how many points it drains. Temporary hit points 
gained in this fashion last for a maximum of 1 hour.

Some ability drain attacks allow a Fortitude save (DC 10 + 1/2 draining creature's 
racial HD + draining creature's Cha modifier; the exact DC is given in the creature's 
descriptive text). If no saving throw is mentioned, none is allowed.

\vspace{12pt}
\textbf{Alternate Form (Su):} A creature with this special quality has the ability 
to assume one or more specific alternate forms. This ability works much like the 
\textit{polymorph }spell, except that the creature is limited to the forms specified, 
and does not regain any hit points for changing its form. Assuming an alternate 
form results in the following changes to the creature:---

The creature retains the type and subtype of its original form. It gains the size 
of its new form.---

The creature loses the natural weapons, natural armor, movement modes, and extraordinary 
special attacks of its original form.---

The creature gains the natural weapons, natural armor, movement modes, and extraordinary 
special attacks of its new form.---

The creature retains the special qualities of its original form. It does not gain 
any special qualities of its new form.---

The creature retains the spell-like abilities and supernatural attacks of its old 
form (except for breath weapons and gaze attacks). It does not gain the spell-like 
abilities or supernatural attacks of its new form.---

The creature gains the physical ability scores (Str, Dex, Con) of its new form. 
It retains the mental ability scores (Int, Wis, Cha) of its original form.---

The creature retains its hit points and save bonuses, although its save modifiers 
may change due to a change in ability scores.---

The creature retains any spellcasting ability it had in its original form, although 
it must be able to speak intelligibly to cast spells with verbal components and 
it must have humanlike hands to cast spells with somatic components.---

The creature is effectively camouflaged as a creature of its new form, and it gains 
a +10 bonus on Disguise checks if it uses this ability to create a disguise.

\vspace{12pt}
\textbf{Air Subtype:} This subtype usually is used for elementals and outsiders 
with a connection to the Elemental Plane Air. Air creatures always have fly speeds 
and usually have perfect maneuverability.

\vspace{12pt}
\textbf{Angel Subtype:} Angels are a race of celestials, or good outsiders, native 
to the good-aligned Outer Planes.

\textit{Traits: }An angel possesses the following traits (unless otherwise noted 
in a creature's entry).---

Darkvision out to 60 feet and low-light vision.---

Immunity to acid, cold, and petrification.---

Resistance to electricity 10 and fire 10.---

+4 racial bonus on saves against poison.---

Protective Aura (Su): Against attacks made or effects created by evil creatures, 
\textit{t}his ability provides a +4 deflection bonus to AC and a +4 resistance 
bonus on saving throws to anyone within 20 feet of the angel. Otherwise, it functions 
as a \textit{magic circle against evil }effect and a \textit{lesser globe of invulnerability, 
}both with a radius of 20 feet (caster level equals angel's HD). (The defensive 
benefits from the circle are not included in an angel's statistics block.) ---

Tongues (Su): All angels can speak with any creature that has a language, as though 
using a \textit{tongues }spell (caster level equal to angel's Hit Dice). This ability 
is always active.

\vspace{12pt}
\textbf{Animal Type:} An animal is a living, nonhuman creature, usually a vertebrate 
with no magical abilities and no innate capacity for language or culture.

\textit{Features: }An animal has the following features (unless otherwise noted 
in a creature's entry).---

d8 Hit Dice.---

Base attack bonus equal to 3/4 total Hit Dice (as cleric).---

Good Fortitude and Reflex saves (certain animals have different good saves).---

Skill points equal to (2 + Int modifier, minimum 1) per Hit Die, with quadruple 
skill points for the first Hit Die.

\textit{Traits: }An animal possesses the following traits (unless otherwise noted 
in a creature's entry).---

Intelligence score of 1 or 2 (no creature with an Intelligence score of 3 or higher 
can be an animal).---

Low-light vision.---

Alignment: Always neutral.---

Treasure: None.---

Proficient with its natural weapons only. A noncombative herbivore uses its natural 
weapons as a secondary attack. Such attacks are made with a -5 penalty on the creature's 
attack rolls, and the animal receives only 1/2 its Strength modifier as a damage 
adjustment.---

Proficient with no armor unless trained for war.---

Animals eat, sleep, and breathe.

\vspace{12pt}
\textbf{Aquatic Subtype:} These creatures always have swim speeds and thus can 
move in water without making Swim checks. An aquatic creature can breathe underwater. 
It cannot also breathe air unless it has the amphibious special quality. 

\vspace{12pt}
\textbf{Archon Subtype:} Archons are a race of celestials, or good outsiders, native 
to lawful good-aligned Outer Planes.

\textit{Traits: }An archon possesses the following traits (unless otherwise noted 
in a creature's entry).---

Darkvision out to 60 feet and low-light vision.---

Aura of Menace (Su): A righteous aura surrounds archons that fight or get angry. 
Any hostile creature within a 20-foot radius of an archon must succeed on a Will 
save to resist its effects. The save DC varies with the type of archon, is Charisma-based, 
and includes a +2 racial bonus. Those who fail take a -2 penalty on attacks, AC, 
and saves for 24 hours or until they successfully hit the archon that generated 
the aura. A creature that has resisted or broken the effect cannot be affected 
again by the same archon's aura for 24 hours.---

Immunity to electricity and petrification.---

+4 racial bonus on saves against poison.---

Magic Circle against Evil (Su): A magic circle against evil effect always surrounds 
an archon (caster level equals the archon's Hit Dice). (The defensive benefits 
from the circle are not included in an archon's statistics block.)---

Teleport (Su): Archons can use greater teleport at will, as the spell (caster level 
14th), except that the creature can transport only itself and up to 50 pounds of 
objects.---

Tongues (Su): All archons can speak with any creature that has a language, as though 
using a \textit{tongues }spell (caster level 14th). This ability is always active.

\vspace{12pt}
\textbf{Augmented Subtype:} A creature receives this subtype whenever something 
happens to change its original type. Some creatures (those with an inherited template) 
are born with this subtype; others acquire it when they take on an acquired template. 
The augmented subtype is always paired with the creature's original type. A creature 
with the augmented subtype usually has the traits of its current type, but the 
features of its original type.

\vspace{12pt}
\textbf{Blindsense (Ex):} Using nonvisual senses, such as acute smell or hearing, 
a creature with blindsense notices things it cannot see. The creature usually does 
not need to make Spot or Listen checks to pinpoint the location of a creature within 
range of its blindsense ability, provided that it has line of effect to that creature. 
Any opponent the creature cannot see still has total concealment against the creature 
with blindsense, and the creature still has the normal miss chance when attacking 
foes that have concealment. Visibility still affects the movement of a creature 
with blindsense. A creature with blindsense is still denied its Dexterity bonus 
to Armor Class against attacks from creatures it cannot see.

\vspace{12pt}
\textbf{Blindsight (Ex):} This ability is similar to blindsense, but is far more 
discerning. Using nonvisual senses, such as sensitivity to vibrations, keen smell, 
acute hearing, or echolocation, a creature with blindsight maneuvers and fights 
as well as a sighted creature. Invisibility, darkness, and most kinds of concealment 
are irrelevant, though the creature must have line of effect to a creature or object 
to discern that creature or object. The ability's range is specified in the creature's 
descriptive text. The creature usually does not need to make Spot or Listen checks 
to notice creatures within range of its blindsight ability. Unless noted otherwise, 
blindsight is continuous, and the creature need do nothing to use it. Some forms 
of blindsight, however, must be triggered as a free action. If so, this is noted 
in the creature's description. If a creature must trigger its blindsight ability, 
the creature gains the benefits of blindsight only during its turn.

\vspace{12pt}
\textbf{Breath Weapon (Su): }A breath weapon attack usually deals damage and is 
often based on some type of energy.

Such breath weapons allow a Reflex save for half damage (DC 10 + 1/2 breathing 
creature's racial HD + breathing creature's Con modifier; the exact DC is given 
in the creature's descriptive text). A creature is immune to its own breath weapon 
unless otherwise noted. Some breath weapons allow a Fortitude save or a Will save 
instead of a Reflex save.

\vspace{12pt}
\textbf{Change Shape (Su):} A creature with this special quality has the ability 
to assume the appearance of a specific creature or type of creature (usually a 
humanoid), but retains most of its own physical qualities. A creature cannot change 
shape to a form more than one size category smaller or larger than its original 
form. Changing shape results in the following changes to the creature:---

The creature retains the type and subtype of its original form. It gains the size 
of its new form.---

The creature loses the natural weapons, movement modes, and extraordinary special 
attacks of its original form.---

The creature gains the natural weapons, movement modes, and extraordinary special 
attacks of its new form.---

The creature retains all other special attacks and qualities of its original form, 
except for breath weapons and gaze attacks.---

The creature retains the ability scores of its original form.---

The creature retains its hit points and saves.---

The creature retains any spellcasting ability it had in its original form, although 
it must be able to speak intelligibly to cast spells with verbal components and 
it must have humanlike hands to cast spells with somatic components.---

The creature is effectively camouflaged as a creature of its new form, and gains 
a +10 bonus on Disguise checks if it uses this ability to create a disguise.

\vspace{12pt}
\textbf{Chaotic Subtype: }A subtype usually applied only to outsiders native to 
the chaotic-aligned Outer Planes. Most creatures that have this subtype also have 
chaotic alignments; however, if their alignments change they still retain the subtype. 
Any effect that depends on alignment affects a creature with this subtype as if 
the creature has a chaotic alignment, no matter what its alignment actually is. 
The creature also suffers effects according to its actual alignment. A creature 
with the chaotic subtype overcomes damage reduction as if its natural weapons and 
any weapons it wields were chaotic-aligned (see Damage Reduction, below).

\vspace{12pt}
\textbf{Cold Subtype: }A creature with the cold subtype has immunity to cold. It 
has vulnerability to fire, which means it takes half again as much (+50\%) damage 
as normal from fire, regardless of whether a saving throw is allowed, or if the 
save is a success or failure. 

\vspace{12pt}
\textbf{Constrict (Ex):} A creature with this special attack can crush an opponent, 
dealing bludgeoning damage, after making a successful grapple check. The amount 
of damage is given in the creature's entry. If the creature also has the improved 
grab ability it deals constriction damage in addition to damage dealt by the weapon 
used to grab.

\vspace{12pt}
\textbf{Construct Type: }A construct is an animated object or artificially constructed 
creature.

\textit{Features: }A construct has the following features.---

10-sided Hit Dice.---

Base attack bonus equal to 3/4 total Hit Dice (as cleric).---

No good saving throws.---

Skill points equal to (2 + Int modifier, minimum 1) per Hit Die, with quadruple 
skill points for the first Hit Die, if the construct has an Intelligence score. 
However, most constructs are mindless and gain no skill points or feats.

\textit{Traits: }A construct possesses the following traits (unless otherwise noted 
in a creature's entry).---

No Constitution score.---

Low-light vision.---

Darkvision out to 60 feet.---

Immunity to all mind-affecting effects (charms, compulsions, phantasms, patterns, 
and morale effects).---

Immunity to poison, sleep effects, paralysis, stunning, disease , death effects, 
and necromancy effects.---

Cannot heal damage on their own, but often can be repaired by exposing them to 
a certain kind of effect (see the creature's description for details) or through 
the use of the Craft Construct feat. A construct with the fast healing special 
quality still benefits from that quality.---

Not subject to critical hits, nonlethal damage, ability damage, ability drain, 
fatigue, exhaustion, or energy drain.---

Immunity to any effect that requires a Fortitude save (unless the effect also works 
on objects, or is harmless).---

Not at risk of death from massive damage. Immediately destroyed when reduced to 
0 hit points or less.---

Since it was never alive, a construct cannot be raised or resurrected.---

Because its body is a mass of unliving matter, a construct is hard to destroy. 
It gains bonus hit points based on size, as shown on the following table.

\begin{longtable}{llllllll}
\hline
% ROW 1
\multicolumn{1}{|p{1.018in}|}{\begin{minipage}[t]{1.018in}\raggedright
\textbf{Construct Size}\end{minipage}} & \multicolumn{1}{p{0.807in}|}{\begin{minipage}[t]{0.807in}\centering
\textbf{Bonus Hit Points}\end{minipage}} & \multicolumn{1}{p{1.000in}|}{\begin{minipage}[t]{1.000in}\centering
\textbf{Construct Size}\end{minipage}} & \multicolumn{1}{p{0.750in}|}{\begin{minipage}[t]{0.750in}\centering
\textbf{Bonus Hit Points}\end{minipage}}\\
\hline
% ROW 2
\multicolumn{1}{p{0.069in}|}{\begin{minipage}[t]{0.069in}\centering
Fine\end{minipage}} & \multicolumn{1}{p{0.069in}|}{\begin{minipage}[t]{0.069in}\centering
---\end{minipage}} & \multicolumn{1}{p{0.069in}|}{\begin{minipage}[t]{0.069in}\centering
Large\end{minipage}} & \multicolumn{1}{p{0.069in}|}{\begin{minipage}[t]{0.069in}\centering
30\end{minipage}}\\
\hline
% ROW 3
\multicolumn{1}{|p{1.018in}|}{\begin{minipage}[t]{1.018in}\centering
Diminutive\end{minipage}} & \multicolumn{1}{p{0.807in}|}{\begin{minipage}[t]{0.807in}\centering
---\end{minipage}} & \multicolumn{1}{p{1.000in}|}{\begin{minipage}[t]{1.000in}\centering
Huge\end{minipage}} & \multicolumn{1}{p{0.750in}|}{\begin{minipage}[t]{0.750in}\centering
40\end{minipage}}\\
\hline
% ROW 4
\multicolumn{1}{p{0.069in}|}{\begin{minipage}[t]{0.069in}\centering
Tiny\end{minipage}} & \multicolumn{1}{p{0.069in}|}{\begin{minipage}[t]{0.069in}\centering
---\end{minipage}} & \multicolumn{1}{p{0.069in}|}{\begin{minipage}[t]{0.069in}\centering
Gargantuan\end{minipage}} & \multicolumn{1}{p{0.069in}|}{\begin{minipage}[t]{0.069in}\centering
60\end{minipage}}\\
\hline
% ROW 5
\multicolumn{1}{|p{1.018in}|}{\begin{minipage}[t]{1.018in}\centering
Small\end{minipage}} & \multicolumn{1}{p{0.807in}|}{\begin{minipage}[t]{0.807in}\centering
10\end{minipage}} & \multicolumn{1}{p{1.000in}|}{\begin{minipage}[t]{1.000in}\centering
Colossal\end{minipage}} & \multicolumn{5}{p{1.028in}|}{\begin{minipage}[t]{1.028in}\centering
80\end{minipage}}\\
\hline
% ROW 6
\multicolumn{1}{p{0.069in}|}{\begin{minipage}[t]{0.069in}\centering
Medium\end{minipage}} & \multicolumn{1}{p{0.069in}|}{\begin{minipage}[t]{0.069in}\centering
20\end{minipage}} & \multicolumn{1}{p{0.069in}|}{\begin{minipage}[t]{0.069in}\centering
\end{minipage}} & \multicolumn{1}{p{0.069in}|}{\begin{minipage}[t]{0.069in}\centering
\end{minipage}}\\
\hline
\end{longtable}
---

Proficient with its natural weapons only, unless generally humanoid in form, in 
which case proficient with any weapon mentioned in its entry.---

Proficient with no armor.---

Constructs do not eat, sleep, or breathe.

\vspace{12pt}
\textbf{Damage Reduction (Ex or Su):} A creature with this special quality ignores 
damage from most weapons and natural attacks. Wounds heal immediately, or the weapon 
bounces off harmlessly (in either case, the opponent knows the attack was ineffective). 
The creature takes normal damage from energy attacks (even nonmagical ones), spells, 
spell-like abilities, and supernatural abilities. A certain kind of weapon can 
sometimes damage the creature normally, as noted below.

The entry indicates the amount of damage ignored (usually 5 to 15 points) and the 
type of weapon that negates the ability. 

Some monsters are vulnerable to piercing, bludgeoning, or slashing damage. 

Some monsters are vulnerable to certain materials, such as alchemical silver, adamantine, 
or cold-forged iron. Attacks from weapons that are not made of the correct material 
have their damage reduced, even if the weapon has an enhancement bonus.

Some monsters are vulnerable to magic weapons. Any weapon with at least a +1 magical 
enhancement bonus on attack and damage rolls overcomes the damage reduction of 
these monsters. Such creatures' natural weapons (but not their attacks with weapons) 
are treated as magic weapons for the purpose of overcoming damage reduction.

A few very powerful monsters are vulnerable only to epic weapons; that is, magic 
weapons with at least a +6 enhancement bonus. Such creatures' natural weapons are 
also treated as epic weapons for the purpose of overcoming damage reduction.

Some monsters are vulnerable to chaotic-, evil-, good-, or lawful-aligned weapons. 
When a cleric casts \textit{align weapon, }affected weapons might gain one or more 
of these properties, and certain magic weapons have these properties as well. A 
creature with an alignment subtype (chaotic, evil, good, or lawful) can overcome 
this type of damage reduction with its natural weapons and weapons it wields as 
if the weapons or natural weapons had an alignment (or alignments) that match the 
subtype(s) of the creature. 

When a damage reduction entry has a dash (-) after the slash, no weapon negates 
the damage reduction.

A few creatures are harmed by more than one kind of weapon. A weapon of either 
type overcomes this damage reduction.

A few other creatures require combinations of different types of attacks to overcome 
their damage reduction. A weapon must be both types to overcome this damage reduction. 
A weapon that is only one type is still subject to damage reduction.

\vspace{12pt}
\textbf{Dragon Type:} A dragon is a reptilelike creature, usually winged, with 
magical or unusual abilities.

\textit{Features: }A dragon has the following features.---

12-sided Hit Dice.---

Base attack bonus equal to total Hit Dice (as fighter).---

Good Fortitude, Reflex, and Will saves.---

Skill points equal to (6 + Int modifier, minimum 1) per Hit Die, with quadruple 
skill points for the first Hit Die.

\textit{Traits: }A dragon possesses the following traits (unless otherwise noted 
in the description of a particular kind).---

Darkvision out to 60 feet and low-light vision.---

Immunity to magic sleep effects and paralysis effects.---

Proficient with its natural weapons only unless humanoid in form (or capable of 
assuming humanoid form), in which case proficient with all simple weapons and any 
weapons mentioned in its entry.---

Proficient with no armor.---

Dragons eat, sleep, and breathe.

\vspace{12pt}
\textbf{Earth Subtype:} This subtype usually is used for elementals and outsiders 
with a connection to the Elemental Plane of Earth. Earth creatures usually have 
burrow speeds, and most earth creatures can burrow through solid rock.

\vspace{12pt}
\textbf{Elemental Type:} An elemental is a being composed of one of the four classical 
elements: air, earth, fire, or water.

\textit{Features: }An elemental has the following features.---

8-sided Hit Dice.---

Base attack bonus equal to 3/4 total Hit Dice (as cleric).---

Good saves depend on the element: Fortitude (earth, water) or Reflex (air, fire).---

Skill points equal to (2 + Int modifier, minimum 1) per Hit Die, with quadruple 
skill points for the first Hit Die.

\textit{Traits: }An elemental possesses the following traits (unless otherwise 
noted in a creature's entry).---

Darkvision out to 60 feet.---

Immunity to poison, sleep effects, paralysis, and stunning.---

Not subject to critical hits or flanking.---

Unlike most other living creatures, an elemental does not have a dual nature---its 
soul and body form one unit. When an elemental is slain, no soul is set loose. 
Spells that restore souls to their bodies, such as \textit{raise dead, reincarnate, 
}and \textit{resurrection, }don't work on an elemental. It takes a different magical 
effect, such as \textit{limited wish, wish, miracle, }or \textit{true resurrection, 
}to restore it to life.---

Proficient with natural weapons only, unless generally humanoid in form, in which 
case proficient with all simple weapons and any weapons mentioned in its entry.---

Proficient with whatever type of armor (light, medium, or heavy) that it is described 
as wearing, as well as all lighter types. Elementals not indicated as wearing armor 
are not proficient with armor. Elementals are proficient with shields if they are 
proficient with any form of armor.---

Elementals do not eat, sleep, or breathe.

\vspace{12pt}
\textbf{Energy Drain (Su):} This attack saps a living opponent's vital energy and 
happens automatically when a melee or ranged attack hits. Each successful energy 
drain bestows one or more negative levels (the creature's description specifies 
how many). If an attack that includes an energy drain scores a critical hit, it 
drains twice the given amount. Unless otherwise specified in the creature's description, 
a draining creature gains 5 temporary hit points (10 on a critical hit) for each 
negative level it bestows on an opponent. These temporary hit points last for a 
maximum of 1 hour. An affected opponent takes a -1 penalty on all skill checks 
and ability checks, attack rolls, and saving throws, and loses one effective level 
or Hit Die (whenever level is used in a die roll or calculation) for each negative 
level. A spellcaster loses one spell slot of the highest level of spells she can 
cast and (if applicable) one prepared spell of that level; this loss persists until 
the negative level is removed. Negative levels remain until 24 hours have passed 
or until they are removed with a spell, such as \textit{restoration}. If a negative 
level is not removed before 24 hours have passed, the affected creature must attempt 
a Fortitude save (DC 10 + 1/2 draining creature's racial HD + draining creature's 
Cha modifier; the exact DC is given in the creature's descriptive text). On a success, 
the negative level goes away with no harm to the creature. On a failure, the negative 
level goes away, but the creature's level is also reduced by one. A separate saving 
throw is required for each negative level.

\vspace{12pt}
\textbf{Evil Subtype: }A subtype usually applied only to outsiders native to the 
evil-aligned Outer Planes. Evil outsiders are also called fiends. Most creatures 
that have this subtype also have evil alignments; however, if their alignments 
change, they still retain the subtype. Any effect that depends on alignment affects 
a creature with this subtype as if the creature has an evil alignment, no matter 
what its alignment actually is. The creature also suffers effects according to 
its actual alignment. A creature with the evil subtype overcomes damage reduction 
as if its natural weapons and any weapons it wields were evil-aligned (see Damage 
Reduction, above).

\vspace{12pt}
\textbf{Extraplanar Subtype:} A subtype applied to any creature when it is on a 
plane other than its native plane. A creature that travels the planes can gain 
or lose this subtype as it goes from plane to plane. Monster entries assume that 
encounters with creatures take place on the Material Plane, and every creature 
whose native plane is not the Material Plane has the extraplanar subtype (but would 
not have when on its home plane). Every extraplanar creature in this book has a 
home plane mentioned in its description. Creatures not labeled as extraplanar are 
natives of the Material Plane, and they gain the extraplanar subtype if they leave 
the Material Plane. No creature has the extraplanar subtype when it is on a transitive 
plane, such as the Astral Plane, the Ethereal Plane, and the Plane of Shadow.

\vspace{12pt}
\textbf{Fast Healing (Ex):} A creature with the fast healing special quality regains 
hit points at an exceptionally fast rate, usually 1 or more hit points per round, 
as given in the creature's entry. Except where noted here, fast healing is just 
like natural healing. Fast healing does not restore hit points lost from starvation, 
thirst, or suffocation, and it does not allow a creature to regrow lost body parts. 
Unless otherwise stated, it does not allow lost body parts to be reattached.

\vspace{12pt}
\textbf{Fear (Su or Sp):} Fear attacks can have various effects.

\textit{Fear Aura (Su): }The use of this ability is a free action. The aura can 
freeze an opponent (such as a mummy's despair) or function like the \textit{fear 
}spell. Other effects are possible. A fear aura is an area effect. The descriptive 
text gives the size and kind of area.

\textit{Fear Cones (Sp) and Rays (Su): }These effects usually work like the \textit{fear 
}spell. 

If a fear effect allows a saving throw, it is a Will save (DC 10 + 1/2 fearsome 
creature's racial HD + creature's Cha modifier; the exact DC is given in the creature's 
descriptive text). All fear attacks are mind-affecting fear effects.

\vspace{12pt}
\textbf{Fey Type:} A fey is a creature with supernatural abilities and connections 
to nature or to some other force or place. Fey are usually human-shaped.

\textit{Features: }A fey has the following features.---

6-sided Hit Dice.---

Base attack bonus equal to 1/2 total Hit Dice (as wizard).---

Good Reflex and Will saves.---

Skill points equal to (6 + Int modifier, minimum 1) per Hit Die, with quadruple 
skill points for the first Hit Die.

\textit{Traits: }A fey possesses the following traits (unless otherwise noted in 
a creature's entry).---

Low-light vision.---

Proficient with all simple weapons and any weapons mentioned in its entry.---

Proficient with whatever type of armor (light, medium, or heavy) that it is described 
as wearing, as well as all lighter types. Fey not indicated as wearing armor are 
not proficient with armor. Fey are proficient with shields if they are proficient 
with any form of armor.---

Fey eat, sleep, and breathe.

\vspace{12pt}
\textbf{Fire Subtype:} A creature with the fire subtype has immunity to fire. It 
has vulnerability to cold, which means it takes half again as much (+50\%) damage 
as normal from cold, regardless of whether a saving throw is allowed, or if the 
save is a success or failure.

\vspace{12pt}
\textbf{Flight (Ex or Su):} A creature with this ability can cease or resume flight 
as a free action. If the ability is supernatural, it becomes ineffective in an 
antimagic field, and the creature loses its ability to fly for as long as the antimagic 
effect persists.

\vspace{12pt}
\textbf{Frightful Presence (Ex):} This special quality makes a creature's very 
presence unsettling to foes. It takes effect automatically when the creature performs 
some sort of dramatic action (such as charging, attacking, or snarling). Opponents 
within range who witness the action may become frightened or shaken. Actions required 
to trigger the ability are given in the creature's descriptive text. The range 
is usually 30 feet, and the duration is usually 5d6 rounds. This ability affects 
only opponents with fewer Hit Dice or levels than the creature has. An affected 
opponent can resist the effects with a successful Will save (DC 10 + 1/2 frightful 
creature's racial HD + frightful creature's Cha modifier; the exact DC is given 
in the creature's descriptive text). An opponent that succeeds on the saving throw 
is immune to that same creature's frightful presence for 24 hours. Frightful presence 
is a mind-affecting fear effect. 

\vspace{12pt}
\textbf{Gaze (Su):} A gaze special attack takes effect when opponents look at the 
creature's eyes. The attack can have almost any sort of effect: petrification, 
death, charm, and so on. The typical range is 30 feet, but check the creature's 
entry for details. The type of saving throw for a gaze attack varies, but it is 
usually a Will or Fortitude save (DC 10 + 1/2 gazing creature's racial HD + gazing 
creature's Cha modifier; the exact DC is given in the creature's descriptive text). 
A successful saving throw negates the effect. A monster's gaze attack is described 
in abbreviated form in its description. Each opponent within range of a gaze attack 
must attempt a saving throw each round at the beginning of his or her turn in the 
initiative order. Only looking directly at a creature with a gaze attack leaves 
an opponent vulnerable. Opponents can avoid the need to make the saving throw by 
not looking at the creature, in one of two ways. 

\textit{Averting Eyes: }The opponent avoids looking at the creature's face, instead 
looking at its body, watching its shadow, tracking it in a reflective surface, 
and so on. Each round, the opponent has a 50\% chance to not need to make a saving 
throw against the gaze attack. The creature with the gaze attack, however, gains 
concealment against that opponent.

\textit{Wearing a Blindfold: }The opponent cannot see the creature at all (also 
possible to achieve by turning one's back on the creature or shutting one's eyes). 
The creature with the gaze attack gains total concealment against the opponent. 

A creature with a gaze attack can actively gaze as an attack action by choosing 
a target within range. That opponent must attempt a saving throw but can try to 
avoid this as described above. Thus, it is possible for an opponent to save against 
a creature's gaze twice during the same round, once before the opponent's action 
and once during the creature's turn. 

Gaze attacks can affect ethereal opponents. A creature is immune to gaze attacks 
of others of its kind unless otherwise noted.

Allies of a creature with a gaze attack might be affected. All the creature's allies 
are considered to be averting their eyes from the creature with the gaze attack, 
and have a 50\% chance to not need to make a saving throw against the gaze attack 
each round. The creature also can veil its eyes, thus negating its gaze ability.

\vspace{12pt}
\textbf{Giant Type:} A giant is a humanoid-shaped creature of great strength, usually 
of at least Large size.

\textit{Features: }A giant has the following features.---

8-sided Hit Dice.---

Base attack bonus equal to 3/4 total Hit Dice (as cleric).---

Good Fortitude saves.---

Skill points equal to (2 + Int modifier, minimum 1) per Hit Die, with quadruple 
skill points for the first Hit Die.

\textit{Traits: }A giant possesses the following traits (unless otherwise noted 
in a creature's entry).---

Low-light vision.---

Proficient with all simple and martial weapons, as well as any natural weapons.---

Proficient with whatever type of armor (light, medium or heavy) it is described 
as wearing, as well as all lighter types. Giants not described as wearing armor 
are not proficient with armor. Giants are proficient with shields if they are proficient 
with any form of armor.---

Giants eat, sleep, and breathe.

\vspace{12pt}
\textbf{Goblinoid Subtype:} Goblinoids are stealthy humanoids who live by hunting 
and raiding and who all speak Goblin.

\vspace{12pt}
\textbf{Good Subtype:} A subtype usually applied only to outsiders native to the 
good-aligned Outer Planes. Most creatures that have this subtype also have good 
alignments; however, if their alignments change, they still retain the subtype. 
Any effect that depends on alignment affects a creature with this subtype as if 
the creature has a good alignment, no matter what its alignment actually is. The 
creature also suffers effects according to its actual alignment. A creature with 
the good subtype overcomes damage reduction as if its natural weapons and any weapons 
it wields were good-aligned (see Damage Reduction, above).

\vspace{12pt}
\textbf{Humanoid Type: }A humanoid usually has two arms, two legs, and one head, 
or a humanlike torso, arms, and a head. Humanoids have few or no supernatural or 
extraordinary abilities, but most can speak and usually have well-developed societies. 
They usually are Small or Medium. Every humanoid creature also has a subtype.

Humanoids with 1 Hit Die exchange the features of their humanoid Hit Die for the 
class features of a PC or NPC class. Humanoids of this sort are presented as 1st-level 
warriors, which means that they have average combat ability and poor saving throws.

Humanoids with more than 1 Hit Die are the only humanoids who make use of the features 
of the humanoid type.

\textit{Features: }A humanoid has the following features (unless otherwise noted 
in a creature's entry).---

8-sided Hit Dice, or by character class.---

Base attack bonus equal to 3/4 total Hit Dice (as cleric).---

Good Reflex saves (usually; a humanoid's good save varies).---

Skill points equal to (2 + Int modifier, minimum 1) per Hit Die, with quadruple 
skill points for the first Hit Die, or by character class.

\textit{Traits: }A humanoid possesses the following traits (unless otherwise noted 
in a creature's entry).---

Proficient with all simple weapons, or by character class.---

Proficient with whatever type of armor (light, medium, or heavy) it is described 
as wearing, or by character class. If a humanoid does not have a class and wears 
armor, it is proficient with that type of armor and all lighter types. Humanoids 
not indicated as wearing armor are not proficient with armor. Humanoids are proficient 
with shields if they are proficient with any form of armor.---

Humanoids breathe, eat, and sleep.

\vspace{12pt}
\textbf{Improved Grab (Ex):} If a creature with this special attack hits with a 
melee weapon (usually a claw or bite attack), it deals normal damage and attempts 
to start a grapple as a free action without provoking an attack of opportunity. 
No initial touch attack is required. Unless otherwise noted, improved grab works 
only against opponents at least one size category smaller than the creature. The 
creature has the option to conduct the grapple normally, or simply use the part 
of its body it used in the improved grab to hold the opponent. If it chooses to 
do the latter, it takes a -20 penalty on grapple checks, but is not considered 
grappled itself; the creature does not lose its Dexterity bonus to AC, still threatens 
an area, and can use its remaining attacks against other opponents. A successful 
hold does not deal any extra damage unless the creature also has the constrict 
special attack. If the creature does not constrict, each successful grapple check 
it makes during successive rounds automatically deals the damage indicated for 
the attack that established the hold. Otherwise, it deals constriction damage as 
well (the amount is given in the creature's descriptive text). When a creature 
gets a hold after an improved grab attack, it pulls the opponent into its space. 
This act does not provoke attacks of opportunity. It can even move (possibly carrying 
away the opponent), provided it can drag the opponent's weight. 

\vspace{12pt}
\textbf{Incorporeal Subtype:} An incorporeal creature has no physical body. It 
can be harmed only by other incorporeal creatures, magic weapons or creatures that 
strike as magic weapons, and spells, spell-like abilities, or supernatural abilities. 
It is immune to all nonmagical attack forms. Even when hit by spells or magic weapons, 
it has a 50\% chance to ignore any damage from a corporeal source (except for positive 
energy, negative energy, force effects such as \textit{magic missile, }or attacks 
made with \textit{ghost touch }weapons). Although it is not a magical attack, holy 
water can affect incorporeal undead, but a hit with holy water has a 50\% chance 
of not affecting an incorporeal creature.

An incorporeal creature has no natural armor bonus but has a deflection bonus equal 
to its Charisma bonus (always at least +1, even if the creature's Charisma score 
does not normally provide a bonus). 

An incorporeal creature can enter or pass through solid objects, but must remain 
adjacent to the object's exterior, and so cannot pass entirely through an object 
whose space is larger than its own. It can sense the presence of creatures or objects 
within a square adjacent to its current location, but enemies have total concealment 
(50\% miss chance) from an incorporeal creature that is inside an object. In order 
to see farther from the object it is in and attack normally, the incorporeal creature 
must emerge. An incorporeal creature inside an object has total cover, but when 
it attacks a creature outside the object it only has cover, so a creature outside 
with a readied action could strike at it as it attacks. An incorporeal creature 
cannot pass through a force effect.

An incorporeal creature's attacks pass through (ignore) natural armor, armor, and 
shields, although deflection bonuses and force effects (such as \textit{mage armor}) 
work normally against it. Incorporeal creatures pass through and operate in water 
as easily as they do in air. Incorporeal creatures cannot fall or take falling 
damage. Incorporeal creatures cannot make trip or grapple attacks, nor can they 
be tripped or grappled. In fact, they cannot take any physical action that would 
move or manipulate an opponent or its equipment, nor are they subject to such actions. 
Incorporeal creatures have no weight and do not set off traps that are triggered 
by weight.

An incorporeal creature moves silently and cannot be heard with Listen checks if 
it doesn't wish to be. It has no Strength score, so its Dexterity modifier applies 
to both its melee attacks and its ranged attacks. Nonvisual senses, such as scent 
and blindsight, are either ineffective or only partly effective with regard to 
incorporeal creatures. Incorporeal creatures have an innate sense of direction 
and can move at full speed even when they cannot see.

\vspace{12pt}
\textbf{Lawful: }A subtype usually applied only to outsiders native to the lawful-aligned 
Outer Planes. Most creatures that have this subtype also have lawful alignments; 
however, if their alignments change, they still retain the subtype. Any effect 
that depends on alignment affects a creature with this subtype as if the creature 
has a lawful alignment, no matter what its alignment actually is. The creature 
also suffers effects according to its actual alignment. A creature with the lawful 
subtype overcomes damage reduction as if its natural weapons and any weapons it 
wields were lawful-aligned (see Damage Reduction, above).

\vspace{12pt}
\textbf{Low-Light Vision (Ex):} A creature with low-light vision can see twice 
as far as a human in starlight, moonlight, torchlight, and similar conditions of 
shadowy illumination. It retains the ability to distinguish color and detail under 
these conditions.

\vspace{12pt}
\textbf{Magical Beast Type:} Magical beasts are similar to animals but can have 
Intelligence scores higher than 2. Magical beasts usually have supernatural or 
extraordinary abilities, but sometimes are merely bizarre in appearance or habits.

\textit{Features: }A magical beast has the following features.---

10-sided Hit Dice.---

Base attack bonus equal to total Hit Dice (as fighter).---

Good Fortitude and Reflex saves.---

Skill points equal to (2 + Int modifier, minimum 1) per Hit Die, with quadruple 
skill points for the first Hit Die.

\textit{Traits: }A magical beast possesses the following traits (unless otherwise 
noted in a creature's entry).---

Darkvision out to 60 feet and low-light vision.---

Proficient with its natural weapons only.---

Proficient with no armor.---

Magical beasts eat, sleep, and breathe.

\vspace{12pt}
\textbf{Manufactured Weapons:} Some monsters employ manufactured weapons when they 
attack. Creatures that use swords, bows, spears, and the like follow the same rules 
as characters, including those for additional attacks from a high base attack bonus 
and two-weapon fighting penalties. This category also includes ``found items,'' 
such as rocks and logs, that a creature wields in combat--- in essence, any weapon 
that is not intrinsic to the creature.

Some creatures combine attacks with natural and manufactured weapons when they 
make a full attack. When they do so, the manufactured weapon attack is considered 
the primary attack unless the creature's description indicates otherwise and any 
natural weapons the creature also uses are considered secondary natural attacks. 
These secondary attacks do not interfere with the primary attack as attacking with 
an off-hand weapon does, but they take the usual -5 penalty (or -2 with the Multiattack 
feat) for such attacks, even if the natural weapon used is normally the creature's 
primary natural weapon.

\vspace{12pt}
\textbf{Monstrous Humanoid Type: }Monstrous humanoids are similar to humanoids, 
but with monstrous or animalistic features. They often have magical abilities as 
well.

\textit{Features: }A monstrous humanoid has the following features.---

8-sided Hit Dice.---

Base attack bonus equal to total Hit Dice (as fighter).---

Good Reflex and Will saves.---

Skill points equal to (2 + Int modifier, minimum 1) per Hit Die, with quadruple 
skill points for the first Hit Die.

\textit{Traits: }A monstrous humanoid possesses the following traits (unless otherwise 
noted in a creature's entry).---

Darkvision out to 60 feet.---

Proficient with all simple weapons and any weapons mentioned in its entry.---

Proficient with whatever type of armor (light, medium, or heavy) it is described 
as wearing, as well as all lighter types. Monstrous humanoids not indicated as 
wearing armor are not proficient with armor. Monstrous humanoids are proficient 
with shields if they are proficient with any form of armor.---

Monstrous humanoids eat, sleep, and breathe.

\vspace{12pt}
\textbf{Movement Modes: }Creatures may have modes of movement other than walking 
and running. These are natural, not magical, unless specifically noted in a monster 
description.

\textit{Burrow: }A creature with a burrow speed can tunnel through dirt, but not 
through rock unless the descriptive text says otherwise. Creatures cannot charge 
or run while burrowing. Most burrowing creatures do not leave behind tunnels other 
creatures can use (either because the material they tunnel through fills in behind 
them or because they do not actually dislocate any material when burrowing); see 
the individual creature descriptions for details.

\textit{Climb: }A creature with a climb speed has a +8 racial bonus on all Climb 
checks. The creature must make a Climb check to climb any wall or slope with a 
DC of more than 0, but it always can choose to take 10 even if rushed or threatened 
while climbing. The creature climbs at the given speed while climbing. If it chooses 
an accelerated climb it moves at double the given climb speed (or its base land 
speed, whichever is lower) and makes a single Climb check at a -5 penalty. Creatures 
cannot run while climbing. A creature retains its Dexterity bonus to Armor Class 
(if any) while climbing, and opponents get no special bonus on their attacks against 
a climbing creature.

\textit{Fly: }A creature with a fly speed can move through the air at the indicated 
speed if carrying no more than a light load. (Note that medium armor does not necessarily 
constitute a medium load.) All fly speeds include a parenthetical note indicating 
maneuverability, as follows:---

Perfect: The creature can perform almost any aerial maneuver it wishes. It moves 
through the air as well as a human moves over smooth ground.---

Good: The creature is very agile in the air (like a housefly or a hummingbird), 
but cannot change direction as readily as those with perfect maneuverability.---

Average: The creature can fly as adroitly as a small bird. ---

Poor: The creature flies as well as a very large bird.---

Clumsy: The creature can barely maneuver at all.

A creature that flies can make dive attacks. A dive attack works just like a charge, 
but the diving creature must move a minimum of 30 feet and descend at least 10 
feet. It can make only claw or talon attacks, but these deal double damage. A creature 
can use the run action while flying, provided it flies in a straight line. 

\textit{Swim: }A creature with a swim speed can move through water at its swim 
speed without making Swim checks. It has a +8 racial bonus on any Swim check to 
perform some special action or avoid a hazard. The creature can always can choose 
to take 10 on a Swim check, even if distracted or endangered. The creature can 
use the run action while swimming, provided it swims in a straight line. 

\vspace{12pt}
\textbf{Native Subtype:} A subtype applied only to outsiders. These creatures have 
mortal ancestors or a strong connection to the Material Plane and can be raised, 
reincarnated, or resurrected just as other living creatures can be. Creatures with 
this subtype are native to the Material Plane (hence the subtype's name). Unlike 
true outsiders, native outsiders need to eat and sleep. 

\vspace{12pt}
\textbf{Natural Weapons:} Natural weapons are weapons that are physically a part 
of a creature. A creature making a melee attack with a natural weapon is considered 
armed and does not provoke attacks of opportunity. Likewise, it threatens any space 
it can reach. Creatures do not receive additional attacks from a high base attack 
bonus when using natural weapons. The number of attacks a creature can make with 
its natural weapons depends on the type of the attack---generally, a creature can 
make one bite attack, one attack per claw or tentacle, one gore attack, one sting 
attack, or one slam attack (although Large creatures with arms or arm-like limbs 
can make a slam attack with each arm). Refer to the individual monster descriptions.

Unless otherwise noted, a natural weapon threatens a critical hit on a natural 
attack roll of 20.

When a creature has more than one natural weapon, one of them (or sometimes a pair 
or set of them) is the primary weapon. All the creature's remaining natural weapons 
are secondary. 

The primary weapon is given in the creature's Attack entry, and the primary weapon 
or weapons is given first in the creature's Full Attack entry. A creature's primary 
natural weapon is its most effective natural attack, usually by virtue of the creature's 
physiology, training, or innate talent with the weapon. An attack with a primary 
natural weapon uses the creature's full attack bonus. Attacks with secondary natural 
weapons are less effective and are made with a -5 penalty on the attack roll, no 
matter how many there are. (Creatures with the Multiattack feat take only a -2 
penalty on secondary attacks.) This penalty applies even when the creature makes 
a single attack with the secondary weapon as part of the attack action or as an 
attack of opportunity.

Natural weapons have types just as other weapons do. The most common are summarized 
below.

\textit{Bite: }The creature attacks with its mouth, dealing piercing, slashing, 
and bludgeoning damage.

\textit{Claw or Talon: }The creature rips with a sharp appendage, dealing piercing 
and slashing damage.

\textit{Gore: }The creature spears the opponent with an antler, horn, or similar 
appendage, dealing piercing damage.

\textit{Slap or Slam: }The creature batters opponents with an appendage, dealing 
bludgeoning damage.

\textit{Sting: }The creature stabs with a stinger, dealing piercing damage. Sting 
attacks usually deal damage from poison in addition to hit point damage.

\textit{Tentacle: }The creature flails at opponents with a powerful tentacle, dealing 
bludgeoning (and sometimes slashing) damage. 

\vspace{12pt}
\textbf{Nonabilities:} Some creatures lack certain ability scores. These creatures 
do not have an ability score of 0---they lack the ability altogether. The modifier 
for a nonability is +0. Other effects of nonabilities are detailed below.

\textit{Strength: }Any creature that can physically manipulate other objects has 
at least 1 point of Strength. A creature with no Strength score can't exert force, 
usually because it has no physical body or because it doesn't move. The creature 
automatically fails Strength checks. If the creature can attack, it applies its 
Dexterity modifier to its base attack bonus instead of a Strength modifier.

\textit{Dexterity: }Any creature that can move has at least 1 point of Dexterity. 
A creature with no Dexterity score can't move. If it can perform actions (such 
as casting spells), it applies its Intelligence modifier to initiative checks instead 
of a Dexterity modifier. The creature automatically fails Reflex saves and Dexterity 
checks.

\textit{Constitution: }Any living creature has at least 1 point of Constitution. 
A creature with no Constitution has no body or no metabolism. It is immune to any 
effect that requires a Fortitude save unless the effect works on objects or is 
harmless. The creature is also immune to ability damage, ability drain, and energy 
drain, and automatically fails Constitution checks. A creature with no Constitution 
cannot tire and thus can run indefinitely without tiring (unless the creature's 
description says it cannot run).

\textit{Intelligence: }Any creature that can think, learn, or remember has at least 
1 point of Intelligence. A creature with no Intelligence score is mindless, an 
automaton operating on simple instincts or programmed instructions. It has immunity 
to mind-affecting effects (charms, compulsions, phantasms, patterns, and morale 
effects) and automatically fails Intelligence checks.

Mindless creatures do not gain feats or skills, although they may have bonus feats 
or racial skill bonuses.

\textit{Wisdom: }Any creature that can perceive its environment in any fashion 
has at least 1 point of Wisdom. Anything with no Wisdom score is an object, not 
a creature. Anything without a Wisdom score also has no Charisma score.

\textit{Charisma: }Any creature capable of telling the difference between itself 
and things that are not itself has at least 1 point of Charisma. Anything with 
no Charisma score is an object, not a creature. Anything without a Charisma score 
also has no Wisdom score.

\vspace{12pt}
\textbf{Ooze Type:} An ooze is an amorphous or mutable creature, usually mindless.

\textit{Features: }An ooze has the following features.---

10-sided Hit Dice.---

Base attack bonus equal to 3/4 total Hit Dice (as cleric).---

No good saving throws.---

Skill points equal to (2 + Int modifier, minimum 1) per Hit Die, with quadruple 
skill points for the first Hit Die, if the ooze has an Intelligence score. However, 
most oozes are mindless and gain no skill points or feats.

\textit{Traits: }An ooze possesses the following traits (unless otherwise noted 
in a creature's entry).---

Mindless: No Intelligence score, and immunity to all mind-affecting effects (charms, 
compulsions, phantasms, patterns, and morale effects).---

Blind (but have the blindsight special quality), with immunity to gaze attacks, 
visual effects, illusions, and other attack forms that rely on sight.---

Immunity to poison, sleep effects, paralysis, polymorph, and stunning.---

Some oozes have the ability to deal acid damage to objects. In such a case, the 
amount of damage is equal to 10 + 1/2 ooze's HD + ooze's Con modifier per full 
round of contact.---

Not subject to critical hits or flanking.---

Proficient with its natural weapons only.---

Proficient with no armor.---

Oozes eat and breathe, but do not sleep.

\vspace{12pt}
\textbf{Outsider Type: }An outsider is at least partially composed of the essence 
(but not necessarily the material) of some plane other than the Material Plane. 
Some creatures start out as some other type and become outsiders when they attain 
a higher (or lower) state of spiritual existence.

\textit{Features: }An outsider has the following features.---

8-sided Hit Dice.---

Base attack bonus equal to total Hit Dice (as fighter).---

Good Fortitude, Reflex, and Will saves.---

Skill points equal to (8 + Int modifier, minimum 1) per Hit Die, with quadruple 
skill points for the first Hit Die.

\textit{Traits: }An outsider possesses the following traits (unless otherwise noted 
in a creature's entry).---

Darkvision out to 60 feet.---

Unlike most other living creatures, an outsider does not have a dual nature---its 
soul and body form one unit. When an outsider is slain, no soul is set loose. Spells 
that restore souls to their bodies, such as \textit{raise dead, reincarnate, }and 
\textit{resurrection, }don't work on an outsider. It takes a different magical 
effect, such as \textit{limited wish, wish, miracle, }or \textit{true resurrection 
}to restore it to life. An outsider with the native subtype can be raised, reincarnated, 
or resurrected just as other living creatures can be.---

Proficient with all simple and martial weapons and any weapons mentioned in its 
entry.---

Proficient with whatever type of armor (light, medium, or heavy) it is described 
as wearing, as well as all lighter types. Outsiders not indicated as wearing armor 
are not proficient with armor. Outsiders are proficient with shields if they are 
proficient with any form of armor.---

Outsiders breathe, but do not need to eat or sleep (although they can do so if 
they wish). Native outsiders breathe, eat, and sleep. 

\vspace{12pt}
\textbf{Paralysis (Ex or Su): }This special attack renders the victim immobile. 
Paralyzed creatures cannot move, speak, or take any physical actions. The creature 
is rooted to the spot, frozen and helpless. Paralysis works on the body, and a 
character can usually resist it with a Fortitude saving throw (the DC is given 
in the creature's description). Unlike \textit{hold person }and similar effects, 
a paralysis effect does not allow a new save each round. A winged creature flying 
in the air at the time that it is paralyzed cannot flap its wings and falls. A 
swimmer can't swim and may drown. 

\vspace{12pt}
\textbf{Plant Type:} This type comprises vegetable creatures. Note that regular 
plants, such as one finds growing in gardens and fields, lack Wisdom and Charisma 
scores (see Nonabilities, above) and are not creatures, but objects, even though 
they are alive. 

\textit{Features: }A plant creature has the following features. ---

8-sided Hit Dice.---

Base attack bonus equal to 3/4 total Hit Dice (as cleric).---

Good Fortitude saves.---

Skill points equal to (2 + Int modifier, minimum 1) per Hit Die, with quadruple 
skill points for the first Hit Die, if the plant creature has an Intelligence score. 
However, some plant creatures are mindless and gain no skill points or feats.

\textit{Traits: }A plant creature possesses the following traits (unless otherwise 
noted in a creature's entry).---

Low-light vision.---

Immunity to all mind-affecting effects (charms, compulsions, phantasms, patterns, 
and morale effects).---

Immunity to poison, sleep effects, paralysis, polymorph, and stunning.---

Not subject to critical hits.---

Proficient with its natural weapons only.---

Proficient with no armor.---

Plants breathe and eat, but do not sleep.

\vspace{12pt}
\textbf{Poison (Ex):} Poison attacks deal initial damage, such as ability damage 
(see page 305) or some other effect, to the opponent on a failed Fortitude save. 
Unless otherwise noted, another saving throw is required 1 minute later (regardless 
of the first save's result) to avoid secondary damage. A creature's descriptive 
text provides the details.

A creature with a poison attack is immune to its own poison and the poison of others 
of its kind.

The Fortitude save DC against a poison attack is equal to 10 + 1/2 poisoning creature's 
racial HD + poisoning creature's Con modifier (the exact DC is given in the creature's 
descriptive text).

A successful save avoids (negates) the damage.

\vspace{12pt}
\textbf{Pounce (Ex):} When a creature with this special attack makes a charge, 
it can follow with a full attack---including rake attacks if the creature also 
has the rake ability.

\vspace{12pt}
\textbf{Powerful Charge (Ex):} When a creature with this special attack makes a 
charge, its attack deals extra damage in addition to the normal benefits and hazards 
of a charge. The amount of damage from the attack is given in the creature's description.

\vspace{12pt}
\textbf{Psionics (Sp):} These are spell-like abilities that a creature generates 
with the power of its mind. Psionic abilities are usually usable at will.

\vspace{12pt}
\textbf{Rake (Ex):} A creature with this special attack gains extra natural attacks 
when it grapples its foe. Normally, a monster can attack with only one of its natural 
weapons while grappling, but a monster with the rake ability usually gains two 
additional claw attacks that it can use only against a grappled foe. Rake attacks 
are not subject to the usual -4 penalty for attacking with a natural weapon in 
a grapple.

A monster with the rake ability must begin its turn grappling to use its rake---it 
can't begin a grapple and rake in the same turn.

\vspace{12pt}
\textbf{Ray (Su or Sp):} This form of special attack works like a ranged attack. 
Hitting with a ray attack requires a successful ranged touch attack roll, ignoring 
armor, natural armor, and shield and using the creature's ranged attack bonus. 
Ray attacks have no range increment. The creature's descriptive text specifies 
the maximum range, effects, and any applicable saving throw.

\vspace{12pt}
\textbf{Regeneration (Ex): }A creature with this ability is difficult to kill. 
Damage dealt to the creature is treated as nonlethal damage. The creature automatically 
heals nonlethal damage at a fixed rate per round, as given in the entry. Certain 
attack forms, typically fire and acid, deal lethal damage to the creature, which 
doesn't go away. The creature's descriptive text describes the details. A regenerating 
creature that has been rendered unconscious through nonlethal damage can be killed 
with a coup de grace. The attack cannot be of a type that automatically converts 
to nonlethal damage.

Attack forms that don't deal hit point damage ignore regeneration. Regeneration 
also does not restore hit points lost from starvation, thirst, or suffocation. 
Regenerating creatures can regrow lost portions of their bodies and can reattach 
severed limbs or body parts; details are in the creature's descriptive text. Severed 
parts that are not reattached wither and die normally.

A creature must have a Constitution score to have the regeneration ability.

\vspace{12pt}
\textbf{Reptilian Subtype: }These creatures are scaly and usually coldblooded. 
The reptilian subtype is only used to describe a set of humanoid races, not all 
animals and monsters that are truly reptiles.

\vspace{12pt}
\textbf{Resistance to Energy (Ex):} A creature with this special quality ignores 
some damage of the indicated type each time it takes damage of that kind (commonly 
acid, cold, fire, or electricity). The entry indicates the amount and type of damage 
ignored.

\vspace{12pt}
\textbf{Scent (Ex):} This special quality allows a creature to detect approaching 
enemies, sniff out hidden foes, and track by sense of smell. Creatures with the 
scent ability can identify familiar odors just as humans do familiar sights.

The creature can detect opponents within 30 feet by sense of smell. If the opponent 
is upwind, the range increases to 60 feet; if downwind, it drops to 15 feet. Strong 
scents, such as smoke or rotting garbage, can be detected at twice the ranges noted 
above. Overpowering scents, such as skunk musk or troglodyte stench, can be detected 
at triple normal range.

When a creature detects a scent, the exact location of the source is not revealed---only 
its presence somewhere within range. The creature can take a move action to note 
the direction of the scent.

Whenever the creature comes within 5 feet of the source, the creature pinpoints 
the source's location.

A creature with the Track feat and the scent ability can follow tracks by smell, 
making a Wisdom (or Survival) check to find or follow a track. The typical DC for 
a fresh trail is 10 (no matter what kind of surface holds the scent). This DC increases 
or decreases depending on how strong the quarry's odor is, the number of creatures, 
and the age of the trail. For each hour that the trail is cold, the DC increases 
by 2. The ability otherwise follows the rules for the Track feat. Creatures tracking 
by scent ignore the effects of surface conditions and poor visibility. 

\vspace{12pt}
\textbf{Shapechanger Subtype:} A shapechanger has the supernatural ability to assume 
one or more alternate forms. Many magical effects allow some kind of shape shifting, 
and not every creature that can change shapes has the shapechanger subtype. 

\textit{Traits: }A shapechanger possesses the following traits (unless otherwise 
noted in a creature's entry).---

Proficient with its natural weapons, with simple weapons, and with any weapons 
mentioned in the creature's description.---

Proficient with any armor mentioned in the creature's description, as well as all 
lighter forms. If no form of armor is mentioned, the shapechanger is not proficient 
with armor. A shapechanger is proficient with shields if it is proficient with 
any type of armor.

\vspace{12pt}
\textbf{Sonic Attacks (Su):} Unless otherwise noted, a sonic attack follows the 
rules for spreads. The range of the spread is measured from the creature using 
the sonic attack. Once a sonic attack has taken effect, deafening the subject or 
stopping its ears does not end the effect. Stopping one's ears ahead of time allows 
opponents to avoid having to make saving throws against mind-affecting sonic attacks, 
but not other kinds of sonic attacks (such as those that deal damage). Stopping 
one's ears is a full-round action and requires wax or other soundproof material 
to stuff into the ears.

\vspace{12pt}
\textbf{Special Abilities:} A special ability is either extraordinary (Ex), spell-like 
(Sp), or supernatural (Su).

\textit{Extraordinary: }Extraordinary abilities are nonmagical, don't become ineffective 
in an \textit{antimagic field, }and are not subject to any effect that disrupts 
magic. Using an extraordinary ability is a free action unless otherwise noted.

\textit{Spell-Like: }Spell-like abilities are magical and work just like spells 
(though they are not spells and so have no verbal, somatic, material, focus, or 
XP components). They go away in an \textit{antimagic field }and are subject to 
spell resistance if the spell the ability resembles or duplicates would be subject 
to spell resistance.

A spell-like ability usually has a limit on how often it can be used. A spell-like 
ability that can be used at will has no use limit. Using a spell-like ability is 
a standard action unless noted otherwise, and doing so while threatened provokes 
attacks of opportunity. It is possible to make a Concentration check to use a spell-like 
ability defensively and avoid provoking an attack of opportunity, just as when 
casting a spell. A spell-like ability can be disrupted just as a spell can be. 
Spell-like abilities cannot be used to counterspell, nor can they be counterspelled.

For creatures with spell-like abilities, a designated caster level defines how 
difficult it is to dispel their spell-like effects and to define any level-dependent 
variables (such as range and duration) the abilities might have. The creature's 
caster level never affects which spell-like abilities the creature has; sometimes 
the given caster level is lower than the level a spellcasting character would need 
to cast the spell of the same name. If no caster level is specified, the caster 
level is equal to the creature's Hit Dice. The saving throw (if any) against a 
spell-like ability is 10 + the level of the spell the ability resembles or duplicates 
+ the creature's Cha modifier.

Some spell-like abilities duplicate spells that work differently when cast by characters 
of different classes\textit{. }A monster's spell-like abilities are presumed to 
be the sorcerer/wizard versions. If the spell in question is not a sorcerer/wizard 
spell, then default to cleric, druid, bard, paladin, and ranger, in that order.

\textit{Supernatural: }Supernatural abilities are magical and go away in an \textit{antimagic 
field }but are not subject to spell resistance. Supernatural abilities cannot be 
dispelled. Using a supernatural ability is a standard action unless noted otherwise. 
Supernatural abilities may have a use limit or be usable at will, just like spell-like 
abilities. However, supernatural abilities do not provoke attacks of opportunity 
and never require Concentration checks. Unless otherwise noted, a supernatural 
ability has an effective caster level equal to the creature's Hit Dice. The saving 
throw (if any) against a supernatural ability is 10 + 1/2 the creature's HD + the 
creature's ability modifier (usually Charisma).

\vspace{12pt}
\textbf{Spell Immunity (Ex): }A creature with spell immunity avoids the effects 
of spells and spell-like abilities that directly affect it. This works exactly 
like spell resistance, except that it cannot be overcome. Sometimes spell immunity 
is conditional or applies to only spells of a certain kind or level. Spells that 
do not allow spell resistance are not affected by spell immunity.

\vspace{12pt}
\textbf{Spell Resistance (Ex): }A creature with spell resistance can avoid the 
effects of spells and spell-like abilities that directly affect it.To determine 
if a spell or spell-like ability works against a creature with spell resistance, 
the caster must make a caster level check (1d20 + caster level). If the result 
equals or exceeds the creature's spell resistance, the spell works normally, although 
the creature is still allowed a saving throw.

\vspace{12pt}
\textbf{Spells:} Sometimes a creature can cast arcane or divine spells just as 
a member of a spellcasting class can (and can activate magic items accordingly). 
Such creatures are subject to the same spellcasting rules that characters are, 
except as follows. 

A spellcasting creature that lacks hands or arms can provide any somatic component 
a spell might require by moving its body. Such a creature also does need material 
components for its spells. The creature can cast the spell by either touching the 
required component (but not if the component is in another creature's possession) 
or having the required component on its person. Sometimes spellcasting creatures 
utilize the Eschew Materials feat to avoid fussing with noncostly components.

A spellcasting creature is not actually a member of a class unless its entry says 
so, and it does not gain any class abilities. A creature with access to cleric 
spells must prepare them in the normal manner and receives domain spells if noted, 
but it does not receive domain granted powers unless it has at least one level 
in the cleric class.

\vspace{12pt}
\textit{\textbf{Summon }}\textbf{(Sp):} A creature with the \textit{summon }ability 
can summon specific other creatures of its kind much as though casting a \textit{summon 
monster }spell, but it usually has only a limited chance of success (as specified 
in the creature's entry). Roll d\%: On a failure, no creature answers the summons. 
Summoned creatures automatically return whence they came after 1 hour. A creature 
that has just been summoned cannot use its own summon ability for 1 hour. Most 
creatures with the ability to summon do not use it lightly, since it leaves them 
beholden to the summoned creature. In general, they use it only when necessary 
to save their own lives. An appropriate spell level is given for each summoning 
ability for purposes of Concentration checks and attempts to dispel the summoned 
creature. No experience points are awarded for summoned monsters.

\vspace{12pt}
\textbf{Swallow Whole (Ex):} If a creature with this special attack begins its 
turn with an opponent held in its mouth (see Improved Grab), it can attempt a new 
grapple check (as though attempting to pin the opponent). If it succeeds, it swallows 
its prey, and the opponent takes bite damage. Unless otherwise noted, the opponent 
can be up to one size category smaller than the swallowing creature. Being swallowed 
has various consequences, depending on the creature doing the swallowing. A swallowed 
creature is considered to be grappled, while the creature that did the swallowing 
is not. A swallowed creature can try to cut its way free with any light slashing 
or piercing weapon (the amount of cutting damage required to get free is noted 
in the creature description), or it can just try to escape the grapple. The Armor 
Class of the interior of a creature that swallows whole is normally 10 + 1/2 its 
natural armor bonus, with no modifiers for size or Dexterity. If the swallowed 
creature escapes the grapple, success puts it back in the attacker's mouth, where 
it may be bitten or swallowed again.

\vspace{12pt}
\textbf{Swarm Subtype:} A swarm is a collection of Fine, Diminutive, or Tiny creatures 
that acts as a single creature. A swarm has the characteristics of its type, except 
as noted here. A swarm has a single pool of Hit Dice and hit points, a single initiative 
modifier, a single speed, and a single Armor Class. A swarm makes saving throws 
as a single creature. A single swarm occupies a square (if it is made up of nonflying 
creatures) or a cube (of flying creatures) 10 feet on a side, but its reach is 
0 feet, like its component creatures. In order to attack, it moves into an opponent's 
space, which provokes an attack of opportunity. It can occupy the same space as 
a creature of any size, since it crawls all over its prey. A swarm can move through 
squares occupied by enemies and vice versa without impediment, although the swarm 
provokes an attack of opportunity if it does so. A swarm can move through cracks 
or holes large enough for its component creatures.

A swarm of Tiny creatures consists of 300 nonflying creatures or 1,000 flying creatures. 
A swarm of Diminutive creatures consists of 1,500 nonflying creatures or 5,000 
flying creatures. A swarm of Fine creatures consists of 10,000 creatures, whether 
they are flying or not. Swarms of nonflying creatures include many more creatures 
than could normally fit in a 10-foot square based on their normal space, because 
creatures in a swarm are packed tightly together and generally crawl over each 
other and their prey when moving or attacking. Larger swarms are represented by 
multiples of single swarms. The area occupied by a large swarm is completely shapeable, 
though the swarm usually remains in contiguous squares.

\textit{Traits: }A swarm has no clear front or back and no discernable anatomy, 
so it is not subject to critical hits or flanking. A swarm made up of Tiny creatures 
takes half damage from slashing and piercing weapons. A swarm composed of Fine 
or Diminutive creatures is immune to all weapon damage. Reducing a swarm to 0 hit 
points or lower causes it to break up, though damage taken until that point does 
not degrade its ability to attack or resist attack. Swarms are never staggered 
or reduced to a dying state by damage. Also, they cannot be tripped, grappled, 
or bull rushed, and they cannot grapple an opponent.

A swarm is immune to any spell or effect that targets a specific number of creatures 
(including single-target spells such as \textit{disintegrate}), with the exception 
of mind-affecting effects (charms, compulsions, phantasms, patterns, and morale 
effects) if the swarm has an Intelligence score and a hive mind. A swarm takes 
half again as much damage (+50\%) from spells or effects that affect an area, such 
as splash weapons and many evocation spells.

Swarms made up of Diminutive or Fine creatures are susceptible to high winds such 
as that created by a \textit{gust of wind }spell. For purposes of determining the 
effects of wind on a swarm, treat the swarm as a creature of the same size as its 
constituent creatures. A swarm rendered unconscious by means of nonlethal damage 
becomes disorganized and dispersed, and does not reform until its hit points exceed 
its nonlethal damage.

\textit{Swarm Attack: }Creatures with the swarm subtype don't make standard melee 
attacks. Instead, they deal automatic damage to any creature whose space they occupy 
at the end of their move, with no attack roll needed. Swarm attacks are not subject 
to a miss chance for concealment or cover. A swarm's statistics block has ``swarm'' 
in the Attack and Full Attack entries, with no attack bonus given. The amount of 
damage a swarm deals is based on its Hit Dice, as shown below.

\begin{longtable}{llll}
\hline
% ROW 1
\multicolumn{1}{|p{0.818in}|}{\begin{minipage}[t]{0.818in}\raggedright
\textbf{Swarm HD}\end{minipage}} & \multicolumn{1}{p{1.408in}|}{\begin{minipage}[t]{1.408in}\centering
\textbf{Swarm Base Damage}\end{minipage}}\\
\hline
% ROW 2
\multicolumn{1}{p{0.069in}|}{\begin{minipage}[t]{0.069in}\centering
1-5\end{minipage}} & \multicolumn{1}{p{0.069in}|}{\begin{minipage}[t]{0.069in}\centering
1d6\end{minipage}}\\
\hline
% ROW 3
\multicolumn{1}{|p{0.818in}|}{\begin{minipage}[t]{0.818in}\centering
6-10\end{minipage}} & \multicolumn{1}{p{1.408in}|}{\begin{minipage}[t]{1.408in}\centering
2d6\end{minipage}}\\
\hline
% ROW 4
\multicolumn{1}{p{0.069in}|}{\begin{minipage}[t]{0.069in}\centering
11-15\end{minipage}} & \multicolumn{1}{p{0.069in}|}{\begin{minipage}[t]{0.069in}\centering
3d6\end{minipage}}\\
\hline
% ROW 5
\multicolumn{1}{|p{0.818in}|}{\begin{minipage}[t]{0.818in}\centering
16-20\end{minipage}} & \multicolumn{3}{p{1.547in}|}{\begin{minipage}[t]{1.547in}\centering
4d6\end{minipage}}\\
\hline
% ROW 6
\multicolumn{1}{p{0.069in}|}{\begin{minipage}[t]{0.069in}\centering
21 or more\end{minipage}} & \multicolumn{1}{p{0.069in}|}{\begin{minipage}[t]{0.069in}\centering
5d6\end{minipage}}\\
\hline
\end{longtable}

A swarm's attacks are nonmagical, unless the swarm's description states otherwise. 
Damage reduction sufficient to reduce a swarm attack's damage to 0, being incorporeal, 
and other special abilities usually give a creature immunity (or at least resistance) 
to damage from a swarm. Some swarms also have acid, poison, blood drain, or other 
special attacks in addition to normal damage.

Swarms do not threaten creatures in their square, and do not make attacks of opportunity 
with their swarm attack. However, they distract foes whose squares they occupy, 
as described below.

\textit{Distraction (Ex): }Any living creature vulnerable to a swarm's damage that 
begins its turn with a swarm in its square is nauseated for 1 round; a Fortitude 
save (DC 10 + 1/2 swarm's HD + swarm's Con modifier; the exact DC is given in a 
swarm's description) negates the effect. Spellcasting or concentrating on spells 
within the area of a swarm requires a Concentration check (DC 20 + spell level). 
Using skills that involve patience and concentration requires a DC 20 Concentration 
check.

\vspace{12pt}
\textbf{Telepathy (Su):} A creature with this ability can communicate telepathically 
with any other creature within a certain range (specified in the creature's entry, 
usually 100 feet) that has a language. It is possible to address multiple creatures 
at once telepathically, although maintaining a telepathic conversation with more 
than one creature at a time is just as difficult as simultaneously speaking and 
listening to multiple people at the same time.

Some creatures have a limited form of telepathy, while others have a more powerful 
form of the ability.

\vspace{12pt}
\textbf{Trample (Ex):} As a full-round action, a creature with this special attack 
can move up to twice its speed and literally run over any opponents at least one 
size category smaller than itself. The creature merely has to move over the opponents 
in its path; any creature whose space is completely covered by the trampling creature's 
space is subject to the trample attack. If a target's space is larger than 5 feet, 
it is only considered trampled if the trampling creature moves over all the squares 
it occupies. If the trampling creature moves over only some of a target's space, 
the target can make an attack of opportunity against the trampling creature at 
a -4 penalty. A trampling creature that accidentally ends its movement in an illegal 
space returns to the last legal position it occupied, or the closest legal position, 
if there's a legal position that's closer.

A trample attack deals bludgeoning damage (the creature's slam damage + 1-1/2 times 
its Str modifier). The creature's descriptive text gives the exact amount.

Trampled opponents can attempt attacks of opportunity, but these take a -4 penalty. 
If they do not make attacks of opportunity, trampled opponents can attempt Reflex 
saves to take half damage.

The save DC against a creature's trample attack is 10 + 1/2 creature's HD + creature's 
Str modifier (the exact DC is given in the creature's descriptive text). A trampling 
creature can only deal trampling damage to each target once per round, no matter 
how many times its movement takes it over a target creature.

\vspace{12pt}
\textbf{Tremorsense (Ex): }A creature with tremorsense is sensitive to vibrations 
in the ground and can automatically pinpoint the location of anything that is in 
contact with the ground. Aquatic creatures with tremorsense can also sense the 
location of creatures moving through water. The ability's range is specified in 
the creature's descriptive text.

\vspace{12pt}
\textbf{Treasure:} This entry in a monster description describes how much wealth 
a creature owns. In most cases, a creature keeps valuables in its home or lair 
and has no treasure with it when it travels. Intelligent creatures that own useful, 
portable treasure (such as magic items) tend to carry and use these, leaving bulky 
items at home. Treasure can include coins, goods, and items. Creatures can have 
varying amounts of each, as follows.

\textit{Standard: }Refer to the treasure tables\textit{ }and roll d\% once for 
each type of treasure (Coins, Goods, Items) on the Level section of the table that 
corresponds to the creature's Challenge Rating (for groups of creatures, use the 
Encounter Level for the encounter instead). Some creatures have double, triple, 
or even quadruple standard treasure; in these cases, roll for each type of treasure 
two, three, or four times.

\textit{None: }The creature collects no treasure of its own.

\textit{Nonstandard: }Some creatures have quirks or habits that affect the types 
of treasure they collect. These creatures use the same

treasure tables, but with special adjustments.

\textit{Fractional Coins: }Roll on the Coins column in the section corresponding 
to the creature's Challenge Rating, but divide the result as indicated.

\textit{\% Goods or Items: }The creature has goods or items only some of the time. 
Before checking for goods or items, roll d\% against the given percentage. On a 
success, make a normal roll on the appropriate Goods or Items column (which may 
still result in no goods or items).

\textit{Double Goods or Items: }Roll twice on the appropriate Goods or Items column.

\textit{Parenthetical Notes: }Some entries for goods or items include notes that 
limit the types of treasure a creature collects.

When a note includes the word ``no,'' it means the creature does not collect or 
cannot keep that thing. If a random roll generates such a result, treat the result 
as ``none'' instead. 

When a note includes the word ``only,'' the creature goes out of its way to collect 
treasure of the indicated type. Treat all results from that column as the indicated 
type of treasure.

It's sometimes necessary to reroll until the right sort of item appears. 

\vspace{12pt}
\textbf{Turn Resistance (Ex): }A creature with this special quality (usually an 
undead) is less easily affected by clerics or paladins. When resolving a turn, 
rebuke, command, or bolster attempt, add the indicated number to the creature's 
Hit Dice total.

\vspace{12pt}
\textbf{Undead Type:} Undead are once-living creatures animated by spiritual or 
supernatural forces.

\textit{Features: }An undead creature has the following features.---

12-sided Hit Dice.---

Base attack bonus equal to 1/2 total Hit Dice (as wizard).---

Good Will saves.---

Skill points equal to (4 + Int modifier, minimum 1) per Hit Die, with quadruple 
skill points for the first Hit Die, if the undead creature has an Intelligence 
score. However, many undead are mindless and gain no skill points or feats.

\textit{Traits: }An undead creature possesses the following traits (unless otherwise 
noted in a creature's entry).---

No Constitution score.---

Darkvision out to 60 feet.---

Immunity to all mind-affecting effects (charms, compulsions, phantasms, patterns, 
and morale effects).---

Immunity to poison, sleep effects, paralysis, stunning, disease, and death effects.---

Not subject to critical hits, nonlethal damage, ability drain, or energy drain. 
Immune to damage to its physical ability scores (Strength, Dexterity, and Constitution), 
as well as to fatigue and exhaustion effects.---

Cannot heal damage on its own if it has no Intelligence score, although it can 
be healed. Negative energy (such as an \textit{inflict }spell) can heal undead 
creatures. The fast healing special quality works regardless of the creature's 
Intelligence score.---

Immunity to any effect that requires a Fortitude save (unless the effect also works 
on objects or is harmless).---

Uses its Charisma modifier for Concentration checks.---

Not at risk of death from massive damage, but when reduced to 0 hit points or less, 
it is immediately destroyed.---

Not affected by \textit{raise dead }and \textit{reincarnate }spells or abilities. 
\textit{Resurrection }and \textit{true resurrection }can affect undead creatures. 
These spells turn undead creatures back into the living creatures they were before 
becoming undead.---

Proficient with its natural weapons, all simple weapons, and any weapons mentioned 
in its entry.---

Proficient with whatever type of armor (light, medium, or heavy) it is described 
as wearing, as well as all lighter types. Undead not indicated as wearing armor 
are not proficient with armor. Undead are proficient with shields if they are proficient 
with any form of armor.---

Undead do not breathe, eat, or sleep.

\vspace{12pt}
\textbf{Vermin Type:} This type includes insects, arachnids, other arthropods, 
worms, and similar invertebrates.

\textit{Features: }Vermin have the following features.---

8-sided Hit Dice.---

Base attack bonus equal to 3/4 total Hit Dice (as cleric).---

Good Fortitude saves.---

Skill points equal to (2 + Int modifier, minimum 1) per Hit Die, with quadruple 
skill points for the first Hit Die, if the vermin has an Intelligence score. However, 
most vermin are mindless and gain no skill points or feats.

\textit{Traits: }Vermin possess the following traits (unless otherwise noted in 
a creature's entry).---

Mindless: No Intelligence score, and immunity to all mind-affecting effects (charms, 
compulsions, phantasms, patterns, and morale effects).---

Darkvision out to 60 feet.---

Proficient with their natural weapons only.---

Proficient with no armor.---

Vermin breathe, eat, and sleep.

\vspace{12pt}
\textbf{Vulnerability to Energy:} Some creatures have vulnerability to a certain 
kind of energy effect (typically either cold or fire). Such a creature takes half 
again as much (+50\%) damage as normal from the effect, regardless of whether a 
saving throw is allowed, or if the save is a success or failure.

\vspace{12pt}
\textbf{Water Subtype:} This subtype usually is used for elementals and outsiders 
with a connection to the Elemental Plane of Water. Creatures with the water subtype 
always have swim speeds and can move in water without making Swim checks. A water 
creature can breathe underwater and usually can breathe air as well. 

\end{document}
