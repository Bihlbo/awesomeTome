\spellentry{Dispel Magic}

Abjuration

\textbf{Level:} Brd 3, Clr 3, Drd 4, Magic 3, Pal 3, Sor/Wiz 3

\textbf{Components:} V, S

\textbf{Casting Time:} 1 standard action

\textbf{Range:} Medium (100 ft. + 10 ft./level)

\textbf{Target or Area:} One spellcaster, creature, or object; or 20-ft.-radius 
burst

\textbf{Duration:} Instantaneous

\textbf{Saving Throw:} None

\textbf{Spell Resistance:} No

You can use \textit{dispel magic} to end ongoing spells that have been cast on 
a creature or object, to temporarily suppress the magical abilities of a magic 
item, to end ongoing spells (or at least their effects) within an area, or to counter 
another spellcaster's spell. A dispelled spell ends as if its duration had expired. 
Some spells, as detailed in their descriptions, can't be defeated by \textit{dispel 
magic}. \textit{Dispel magic} can dispel (but not counter) spell-like effects just 
as it does spells.

\textit{Note:} The effect of a spell with an instantaneous duration can't be dispelled, 
because the magical effect is already over before the \textit{dispel magic} can 
take effect. 

You choose to use \textit{dispel magic} in one of three ways: a targeted dispel, 
an area dispel, or a counterspell:

\textit{Targeted Dispel:} One object, creature, or spell is the target of the \textit{dispel 
magic} spell. You make a dispel check (1d20 + your caster level, maximum +10) against 
the spell or against each ongoing spell currently in effect on the object or creature. 
The DC for this dispel check is 11 + the spell's caster level. If you succeed on 
a particular check, that spell is dispelled; if you fail, that spell remains in 
effect.

If you target an object or creature that is the effect of an ongoing spell (such 
as a monster summoned by \textit{monster summoning}), you make a dispel check to 
end the spell that conjured the object or creature.

If the object that you target is a magic item, you make a dispel check against 
the item's caster level. If you succeed, all the item's magical properties are 
suppressed for 1d4 rounds, after which the item recovers on its own. A suppressed 
item becomes nonmagical for the duration of the effect. An interdimensional interface 
(such as a \textit{bag of holding}) is temporarily closed. A magic item's physical 
properties are unchanged: A suppressed magic sword is still a sword (a masterwork 
sword, in fact). Artifacts and deities are unaffected by mortal magic such as this.

You automatically succeed on your dispel check against any spell that you cast 
yourself.

\textit{Area Dispel:} When \textit{dispel magic} is used in this way, the spell 
affects everything within a 20-foot radius.

For each creature within the area that is the subject of one or more spells, you 
make a dispel check against the spell with the highest caster level. If that check 
fails, you make dispel checks against progressively weaker spells until you dispel 
one spell (which discharges the \textit{dispel magic} spell so far as that target 
is concerned) or until you fail all your checks. The creature's magic items are 
not affected.

For each object within the area that is the target of one or more spells, you make 
dispel checks as with creatures. Magic items are not affected by an area dispel.

For each ongoing area or effect spell whose point of origin is within the area 
of the \textit{dispel magic} spell, you can make a dispel check to dispel the spell.

For each ongoing spell whose area overlaps that of the \textit{dispel magic} spell, 
you can make a dispel check to end the effect, but only within the overlapping 
area.

If an object or creature that is the effect of an ongoing spell (such as a monster 
summoned by \textit{monster summoning}) is in the area, you can make a dispel check 
to end the spell that conjured that object or creature (returning it whence it 
came) in addition to attempting to dispel spells targeting the creature or object.

You may choose to automatically succeed on dispel checks against any spell that 
you have cast.

\textit{Counterspell:} When \textit{dispel magic} is used in this way, the spell 
targets a spellcaster and is cast as a counterspell. Unlike a true counterspell, 
however, \textit{dispel magic} may not work; you must make a dispel check to counter 
the other spellcaster's spell.

