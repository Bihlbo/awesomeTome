\spellentry{Fire Trap}

Abjuration [Fire]

\textbf{Level:} Drd 2, Sor/Wiz 4

\textbf{Components:} V, S, M

\textbf{Casting Time:} 10 minutes

\textbf{Range:} Touch

\textbf{Target:} Object touched

\textbf{Duration:} Permanent until discharged (D)

\textbf{Saving Throw:} Reflex half; see text

\textbf{Spell Resistance:} Yes

\textit{Fire trap} creates a fiery explosion when an intruder opens the item that 
the trap protects. A \textit{fire trap} can ward any object that can be opened 
and closed.

When casting \textit{fire trap}, you select a point on the object as the spell's 
center. When someone other than you opens the object, a fiery explosion fills the 
area within a 5-foot radius around the spell's center. The flames deal 1d4 points 
of fire damage +1 point per caster level (maximum +20). The item protected by the 
trap is not harmed by this explosion.

A \textit{fire trapped} item cannot have a second closure or warding spell placed 
on it.

A \textit{knock} spell does not bypass a \textit{fire trap}.  An unsuccessful \textit{dispel 
magic} spell does not detonate the spell.

Underwater, this ward deals half damage and creates a large cloud of steam.

You can use the \textit{fire trapped} object without discharging it, as can any 
individual to whom the object was specifically attuned when cast. Attuning a \textit{fire 
trapped} object to an individual usually involves setting a password that you can 
share with friends.

\textit{Note:} Magic traps such as \textit{fire trap} are hard to detect and disable. 
A rogue (only) can use the Search skill to find a \textit{fire trap} and Disable 
Device to thwart it. The DC in each case is 25 + spell level (DC 27 for a druid's 
\textit{fire trap} or DC 29 for the arcane version).

\textit{Material Component:} A half-pound of gold dust (cost 25 gp) sprinkled on 
the warded object.

