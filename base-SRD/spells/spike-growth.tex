\spellentry{Spike Growth}

Transmutation

\textbf{Level:} Drd 3, Rgr 2

\textbf{Components:} V, S, DF

\textbf{Casting Time:} 1 standard action

\textbf{Range:} Medium (100 ft. + 10 ft./level)

\textbf{Area:} One 20-ft. square/level

\textbf{Duration:} 1 hour/level (D)

\textbf{Saving Throw:} Reflex partial

\textbf{Spell Resistance:} Yes

Any ground-covering vegetation in the spell's area becomes very hard and sharply 
pointed without changing its appearance.

In areas of bare earth, roots and rootlets act in the same way. Typically, \textit{spike 
growth} can be cast in any outdoor setting except open water, ice, heavy snow, 
sandy desert, or bare stone. Any creature moving on foot into or through the spell's 
area takes 1d4 points of piercing damage for each 5 feet of movement through the 
spiked area.

Any creature that takes damage from this spell must also succeed on a Reflex save 
or suffer injuries to its feet and legs that slow its land speed by one-half. This 
speed penalty lasts for 24 hours or until the injured creature receives a \textit{cure 
}spell (which also restores lost hit points). Another character can remove the 
penalty by taking 10 minutes to dress the injuries and succeeding on a Heal check 
against the spell's save DC.

\textit{Spike growth} can't be disabled with the Disable Device skill.

\textit{Note:} Magic traps such as \textit{spike growth} are hard to detect. A 
rogue (only) can use the Search skill to find a \textit{spike growth}. The DC is 
25 + spell level, or DC 28 for \textit{spike growth} (or DC 27 for \textit{spike 
growth} cast by a ranger).

