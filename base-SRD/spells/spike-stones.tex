\spellentry{Spike Stones}

Transmutation [Earth]

\textbf{Level:} Drd 4, Earth 4

\textbf{Components:} V, S, DF

\textbf{Casting Time:} 1 standard action

\textbf{Range:} Medium (100 ft. + 10 ft./level)

\textbf{Area:} One 20-ft. square/level

\textbf{Duration:} 1 hour/level (D)

\textbf{Saving Throw:} Reflex partial

\textbf{Spell Resistance:} Yes

Rocky ground, stone floors, and similar surfaces shape themselves into long, sharp 
points that blend into the background.

\textit{Spike stones} impede progress through an area and deal damage. Any creature 
moving on foot into or through the spell's area moves at half speed.

In addition, each creature moving through the area takes 1d8 points of piercing 
damage for each 5 feet of movement through the spiked area.

Any creature that takes damage from this spell must also succeed on a Reflex save 
to avoid injuries to its feet and legs. A failed save causes the creature's speed 
to be reduced to half normal for 24 hours or until the injured creature receives 
a \textit{cure} spell (which also restores lost hit points). Another character 
can remove the penalty by taking 10 minutes to dress the injuries and succeeding 
on a Heal check against the spell's save DC.

\textit{Spike stones} is a magic trap that can't be disabled with the Disable Device 
skill.

\textit{Note:} Magic traps such as \textit{spike stones} are hard to detect. A 
rogue (only) can use the Search skill to find \textit{spike stones}. The DC is 
25 + spell level, or DC 29 for \textit{spike stones}.

