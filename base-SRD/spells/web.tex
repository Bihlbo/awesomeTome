\spellentry{Web}

Conjuration (Creation)

\textbf{Level:} Sor/Wiz 2

\textbf{Components:} V, S, M

\textbf{Casting Time:} 1 standard action

\textbf{Range:} Medium (100 ft. + 10 ft./level)

\textbf{Effect:} Webs in a 20-ft.-radius spread

\textbf{Duration:} 10 min./level (D)

\textbf{Saving Throw:} Reflex negates; see text

\textbf{Spell Resistance:} No

\textit{Web} creates a many-layered mass of strong, sticky strands. These strands 
trap those caught in them. The strands are similar to spider webs but far larger 
and tougher. These masses must be anchored to two or more solid and diametrically 
opposed points or else the web collapses upon itself and disappears. Creatures 
caught within a \textit{web} become entangled among the gluey fibers. Attacking 
a creature in a web won't cause you to become entangled.

Anyone in the effect's area when the spell is cast must make a Reflex save. If 
this save succeeds, the creature is entangled, but not prevented from moving, though 
moving is more difficult than normal for being entangled (see below). If the save 
fails, the creature is entangled and can't move from its space, but can break loose 
by spending 1 round and making a DC 20 Strength check or a DC 25 Escape Artist 
check. Once loose (either by making the initial Reflex save or a later Strength 
check or Escape Artist check), a creature remains entangled, but may move through 
the \textit{web} very slowly. Each round devoted to moving allows the creature 
to make a new Strength check or Escape Artist check. The creature moves 5 feet 
for each full 5 points by which the check result exceeds 10.

If you have at least 5 feet of web between you and an opponent, it provides cover. 
If you have at least 20 feet of web between you, it provides total cover.

The strands of a \textit{web} spell are flammable. A magic \textit{flaming sword 
}can slash them away as easily as a hand brushes away cobwebs. Any fire can set 
the webs alight and burn away 5 square feet in 1 round. All creatures within flaming 
webs take 2d4 points of fire damage from the flames.

\textit{Web} can be made permanent with a \textit{permanency} spell. A permanent 
\textit{web} that is damaged (but not destroyed) regrows in 10 minutes.

\textit{Material Component:} A bit of spider web.

