\classentry{Cleric}
\modebab
\goodfor
\poorref
\goodwil
\quot{``Fear my righteous shining holy beacon of... righteousness?''}

\begin{classpreamble}
\desc{\class s are the holy (or unholy) warriors, standing fast against the darkness (or light). They are also made of cheese, and thus a prime target for minmaxxers.}
\playingaclass{The \class ~class can fit many different playstyles, but all \class s should have a high Wisdom score.}
\alignment{A \class 's alignment must be within one step of his deity's (that is, it may be one step away on either the lawful-chaotic axis or the good-evil axis, but not both). A \class ~may not be neutral unless his deity's alignment is also neutral.}
\races{Any.}
\startinggold{5d4x10	gp (125 Gold)}
%\startingage{ <-starting age, often written as a class reference like "As Rogue."-> }
\hitdie{d8}
\classskills{Concentration (Con), Craft (Int), Diplomacy (Cha), Heal (Wis), Knowledge (arcana) (Int), Knowledge (history) (Int), Knowledge (religion) (Int), Knowledge (the planes) (Int), Profession (Wis), and Spellcraft (Int).}
\ability{Domains and Class Skills:}{A \class ~who chooses the Animal or Plant domain adds Knowledge (nature) (Int) to the \class ~class skills listed above. A \class ~who chooses the Knowledge domain adds all Knowledge (Int) skills to the list. A \class ~who chooses the Travel domain adds Survival (Wis) to the list. A \class ~who chooses the Trickery domain adds Bluff (Cha), Disguise (Cha), and Hide (Dex) to the list. See Deity, Domains, and Domain Spells, below, for more information.}
\skillpoints{2}
\end{classpreamble}

\begin{fullcastingclasstable}
\levelone{Turn or Rebuke Undead  &3 &1 &- &- &- &- &- &- &- &- }
\leveltwo{&4 &2 &- &- &- &- &- &- &- &- }
\levelthree{&4 &2 &1 &- &- &- &- &- &- &- }
\levelfour{&5 &3 &2 &- &- &- &- &- &- &- }
\levelfive{&5 &3 &2 &1 &- &- &- &- &- &- }
\levelsix{&5 &3 &3 &2 &- &- &- &- &- &- }
\levelseven{&6 &4 &3 &2 &1 &- &- &- &- &- }
\leveleight{&6 &4 &3 &3 &2 &- &- &- &- &- }
\levelnine{&6 &4 &4 &3 &2 &1 &- &- &- &- }
\levelten{&6 &4 &4 &3 &3 &2 &- &- &- &- }
\leveleleven{&6 &5 &4 &4 &3 &2 &1 &- &- &- }
\leveltwelve{&6 &5 &4 &4 &3 &3 &2 &- &- &- }
\levelthirteen{&6 &5 &5 &4 &4 &3 &2 &1 &- &- }
\levelfourteen{&6 &5 &5 &4 &4 &3 &3 &2 &- &- }
\levelfifteen{&6 &5 &5 &5 &4 &4 &3 &2 &1 &- }
\levelsixteen{&6 &5 &5 &5 &4 &4 &3 &3 &2 &- }
\levelseventeen{&6 &5 &5 &5 &5 &4 &4 &3 &2 &1 }
\leveleighteen{&6 &5 &5 &5 &5 &4 &4 &3 &3 &2 }
\levelnineteen{&6 &5 &5 &5 &5 &5 &4 &4 &3 &3 }
\leveltwenty{&6 &5 &5 &5 &5 &5 &4 &4 &4 &4 }
\end{fullcastingclasstable}

%Theres no way the table would fit with the '+1's in the spells per day section, so I took it out, the text is pretty clear that you get an extra spell slot for your domain.  Feel free to break it into two tables if you think its best.

\startclassfeatures

\proficiencies{all types of armor (light, medium, and heavy), and with shields (except tower shields).

\smallskip\noindent A \class ~who chooses the War domain receives the Weapon Focus feat related to his deity's weapon as a bonus feat. He also receives the appropriate Martial Weapon Proficiency feat as a bonus feat, if the weapon falls into that category.}

\classfeature{Aura (Ex):}{A \class ~of a chaotic, evil, good, or lawful deity has a particularly powerful aura corresponding to the deity's alignment (see the detect evil spell for details). \class s who don't worship a specific deity but choose the Chaotic, Evil, Good, or Lawful domain have a similarly powerful aura of the corresponding alignment.}

\classfeature{Spells:}{A \class ~casts divine spells, which are drawn from the \class ~spell list. However, his alignment may restrict him from casting certain spells opposed to his moral or ethical beliefs; see Chaotic, Evil, Good, and Lawful Spells, below. A \class ~must choose and prepare his spells in advance (see below). To prepare or cast a spell, a \class ~must have a Wisdom score equal to at least 10 + the spell level. The Difficulty Class for a saving throw against a \class 's spell is 10 + the spell level + the \class 's Wisdom modifier. Like other spellcasters, a \class ~can cast only a certain number of spells of each spell level per day. His base daily spell allotment is given on Table: The \class. In addition, he receives bonus spells per day if he has a high Wisdom score. 

\smallskip\noindent A \class ~also gets one domain spell of each spell level he can cast, starting at 1st level. When a \class ~prepares a spell in a domain spell slot, it must come from one of his two domains (see Deities, Domains, and Domain Spells, below). \class smeditate or pray for their spells. Each \class ~must choose a time at which he must spend 1 hour each day in quiet contemplation or supplication to regain his daily allotment of spells. 

\smallskip\noindent Time spent resting has no effect on whether a \class ~can prepare spells. A \class ~may prepare and cast any spell on the \class ~spell list, provided that he can cast spells of that level, but he must choose which spells to prepare during his daily meditation.}

\classfeature{Deity, Domains, and Domain Spells:}{A \class 's deity influences his alignment, what magic he can perform, his values, and how others see him. A \class ~chooses two domains from among those belonging to his deity. A \class ~can select an alignment domain (Chaos, Evil, Good, or Law) only if his alignment matches that domain. If a \class ~is not devoted to a particular deity, he still selects two domains to represent his spiritual inclinations and abilities. The restriction on alignment domains still applies. Each domain gives the \class ~access to a domain spell at each spell level he can cast, from 1st on up, as well as a granted power. The \class ~gets the granted powers of both the domains selected. With access to two domain spells at a given spell level, a \class ~prepares one or the other each day in his domain spell slot. If a domain spell is not on the \class ~spell list, a \class ~can prepare it only in his domain spell slot.}

\classfeature{Spontaneous Casting:}{A good \class ~(or a neutral \class ~of a good deity) can channel stored spell energy into healing spells that the \class ~did not prepare ahead of time. The \class ~can ``lose'' any prepared spell that is not a domain spell in order to cast any cure spell of the same spell level or lower (a cure spell is any spell with ``cure'' in its name). An evil \class ~(or a neutral \class ~of an evil deity), can't convert prepared spells to cure spells but can convert them to inflict spells (an inflict spell is one with ``inflict'' in its name). A \class ~who is neither good nor evil and whose deity is neither good nor evil can convert spells to either cure spells or inflict spells (player's choice). Once the player makes this choice, it cannot be reversed. This choice also determines whether the \class ~turns or commands undead (see below).}

\classfeature{Chaotic, Evil, Good, and Lawful Spells:}{A \class ~can't cast spells of an alignment opposed to his own or his deity's (if he has one). Spells associated with particular alignments are indicated by the chaos, evil, good, and law descriptors in their spell descriptions.}

\classfeature{Turn or Rebuke Undead (Su):}{Any \class, regardless of alignment, has the power to affect undead creatures by channeling the power of his faith through his holy (or unholy) symbol (see Turn or Rebuke Undead). A good \class ~(or a neutral \class ~who worships a good deity) can turn or destroy undead creatures. An evil \class ~(or a neutral \class ~who worships an evil deity) instead rebukes or commands such creatures. A neutral \class ~of a neutral deity must choose whether his turning ability functions as that of a good \class ~or an evil \class. Once this choice is made, it cannot be reversed. This decision also determines whether the \class ~can cast spontaneous cure or inflict spells (see above). A \class ~may attempt to turn undead a number of times per day equal to 3 + his Charisma modifier. A \class ~with 5 or more ranks in Knowledge (religion) gets a +2 bonus on turning checks against undead.}

\classfeature{Bonus Languages:}{A \class s bonus language options include Celestial, Abyssal, and Infernal (the languages of good, chaotic evil, and lawful evil outsiders, respectively). These choices are in addition to the bonus languages available to the character because of his race.}

\classfeature{Ex-\class ~s:}{A \class ~who grossly violates the code of conduct required by his god loses all spells and class features, except for armor and shield proficiencies and proficiency with simple weapons. He cannot thereafter gain levels as a \class ~of that god until he atones (see the atonement spell description).}
