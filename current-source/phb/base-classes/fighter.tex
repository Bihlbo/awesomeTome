\classentry{Fighter}
\goodbab
\goodfor
\goodref
\goodwil
\quot{``I've seen this kind of fire-breathing chicken-demon before. We're going to need more rope. Also a bigger cart.''}

\begin{classpreamble}
\desc{The Fighter is a versatile combatant who is able to actively disrupt the activities of his enemies. Fighters represent plucky heroes and grizzled veterans, but they always appear to surmount impossible odds. Which means in retrospect that the odds weren't all that impossible. At least, not for someone with a Fighter's talents.}
\playingaclass{Fighters are often handed to beginning players in order to help them learn the ropes. This is a cruel practice that dates back to when the Fighter was explicitly a weak class that players were forced to play to the (quit proximate) death if for whatever reason they didn't roll well enough on their stats to play a real character. The Fighter described here is not the hazing ritual of old, but it is a more complicated character than many others, being the non-magical equivalent to the Wizard. Beginning characters should probably be given a Barbarian, Conduit, or Rogue character to introduce them to the game mechanics of D\&D.
\vspace{8pt}
\newline
A Fighter has an answer for virtually any circumstance and a great deal of adaptability and flexibility, and benefits greatly from being played by a player who actually knows how far a Roper's strands or a Beholder's rays reach. The Fighter character is archetypically a character who uses her opponent's limitations against them, and it really slows down play if the player needs to have those limitations explained during combat. As such, a full classed Fighter is recommended for experienced players of the game.
\vspace{8pt}
\newline
That being said, a level or two of Fighter can give some breadth and resilience to almost any martial build, and makes a good multiclassing dip even (sometimes especially) for inexperienced players.}
\alignment{Every alignment has its share of Fighters, however more Fighters are of Lawful alignment than of Chaotic Alignment.}
\races{Every humanoid race has warriors, but actual Fighters are rarer in societies that don't value logistics and planning. So while there are many Fighters among the Hobgoblins, Dwarves, and Fire Giants, a Fighter is rarely seen among the ranks of the Orcs, Gnomes, or Ogres.}
\startinggold{6d6x10 gp (210 gold)}
%\startingage{ <-starting age, often written as a class reference like "As Rogue."-> }
\hitdie{d10}
\classskills{Balance (Dex), Bluff (Cha), Climb (Str), Craft (Int), Diplomacy (Cha), Escape Artist (Dex), Handle Animal (Cha), Intimidate (Cha), Jump (Str), Knowledge (all skills individually) (Int), Listen (Wis), Move Silently (Dex), Profession (Wis), Ride (Dex), Sense Motive (Wis), Spot (Wis), Survival (Wis), Swim (Str), Tumble (Dex), and Use Rope (Dex).}
\skillpoints{6}
\end{classpreamble}

\begin{classtable}{}
\levelone{Weapons Training, Combat Focus}
\leveltwo{Bonus Feat}
\levelthree{Problem Solver, Pack Mule}
\levelfour{Bonus Feat}
\levelfive{Logistics Mastery, Active Assualt}
\levelsix{Bonus Feat}
\levelseven{Forge Lore, Improved Delay}
\leveleight{Bonus Feat}
\levelnine{Foil Action}
\levelten{Bonus Feat}
\leveleleven{Lunging Attacks}
\leveltwelve{Bonus Feat}
\levelthirteen{Array of Stunts}
\levelfourteen{Bonus Feat}
\levelfifteen{Greater Combat Focus}
\levelsixteen{Bonus Feat}
\levelseventeen{Improved Foil Action}
\leveleighteen{Bonus Feat}
\levelnineteen{Intense Focus, Supreme Combat Focus}
\leveltwenty{Bonus Feat}
\end{classtable}

\startclassfeatures

\proficiencies{all Simple and Martial Weapons. Fighters are proficient with Light, Medium, and Heavy Armor and with Shields and Great Shields.}

\classfeature{Weapons Training (Ex):}{Fighters train obsessively with armor and weapons of all kinds, and using a new weapon is easy and fun. By practicing with a weapon he is not proficient with for a day, a Fighter may permanently gain proficiency with that weapon by succeeding at an Intelligence check DC 10 (you may not take 10 on this check).}

\classfeature{Combat Focus (Ex):}{A Fighter is at his best when the chips are down and everything is going to Baator in a handbasket. When the world is on fire, a Fighter keeps his head better than anyone. If the Fighter is in a situation that is stressful and/or dangerous enough that he would normally be unable to ``take 10" on skill checks, he may spend a Swift Action to gain Combat Focus. A Fighter may end his Combat Focus at any time to reroll any die roll he makes, and if not used it ends on its own after a number of rounds equal to his Base Attack Bonus.}

\classfeature{Problem Solver (Ex):}{A Fighter of 3rd level can draw upon his intense and diverse training to respond to almost any situation. As a Swift action, he may choose any [Combat] feat he meets the prerequisites for and use it for a number of rounds equal to his base attack bonus. This ability may be used once per hour.}

\classfeature{Pack Mule (Ex):}{Fighters are used to long journeys with a heavy pack and the use of a wide variety of weaponry and equipment. A 3rd level Fighter suffers no penalties for carrying a medium load, and may retrieve stored items from his person without provoking an attack of opportunity.}

\classfeature{Logistics Mastery (Ex):}{Fighters are excellent and efficient logisticians. When a Fighter reaches 5th level, he gains a bonus to his Command Rating equal to one third his Fighter Level.}

\classfeature{Active Assault (Ex):}{A 5th level Fighter can flawlessly place himself where he is most needed in combat. He may take a 5 foot step as an immediate action. This is in addition to any other movement he takes during his turn, even another 5 foot step.}

\classfeature{Forge Lore:}{A 7th level Fighter can produce magical weapons and equipment as if he had a Caster Level equal to his ranks in Craft.}

\classfeature{Improved Delay (Ex):}{A Fighter of 7th level may delay his action in one round without compromising his Initiative in the next round. In addition, a Fighter may interrupt another action with his delayed action like it was a readied action (though he does not have to announce his intentions before hand).}

\classfeature{Foil Action (Ex):}{A 9th level Fighter may attempt to monkeywrench any action an opponent is taking. The Fighter may throw sand into a beholder's eye, bat aside a key spell component, or strike a weapon hand with a thrown object, but the result is the same: the opponent's action is wasted, and any spell slots, limited ability uses, or the like used to power it are expended. A Fighter must be within 30 feet of his opponent to use this ability, and must hit with a touch attack or ranged touch attack. Using Foil Action is an Immediate action. At 17th level, Foil Action may be used at up to 60 feet.}

\classfeature{Lunging Attacks (Ex):}{The battlefield is an extremely dangerous place, and 11th level Fighters are expected to hold off Elder Elementals, Hezrous, and Hamatulas. Fighters of this level may add 5 feet to the reach of any of their weapons.}

\classfeature{Array of Stunts (Ex):}{A 13th level Fighter may take one extra Immediate Action between his turns without sacrificing a Swift action during his next turn.}

\classfeature{Greater Combat Focus (Ex):}{At 15th level, a Fighter may voluntarily expend his Combat Focus as a non-action to suppress any status effect or ongoing spell effect on himself for his Base Attack Bonus in rounds.}

\classfeature{Intense Focus (Ex):}{A 19th level Fighter may take an extra Swift Action each round (in addition to the extra Immediate Action he can take from Array of Stunts).}

\classfeature{Supreme Combat Focus (Ex):}{A 19th level Fighter may expend his Combat Focus as a non-action to take 20 on any die roll. He must elect to use Supreme Combat Focus before rolling the die.}
