\begin{table}
\rowcolors{1}{colorone}{colortwo}
\caption{Adventuring Gear}
\begin{tabu}to \linewidth{X c c | X c c}
\header\textbf{Item} & \textbf{Cost} & \textbf{Weight} & \textbf{Item} & \textbf{Cost} & \textbf{Weight}\\ \hline
10ft Ladder & 5 cp & 20 lb. & Lantern (bullseye) & 12 gp & 3 lb. \\
10ft Pole & 2 sp & 8 lb. & Lantern (hooded) & 7 gp & 2 lb. \\
Backpack (empty) & 2 gp & 2 lb.\textsuperscript{1} & Lock & -- & 1 lb. \\
Barrel (empty) & 2 gp & 30 lb. & \hspace{.25cm}Very simple & 20 gp & 1 lb. \\
Basket (empty) & 4 sp & 1 lb. & \hspace{.25cm}Average & 40 gp & 1 lb. \\
Bedroll & 1 sp & 5 lb.\textsuperscript{1} & \hspace{.25cm}Good & 80 gp & 1 lb. \\
Bell & 1 gp & -- & \hspace{.25cm}Amazing & 150 gp & 1 lb. \\
Belt Pouch (empty) & 1 gp & \sfrac{1}{2} lb.\textsuperscript{1} & Manacles (common) & 15 gp & 2 lb. \\
Block and tackle & 5 gp & 5 lb. & Manacles (masterwork) & 50 gp & 2 lb. \\
Bucket (empty) & 5 sp & 2 lb. & Miner's Pick & 3 gp & 10 lb. \\
Caltrops & 1 gp & 2 lb. & Oil (1-pint flask) & 1 sp & 1 lb. \\
Candle & 1 cp & -- & Paper (sheet) & 4 sp & -- \\
Canvas (sq. yd.) & 1 sp & 1 lb. & Parchment (sheet) & 2 sp & -- \\
Case, map or scroll & 1 gp & \sfrac{1}{2} lb. & Piton & 1 sp & \sfrac{1}{2} lb. \\
Chain (10 ft.) & 30 gp & 2 lb. & Portable Ram & 10 gp & 20 lb. \\
Chalk, 1 piece & 1 cp & -- & Rope (hempen, 50 ft.) & 1 gp & 10 lb. \\
Chest (empty) & 2 gp & 25 lb. & Rope (silk, 50 ft.) & 10 gp & 5 lb. \\
Clay Jug & 3 cp & 9 lb. & Sack (empty) & 1 sp & \sfrac{1}{2} lb.\textsuperscript{1} \\
Clay Mug/Tankard & 2 cp & 1 lb. & Sealing wax & 1 gp & 1 lb. \\
Clay Pitcher & 2 cp & 5 lb. & Sewing needle & 5 sp & -- \\
Common Lamp & 1 sp & 1 lb. & Signal whistle & 8 sp & -- \\
Crowbar & 2 gp & 5 lb. & Signet ring & 5 gp & -- \\
Firewood (per day) & 1 cp & 20 lb. & Sledge & 1 gp & 10 lb. \\
Fishhook & 1 sp & -- & Small Steel Mirror & 10 gp & \sfrac{1}{2} lb. \\
Fishing net, 25 sq. ft. & 4 gp & 5 lb. & Soap (per lb.) & 5 sp & 1 lb. \\
Flask (empty) & 3 cp & 1.5 lb. & Spade or shovel & 2 gp & 8 lb. \\
Flint and steel & 1 gp & -- & Spyglass & 1,000 gp & 1 lb. \\
Glass Wine Bottle & 2 gp & -- & Tent & 10 gp & 20 lb.\textsuperscript{1} \\
Grappling hook & 1 gp & 4 lb. & Torch & 1 cp & 1 lb. \\
Hammer & 5 sp & 2 lb. & Trail Rations (per day) & 5 sp & 1 lb.\textsuperscript{1} \\
Ink (1 oz. vial) & 8 gp & -- & Vial, ink or potion & 1 gp & \sfrac{1}{10} lb. \\
Inkpen & 1 sp & -- & Waterskin & 1 gp & 4 lb.\textsuperscript{1} \\
Iron Pot & 5 sp & 10 lb. & Whetstone & 2 cp & 1 lb. \\
&&&Winter Blanket & 5 sp & 3 lb.\textsuperscript{1} \\ \hline
\multicolumn{6}{p{\linewidth}}{\textsuperscript{1} These items weigh one-quarter this amount when made for Small characters. Containers for Small characters also carry one-quarter the normal amount.}\\ \hline
\end{tabu}
\end{table}

A few of the pieces of adventuring gear found on Table: Adventuring Gear are described 
below, along with any special benefits they confer on the user ("you").

\textbf{Caltrops:} A caltrop is a four-pronged iron spike crafted so that one prong 
faces up no matter how the caltrop comes to rest. You scatter caltrops on the ground 
in the hope that your enemies step on them or are at least forced to slow down 
to avoid them. One 2- pound bag of caltrops covers an area 5 feet square.

Each time a creature moves into an area covered by caltrops (or spends a round 
fighting while standing in such an area), it might step on one. The caltrops make 
an attack roll (base attack bonus +0) against the creature. For this attack, the 
creature's shield, armor, and deflection bonuses do not count. If the creature 
is wearing shoes or other footwear, it gets a +2 armor bonus to AC. If the caltrops 
succeed on the attack, the creature has stepped on one. The caltrop deals 1 point 
of damage, and the creature's speed is reduced by one-half because its foot is 
wounded. This movement penalty lasts for 24 hours, or until the creature is successfully 
treated with a DC 15 Heal check, or until it receives at least 1 point of magical 
curing. A charging or running creature must immediately stop if it steps on a caltrop. 
Any creature moving at half speed or slower can pick its way through a bed of caltrops 
with no trouble.

Caltrops may not be effective against unusual opponents.

\textbf{Candle:} A candle dimly illuminates a 5-foot radius and burns for 1 hour.

\textbf{Chain:} Chain has hardness 10 and 5 hit points. It can be burst with a 
DC 26 Strength check.

\textbf{Clay Jug:} This basic ceramic jug is fitted with a stopper and holds 1 gallon of liquid.

\textbf{Common Lamp:} A lamp clearly illuminates a 15-foot radius, provides shadowy 
illumination out to a 30-foot radius, and burns for 6 hours on a pint of oil. You 
can carry a lamp in one hand. 

\textbf{Crowbar:} A crowbar it grants a +2 circumstance bonus on Strength checks 
made for such purposes. If used in combat, treat a crowbar as a one-handed improvised 
weapon that deals bludgeoning damage equal to that of a club of its size.

\textbf{Flint and Steel:} Lighting a torch with flint and steel is a full-round 
action, and lighting any other fire with them takes at least that long.

\textbf{Grappling Hook:} Throwing a grappling hook successfully requires a Use 
Rope check (DC 10, +2 per 10 feet of distance thrown).

\textbf{Hammer:} If a hammer is used in combat, treat it as a one-handed improvised 
weapon that deals bludgeoning damage equal to that of a spiked gauntlet of its 
size.

\textbf{Ink:} This is black ink. You can buy ink in other colors, but it costs 
twice as much.

\textbf{Lantern, Bullseye:} A bullseye lantern provides clear illumination in a 
60-foot cone and shadowy illumination in a 120-foot cone. It burns for 6 hours 
on a pint of oil. You can carry a bullseye lantern in one hand.

\textbf{Lantern, Hooded:} A hooded lantern clearly illuminates a 30-foot radius 
and provides shadowy illumination in a 60-foot radius. It burns for 6 hours on 
a pint of oil. You can carry a hooded lantern in one hand.

\textbf{Lock:} The DC to open a lock with the Open Lock skill depends on the lock's 
quality: simple (DC 20), average (DC 25), good (DC 30), or superior (DC 40).

\textbf{Manacles and Manacles, Masterwork:} Manacles can bind a Medium creature. 
A manacled creature can use the Escape Artist skill to slip free (DC 30, or DC 
35 for masterwork manacles). Breaking the manacles requires a Strength check (DC 
26, or DC 28 for masterwork manacles). Manacles have hardness 10 and 10 hit points.

Most manacles have locks; add the cost of the lock you want to the cost of the 
manacles.

For the same cost, you can buy manacles for a Small creature.

For a Large creature, manacles cost ten times the indicated amount, and for a Huge 
creature, one hundred times this amount. Gargantuan, Colossal, Tiny, Diminutive, 
and Fine creatures can be held only by specially made manacles.

\textbf{Oil:} A pint of oil burns for 6 hours in a lantern. You can use a flask 
of oil as a splash weapon. Use the rules for alchemist's fire, except that it takes 
a full round action to prepare a flask with a fuse. Once it is thrown, there is 
a 50\% chance of the flask igniting successfully.

You can pour a pint of oil on the ground to cover an area 5 feet square, provided 
that the surface is smooth. If lit, the oil burns for 2 rounds and deals 1d3 points 
of fire damage to each creature in the area.

\textbf{Portable Ram:} This iron-shod wooden beam gives you a +2 circumstance 
bonus on Strength checks made to break open a door and it allows a second person 
to help you without having to roll, increasing your bonus by 2.

\textbf{Rope, Hempen:} This rope has 2 hit points and can be burst with a DC 23 
Strength check.

\textbf{Rope, Silk:} This rope has 4 hit points and can be burst with a DC 24 Strength 
check. It is so supple that it provides a +2 circumstance bonus on Use Rope checks.

\textbf{Spyglass:} Objects viewed through a spyglass are magnified to twice their 
size.

\textbf{Torch:} A torch burns for 1 hour, clearly illuminating a 20-foot radius 
and providing shadowy illumination out to a 40- foot radius. If a torch is used 
in combat, treat it as a one-handed improvised weapon that deals bludgeoning damage 
equal to that of a gauntlet of its size, plus 1 point of fire damage.

\textbf{Vial:} A vial holds 1 ounce of liquid. The stoppered container usually 
is no more than 1 inch wide and 3 inches high.

%%%%%%%%%%%%%%%%%%%%%%%%%
\subsection{Special Substances and Items}
%%%%%%%%%%%%%%%%%%%%%%%%%

Any of these substances except for the everburning torch and holy water can be 
made by a character with the Craft (alchemy) skill.

\textbf{Acid:} You can throw a flask of acid as a splash weapon. Treat this attack 
as a ranged touch attack with a range increment of 10 feet. A direct hit deals 
1d6 points of acid damage. Every creature within 5 feet of the point where the 
acid hits takes 1 point of acid damage from the splash.

\textbf{Alchemist's Fire:} You can throw a flask of alchemist's fire as a splash 
weapon. Treat this attack as a ranged touch attack with a range increment of 10 
feet.

A direct hit deals 1d6 points of fire damage. Every creature within 5 feet of the 
point where the flask hits takes 1 point of fire damage from the splash. On the 
round following a direct hit, the target takes an additional 1d6 points of damage. 
If desired, the target can use a full-round action to attempt to extinguish the 
flames before taking this additional damage. Extinguishing the flames requires 
a DC 15 Reflex save. Rolling on the ground provides the target a +2 bonus on the 
save. Leaping into a lake or magically extinguishing the flames automatically smothers 
the fire.

\textbf{Antitoxin:} If you drink antitoxin, you get a +5 alchemical bonus on Fortitude 
saving throws against poison for 1 hour.

\textbf{Everburning Torch:} This otherwise normal torch has a \textit{continual 
flame }spell cast upon it. An everburning torch clearly illuminates a 20-foot radius 
and provides shadowy illumination out to a 40-foot radius.

\textbf{Holy Water:} Holy water damages undead creatures and evil outsiders almost 
as if it were acid. A flask of holy water can be thrown as a splash weapon.

Treat this attack as a ranged touch attack with a range increment of 10 feet. A 
flask breaks if thrown against the body of a corporeal creature, but to use it 
against an incorporeal creature, you must open the flask and pour the holy water 
out onto the target. Thus, you can douse an incorporeal creature with holy water 
only if you are adjacent to it. Doing so is a ranged touch attack that does not 
provoke attacks of opportunity.

\begin{wraptable}{r}{.5\linewidth}
\rowcolors{1}{colorone}{colortwo}
\caption{Special Substances and Items}
\begin{tabu}to \linewidth{X c c}
\header\textbf{Item} & \textbf{Cost} & \textbf{Weight}\\ \hline
Acid (flask) & 10 gp & 1 lb.\\
Alchemist's fire (flask) & 20 gp & 1 lb.\\
Antitoxin (vial) & 50 gp & --\\
Everburning torch & 110 gp & 1 lb.\\
Holy water (flask) & 25 gp & 1 lb.\\
Smokestick & 20 gp & \sfrac{1}{2} lb.\\
Sunrod & 2 gp & 1 lb.\\
Tanglefoot bag & 50 gp & 4 lb.\\
Thunderstone & 30 gp & 1 lb.\\
Tindertwig & 1 gp & --\\ \hline
\end{tabu}
\end{wraptable}

A direct hit by a flask of holy water deals 2d4 points of damage to an undead creature 
or an evil outsider. Each such creature within 5 feet of the point where the flask 
hits takes 1 point of damage from the splash.

Temples to good deities sell holy water at cost (making no profit).

\textbf{Smokestick:} This alchemically treated wooden stick instantly creates thick, 
opaque smoke when ignited. The smoke fills a 10- foot cube (treat the effect as 
a \textit{fog cloud }spell, except that a moderate or stronger wind dissipates 
the smoke in 1 round). The stick is consumed after 1 round, and the smoke dissipates 
naturally.

\textbf{Sunrod:} This 1-foot-long, gold-tipped, iron rod glows brightly when struck. 
It clearly illuminates a 30-foot radius and provides shadowy illumination in a 
60-foot radius. It glows for 6 hours, after which the gold tip is burned out and 
worthless.

\textbf{Tanglefoot Bag:} When you throw a tanglefoot bag at a creature (as a ranged 
touch attack with a range increment of 10 feet), the bag comes apart and the goo 
bursts out, entangling the target and then becoming tough and resilient upon exposure 
to air. An entangled creature takes a -2 penalty on attack rolls and a -4 penalty 
to Dexterity and must make a DC 15 Reflex save or be glued to the floor, unable 
to move. Even on a successful save, it can move only at half speed. Huge or larger 
creatures are unaffected by a tanglefoot bag. A flying creature is not stuck to 
the floor, but it must make a DC 15 Reflex save or be unable to fly (assuming it 
uses its wings to fly) and fall to the ground. A tanglefoot bag does not function 
underwater.

A creature that is glued to the floor (or unable to fly) can break free by making 
a DC 17 Strength check or by dealing 15 points of damage to the goo with a slashing 
weapon. A creature trying to scrape goo off itself, or another creature assisting, 
does not need to make an attack roll; hitting the goo is automatic, after which 
the creature that hit makes a damage roll to see how much of the goo was scraped 
off. Once free, the creature can move (including flying) at half speed. A character 
capable of spellcasting who is bound by the goo must make a DC 15 Concentration 
check to cast a spell. The goo becomes brittle and fragile after 2d4 rounds, cracking 
apart and losing its effectiveness. An application of \textit{universal solvent} 
to a stuck creature dissolves the alchemical goo immediately.

\textbf{Thunderstone:} You can throw this stone as a ranged attack with a range 
increment of 20 feet. When it strikes a hard surface (or is struck hard), it creates 
a deafening bang that is treated as a sonic attack. Each creature within a 10-foot-radius 
spread must make a DC 15 Fortitude save or be deafened for 1 hour. A deafened creature, 
in addition to the obvious effects, takes a -4 penalty on initiative and has a 
20\% chance to miscast and lose any spell with a verbal component that it tries 
to cast.

Since you don't need to hit a specific target, you can simply aim at a particular 
5-foot square. Treat the target square as AC 5.

\textbf{Tindertwig:} The alchemical substance on the end of this small, wooden 
stick ignites when struck against a rough surface. Creating a flame with a tindertwig 
is much faster than creating a flame with flint and steel (or a magnifying glass) 
and tinder. Lighting a torch with a tindertwig is a standard action (rather than 
a full-round action), and lighting any other fire with one is at least a standard 
action.

%%%%%%%%%%%%%%%%%%%%%%%%%
\subsection{Tools and Skill Kits}
%%%%%%%%%%%%%%%%%%%%%%%%%

\textbf{Alchemist's Lab:} An alchemist's lab always has the perfect tool for making 
alchemical items, so it provides a +2 circumstance bonus on Craft (alchemy) checks. 
It has no bearing on the costs related to the Craft (alchemy) skill. Without this 
lab, a character with the Craft (alchemy) skill is assumed to have enough tools 
to use the skill but not enough to get the +2 bonus that the lab provides.

\textbf{Artisan's Tools (common):} These special tools include the items needed to pursue 
any craft. Without them, you have to use improvised tools (-2 penalty on Craft 
checks), if you can do the job at all.

\textbf{Artisan's Tools (masterwork):} These tools serve the same purpose as artisan's 
tools (above), but masterwork artisan's tools are the perfect tools for the job, 
so you get a +2 circumstance bonus on Craft checks made with them.

\textbf{Climber's Kit:} This is the perfect tool for climbing and gives you a +2 
circumstance bonus on Climb checks.

\begin{wraptable}{r}{.5\linewidth}
\rowcolors{1}{colorone}{colortwo}
\caption{Tools and Skill Kits}
\begin{tabu}to \linewidth{X c c}
\header\textbf{Item} & \textbf{Cost} & \textbf{Weight}\\ \hline
Alchemist's lab & 500 gp & 40 lb.\\
Artisan's tools (common) & 5 gp & 5 lb.\\
Artisan's tools (masterwork) & 55 gp & 5 lb.\\
Climber's kit & 80 gp & 5 lb.\textsuperscript{1}\\
Disguise kit & 50 gp & 8 lb.\textsuperscript{1}\\
Healer's kit & 50 gp & 1 lb.\\
Holly and mistletoe & -- & --\\
Holy symbol (silver) & 25 gp & 1 lb.\\
Holy symbol (wooden) & 1 gp & --\\
Hourglass & 25 gp & 1 lb.\\
Magnifying glass & 100 gp & --\\
Masterwork  Tool & 50 gp & 1 lb.\\
Merchant's Scale & 2 gp & 1 lb.\\
Musical instrument (common) & 5 gp & 3 lb.\textsuperscript{1}\\
Musical instrument (masterwork) & 100 gp & 3 lb.\textsuperscript{1}\\
Spell component pouch & 5 gp & 2 lb.\\
Thieves' tools (common) & 30 gp & 1 lb.\\
Thieves' tools (masterwork) & 100 gp & 2 lb.\\
Water clock & 1,000 gp & 200 lb.\\
Wizard's Spellbook (blank) & 15 gp & 3 lb.\\ \hline
\multicolumn{3}{p{\linewidth}}{\textsuperscript{1} These items weigh one-quarter this amount when made for Small characters.}\\
\hline
\end{tabu}
\end{wraptable}

\textbf{Disguise Kit:} The kit is the perfect tool for disguise and provides a 
+2 circumstance bonus on Disguise checks. A disguise kit is exhausted after ten 
uses.

\textbf{Healer's Kit:} It is the perfect tool for healing and provides a +2 circumstance 
bonus on Heal checks. A healer's kit is exhausted after ten uses.

\textbf{Holy Symbol, Silver or Wooden:} A holy symbol focuses positive energy. 
A cleric or paladin uses it as the focus for his spells and as a tool for turning 
undead. Each religion has its own holy symbol.

\textbf{Magnifying Glass:} This simple lens allows a closer look at small objects. 
It is also useful as a substitute for flint and steel when starting fires. Lighting 
a fire with a magnifying glass requires light as bright as sunlight to focus, tinder 
to ignite, and at least a full-round action. A magnifying glass grants a +2 circumstance 
bonus on Appraise checks involving any item that is small or highly detailed.

\textbf{Masterwork Tool:} This well-made item is the perfect tool for the job. 
It grants a +2 circumstance bonus on a related skill check (if any). Bonuses provided 
by multiple masterwork items used toward the same skill check do not stack.

\textbf{Merchant's Scale:} A scale grants a +2 circumstance bonus on Appraise 
checks involving items that are valued by weight, including anything made of precious 
metals.

\textbf{Musical Instrument, Common or Masterwork:} A masterwork instrument grants 
a +2 circumstance bonus on Perform checks involving its use.

\textbf{Spell Component Pouch:} A spellcaster with a spell component pouch is assumed 
to have all the material components and focuses needed for spellcasting, except 
for those components that have a specific cost, divine focuses, and focuses that 
wouldn't fit in a pouch.

\textbf{Thieves' Tools (common):} This kit contains the tools you need to use the Disable 
Device and Open Lock skills. Without these tools, you must improvise tools, and 
you take a -2 circumstance penalty on Disable Device and Open Locks checks.

\textbf{Thieves' Tools (masterwork):} This kit contains extra tools and tools of 
better make, which grant a +2 circumstance bonus on Disable Device and Open Lock 
checks.

\textit{Unholy Symbols:} An unholy symbol is like a holy symbol except that it 
focuses negative energy and is used by evil clerics (or by neutral clerics who 
want to cast evil spells or command undead).

\textbf{Water Clock:} This large, bulky contrivance gives the time accurate to 
within half an hour per day since it was last set. It requires a source of water, 
and it must be kept still because it marks time by the regulated flow of droplets 
of water.

\textbf{Wizard's Spellbook (blank):} A spellbook has 100 pages of parchment, and 
each spell takes up one page per spell level (one page each for 0-level spells).

%%%%%%%%%%%%%%%%%%%%%%%%%
\subsection{Clothing}
%%%%%%%%%%%%%%%%%%%%%%%%%

\textbf{Artisan's Outfit:} This outfit includes a shirt with buttons, a skirt or 
pants with a drawstring, shoes, and perhaps a cap or hat. It may also include a 
belt or a leather or cloth apron for carrying tools.

\textbf{Cleric's Vestments:} These ecclesiastical clothes are for performing priestly 
functions, not for adventuring.

\textbf{Cold Weather Outfit:} A cold weather outfit includes a wool coat, linen 
shirt, wool cap, heavy cloak, thick pants or skirt, and

boots. This outfit grants a +5 circumstance bonus on Fortitude saving throws against 
exposure to cold weather.

\textbf{Courtier's Outfit:} This outfit includes fancy, tailored clothes in whatever 
fashion happens to be the current style in the courts of the nobles. Anyone trying 
to influence nobles or courtiers while wearing street dress will have a hard time 
of it (-2 penalty on Charisma-based skill checks to influence such individuals). 
If you wear this outfit without jewelry (costing an additional 50 gp), you look 
like an out-of-place commoner.

\begin{wraptable}{r}{.5\linewidth}
\rowcolors{1}{colorone}{colortwo}
\caption{Clothing}
\begin{tabu}to \linewidth{X c c}
\header\textbf{Item} & \textbf{Cost} & \textbf{Weight}\\ \hline
Artisan's outfit & 1 gp & 4 lb\textsuperscript{1}\\
Cleric's vestments & 5 gp & 6 lb\textsuperscript{1}\\
Cold weather outfit & 8 gp & 7 lb\textsuperscript{1}\\
Courtier's outfit & 30 gp & 6 lb\textsuperscript{1}\\
Entertainer's outfit & 3 gp & 4 lb\textsuperscript{1}\\
Explorer's outfit & 10 gp & 8 lb\textsuperscript{1}\\
Monk's outfit & 5 gp & 2 lb\textsuperscript{1}\\
Noble's outfit & 75 gp & 10 lb\textsuperscript{1}\\
Peasant's outfit & 1 sp & 2 lb\textsuperscript{1}\\
Royal outfit & 200 gp & 15 lb\textsuperscript{1}\\
Scholar's outfit & 5 gp & 6 lb\textsuperscript{1}\\
Traveler's outfit & 1 gp & 5 lb\textsuperscript{1}\\ \hline
\multicolumn{3}{p{\linewidth}}{\textsuperscript{1} These items weigh one-quarter this amount when made for Small characters.}\\
\hline
\end{tabu}
\end{wraptable}

\textbf{Entertainer's Outfit:} This set of flashy, perhaps even gaudy, clothes 
is for entertaining. While the outfit looks whimsical, its practical design lets 
you tumble, dance, walk a tightrope, or just run (if the audience turns ugly).

\textbf{Explorer's Outfit:} This is a full set of clothes for someone who never 
knows what to expect. It includes sturdy boots, leather breeches or a skirt, a 
belt, a shirt (perhaps with a vest or jacket), gloves, and a cloak. Rather than 
a leather skirt, a leather overtunic may be worn over a cloth skirt. The clothes 
have plenty of pockets (especially the cloak). The outfit also includes any extra 
items you might need, such as a scarf or a wide-brimmed hat.

\textbf{Monk's Outfit:} This simple outfit includes sandals, loose breeches, and 
a loose shirt, and is all bound together with sashes. The outfit is designed to 
give you maximum mobility, and it's made of high-quality fabric. You can hide small 
weapons in pockets hidden in the folds, and the sashes are strong enough to serve 
as short ropes.

\textbf{Noble's Outfit:} This set of clothes is designed specifically to be expensive 
and to show it. Precious metals and gems are worked into the clothing. To fit into 
the noble crowd, every would-be noble also needs a signet ring (see Adventuring 
Gear, above) and jewelry (worth at least 100 gp).

\textbf{Peasant's Outfit:} This set of clothes consists of a loose shirt and baggy 
breeches, or a loose shirt and skirt or overdress. Cloth wrappings are used for 
shoes.

\textbf{Royal Outfit:} This is just the clothing, not the royal scepter, crown, 
ring, and other accoutrements. Royal clothes are ostentatious, with gems, gold, 
silk, and fur in abundance.

\textbf{Scholar's Outfit:} Perfect for a scholar, this outfit includes a robe, 
a belt, a cap, soft shoes, and possibly a cloak.

\textbf{Traveler's Outfit:} This set of clothes consists of boots, a wool skirt 
or breeches, a sturdy belt, a shirt (perhaps with a vest or jacket), and an ample 
cloak with a hood.