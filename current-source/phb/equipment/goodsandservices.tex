%%%%%%%%%%%%%%%%%%%%%%%%%
\subsection{Food, Drink, and Lodging}
%%%%%%%%%%%%%%%%%%%%%%%%%

\textbf{Inn:} Poor accommodations at an inn amount to a place on the floor near 
the hearth. Common accommodations consist of a place on a raised, heated floor, 
the use of a blanket and a pillow. Good accommodations consist of a small, private 
room with one bed, some amenities, and a covered chamber pot in the corner.

\textbf{Meals:} Poor meals might be composed of bread, baked turnips, onions, and 
water. Common meals might consist of bread, chicken stew, carrots, and watered-down 
ale or wine. Good meals might be composed of bread and pastries, beef, peas, and 
ale or wine.

\begin{table}[h]
\rowcolors{1}{colorone}{colortwo}
\caption{Food, Drink, and Lodging}
\centering
{\tabulinesep=1mm
\begin{tabu}to \linewidth{X[2] X X | X[2] X X}
\header \textbf{Item} & \textbf{Cost} & \textbf{Weight} & \textbf{Item} & \textbf{Cost} & \textbf{Weight}\\ \hline
Banquet (per person) & 10 gp & --&Meals (per day)&&\\
Chunk of Meat & 3 sp & \sfrac{1}{2} lb.&\hspace{.5cm}Good & 5 sp & --\\
Hunk of Cheese & 1 sp & \sfrac{1}{2} lb.&\hspace{.5cm}Common & 3 sp & --\\
Loaf of Bread & 2 cp & \sfrac{1}{2} lb.&\hspace{.5cm}Poor & 1 sp & --\\
Ale&&&Wine&&\\
\hspace{.5cm}Gallon & 2 sp & 8 lb.&\hspace{.5cm}Common (pitcher) & 2 sp & 6 lb.\\
\hspace{.5cm}Mug & 4 cp & 1 lb. &\hspace{.5cm}Fine (bottle) & 10 gp & 1.5 lb.\\
Inn stay (per day)&&&&&\\
\hspace{.5cm}Good & 2 gp & --&&&\\
\hspace{.5cm}Common & 5 sp & --&&&\\
\hspace{.5cm}Poor & 2 sp & --&&&\\ \hline
\end{tabu}}
\end{table}

%%%%%%%%%%%%%%%%%%%%%%%%%
\subsection{Transport}
%%%%%%%%%%%%%%%%%%%%%%%%%

\textbf{Carriage}: This four-wheeled vehicle can transport as many as four people 
within an enclosed cab, plus two drivers. In general, two horses (or other beasts 
of burden) draw it. A carriage comes with the harness needed to pull it.

\textbf{Cart:} This two-wheeled vehicle can be drawn by a single horse (or other 
beast of burden). It comes with a harness.

\textbf{Galley:} This three-masted ship has seventy oars on either side and requires 
a total crew of 200. A galley is 130 feet long and 20 feet wide, and it can carry 
150 tons of cargo or 250 soldiers. For 8,000 gp more, it can be fitted with a ram 
and castles with firing platforms fore, aft, and amidships. This ship cannot make 
sea voyages and sticks to the coast. It moves about 4 miles per hour when being 
rowed or under sail.

\textbf{Keelboat:} This 50- to 75-foot-long ship is 15 to 20 feet wide and has 
a few oars to supplement its single mast with a square sail. It has a crew of eight 
to fifteen and can carry 40 to 50 tons of cargo or 100 soldiers. It can make sea 
voyages, as well as sail down rivers (thanks to its flat bottom). It moves about 
1 mile per hour.

\begin{wraptable}{r}{.4\linewidth}
\rowcolors{1}{colorone}{colortwo}
\caption{Transport}
\centering
{\tabulinesep=1mm
\begin{tabu}to \linewidth{X r r}
\header\textbf{Item} & \textbf{Cost} & \textbf{Weight}\\\hline
Carriage & 100 gp & 600 lb.\\
Cart & 15 gp & 200 lb.\\
Galley & 30,000 gp & --\\
Keelboat & 3,000 gp & --\\
Longship & 10,000 gp & --\\
Rowboat & 50 gp & 100 lb.\\
Oar & 2 gp & 10 lb.\\
Sailing ship & 10,000 gp & --\\
Sled & 20 gp & 300 lb.\\
Wagon & 35 gp & 400 lb.\\
Warship & 25,000 gp & --\\ \hline
\end{tabu}}
\end{wraptable}

\textbf{Longship:} This 75-foot-long ship with forty oars requires a total crew 
of 50. It has a single mast and a square sail, and it can carry 50 tons of cargo 
or 120 soldiers. A longship can make sea voyages. It moves about 3 miles per hour 
when being rowed or under sail.

\textbf{Rowboat:} This 8- to 12-foot-long boat holds two or three Medium passengers. 
It moves about 1.5 miles per hour.

\textbf{Sailing Ship:} This larger, seaworthy ship is 75 to 90 feet long and 20 
feet wide and has a crew of 20. It can carry 150 tons of cargo. It has square sails 
on its two masts and can make sea voyages. It moves about 2 miles per hour.

\textbf{Sled:} This is a wagon on runners for moving through snow and over ice. 
In general, two horses (or other beasts of burden) draw it. A sled comes with the 
harness needed to pull it.

\textbf{Wagon:} This is a four-wheeled, open vehicle for transporting heavy loads. 
In general, two horses (or other beasts of burden) draw it. A wagon comes with 
the harness needed to pull it.

\textbf{Warship:} This 100-foot-long ship has a single mast, although oars can 
also propel it. It has a crew of 60 to 80 rowers. This ship can carry 160 soldiers, 
but not for long distances, since there isn't room for supplies to support that 
many people. The warship cannot make sea voyages and sticks to the coast. It is 
not used for cargo. It moves about 2.5 miles per hour when being rowed or under 
sail.

%%%%%%%%%%%%%%%%%%%%%%%%%
\subsection{Spellcasting and Services}
%%%%%%%%%%%%%%%%%%%%%%%%%

\begin{table}[b]
\rowcolors{1}{colorone}{colortwo}
\caption{Spellcasting and Services}
{\tabulinesep=1mm
\begin{tabu}to \textwidth{X X}
\header\textbf{Service} & \textbf{Cost}\\ \hline
Coach cab & 3 cp per mile\\
Messenger & 2 cp per mile\\
Road or gate toll & 1 cp\\
Ship's passage & 1 sp per mile\\
Spell, 0th-level & Caster level x 5 gp\textsuperscript{1}\\
Spell, 1st-level & Caster level x 10 gp\textsuperscript{1}\\
Spell, 2nd-level & Caster level x 20 gp\textsuperscript{1}\\
Spell, 3rd-level & Caster level x 30 gp\textsuperscript{1}\\
Spell, 4th-level & Caster level x 40 gp\textsuperscript{1}\\
Spell, 5th-level & Caster level x 50 gp\textsuperscript{1}\\
Spell, 6th-level & Caster level x 60 gp\textsuperscript{1}\\
Spell, 7th-level & Caster level x 70 gp\textsuperscript{1}\\
Spell, 8th-level & Caster level x 80 gp\textsuperscript{1}\\
Spell, 9th-level & Caster level x 90 gp\textsuperscript{1}\\
Trained Hireling & 3 sp per day\\
Untrained Hireling & 1 sp per day\\\hline
\end{tabu}
\begin{tabu}to \linewidth{X}
\rowcolor{colortwo}\textsuperscript{1} See spell description for additional costs. If the additional costs put the spell's total cost above 3,000 gp, that spell is not generally available.\\ \hline
\end{tabu}}
\end{table}

Sometimes the best solution for a problem is to hire someone else to take care 
of it.

\textbf{Coach Cab:} The price given is for a ride in a coach that transports people 
(and light cargo) between towns. For a ride in a cab that transports passengers 
within a city, 1 copper piece usually takes you anywhere you need to go.

\textbf{Hireling, Trained:} The amount given is the typical daily wage for mercenary 
warriors, masons, craftsmen, scribes, teamsters, and other trained hirelings. This 
value represents a minimum wage; many such hirelings require significantly higher 
pay.

\textbf{Hireling, Untrained:} The amount shown is the typical daily wage for laborers, 
porters, cooks, maids, and other menial workers.

\textbf{Messenger:} This entry includes horse-riding messengers and runners. Those 
willing to carry a message to a place they were going anyway may ask for only half 
the indicated amount.

\textbf{Road or Gate Toll:} A toll is sometimes charged to cross a well-trodden, 
well-kept, and well-guarded road to pay for patrols on it and for its upkeep. Occasionally, 
a large walled city charges a toll to enter or exit (or sometimes just to enter).

\textbf{Ship's Passage:} Most ships do not specialize in passengers, but many have 
the capability to take a few along when transporting cargo. Double the given cost 
for creatures larger than Medium or creatures that are otherwise difficult to bring 
aboard a ship.

\textbf{Spell:} The indicated amount is how much it costs to get a spellcaster 
to cast a spell for you. This cost assumes that you can go to the spellcaster and 
have the spell cast at his or her convenience (generally at least 24 hours later, 
so that the spellcaster has time to prepare the spell in question). If you want 
to bring the spellcaster somewhere to cast a spell you need to negotiate with him 
or her, and the default answer is no.

The cost given is for a spell with no cost for a material component or focus component 
and no XP cost. If the spell includes a material component, add the cost of that 
component to the cost of the spell.

If the spell has a focus component (other than a divine focus), add 1/10 the cost 
of that focus to the cost of the spell. If the spell has an XP cost, add 5 gp per 
XP lost. 

Furthermore, if a spell has dangerous consequences, the spellcaster will certainly 
require proof that you can and will pay for dealing with any such consequences 
(that is, assuming that the spellcaster even agrees to cast such a spell, which 
isn't certain). In the case of spells that transport the caster and characters 
over a distance, you will likely have to pay for two castings of the spell, even 
if you aren't returning with the caster.

In addition, not every town or village has a spellcaster of sufficient level to 
cast any spell. In general, you must travel to a small town (or larger settlement) 
to be reasonably assured of finding a spellcaster capable of casting 1st-level 
spells, a large town for 2nd-level spells, a small city for 3rd- or 4th-level spells, 
a large city for 5th- or 6th-level spells, and a metropolis for 7th- or 8th-level 
spells. Even a metropolis isn't guaranteed to have a local spellcaster able to 
cast 9th-level spells.