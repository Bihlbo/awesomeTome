At their core, a weapon is just an object of a particular size and complexity that you wield against a foe in an attempt to disable them. The size, and complexity of that object are not negligible parts of it though, and are in fact the basis of an effective weapon.

\subsection{Weapon Descriptors}

\subsubsection{Melee Weapons}
Melee weapons are used for making attacks against nearby foes, and generally threaten every space within a creature's natural reach.

\textbf{Reach Weapons:} A reach weapon is a melee weapon with a much longer haft than normal, allowing its wielder to strike at targets that aren't adjacent to him or her. Most reach weapons double the wielder's natural reach, meaning that a typical Small or Medium wielder of such a weapon can attack a creature 10 feet away, but not a creature in an adjacent square. A typical Large character wielding a reach weapon of the appropriate size can attack a creature 15 or 20 feet away, but not adjacent creatures or creatures up to 10 feet away. There may be limits on how you can use a reach weapon, consult each weapons individual entry. 

\textbf{Hurled Weapons:} Some melee weapons also list a range. These may be hurled at a target, and function as a ranged weapon when they are used in this way. See "Thrown Weapons" for more information about using a hurled melee weapon in this way.

\textbf{Double Weapons:} A double weapon has a damaging head on both the ends of the weapon. A character can fight with both ends of a double weapon as if fighting with two weapons, but they incur all the normal attack penalties associated with two-weapon combat as if they were wielding a one-handed weapon and a light weapon. The character can also choose to use a double weapon two handed, attacking with only one end of it. A creature wielding a double weapon in one hand can't use it as a double weapon-only one end of the weapon can be used in any given round. 

\subsubsection{Ranged Weapons}
Ranged weapons are suited for striking distant foes, and threaten no spaces. Using a ranged weapon within a threat range generally provokes attacks of opportunity, making them well suited to firing from out of the fray.

\textbf{Projectile Weapons:} Crossbows, repeating crossbows, bows, compound bows, and slings are projectile weapons. Most projectile weapons require two hands to use (see specific weapon descriptions). A character does not add their Strength bonus on damage rolls with a projectile weapon unless it's a composite bow or sling. If the character has a penalty for low Strength, it is added to damage rolls when they use a projectile weapon other than a crossbow.

\textbf{Ammunition:} Projectile weapons use ammunition: arrows (for bows), bolts (for crossbows), or sling bullets (for slings). When using a bow, a character can draw ammunition as a free action; crossbows and slings require an action for reloading. Generally speaking, ammunition that hits its target is destroyed or rendered useless, while normal ammunition that misses has a 50\% chance of being destroyed or lost. 

Attempting to use an arrow or bolt as a melee weapon incurs a -4 non-proficiency penalty, and deals damage equal to the bow or crossbow it was designed for. Sling bullets may not be used as a melee weapon.

\textbf{Ranged weapons and Mounts:} Thrown weapons and crossbows can be used from mounts without complication (aside from the normal penalties for using ranged weapons from mounts). Bows must be at least one size category smaller than the wielder to be used on a mount.

\textbf{Thrown Weapons:} In order to use a thrown weapon properly, it must be small enough for the wielder to use one handed. Ranged weapons the same size as the wielder can be thrown with two hands, but doing so incurs a -4 penalty on the attack roll. The wielder applies his or her Strength modifier to damage dealt by thrown weapons (except for splash weapons). It is possible to throw a weapon that isn't designed to be thrown (that is, a melee weapon that doesn't have a numeric entry in the Range Increment column on Table: Weapons), but a character who does so takes a -4 penalty on the attack roll. Throwing a light or one-handed weapon is a standard action, while throwing a two-handed weapon is a full-round action. Regardless of the type of weapon, such an attack scores a threat only on a natural roll of 20 and deals double damage on a critical hit. Such a weapon has a range increment of 10 feet. Any weapon three sizes smaller than the wielder can be thrown with a 10 foot range increment without penalty. 

\subsubsection{Improvised Weapons}
Sometimes objects not crafted to be weapons nonetheless see use in combat. Because such objects are not designed for this use, any creature that uses one in combat is considered to be nonproficient with it and takes a -4 penalty on attack rolls made with that object. To determine the size category and appropriate damage for an improvised weapon, compare its relative size and damage potential to the weapon list to find a reasonable match. An improvised weapon scores a threat on a natural roll of 20 and deals double damage on a critical hit. An improvised thrown weapon has a range increment of 10 feet. Objects heaver than a character's light load cannot be used as weapons.

\subsubsection{Weapon Size}
Every weapon, like every object and creature, has a size category that indicates how different sized creatures can interact with it.

\textbf{Two-Handed:} A two-handed weapon is one that is the same size category as the wielder. Two-handed weapons must be wielded with both the primary and off hand to be effective. Attacks with a two-handed melee weapon add 1-1/2 times the character's Strength bonus to damage rolls.

\textbf{One-Handed:} A one-handed weapon is one that is one size category smaller than the wielder. One-handed weapons can be used in either the primary hand or the off hand. Attacks with a one-handed melee weapon add the wielder's Strength bonus to damage rolls if it's used in the primary hand, or 1/2 their Strength bonus if it's used in the off hand. If a one-handed melee weapon is wielded with two hands during combat, 1-1/2 times the character's Strength bonus is added to damage rolls. 

\textbf{Light:} A light weapon is one that is two or more size categories smaller than the wielder. Light weapons can be used in either the primary hand or the off hand. Attacks with a light weapon add the wielder's Strength bonus (if any) to damage rolls for melee attacks with a light weapon if it's used in the primary hand, or one-half the wielder's Strength bonus if it's used in the off hand. Using two hands to wield a light weapon gives no advantage on damage; the Strength bonus applies as though the weapon were held in the wielder's primary hand only. It is even easier to use in one's off hand than a one-handed weapon is, however, and they are well suited for two-weapon fighting styles. Light melee weapons can also be used while grappling. An unarmed strike is always considered a light weapon.

\textbf{Inappropriately Sized Weapons:} A creature can't use weapons larger than itself.

\subsubsection{Weapon Complexity}
In addition to their size, every weapon is grouped according to how difficult the weapon is to master. Every weapon falls into one of three broad categories: simple, martial, and exotic. 

\textbf{Simple Weapons:} Simple weapons are those that require the least training and practice to use effectively. Many are simple 'swing and pray' or 'point and thrust' weapons that are hard to mess up, or simply weapons that are so small that they only work in very straightforward ways. Most classes are proficient in all simple weapons.

\textbf{Martial Weapons:} Martial weapons are those that require dedication and training to use effectively. 

Because of this martial weapons that are three sizes smaller than the wielder are always treated as simple weapons for the wielder.

\textbf{Exotic Weapons:} 

\subsection{Weapon Description}
Weapon entries follow the following format.

\noindent\textbf{Cost:} This value is the weapon's cost in gold pieces (gp) or silver pieces (sp). The cost includes miscellaneous gear that goes with the weapon. The cost is the same for a Small or Medium version of the weapon, while a Large version costs twice the listed price. Versions smaller than Small cost half as much for each size category reduction.

\textbf{Damage:} Each type of weapon deals damage based on its size.

\textbf{Critical:} The entry in this column notes how the weapon is used with the rules for critical hits. The first number indicates the minimum d20 roll that will generate a critical threat. When your character confirms a threat and scores a critical hit, roll the weapon damage two, three, or four times as indicated by its critical multiplier (using all applicable modifiers to weapon damage on each roll), and add all the results together. Extra damage over and above a weapon's normal damage, such as sneak attack damage, is not multiplied when you score a critical hit.

\begin{itemize}
  \item\noindent\textbf{20x2:}{The weapon deals double damage on a critical hit.}
  \item\noindent\textbf{20x3:}{The weapon deals triple damage on a critical hit.}
  \item\noindent\textbf{x3/x4:}{One head of this double weapon deals triple damage on a critical hit. The other head deals quadruple damage on a critical hit.}
  \item\noindent\textbf{20x4:}{The weapon deals quadruple damage on a critical hit.}
  \item\noindent\textbf{19-20/x2:}{The weapon scores a threat on a natural roll of 19 or 20 (instead of just 20) and deals double damage on a critical hit. (The weapon has a threat range of 19-20.)}
  \item\noindent\textbf{18-20/x2:}{The weapon scores a threat on a natural roll of 18, 19, or 20 (instead of just 20) and deals double damage on a critical hit. (The weapon has a threat range of 18-20.)}
  \item\noindent\textbf{19-20/x3:}{The weapon scores a threat on a natural roll of 19 or 20 (instead of just 20) and deals triple damage on a critical hit. (The weapon has a threat range of 19-20.)}
\end{itemize}

\textbf{Range:} Any attack at less than this distance is not penalized for range. However, each full range increment imposes a cumulative -2 penalty on the attack roll. A thrown weapon has a maximum range of five range increments. A projectile weapon can shoot out to ten range increments.

\textbf{Type:} Weapons are classified according to the type of damage they deal: bludgeoning, piercing, or slashing. Some monsters may be resistant or immune to attacks from certain types of weapons. Some weapons deal damage of multiple types. If a weapon is of two types, the damage it deals is not half one type and half another; all of it is both types. Therefore, a creature would have to be immune to both types of damage to ignore any of the damage from such a weapon.  In other cases, a weapon can deal either of two types of damage. In a situation when the damage type is significant, the wielder can choose which type of damage to deal with such a weapon.

\textbf{Special:} Some weapons have special features. See the weapon descriptions for details.

\afterpage{
\footnotetext[1]{The size and damage for this weapon indicates the size of the creature using it, instead of the actual size of the weapon. These weapons are always considered light weapons.}
\begin{small}
{\tabulinesep=1mm
\rowcolors{1}{colorone}{colortwo}
\begin{longtabu} to \textwidth {X[6, l] X[3, l] X[4, l] X[3, l] X[2, l] X[2, l] X[2, l] X[2, l] X[2, l] X[2, l]}
\header\textbf{Simple Weapons}&&&&&&&&&\\
\hline
\rowcolor{colortwo}\textbf{Weapon} &\textbf{Critical} &\textbf{Type} &\textbf{Range} &\textbf{Fine} &\textbf{Dimin.} &\textbf{Tiny} &\textbf{Small} &\textbf{Medium} &\textbf{Large} \endhead
 Club &20x2 &Bludgeoning &Melee &1d2  &1d3 &1d4  &1d6   &1d8 &2d6 \\[1ex]
 Crowssbow &19-20x2 &Piercing &120 ft. &- &1d4 &1d6 &1d8 &1d10 &2d8 \\[1ex]
 Gauntlet\footnotemark[1] &20x2 &Bludgeoning &Melee &- &- &1 &1d2 &1d3 &1d4 \\[1ex]
 Hammer &20x2 &Bludgeoning &Melee &1d2 &1d3 &1d4 &1d6 &1d8 &2d6 \\[1ex]
 Longspear &20x3 &Piercing &Reach &1d2 &1d3 &1d4 &1d6 &1d8 &2d6 \\[1ex]
 Morning Star &20x2 &Bludgeoning \& Piercing &Melee &1d3 &1d4 &1d6 &1d8 &2d6 &3d6 \\
 Sling &20x2 &Bludgeoning &50 ft. &- &1 &1d2 &1d3 &1d4 &1d6 \\[1ex]
 Spear &20x3 &Piercing &Melee or 20 ft. &1d2 &1d3 &1d4 &1d6 &1d8 &2d6 \\
 Staff &20x2 &Bludgeoning &Melee &1/\newline{}1 &1d2/ 1d2 &1d3/ 1d3 &1d4/ 1d4 &1d6/ 1d6 &1d8/ 1d8 \\
 Spiked Gauntlet\footnotemark[1] &20x2 &Piercing and Bludgeoning &Melee &- &1 &1d2 &1d3 &1d4 &1d6 \\
 Unarmed\footnotemark[1] &20x2 &Bludgeoning &Melee &- &- &1 &1d2 &1d3 &1d4 \\[1ex]
 \hline
\end{longtabu}
%
\rowcolors{1}{colorone}{colortwo}
\begin{longtabu} to \textwidth {X[6, l] X[3, l] X[4, l] X[3, l] X[2, l] X[2, l] X[2, l] X[2, l] X[2, l] X[2, l]}
\header\textbf{Martial Weapons}&&&&&&&&&\\
\hline
\rowcolor{colortwo}\textbf{Weapon} &\textbf{Critical} &\textbf{Type} &\textbf{Range} &\textbf{Fine} &\textbf{Dimin.} &\textbf{Tiny} &\textbf{Small} &\textbf{Medium} &\textbf{Large} \endhead
 Axe &20x3 &Bludgeoning \& Slashing &Melee &1d3 &1d4 &1d6 &1d8 &1d12 &3d6 \\
 Bastard Sword &19-20x2 &Slashing or Piercing &Melee &1d3 &1d4 &1d6 &1d8 &1d10 &2d8 \\
 Bow &20x3 &Piercing &100 ft. &- &1d3 &1d4 &1d6 &1d8 &2d6 \\[1ex]
 Composite Bow &20x3 &Piercing &110 ft.&- &1d3 &1d4 &1d6 &1d8 &2d6 \\[1ex]
 Curved Sword &18-20x2 &Slashing &Melee &1d2 &1d3 &1d4 &1d6 &2d4 &2d6 \\[1ex]
 Dwarven Axe &20x3 &Bludgeoning \& Slashing &Melee &1d3 &1d4 &1d6 &1d8 &1d10 &1d12 \\
 Flail &19-20x2 &Bludgeoning &Melee &1d3 &1d4 &1d6 &1d8 &1d10 &2d6 \\[1ex]
 Glaive &20x3 &Slashing &Reach &1d3 &1d4 &1d6 &1d8 &1d10 &2d6 \\[1ex]
 Greatclub &20x2 &Bludgeoning &Melee &1d3 &1d4 &1d6 &1d8 &1d10 &2d6 \\[1ex]
 Guisarme &20x3 &Slashing &Reach &1d2 &1d3 &1d4 &1d6 &2d4 &2d6 \\[1ex]
 Halberd &20x3 &Percing or Slashing &Melee &1d3 &1d4 &1d6 &1d8 &1d10 &2d6 \\
 Pick &20x4 &Piercing &Melee &1 &1d2 &1d4 &1d6 &1d8 &1d10 \\[1ex]
 Ranseur &20x3 &Piercing &Reach &1d2 &1d3 &1d4 &1d6 &2d4 &2d6 \\[1ex]
 Sap &20x2 &Bludgeoning &Melee &1d3 &1d4 &1d6 &1d8 &1d10 &2d6 \\[1ex]
 Scythe &20x4 &Piercing or Slashing &Melee &1d2 &1d3 &1d4 &1d6 &2d4 &2d6 \\
 Shield &20x2 &Bludgeoning &Melee &1 &1d2 &1d3 &1d4 &1d6 &1d8 \\[1ex]
 Spiked Armor\footnotemark[1] &20x2 &Piercing &Melee &1 &1d2 &1d3 &1d4 &1d6 &1d8 \\[1ex]
 Spiked Shield &20x2 &Bludgeoning \& Piercing &Melee &1d3 &1d4 &1d6 &1d8 &1d10 &2d6 \\[1ex]
 Sword &19-20x2 &Slashing or Piercing &Melee &1d3 &1d4 &1d6 &1d8 &2d6 &3d6 \\
 Thinblade &19-20x3 &Piercing &Melee &1d2 &1d3 &1d4 &1d6 &2d4 &2d6 \\[1ex]
 Throwing Axe &20x2 &Bludgeoning \& Slashing &10 ft. &1d2 &1d3 &1d4 &1d6 &1d8 &1d10 \\
 Throwing Hammer &20x2 &Bludgeoning &20 ft. &1d2 &1d3 &1d4 &1d6 &1d8 &1d10  \\[1ex]
 Trident &20x2 &Piercing &Melee or 10 ft. &1d3 &1d4 &1d6 &1d8 &1d10 &2d6 \\
 Warhammer &20x3 or 20x4 &Bludgeoning or Piercing &Melee &1d3 or 1d2 &1d4 or 1d3 &1d6 or 1d4 &1d8 or 1d6 &2d6 or 1d8 &3d6 or 2d6 \\
 \hline
\end{longtabu}
%
\rowcolors{1}{colorone}{colortwo}
\begin{longtabu} to \textwidth {X[6, l] X[3, l] X[4, l] X[3, l] X[2, l] X[2, l] X[2, l] X[2, l] X[2, l] X[2, l]}
\header\textbf{Exotic Weapons}&&&&&&&&&\\
\hline
\rowcolor{colortwo}\textbf{Weapon} &\textbf{Critical} &\textbf{Type} &\textbf{Range} &\textbf{Fine} &\textbf{Dimin.} &\textbf{Tiny} &\textbf{Small} &\textbf{Medium} &\textbf{Large} \endhead
 Bolas &20x2 &Bludgeoning &10 ft. &- &1 &1d2 &1d3 &1d4 &1d6 \\[1ex]
 Dire Flail &19-20x2/ 19-20x2 &Bludgeoning &Melee &1d2/ 1d2 &1d3/ 1d3 &1d4/ 1d4 &1d6/ 1d6 &1d8/ 1d8 &1d10/ 1d10 \\
 Double Axe &20x3/ 20x3 &Bludgeoning \& Slashing &Melee &1d3/ 1d3 &1d3/ 1d3 &1d4/ 1d4 &1d6/ 1d6 &1d8/ 1d8 &1d10/ 1d10 \\
 Double Sword &19-20x2/ 19-20x2 &Piercing or Slashing &Melee &1d2/ 1d2 &1d3/ 1d3 &1d4/ 1d4 &1d6/ 1d6 &1d8/ 1d8 &1d10/ 1d10 \\
 Hook-Hammer &20x3/ 20x4 &Bludgeoning/ Piercing &Melee &- &1d2/\newline{}1 &1d3/ 1d2 &1d4/ 1d3 &1d6/ 1d4 &1d8/ 1d6 \\
 Kama &20x2 &Slashing &Melee &1d3 &1d4 &1d6 &1d8 &1d10 &2d6 \\[1ex]
 Kasurigama &20x2 &Slashing &Melee or Reach &1 &1d2 &1d3 &1d4 &1d6 &2d4 \\
 Net &N/A &N/A &Reach &- &- &- &- &- &- \\[1ex]
 Nunchaku &20x2 &Bludgeoning &Melee &1d3 &1d4 &1d6 &1d8 &1d10 &2d6 \\[1ex]
 Repeating Crossbow &19-20x2 &Piercing &120 ft. &1d3 &1d4 &1d6 &1d8 &1d10 &2d6\\[1ex]
 Sai &20x2 &Bludgeoning &Melee or 10 ft. &1d2 &1d3 &1d4 &1d6 &1d8 &1d10 \\
 Shuriken &20x2 &Piercing &10 ft &1 &1d2 &1d3 &1d4 &1d6 &1d8 \\[1ex]
 Siangham &20x2 &Piercing &Melee &1d3 &1d4 &1d6 &1d8 &1d10 &2d6 \\[1ex]
 Spiked Chain &20x2 &Piercing &Special &1d2 &1d3 &1d4 &1d6 &2d4 & \\[1ex]
 Urgrosh &20x3/\newline{}20x3 &Slashing/ Piercing &Melee &1d2/\newline{}1 &1d3/ 1d2 &1d4/ 1d3 &1d6/ 1d4 &1d8/ 1d6 &1d10/ 1d8 \\
 Whip &20x2 &Slashing &Special &- &1 &1d2 &1d3 &1d4 &1d6 \\[1ex]
 \hline
\end{longtabu}}
\end{small}
}

\subsection{Individual Weapon Rules}

In addition to the qualities given on the table, some weapons have additional rules, given below.

\textbf{Bastard Sword:} A character with exotic weapon proficiency can wield a bastard sword as if they were one size larger than they are.

\textbf{Bolas:} You can use this weapon to make a ranged trip attack against an opponent. You can't be tripped during your own trip attempt when using a set of bolas. As a thrown weapon, bolas must be one size smaller than you to be used effectively.

\textbf{Bow:} Bows are projectile weapons, the range given is for a medium sized bow. For every size category larger or smaller than medium, add or subtract 30 feet from the bows range. You need at least two hands to use a bow, regardless of its size. A bow the same size as you is too unwieldy to use while you are mounted. If you have a penalty for low Strength, apply it to damage rolls when you use a Bow. If you have a bonus for high Strength, you can apply it to damage rolls when you use a composite bow (see below) but not a regular bow.

\textbf{Composite Bow:} You need at least two hands to use a composite bow, regardless of its size. You can use a composite bow up to your size while mounted. All composite bows are made with a particular minimum strength rating (that is, each requires a minimum Strength score to use with proficiency). If your Strength score is less than the strength rating of the composite bow, you can't use it. The default composite longbow requires a Strength score of 10 or higher to use. A composite longbow can be made with a high strength rating to take advantage of an above-average Strength score; this feature allows you to add your Strength bonus to damage, as long as you meet the strength rating for the bow you can add either your Strength bonus, or the strength bonus that would be derived from the bows strength rating +4, to your damage rolls, whichever is lower.

\textbf{Crossbow:} Crossbows are ranged weapons that use bolts. The range listed for the crossbow is for one of medium size, for every size category larger or smaller than medium increase or decrease the range by 40 ft. Reloading a crossbow provokes an attack of opportunity, Reloading a light and one-handed crossbows is a move action, two-handed crossbows require a full round action to reload. Reloading a crossbow requires two hands.

\textbf{Dire Flail:} A dire flail is a double weapon. You can fight with it as if fighting with two weapons, but if you do, you incur all the normal attack penalties associated with fighting with two weapons, just as if you were using a one-handed weapon and a light weapon. A creature wielding a dire flail in one hand can't use it as a double weapon— only one end of the weapon can be used in any given round. When using a dire flail, you get a +2 bonus on attack rolls made to disarm an enemy. You can also use this weapon to make trip attacks. If you are tripped during your own trip attempt, you can drop the dire flail to avoid being tripped.

\textbf{Double Axe:} A double axe is a double weapon. You can fight with it as if fighting with two weapons, but if you do, you incur all the normal attack penalties associated with fighting with two weapons, just as if you were using a one-handed weapon and a light weapon. A creature wielding an orc double axe in one hand can't use it as a double weapon-only one end of the weapon can be used in any given round.

\textbf{Double Sword:} A double sword is a double weapon. You can fight with it as if fighting with two weapons, but if you do, you incur all the normal attack penalties associated with fighting with two weapons, just as if you were using a one-handed weapon and a light weapon. A creature wielding a two-bladed sword in one hand can't use it as a double weapon-only one end of the weapon can be used in any given round.

\textbf{Dwarven Axe:} A character with exotic proficiency with a Dwarven Axe can wield one as if they were one size category larger than they are. Dwarves only need martial proficiency with them to do this.

\textbf{Flail:} With a flail, you get a +2 bonus on attack rolls made to disarm an enemy. You can also use this weapon to make trip attacks. If you are tripped during your own trip attempt, you can drop the flail to avoid being tripped.

\textbf{Gauntlet:} This metal glove lets you deal lethal damage rather than nonlethal damage with unarmed strikes. A strike with a gauntlet is otherwise considered an unarmed attack. Medium and heavy armors (except breastplate) come with gauntlets. The damage listings given are for a gauntlet made for a creature of the indicated size, instead of fo a gauntlet of the indicated size. You may not wear gauntlets made for a creature of a different size than you.

\textbf{Glaive:} A glaive has reach. The glaives reach property can only be used when it is a two-handed weapon. You can strike opponents 10 feet away with it, but you can't use it against an adjacent foe.

\textbf{Guisarme:} A guisarme has reach. The guisarmes reach property can only be used when it is a two-handed weapon. You can strike opponents 10 feet away with it, but you can't use it against an adjacent foe. You can also use it to make trip attacks. If you are tripped during your own trip attempt, you can drop the guisarme to avoid being tripped.

\textbf{Halberd:} If you use a ready action to set a halberd against a charge, you deal double damage on a successful hit against a charging character. You can use a halberd to make trip attacks. If you are tripped during your own trip attempt, you can drop the halberd to avoid being tripped.

\textbf{Hook-Hammer:} A hook-hammer is a double weapon. You can fight with it as if fighting with two weapons, but if you do, you incur all the normal attack penalties associated with fighting with two weapons, just as if you were using a one-handed weapon and a light weapon. On a medium sized hook-hammer the hammer's blunt head is a bludgeoning weapon that deals 1d6 points of damage (crit ×3) and its hook is a piercing weapon that deals 1d4 points of damage (crit ×4). You can use either head as the primary weapon. The other head is the offhand weapon. A creature wielding a gnome hook-hammer in one hand can't use it as a double weapon-only one end of the weapon can be used in any given round. You can use a hook-hammer to make trip attacks. If you are tripped during your own trip attempt, you can drop the gnome hooked hammer to avoid being tripped. Gnomes treat hook-hammers as martial weapons.

\textbf{Kusarigama:} A kusarigama has reach, so you can strike opponents 10 feet away with it. The kusarigamas reach property can only be used when it is wielded in two hands (though not necessarily a two-handed weapon). In addition, unlike most other weapons with reach, it can be used against an adjacent foe. You can make trip attacks with the chain. If you are tripped during your own trip attempt, you can drop the chain to avoid being tripped. When using a spiked chain, you get a +2 bonus on opposed attack rolls made to disarm an opponent (including the roll to avoid being disarmed if such an attempt fails).

\textbf{Longspear:} A longspear has reach. The longspears reach property can only be used when it is a two-handed weapon. You can strike opponents 10 feet away with it, but you can't use it against an adjacent foe. If you use a ready action to set a longspear against a charge, you deal double damage on a successful hit against a charging character. While mounted, you can wield a lance with one hand. A longspear couched in a military saddle deals double damage on a charge.

\textbf{Net:} A net is a reach weapon used to entangle enemies. Unlike other reach weapons, a net the same size as you can be used with one hand. When you use a net, you make a ranged touch attack against your target. If you hit, the target is entangled. An entangled creature takes a -2 penalty on attack rolls and a -4 penalty on Dexterity, can move at only half speed, and cannot charge or run. If you control the trailing rope by succeeding on an opposed Strength check while holding it, the entangled creature can move only within the limits that the rope allows. If the entangled creature attempts to cast a spell, it must make a DC 15 Concentration check or be unable to cast the spell. An entangled creature can escape with a DC 20 Escape Artist check (a full-round action). The net has 5 hit points and can be burst with a DC 25 Strength check (also a full-round action). A net is useful only against creatures within one size category of you. A net must be folded to be thrown effectively. The first time you throw your net in a fight, you make a normal ranged touch attack roll. After the net is unfolded, you take a -4 penalty on attack rolls with it. It takes 2 rounds for a proficient user to fold a net and twice that long for a nonproficient one to do so.

\textbf{Nunchaku:} The nunchaku is a special monk weapon. This designation gives a monk wielding a nunchaku special options. With a nunchaku, you get a +2 bonus on attack rolls made to disarm an enemy. Nunchakus only count as monk weapons if they are light.

\textbf{Ranseur:} A ranseur has reach. The ranseurs reach property can only be used when it is a two-handed weapon. You can strike opponents 10 feet away with it, but you can't use it against an adjacent foe. With a ranseur, you get a +2 bonus on attack rolls made to disarm an opponent.

\textbf{Repeating Crossbow:} The repeating crossbow holds 5 crossbow bolts. As long as it holds bolts, you can reload it by pulling the reloading lever (a free action). Loading a new case of 5 bolts is a full-round action that provokes attacks of opportunity. A repeating crossbow functions identically to a crossbow in all other ways.

\textbf{Sai:} With a sai, you get a +4 bonus on opposed attack rolls made to disarm an enemy. The sai is a special monk weapon. This designation gives a monk wielding a sai special options. Sais only count as monk weapons if they are light.

\textbf{Shield:} You can bash with a shield instead of using it for defense. Doing so incurs all the normal penalties for two weapon fighting. Great Shields are one size smaller than the size of creature it was designed for, normal shields are two sizes smaller.

\textbf{Shuriken:} A shuriken is a special monk weapon. This designation gives a monk wielding shuriken special options. A shuriken can't be used as a melee weapon. Although they are thrown weapons, shuriken are treated as ammunition for the purposes of drawing them as long as they are three size categories smaller than you.

\textbf{Siangham:} The siangham is a special monk weapon. This designation gives a monk wielding a siangham special options. Siangham must be light to be used as a monk weapon.

\textbf{Sickle:} A sickle can be used to make trip attacks. If you are tripped during your own trip attempt, you can drop the sickle to avoid being tripped.

\textbf{Sling:} Your Strength modifier applies to damage rolls when you use a sling, just as it does for thrown weapons. You can fire, but not load, a sling the same size as you with one hand. Loading a sling is a move action that requires two hands and provokes attacks of opportunity. You can hurl ordinary stones with a sling, but stones are not as dense or as round as bullets. Thus, such an attack deals damage as if the weapon were designed for a creature one size category smaller than you and you take a -1 penalty on attack rolls. The range given is for a sling of medium size, for every size larger or smaller than medium increase or decrease the range by 15 feet.

\textbf{Spear:} If you use a ready action to set a spear against a charge, you deal double damage on a successful hit against a charging character. A spear one size smaller than you can be used as a thrown weapon with a 20 foot range incriment.

\textbf{Spiked Armor:} You can outfit your armor with spikes, which can deal damage in a grapple or as a separate attack. The damage listed is for armor made for a creature of the given size. Spiked armor is a light weapon.

\textbf{Spiked Chain:} A spiked chain has reach, so you can strike opponents 10 feet away with it. The spiked chains reach property can only be used when it is wielded in two hands (though not necessarily a two-handed weapon). In addition, unlike most other weapons with reach, it can be used against an adjacent foe. You can make trip attacks with the chain. If you are tripped during your own trip attempt, you can drop the chain to avoid being tripped. When using a spiked chain, you get a +2 bonus on opposed attack rolls made to disarm an opponent (including the roll to avoid being disarmed if such an attempt fails).

\textbf{Spiked Gauntlet:} Your opponent cannot use a disarm action to disarm you of spiked gauntlets. An attack with a spiked gauntlet is considered an armed attack. The damage listings given are for a spiked gauntlet made for a creature of the indicated size, instead of fo a spiked gauntlet of the indicated size. You may not wear gauntlets made for a creature of a different size than you.

\textbf{Spiked Shield:} You can bash with a spiked shield instead of using it for defense. If you use a ready action to set a spear against a charge, you deal double damage on a successful hit against a charging character.

\textbf{Staff:} A staff is a double weapon. You can fight with it as if fighting with two weapons, but if you do, you incur all the normal attack penalties associated with fighting with two weapons, just as if you were using a one-handed weapon and a light weapon. A creature wielding a quarterstaff in one hand can't use it as a double weapon-only one end of the weapon can be used in any given round. The quarterstaff is a special monk weapon. This designation gives a monk wielding a staff special options.

\textbf{Trident:} This weapon can be thrown as long as it is one size category smaller than you. If you use a ready action to set a trident against a charge, you deal double damage on a successful hit against a charging character.

\textbf{Unarmed Strike:} The damage listed for each size of unarmed strike is the size of the creature using unarmed strike. You can deal leathal or non-leathal damage at your option with an unarmed strike. The damage from an unarmed strike is considered weapon damage for the purposes of effects that give you a bonus on weapon damage rolls. An unarmed strike is always considered a light weapon.

\textbf{Urgrosh:} An urgrosh is a double weapon. You can fight with it as if fighting with two weapons, but if you do, you incur all the normal attack penalties associated with fighting with two weapons, just as if you were using a one-handed weapon and a light weapon. The urgrosh's axe head is a slashing weapon that deals 1d8 points of damage. Its spear head is a piercing weapon that deals 1d6 points of damage. You can use either head as the primary weapon. The other is the off-hand weapon. A creature wielding an urgrosh in one hand can't use it as a double weapon-only one end of the weapon can be used in any given round. If you use a ready action to set an urgrosh against a charge, you deal double damage if you score a hit against a charging character. If you use an urgrosh against a charging character, the spear head is the part of the weapon that deals damage. Dwarves treat urgroshes as martial weapons.

\textbf{Warhammer:} A warhammer has two sides that can be used interchangably. One side deals bludgeoning and has a critical range of 20x3, the other deals piercing damage and has a critical range of 20x4. As a medium weapon the hammer side deals 2d6 damage and the pick side deals 1d8 damage. You can choose which side you make an attack with at the beginning of each attack. It is not a double weapon, and cannot be weilded as one. Enhancements to the weapon effect both sides.

\textbf{Whip:} A whip has a 15 foot reach and can be used to attack any creature within range, including adjacent foes. The whips reach property can only be used when it is a one-handed or light weapon. A whip deals nonlethal damage. It also deals no damage to any creature with an armor bonus of +1 or higher, or a natural armor bonus of +3 or higher. Using a whip provokes an attack of opportunity as if you had used a ranged weapon. You cannot use a whip as a two-handed weapon. You can make trip attacks with a whip. If you are tripped during your own trip attempt, you can drop the whip to avoid being tripped. When using a whip, you get a +2 bonus on opposed attack rolls made to disarm an opponent..