\raceentry{ <-race name, capitalized-> }{`` <-fluff quote about race-> "}
\begin{multicols}{2}
<- intro blurb. maybe a sentence or two->
\racedescription{ <-physical description-> }
\racepersonality{ <-default personality description-> }
\racesociety{ <-default society description-> }
\racealignment{ <-default alignment notes. if you're into that sort of thing-> }

\columnbreak

\begin{racetable}
\type{ <-type and subtypes-> }
\size{ <-size-> }
\scores{+2 Wisdom, +2 Charisma}
\speed{ <-base speed, in feet-> }
\senses{ <-either "Standard" or something like "Low-light vision" or "Darkvision 120'"->
\racialtraits{
\racetrait{ <-name of racial trait. skill bonuses get a name too, like "Skill Bonus". use "\sp", "\su", or "\ex" after the name if needed-> }
{ <-racial trait mechanical description. it's less complicated than it sounds.-> }
} %%% repeat the whole '\racetrait' line as needed for extra abilities %%%
\autolanguages{ <-automatic languages. "Common" and a racial language, if any, are probably here-> }
\bonuslanguages{ <-bonus language options-> }
\favoredclasses{ <-favored classes. plural-> }
\end{racetable}

\vspace{\baselineskip}
\agetable{ <-age of adulthood-> }
{ <-simple class extra years-> }
{ <-moderate class extra years-> }
{ <-complex class extra years-> }

\vspace{\baselineskip}
\begin{heightweighttable}
\male{ <-base height, in [#' #"] format, -> }
{ <-height mod. no units-> }
{ <-base weight, in [# lb.] format-> }
{ <-weight mod. no units-> }
\female{ <-base height, in [#' #"] format, -> }
{ <-height mod. no units-> }
{ <-base weight, in [# lb.] format-> }
{ <-weight mod. no units-> }
\end{heightweighttable}
\end{multicols}