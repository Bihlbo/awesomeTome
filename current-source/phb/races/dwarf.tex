\raceentry{Dwarf}{``I remember that..."}
\begin{multicols}{2}
The great underground kingdoms of dwarven warriors and mystics protect the birthright of the dwarven race: The Stone Ledger. Deep in the mountains, the dwarf people have records that go back to when most of the other races were learning about fire. Second only to the Aboleth themselves, the racial memory of the dwarves extends to days beyond reckoning. The dwarves are known to the other races by their martial ability and toughness, and by the seemingly miraculous connection they have with the arcane secrets of ancient artifacts, magical treasures, and the world below.
\racedescription{Dwarves are distinctively short and broad when compared to most other races. Standing less than five feet tall marks them as tempting targets for larger races, but their hale and stocky build gives them a bulky presence that rivals even large humans. Dwarf women tend to be slighter and shorter than dwarf men. The hair of a dwarf is usually of a dark hue and worn long, their skin ranging from pale to tan. The dwarven man's beard is typically a source of price, and is groomed and adorned as such. Dwarven fashion, decor, and craft highlight the simple, straightforward nature of their culture. Dwarves are considered adults at about age 40, and they can live to be more than 400 years old.}
\racepersonality{Second only to elves in longevity, a dwarf's outlook on life is long and patient. Dwarves appreciate common ground they share with others, and that which is predictable, ritual, and traditional. The loyalty of a dwarven ally is as reliable as their stonecraft. However, since every group has at one time or another been at war with any other race you care to name, and a dwarf is likely to have studied enough to have something against anyone they meet, the loyalty of a dwarf is hard-won and may cost generations of effort. However, true hatred is also hard to earn from a dwarf, who tends to see everything in light of history and understands that sometimes conflict must come and will eventually end in a return to peace. A courageous and determined people, the dwarves are methodical and efficient in war, ever thinking many steps ahead to ensure the safety of their society, their records, and their wealth (which they are zealously driven to increase).}
\racesociety{Dwarven society has seen very little change over the thousands of years it has existed. The main reason for this is that dwarves constantly look to, revere, and study the past; because they can, thanks to the Stone Ledger which measures in exact terms the location of all the great things that the Dwarven people have found, gives tips for dealing with problems that dwarves have overcome, and records in excruciating detail every bad thing that anyone has ever done to the dwarven race. Social norms, language, superstitions, trade agreements, magical knowledge, philosophy and morality, crafting practices, and most imaginable elements of a dwarf's life have been recorded, studied, and maintained throughout well-counted and documented generations. After spending decades studying the history and traditions of the dwarf people, few seek to change those traditions, and those who do usually find a much warmer welcome for their innovative ideas among other races. Therefore, dwarven society is orderly, predictable, and generally safe for those who could be considered a part of that society. The long memory of the dwarves means they are one of the few races who still maintain an omnipresent religious tradition for the god who created them.}
\racealignment{Dwarven society tends to enforce and embrace a more lawful disposition, and holds those who strive for goodness in high regard. Dwarves who are evil are often cast out, and those rare dwarves who seek a less structured and traditional life usually find it by leaving the lands of their people. Dwarven adventurers therefore run the gamut, and if for no other reason than familiarity, usually get along well with lawful and good characters.}
\columnbreak

\begin{racetable}
\type{Humanoid (Dwarf subtype)}
\size{Medium}
\scores{+2 Constitution, --2 Charisma}
\speed{20}
\senses{"Darkvision 60'"}
\racialtraits{
\racetrait{Dwarves can move up to their full speed even when wearing medium or heavy armor or when carrying a medium or heavy load.}
\racetrait{Stonecunning}{This ability grants a dwarf a +2 racial bonus on Search checks to notice unusual stonework, such as sliding walls, stonework traps, new construction (even when built to match the old), unsafe stone surfaces, shaky stone ceilings, and the like. Something that isn’t stone but that is disguised as stone also counts as unusual stonework. A dwarf who merely comes within 10 feet of unusual stonework can make a Search check as if he or she were actively searching, and a dwarf can use the Search skill to find stonework traps as a rogue can. A dwarf can also intuit depth, sensing his or her approximate depth underground as naturally as a human can sense which way is up.}
\racetrait{Weapon Familiarity}{Dwarves may treat dwarven waraxes and dwarven urgroshes as martial weapons, rather than exotic weapons.}
\racetrait{Stability}{A dwarf gains a +4 bonus on ability checks made to resist being bull rushed or tripped when standing on the ground (but not when climbing, flying, riding, or otherwise not standing firmly on the ground).}
\racetrait{Save Bonus}{+2 racial bonus on saving throws against poison.}
\racetrait{Save Bonus}{+2 racial bonus on saving throws against spells and spell-like effects.}
\racetrait{Attack Bonus}{+1 racial bonus on attack rolls against orcs and goblinoids.}
\racetrait{AC Bonus}{+4 dodge bonus to Armor Class against monsters of the giant type.}
\racetrait{Skill Bonus}{+2 racial bonus on Appraise checks that are related to stone or metal items.}
\racetrait{Skill Bonus}{+2 racial bonus on Craft checks that are related to stone or metal.}
\autolanguages{Common, Dwarven}
\bonuslanguages{Giant, Gnome, Goblin, Orc, Terran, and Undercommon}
\favoredclasses{Fighter}
\end{racetable}

\vspace{\baselineskip}
\agetable{40}{+3d6}{+5d6}{+7d6}

\vspace{\baselineskip}
\begin{heightweighttable}
\male{3' 9"}{+2d4}{130 lb.}{x(2d6)}
\female{3' 7"}{+2d4}{100 lb.}{x(2d6)}
\end{heightweighttable}
\end{multicols}
\end{racebox}