\section{Movement, Position, and Distance}

Miniatures are on the 30mm scale - a miniature figure of a six\textendash foot tall human is approximately 30mm tall. A square on the battle grid is 1 inch across, representing a 5\textendash foot by 5\textendash foot area.

\subsection{Tactical Movement}

\subsubsection{How Far Can Your Character Move?}

Your speed is determined by your race and your armor (see Table: Tactical Speed). Your speed while unarmored is your base land speed.

\vspace*{10pt}

\ability{Encumbrance:}A character encumbered by carrying a large amount of gear, treasure, or fallen comrades may move slower than normal.  

\ability{Hampered Movement:}Difficult terrain, obstacles, or poor visibility can hamper movement.

\ability{Movement in Combat:}Generally, you can move your speed in a round and still do something (take a move action and a standard action).
If you do nothing but move (that is, if you use both of your actions in a round to move your speed), you can move double your speed.
If you spend the entire round running, you can move quadruple your speed. If you do something that requires a full round you can only take a 5\textendash foot step.

%This is not the original table from the SRD, but instead a better one partly lifted out of the Rules Compendium
\begin{wraptable}{o}{2in}
\caption{Reduced Speed}
\begin{tabular}[h]{c|c}
Base Speed & Reduced Speed \\ \hline
20 ft. & 15 ft. \\
30 ft. & 20 ft. \\
40 ft. & 30 ft. \\
50 ft. & 35 ft. \\
60 ft. & 40 ft. \\
70 ft. & 50 ft. \\
80 ft. & 55 ft. \\
90 ft. & 60 ft. \\
100 ft. & 70 ft. \\
\end{tabular}
\end{wraptable}

\ability{Bonuses to Speed:}Some class features or magical items may grant bonuses to a characters speed. Always apply any modifiers to a character's speed before adjusting the character's speed based on armor or encumbrance, and remember that multiple bonuses of the same type to a character's speed don't stack.

\ability{Reduced Speed:}Being encumbered up to a medium or heavy load or wearing medium or heavy armor reduces your speed.  The table below gives the reduced speeds for different base land speeds.


\subsubsection{Measuring Distance}

\ability{Diagonals:}When measuring distance, the first diagonal counts as 1 square, the second counts as 2 squares, the third counts as 1, the fourth as 2, and so on. You can't move diagonally past a corner (even by taking a 5\textendash foot step). You can move diagonally past a creature, even an opponent. You can also move diagonally past other impassable obstacles, such as pits.

\ability{Closest Creature:}When it's important to determine the closest square or creature to a location, if two squares or creatures are equally close, randomly determine which one counts as closest by rolling a die.

\subsubsection{Moving through a Square}

\ability{Friend:}You can move through a square occupied by a friendly character, unless you are charging. When you move through a square occupied by a friendly character, that character doesn't provide you with cover.

\ability{Opponent:}You can't move through a square occupied by an opponent, unless the opponent is helpless. You can move through a square occupied by a helpless opponent without penalty. (Some creatures, particularly very large ones, may present an obstacle even when helpless. In such cases, each square you move through counts as 2 squares.)

\ability{Ending Your Movement:}You can't end your movement in the same square as another creature unless it is helpless.

\ability{Overrun:}During your movement you can attempt to move through a square occupied by an opponent.

\ability{Tumbling:}A trained character can attempt to tumble through a square occupied by an opponent (see the Tumble skill).

\ability{Very Small Creature:}A Fine, Diminutive, or Tiny creature can move into or through an occupied square. The creature provokes attacks of opportunity when doing so.
Square Occupied by Creature Three Sizes Larger or Smaller: Any creature can move through a square occupied by a creature three size categories larger than it is. A big creature can move through a square occupied by a creature three size categories smaller than it is.

\ability{Designated Exceptions:}Some creatures break the above rules. A creature that completely fills the squares it occupies cannot be moved past, even with the Tumble skill or similar special abilities.

\subsubsection{Terrain and Obstacles}

\ability{Difficult Terrain:}Difficult terrain hampers movement. Each square of difficult terrain counts as 2 squares of movement. (Each diagonal move into a difficult terrain square counts as 3 squares.) You can't run or charge across difficult terrain. If you occupy squares with different kinds of terrain, you can move only as fast as the most difficult terrain you occupy will allow. Flying and incorporeal creatures are not hampered by difficult terrain.

\ability{Obstacles:}Like difficult terrain, obstacles can hamper movement. If an obstacle hampers movement but doesn't completely block it each obstructed square or obstacle between squares counts as 2 squares of movement. You must pay this cost to cross the barrier, in addition to the cost to move into the square on the other side. If you don't have sufficient movement to cross the barrier and move into the square on the other side, you can't cross the barrier. Some obstacles may also require a skill check to cross. On the other hand, some obstacles block movement entirely. A character can't move through a blocking obstacle. Flying and incorporeal creatures can avoid most obstacles

\ability{Squeezing:}In some cases, you may have to squeeze into or through an area that isn't as wide as the space you take up. You can squeeze through or into a space that is at least half as wide as your normal space. Each move into or through a narrow space counts as if it were 2 squares, and while squeezed in a narrow space you take a \textendash4 penalty on attack rolls and a \textendash4 penalty to AC. When a Large creature (which normally takes up four squares) squeezes into a space that's one square wide, the creature's miniature figure occupies two squares, centered on the line between the two squares. For a bigger creature, center the creature likewise in the area it squeezes into. A creature can squeeze past an opponent while moving but it can't end its movement in an occupied square. To squeeze through or into a space less than half your space's width, you must use the Escape Artist skill. You can't attack while using Escape Artist to squeeze through or into a narrow space, you take a \textendash4 penalty to AC, and you lose any Dexterity bonus to AC.

\subsubsection{Special Movement Rules}

These rules cover special movement situations.

\vspace*{10pt}

\ability{Accidentally Ending Movement in an Illegal Space:}Sometimes a character ends its movement while moving through a space where it's not allowed to stop. When that happens, put your miniature in the last legal position you occupied, or the closest legal position, if there's a legal position that's closer.

\ability{Double Movement Cost:}When your movement is hampered in some way, your movement usually costs double. For example, each square of movement through difficult terrain counts as 2 squares, and each diagonal move through such terrain counts as 3 squares (just as two diagonal moves normally do). If movement cost is doubled twice, then each square counts as 4 squares (or as 6 squares if moving diagonally). If movement cost is doubled three times, then each square counts as 8 squares (12 if diagonal) and so on. This is an exception to the general rule that two doublings are equivalent to a tripling.

\ability{Minimum Movement:}Despite penalties to movement, you can take a full\textendash round action to move 5 feet (1 square) in any direction, even diagonally. (This rule doesn't allow you to move through impassable terrain or to move when all movement is prohibited.) Such movement provokes attacks of opportunity as normal (despite the distance covered, this move isn't a 5\textendash foot step).

\subsection{Big and Little Creatures in Combat}

Creatures smaller than Small or larger than Medium have special rules relating to position. 

\vspace*{10pt}

\begin{wraptable}{O}{3in}
\caption{Creature Size and Scale}
\begin{tabular}[h]{c|c|c}
Creature Size & Space\textsuperscript{1} &	Natural Reach\textsuperscript{1} \\ \hline
Fine & 1/2 ft. & 0 \\
Diminutive & 1 ft. & 0 \\
Tiny & 2\textendash1/2 ft. & 0 \\
Small & 5 ft. & 5 ft.	\\
Medium & 5 ft. & 5 ft. \\
Large (tall) & 10 ft. & 10 ft. \\
Large (long) & 10 ft. & 5 ft. \\
Huge (tall) & 15 ft. & 15 ft. \\
Huge (long) & 15 ft. & 10 ft. \\
Gargantuan (tall) & 20 ft. & 20 ft. \\
Gargantuan (long) & 20 ft. & 15 ft.	\\
Colossal (tall) & 30 ft. & 30 ft. \\
Colossal (long) & 30 ft. & 20 ft. \\ \hline
\multicolumn{3}{p{3in}}{\textsuperscript{1} These values are typical for creatures of the indicated size.}
\end{tabular}
\end{wraptable}

\ability{Tiny, Diminutive, and Fine Creatures:} Very small creatures take up less than 1 square of space. This means that more than one such creature can fit into a single square. A Tiny creature typically occupies a space only 2 \half feet across, so four can fit into a single square. Twenty\textendash five Diminutive creatures or 100 Fine creatures can fit into a single square. Creatures that take up less than 1 square of space typically have a natural reach of 0 feet, meaning they can't reach into adjacent squares. They must enter an opponent's square to attack in melee. This provokes an attack of opportunity from the opponent. You can attack into your own square if you need to, so you can attack such creatures normally. Since they have no natural reach, they do not threaten the squares around them. You can move past them without provoking attacks of opportunity. They also can't flank an enemy.

\ability{Large, Huge, Gargantuan, and Colossal Creatures:} Very large creatures take up more than 1 square. Creatures that take up more than 1 square typically have a natural reach of 10 feet or more, meaning that they can reach targets even if they aren't in adjacent squares. Unlike when someone uses a reach weapon, a creature with greater than normal natural reach (more than 5 feet) still threatens squares adjacent to it. A creature with greater than normal natural reach usually gets an attack of opportunity against you if you approach it, because you must enter and move within the range of its reach before you can attack it. (This attack of opportunity is not provoked if you take a 5\textendash foot step.) Large or larger creatures using reach weapons can strike up to double their natural reach but can't strike at their natural reach or less.

\pagebreak

\section{Combat Modifiers}

\subsection{Favorable and Unfavorable Conditions}

\begin{multicols}{2}

\begin{tabular}[h]{l|cc}
\multicolumn{3}{c}{\textbf{Table: Armor Class Modifiers}} \\
Defender is \ldots & Melee & Ranged	\\ \hline
Behind cover & +4 & +4 \\	   
Blinded & \textendash2\textsuperscript{1} & \textendash2\textsuperscript{1} \\
Concealed or invisible & \multicolumn{2}{c}{� See Concealment �} \\
Cowering & \textendash2\textsuperscript{1} & \textendash2\textsuperscript{1} \\
Entangled & +0\textsuperscript{2} & +0\textsuperscript{2} \\
Flat\textendash footed & +0\textsuperscript{1} & +0\textsuperscript{1} \\
Grappling (attacker is not) & +0\textsuperscript{1} & +0\textsuperscript{1, 3} \\
Helpless & \textendash4\textsuperscript{4} & +0\textsuperscript{4} \\
Kneeling or sitting & \textendash2 & +2 \\
Pinned & \textendash4\textsuperscript{4} & +0\textsuperscript{4} \\
Prone  & \textendash4 & +4 \\
Squeezing through a space & \textendash4 & \textendash4 \\
Stunned & \textendash2\textsuperscript{1} & \textendash2\textsuperscript{1} \\ \hline
\multicolumn{3}{p{3in}}{\textsuperscript{1} The defender loses any Dexterity bonus to AC.} \\
\multicolumn{3}{p{3in}}{\textsuperscript{2} An entangled character takes a \textendash4 penalty to Dexterity.} \\
\multicolumn{3}{p{3in}}{\textsuperscript{3} Roll randomly to see which grappling combatant you strike. That defender loses any Dexterity bonus to AC.} \\
\multicolumn{3}{p{3in}}{\textsuperscript{4} Treat the defender's Dexterity as 0 (\textendash5 modifier). Rogues can sneak attack helpless or pinned defenders.} \\
\end{tabular}

\begin{tabular}[h]{l|cc} 
\multicolumn{3}{c}{\textbf{Table: Attack Roll Modifiers}} \\
Attacker is \ldots & Melee & Ranged \\ \hline	   
Dazzled & \textendash1 & \textendash1 \\	   
Entangled & \textendash2\textsuperscript{1} &	\textendash2\textsuperscript{1} \\	   
Flanking defender & +2 & \textendash \\	   
Invisible & +2\textsuperscript{2} & +2\textsuperscript{2} \\  
On higher ground & +1 & +0 \\	   
Prone & \textendash4 & \textendash3 \\   
Shaken or frightened & \textendash2 & \textendash2 \\	   
Squeezing through a space & \textendash4 & \textendash4 \\ \hline
\multicolumn{3}{p{3in}}{\textsuperscript{1} An entangled character also takes a \textendash4 penalty to Dexterity, which may affect his attack roll.} \\
\multicolumn{3}{p{3in}}{\textsuperscript{2} The defender loses any Dexterity bonus to AC. This bonus doesn't apply if the target is blinded.} \\
\multicolumn{3}{p{3in}}{\textsuperscript{3} Most ranged weapons can't be used while the attacker is prone, but you can use a crossbow or shuriken while prone at no penalty.} \\
\end{tabular}

\end{multicols}

\subsection{Cover}

To determine whether your target has cover from your ranged attack, choose a corner of your square. If any line from this corner to any corner of the target's square passes through a square or border that blocks line of effect or provides cover, or through a square occupied by a creature, the target has cover (+4 to AC).
When making a melee attack against an adjacent target, your target has cover if any line from your square to the target's square goes through a wall (including a low wall). When making a melee attack against a target that isn't adjacent to you (such as with a reach weapon), use the rules for determining cover from ranged attacks.
Low Obstacles and Cover: A low obstacle (such as a wall no higher than half your height) provides cover, but only to creatures within 30 feet (6 squares) of it. The attacker can ignore the cover if he's closer to the obstacle than his target.

\vspace*{10pt}

\ability{Cover and Attacks of Opportunity:}You can't execute an attack of opportunity against an opponent with cover relative to you.

\ability{Cover and Reflex Saves:}Cover grants you a +2 bonus on Reflex saves against attacks that originate or burst out from a point on the other side of the cover from you. Note that spread effects can extend around corners and thus negate this cover bonus.

\ability{Cover and Hide Checks:}You can use cover to make a Hide check. Without cover, you usually need concealment (see below) to make a Hide check.

\ability{Soft Cover:}Creatures, even your enemies, can provide you with cover against ranged attacks, giving you a +4 bonus to AC. However, such soft cover provides no bonus on Reflex saves, nor does soft cover allow you to make a Hide check.

\ability{Big Creatures and Cover:}Any creature with a space larger than 5 feet (1 square) determines cover against melee attacks slightly differently than smaller creatures do. Such a creature can choose any square that it occupies to determine if an opponent has cover against its melee attacks. Similarly, when making a melee attack against such a creature, you can pick any of the squares it occupies to determine if it has cover against you.

\ability{Total Cover:}If you don't have line of effect to your target he is considered to have total cover from you. You can't make an attack against a target that has total cover.

\ability{Varying Degrees of Cover:}In some cases, cover may provide a greater bonus to AC and Reflex saves. In such situations the normal cover bonuses to AC and Reflex saves can be doubled (to +8 and +4, respectively). A creature with this improved cover effectively gains improved evasion against any attack to which the Reflex save bonus applies. Furthermore, improved cover provides a +10 bonus on Hide checks.

\subsection{Concealment}

To determine whether your target has concealment from your ranged attack, choose a corner of your square. If any line from this corner to any corner of the target's square passes through a square or border that provides concealment, the target has concealment. When making a melee attack against an adjacent target, your target has concealment if his space is entirely within an effect that grants concealment. When making a melee attack against a target that isn't adjacent to you use the rules for determining concealment from ranged attacks. In addition, some magical effects provide concealment against all attacks, regardless of whether any intervening concealment exists.

\vspace*{10pt}

\ability{Concealment Miss Chance:}Concealment gives the subject of a successful attack a 20\% chance that the attacker missed because of the concealment. If the attacker hits, the defender must make a miss chance percentile roll to avoid being struck. Multiple concealment conditions do not stack.

\ability{Concealment and Hide Checks:}You can use concealment to make a Hide check. Without concealment, you usually need cover to make a Hide check.

\ability{Total Concealment:}If you have line of effect to a target but not line of sight he is considered to have total concealment from you. You can�t attack an opponent that has total concealment, though you can attack into a square that you think he occupies. A successful attack into a square occupied by an enemy with total concealment has a 50\% miss chance (instead of the normal 20\% miss chance for an opponent with concealment). You can�t execute an attack of opportunity against an opponent with total concealment, even if you know what square or squares the opponent occupies.

\ability{Ignoring Concealment:}Concealment isn't always effective. A shadowy area or darkness doesn't provide any concealment against an opponent with darkvision. Characters with low\textendash light vision can see clearly for a greater distance with the same light source than other characters. Although invisibility provides total concealment, sighted opponents may still make Spot checks to notice the location of an invisible character. An invisible character gains a +20 bonus on Hide checks if moving, or a +40 bonus on Hide checks when not moving (even though opponents can't see you, they might be able to figure out where you are from other visual clues).

\ability{Varying Degrees of Concealment:}Certain situations may provide more or less than typical concealment, and modify the miss chance accordingly.

\subsection{Flanking}

When making a melee attack, you get a +2 flanking bonus if your opponent is threatened by a character or creature friendly to you on the opponent's opposite border or opposite corner. When in doubt about whether two friendly characters flank an opponent in the middle, trace an imaginary line between the two friendly characters' centers. If the line passes through opposite borders of the opponent's space (including corners of those borders), then the opponent is flanked. 
\textbf{Exception:} If a flanker takes up more than 1 square, it gets the flanking bonus if any square it occupies counts for flanking. Only a creature or character that threatens the defender can help an attacker get a flanking bonus. Creatures with a reach of 0 feet can't flank an opponent.

\subsection{Helpless Defenders}

A helpless opponent is someone who is bound, sleeping, paralyzed, unconscious, or otherwise at your mercy.

\vspace*{10pt}

\ability{Regular Attack:}A helpless character takes a \textendash4 penalty to AC against melee attacks, but no penalty to AC against ranged attacks.
A helpless defender can't use any Dexterity bonus to AC. In fact, his Dexterity score is treated as if it were 0 and his Dexterity modifier to AC as if it were \textendash5 (and a rogue can sneak attack him).

\ability{Coup de Grace:}As a full\textendash round action, you can use a melee weapon to deliver a coup de grace to a helpless opponent. You can also use a bow or crossbow, provided you are adjacent to the target. You automatically hit and score a critical hit. If the defender survives the damage, he must make a Fortitude save (DC 10 + damage dealt) or die. A rogue also gets her extra sneak attack damage against a helpless opponent when delivering a coup de grace. Delivering a coup de grace may provoke attacks of opportunity from threatening opponents. You can't deliver a coup de grace against a creature that is immune to critical hits. You can deliver a coup de grace against a creature with total concealment, but doing this requires two consecutive full\textendash round actions (one to ``find'' the creature once you've determined what square it's in, and one to deliver the coup de grace).

\subsection{Attack Options}

Characters have a number of options when they attack their opponents. Expertise and Power Attack can be used on any attacks.

\listone\hypertarget{combat:expertise}{}
\item \textbf{{Expertise}}\\
\emph{You leverage your combat skill into defense rather than offense.}\\
\shortability{Requirement:}{You must make an attack action and have a BAB of at least +1. You need not specifically attack an enemy.}
\shortability{Effect:}{Before making an attack roll, you may take an attack penalty of up to your BAB on this attack and all further attacks until your next turn, and gain an equal Dodge Bonus to AC. You may only use this option once per turn.}\\

\hypertarget{combat:powerattack}{}
\item\textbf{{Power Attack}}\\
\emph{You leverage your combat skill into devastating attacks at the expense of accuracy.}\\
\shortability{Requirement:}{You must make an attack action and have a BAB of at least +1.}
\shortability{Effect:}{Before making an attack roll, you may voluntarily take an attack penalty of up to your BAB, and inflict two times that amount in extra damage with that attack. You may take this option on any or all of your attacks if you wish.}
\end{list}

\section{Special Combat Actions}

The following are special actions that can be performed in combat.\\

\subsection{Combat Maneuvers}
 
%Table: Special Attacks	   
%Special Attack 	Brief Description	   
%Aid another 	Grant an ally a +2 bonus on attacks or AC	   
%Bull rush 	Push an opponent back 5 feet or more	   
%Charge 	Move up to twice your speed and attack with +2 bonus	   
%Disarm 	Knock a weapon from your opponent's hands	   
%Feint 	Negate your opponent's Dex bonus to AC	   
%Grapple 	Wrestle with an opponent	   
%Overrun 	Plow past or over an opponent as you move	   
%Sunder 	Strike an opponent's weapon or shield	   
%Throw splash weapon 	Throw container of dangerous liquid at target	   
%Trip 	Trip an opponent	   
%Turn (rebuke) undead 	Channel positive (or negative) energy to turn away (or awe) undead	   
%Two\textendash weapon fighting 	Fight with a weapon in each hand	 

\ability{Having the Edge:}If you have more BAB than the target of your attacks, you are considered to ``Have the Edge'' on that attack. Some combat maneuvers will perform better when used by someone with the Edge.

\listone

\hypertarget{combat:aidanother}{}
\normalsize\item\textbf{{Aid Another}}\\ \small
In melee combat, you can help a friend attack or defend by distracting or interfering with an opponent. If you're in position to make a melee attack on an opponent that is engaging a friend in melee combat, you can attempt to aid your friend as a standard action. You make an attack roll against AC 10. If you succeed, your friend gains either a +2 bonus on his next attack roll against that opponent or a +2 bonus to AC against that opponent's next attack (your choice), as long as that attack comes before the beginning of your next turn. Multiple characters can aid the same friend, and similar bonuses stack. You can also use this standard action to help a friend in other ways, such as when he is affected by a spell, or to assist another character's skill check.\\

\hypertarget{combat:bullrush}{}
\normalsize\item\textbf{{Bullrush}}\\ \small
If you have not moved your entire allotted distance this turn, you may attempt to push your opponent back as a melee attack. First, you move into your opponent's square (which probably provokes an attack of opportunity, see movement). Then you make an opposed size-modified strength check against a DC of 10 + the target's Strength modifier + the target's size modifier (you do not have to roll to hit). If you succeed, you push your opponent back 5 feet. If you succeed by more than 1, you may move your opponent back a single 5' square for every 2 points your check exceeds the DC.

\ability{Modifiers:} The Size Modifier to both the Bullrush check and the DC is +4 for every size larger than medium and -4 for every size smaller than medium.

\ability{Special:} The movement used during a Bullrush counts against your movement this turn. If you do not take a move or charge action this turn, you will normally be limited to five feet of movement. This movement does not provoke an attack of opportunity from you or the target, but is quite likely to provoke an attack of opportunity from any other creature standing nearby. During a bullrush, both characters provide cover for each other.

\smallskip\emph{\underline{Edge Option:} If you have the edge on your target, you do not provide cover for your opponent even if they are the same size as you. Further, you may move your opponent in a direction up to 45 degrees off from your initial approach, altering your own course to push them more than 5 feet if necessary. If you fail the initial strength check, you may choose which adjacent square you are pushed into.}\\

\hypertarget{combat:charge}{}
\normalsize\item\textbf{{Charge}}\\ \small
Charging is a special full\textendash round action that allows you to move up to twice your speed and attack during the action. However, it carries tight restrictions on how you can move.

\ability{Movement During a Charge}{You must move before your attack, not after. You must move at least 10 feet (2 squares) and may move up to double your speed directly toward the designated opponent. You must have a clear path toward the opponent, and nothing can hinder your movement (such as difficult terrain or obstacles). Here's what it means to have a clear path. First, you must move to the closest space from which you can attack the opponent. (If this space is occupied or otherwise blocked, you can't charge.) Second, if any line from your starting space to the ending space passes through a square that blocks movement, slows movement, or contains a creature (even an ally), you can't charge. (Helpless creatures don't stop a charge.) If you don't have line of sight to the opponent at the start of your turn, you can't charge that opponent. You can't take a 5\textendash foot step in the same round as a charge.

If you are able to take only a standard action or a move action on your turn, you can still charge, but you are only allowed to move up to your speed (instead of up to double your speed). You can't use this option unless you are restricted to taking only a standard action or move action on your turn.}

\ability{Attacking on a Charge}{After moving, you may make a single melee attack. You get a +2 bonus on the attack roll. and take a \textendash2 penalty to your AC until the start of your next turn.A charging character gets a +2 bonus on the Strength check made to bull rush an opponent (see Bull Rush, above). Even if you have extra attacks, such as from having a high enough base attack bonus or from using multiple weapons, you only get to make one attack during a charge.}

\ability{Lances and Charge Attacks:}{	A lance deals double damage if employed by a mounted character in a charge.}
	
\ability{Weapons Readied against a Charge :}{Spears, tridents, and certain other piercing weapons deal double damage when readied (set) and used against a charging character.}\\

\hypertarget{combat:coupdegrace}{}
\normalsize\item\textbf{{Coup de Grace}}\\\small
You may attempt to slay an opponent outright if they are helpless. As a full-round action, you may automatically hit a helpless opponent in melee range. This attack is automatically a critical hit. This action provokes an attack of opportunity.

\ability{Interrupting a Coup de Grace:}{A character who suffers damage during the Coup de Grace must make a Concentration Check (DC 10 + Damage Inflicted) or the action is resolved as a normal attack.}

\smallskip\emph{\underline{Edge Option:} If you have the Edge on an opponent who threatens you during a Coup de Grace, you do not provoke an attack of opportunity from them.}\\

\hypertarget{combat:coveringfire}{}
\normalsize\item\textbf{{Covering Fire}}\\\small
You may use your ranged attacks to provide cover for your allies. Take an attack with your ranged weapon and roll a normal attack roll. Until the beginning of your next turn one of your allies may use the result of your attack roll as their Armor Class against one attack of opportunity.

\smallskip\emph{\underline{Edge Option:} If you have The Edge against an opponent whose attack of opportunity was negated by Covering Fire, your ranged weapon may hit them. Simply compare the attack roll to their armor class as if it was also a normal attack.}\\

\hypertarget{combat:disarm}{}
\normalsize\item\textbf{{Disarm}}\\\small
You may attempt to disarm your opponent with a melee attack. Disarm is a special attack action. Make an attack roll against an ``armor class" of 10 + the target's melee attack bonuses with the item in question. If you succeed, one weapon or held item is snatched out of your opponent's grasp. Failing a Disarm attempt provokes an attack of opportunity from the target. A disarmed item lands in a randomly determined square adjacent to the target.

\ability{Grabbing Items:}{You can use a disarm action to snatch an item worn by the target. If you want to have the item in your hand, the disarm must be made as an unarmed attack. If the item is poorly secured or otherwise easy to snatch or cut away the attacker gets a +4 bonus. Unlike on a normal disarm attempt, failing the attempt doesn't allow the defender to attempt to disarm you. This otherwise functions identically to a disarm attempt, as noted above. You can't snatch an item that is well secured unless you have pinned the wearer (see Grapple). Even then, the defender gains a +4 bonus on his roll to resist the attempt.}

\ability{Defending against a Disarm:}{An item held in two hands is harder to disarm, increasing the DC by +4. An item tied to one's body with a sword-wrap or locked gauntlet is much harder to disarm, increasing the DC by +8.}

\ability{Special:}{A Disarm may be used to attempt to remove a weapon that is presently being used in an attack against the disarmer even if the creature using the weapon is out of range or otherwise not threatened by the character. A Disarm (or any attack) is normally only usable during an attack against such creatures as an Attack of Opportunity or a Readied Action.}

\smallskip\emph{\underline{Edge Option:} If you have the Edge on your target, your Disarm attempt does not provoke an attack of opportunity, and you may choose which adjacent square your opponent's weapon or held item lands in. If you have a free hand, the item may end up in your possession instead.}\\


\hypertarget{combat:feint}{}
\normalsize\item\textbf{{Feint}}\\\small
By performing a distracting maneuver or fencing your opponent into a poor position, you may make an attack against them at their worst. You take an attack action to make a Bluff check with a DC of 10 + your opponent's Wisdom modifier + the higher of your opponent's BAB or ranks in Sense Motive. If you succeed, your opponent does not get their Dexterity Bonus to AC against the next attack you make against them (if it is within the next round).

\smallskip\emph{\underline{Edge Option:} If you have the Edge on your target and you successfully Feint, you may make an attack against that opponent this round as a Swift action.}\\

\hypertarget{combat:grapple}{}
\normalsize\item\textbf{{Grapple}}\\\small
Grapple is collectively 3 separate maneuvers that all fall under the super-heading of ``grappling". Any grapple attempt provokes an attack of opportunity unless your attack has the edge.

\listtwo\hypertarget{combat:grabon}{}
      \normalsize\item\textbf{{Grab On}}\\\small
      Sometimes, you want to attach yourself to a larger creature, getting inside their reach and then repeatedly stabbing them or simply weighing them down.  As an attack action you may attempt to grab on to an opponent.

      Grabbing on to an opponent provokes an attack of opportunity and requires a check with the same bonuses as a melee attack. The DC to grab on to an opponent is their Touch AC plus their BAB. If you have 5 ranks of Climb or Ride, you get a +2 synergy bonus on this maneuver for each skill.

      \ability{Holding on:}{Once you've attached yourself to your opponent, you go wherever they go. Move in to their space, and move where they do automatically (this movement does not provoke attacks of opportunity or count against your movement in any way). You may attack with any light or one handed weapon, and your opponent is denied his Dexterity bonus against you.}

      \ability{Being Held on to:}{If another creature has grabbed on to your character, their weight counts against your carrying capacity. If you're overloaded, you may be unable to move or even collapse until you shake your opponent off. You can attempt to attack a creature holding on to you, but your strength modifier is halved for such attacks and your attacks are at -4. You may attempt to shake your opponent off as an attack action by making a check with a bonus equal to your melee attack or Escape Artist and a DC of 10 + the greatest of your opponent's BAB, Climb Ranks, or Ride Ranks.}

      \smallskip\emph{\underline{Edge Options:} If you have the edge on an opponent when you grab them, they may not attack you at all once you have grabbed on to them. Further, grabbing on to an opponent does not provoke an attack of opportunity.}\\

     \hypertarget{combat:holddown}{}\hypertarget{combat:pin}{}
      \normalsize\item\textbf{{Hold Down}}\\\small
      Sometimes you want to pin an opponent to the ground. First, make a touch attack. Then, make a Grapple Check (BAB + Strength Modifier + Special Size Modifier) with a DC of 10 + Defender's Grapple Check Modifier. If you succeed, your opponent is pinned for one round. They can't move, and you may put ropes or manacles on them if you wish with an attack action. At the end of any turn you are pinning your opponent, you may inflict unarmed or constriction damage. With subsequent attack actions, you may attack with natural weapons or light weapons with no penalty.

      \ability{Escaping a Pin:}{If you're pinned you can attempt to fight back, but you're prone and you suffer an additional -4 penalty to attack the creature pinning you (generally a -8 total penalty to attack your attacker). You can get out with an attack action by making a Grapple or Escape Artist check with a DC of 10 + your opponent's Grapple Modifier.}

      \smallskip\emph{\underline{Edge Options:} If you're pinning an opponent and your attacks have the edge, your opponent cannot attack you or anyone else until they get free. Furthermore, if anyone else attacks them, they are considered helpless.}\\

     \hypertarget{combat:lift}{}
      \normalsize\item\textbf{{Lift}}\\\small
      Sometimes you want to put an opponent in your mouth or carry away a struggling princess. Make a touch attack and then make a Grapple Check with a DC equal to 10 + your opponent's Grapple modifier. If you succeed, your opponent is hefted into the air. You may move around freely while carrying your opponent (their weight counts against your limits of course). You may perform a coup de grace or swallow whole action on a character you have lifted, but doing so ends the lift whether it succeeds or fails.

      \ability{Escaping a Lift:}{When you've been lifted, you cannot move under your own power, but you can continue to attack. Attacks against the creature which has lifted you are at a -4 penalty. You can also attempt to escape with an attack action by making a Grapple or Escape Artist check with a DC of 10 + your opponent's Grapple Modifier.}

      \smallskip\emph{\underline{Edge Options:} If you have the edge on an opponent you have lifted, they may not attack you or anyone else until they escape.}\\
\end{list}

\hypertarget{combat:mountedcombat}{}
\normalsize\item\textbf{{Mounted Combat}}\\\small
\ability{Horses in Combat:}{Warhorses and warponies can serve readily as combat steeds. Light horses, ponies, and heavy horses, however, are frightened by combat. If you don't dismount, you must make a DC 20 Ride check each round as a move action to control such a horse. If you succeed, you can perform a standard action after the move action. If you fail, the move action becomes a full round action and you can't do anything else until your next turn.

Your mount acts on your initiative count as you direct it. You move at its speed, but the mount uses its action to move.

A horse (not a pony) is a Large creature and thus takes up a space 10 feet (2 squares) across. For simplicity, assume that you share your mount's space during combat.}

\ability{Combat while Mounted:}{With a DC 5 Ride check, you can guide your mount with your knees so as to use both hands to attack or defend yourself. This is a free action.

When you attack a creature smaller than your mount that is on foot, you get the +1 bonus on melee attacks for being on higher ground. If your mount moves more than 5 feet, you can only make a single melee attack. Essentially, you have to wait until the mount gets to your enemy before attacking, so you can't make a full attack. Even at your mount's full speed, you don't take any penalty on melee attacks while mounted.

If your mount charges, you also take the AC penalty associated with a charge. If you make an attack at the end of the charge, you receive the bonus gained from the charge. When charging on horseback, you deal double damage with a lance (see Charge).

You can use ranged weapons while your mount is taking a double move, but at a \textendash4 penalty on the attack roll. You can use ranged weapons while your mount is running (quadruple speed), at a \textendash8 penalty. In either case, you make the attack roll when your mount has completed half its movement. You can make a full attack with a ranged weapon while your mount is moving. Likewise, you can take move actions normally.}

\ability{Casting Spells while Mounted:}{You can cast a spell normally if your mount moves up to a normal move (its speed) either before or after you cast. If you have your mount move both before and after you cast a spell, then you're casting the spell while the mount is moving, and you have to make a Concentration check due to the vigorous motion (DC 10 + spell level) or lose the spell. If the mount is running (quadruple speed), you can cast a spell when your mount has moved up to twice its speed, but your Concentration check is more difficult due to the violent motion (DC 15 + spell level).}

\ability{If Your Mount Falls in Battle:}{If your mount falls, you have to succeed on a DC 15 Ride check to make a soft fall and take no damage. If the check fails, you take 1d6 points of damage.}

\ability{If You Are Dropped:}{If you are knocked unconscious, you have a 50\% chance to stay in the saddle (or 75\% if you're in a military saddle). Otherwise you fall and take 1d6 points of damage.

Without you to guide it, your mount avoids combat.}\\

\hypertarget{combat:overrun}{}
\normalsize\item\textbf{{Overrun}}\\\small
You can attempt an overrun as a standard action taken during your move. (In general, you cannot take a standard action during a move; this is an exception.) With an overrun, you attempt to plow past or over your opponent (and move through his square) as you move. You can only overrun an opponent who is one size category larger than you, the same size, or smaller. You can make only one overrun attempt per round.

If you're attempting to overrun an opponent, follow these steps.
\ability{Step 1:}{Attack of Opportunity. Since you begin the overrun by moving into the defender's space, you provoke an attack of opportunity from the defender.}

\ability{Step 2:}{Opponent Avoids? The defender has the option to simply avoid you. If he avoids you, he doesn't suffer any ill effect and you may keep moving (You can always move through a square occupied by someone who lets you by.) The overrun attempt doesn't count against your actions this round (except for any movement required to enter the opponent's square). If your opponent doesn't avoid you, move to Step 3.}

\ability{Step 3:}{ Opponent Blocks? If your opponent blocks you, make a Strength check opposed by the defender's Dexterity or Strength check (whichever ability score has the higher modifier). A combatant gets a +4 bonus on the check for every size category he is larger than Medium or a \textendash4 penalty for every size category he is smaller than Medium. The defender gets a +4 bonus on his check if he has more than two legs or is otherwise more stable than a normal humanoid. If you win, you knock the defender prone. If you lose, the defender may immediately react and make a Strength check opposed by your Dexterity or Strength check (including the size modifiers noted above, but no other modifiers) to try to knock you prone.}

\ability{Step 4:}Consequences. If you succeed in knocking your opponent prone, you can continue your movement as normal. If you fail and are knocked prone in turn, you have to move 5 feet back the way you came and fall prone, ending your movement there. If you fail but are not knocked prone, you have to move 5 feet back the way you came, ending your movement there. If that square is occupied, you fall prone in that square.

%\ability{Improved Overrun:}{If you have the Improved Overrun feat, your target may not choose to avoid you.}

\ability{Mounted Overrun (Trample):}{If you attempt an overrun while mounted, your mount makes the Strength check to determine the success or failure of the overrun attack (and applies its size modifier, rather than yours). If you have the Trample feat and attempt an overrun while mounted, your target may not choose to avoid you, and if you knock your opponent prone with the overrun, your mount may make one hoof attack against your opponent.}\\

\hypertarget{combat:sunder}{}
\normalsize\item\textbf{{Sunder}}\\\small
You can use a melee attack with a slashing or bludgeoning weapon to strike a weapon or shield that your opponent is holding. If you're attempting to sunder a weapon or shield, follow the steps outlined here. (Attacking held objects other than weapons or shields is covered below.)

\ability{Step 1:}{Attack of Opportunity. You provoke an attack of opportunity from the target whose weapon or shield you are trying to sunder. (If you have the Improved Sunder feat, you don't incur an attack of opportunity for making the attempt.)}

\ability{Step 2:}{Opposed Rolls. You and the defender make opposed attack rolls with your respective weapons. The wielder of a two\textendash handed weapon on a sunder attempt gets a +4 bonus on this roll, and the wielder of a light weapon takes a \textendash4 penalty. If the combatants are of different sizes, the larger combatant gets a bonus on the attack roll of +4 per difference in size category.}

\ability{Step 3:}{Consequences. If you beat the defender, roll damage and deal it to the weapon or shield. See Table: Common Armor, Weapon, and Shield Hardness and Hit Points to determine how much damage you must deal to destroy the weapon or shield. If you fail the sunder attempt, you don't deal any damage.}

\ability{Sundering a Carried or Worn Object:}{You don't use an opposed attack roll to damage a carried or worn object. Instead, just make an attack roll against the object's AC. A carried or worn object's AC is equal to 10 + its size modifier + the Dexterity modifier of the carrying or wearing character. Attacking a carried or worn object provokes an attack of opportunity just as attacking a held object does. To attempt to snatch away an item worn by a defender rather than damage it, see Disarm. You can't sunder armor worn by another character.}\\

\begin{table}
\caption{Armor, weapon, and Shield Hardness and Hit Points}
\begin{tabular}[t!]{l|cc} 
Weapon or Shield & Hardness & HP\textsuperscript{1} \\ \hline
Light blade & 10 & 2 \\ 
One\textendash handed blade & 10 & 5 \\
Two\textendash handed blade & 10 & 10	\\   
Light metal\textendash hafted weapon & 10 & 10 \\	   
One\textendash handed metal\textendash hafted weapon & 10 & 20 \\	   
Light hafted weapon & 5 & 2 \\	   
One\textendash handed hafted weapon & 5 & 5 \\	   
Two\textendash handed hafted weapon & 5 & 10 \\   
Projectile weapon & 5 & 5	\\   
Armor &	special\textsuperscript{2} & armor bonus x 5 \\
Buckler & 10 & 5 \\  
Light wooden shield & 5 & 7 \\	   
Heavy wooden shield & 5 & 15 \\   
Light steel shield & 10 & 10 \\	   
Heavy steel shield & 10 & 20 \\	   
Tower shield & 5 & 20 \\ \hline	   
\multicolumn{3}{p{6.75in}}{\textsuperscript{1} The hp value given is for Medium armor, weapons, and shields. Divide by 2 for each size category of the item smaller than Medium, or multiply it by 2 for each size category larger than Medium.} \\
\multicolumn{3}{p{6.75in}}{\textsuperscript{2} Varies by material.} \\
\end{tabular}
\end{table}

\normalsize\item\textbf{{Throw Splash Weapon}}\\\small
A splash weapon is a ranged weapon that breaks on impact, splashing or scattering its contents over its target and nearby creatures or objects. To attack with a splash weapon, make a ranged touch attack against the target. Thrown weapons require no weapon proficiency, so you don't take the \textendash4 nonproficiency penalty. A hit deals direct hit damage to the target, and splash damage to all creatures within 5 feet of the target.

You can instead target a specific grid intersection. Treat this as a ranged attack against AC 5. However, if you target a grid intersection, creatures in all adjacent squares are dealt the splash damage, and the direct hit damage is not dealt to any creature. (You can't target a grid intersection occupied by a creature, such as a Large or larger creature; in this case, you're aiming at the creature.)

If you miss the target (whether aiming at a creature or a grid intersection), roll 1d8. This determines the misdirection of the throw, with 1 being straight back at you and 2 through 8 counting clockwise around the grid intersection or target creature. Then, count a number of squares in the indicated direction equal to the range increment of the throw.

After you determine where the weapon landed, it deals splash damage to all creatures in adjacent squares.\\

\hypertarget{combat:trip}{}
\normalsize\item\textbf{{Trip}}\\\small
As an attack action, you may attempt to knock an opponent prone. Make a touch attack, and if you succeed make a Strength + BAB check against a DC of 10 + your opponent's Strength + BAB or Balance modifier (whichever is greater). Success leaves your opponent prone. Failure provokes an attack of opportunity.

\ability{Modifiers:}{The DC to trip an opponent who has four legs or is otherwise inherently stabile is increased by 4. Radially symmetrical creatures like Oozes cannot be tripped at all.}

\smallskip\emph{\underline{Edge Option:} If you have the edge on your target, you do not provoke an attack of opportunity if your trip attempt fails, but your target provokes an attack of opportunity from you if your trip succeeds.}\\

\hypertarget{combat:turning}{}
\normalsize\item\textbf{{Turn or Rebuke Undead}}\\\small

\subsubsection{Turning Checks}

Good clerics and paladins and some neutral clerics can channel positive energy, which can halt, drive off (rout), or destroy undead.
Evil clerics and some neutral clerics can channel negative energy, which can halt, awe (rebuke), control (command), or bolster undead.
Regardless of the effect, the general term for the activity is ``turning.'' When attempting to exercise their divine control over these creatures, characters make turning checks.

Turning undead is a supernatural ability that a character can perform as a standard action. It does not provoke attacks of opportunity.

You must present your holy symbol to turn undead. Turning is considered an attack.

\ability{Times per Day:}{You may attempt to turn undead a number of times per day equal to 3 + your Charisma modifier. You can increase this number by taking the Extra Turning feat.}

\ability{Range:}{You turn the closest turnable undead first, and you can't turn undead that are more than 60 feet away or that have total cover relative to you. You don't need line of sight to a target, but you do need line of effect.}

\ability{Turning Check:}{The first thing you do is roll a turning check to see how powerful an undead creature you can turn. This is a Charisma check (1d20 + your Charisma modifier). Table: Turning Undead gives you the Hit Dice of the most powerful undead you can affect, relative to your level. On a given turning attempt, you can turn no undead creature whose Hit Dice exceed the result on this table.}

\ability{Turning Damage:}{If your roll on Table: Turning Undead is high enough to let you turn at least some of the undead within 60 feet, roll 2d6 + your cleric level + your Charisma modifier for turning damage. That's how many total Hit Dice of undead you can turn.
If your Charisma score is average or low, it's possible to roll fewer Hit Dice of undead turned than indicated on Table: Turning Undead.

You may skip over already turned undead that are still within range, so that you do not waste your turning capacity on them.}

\ability{Effect and Duration of Turning:}{Turned undead flee from you by the best and fastest means available to them. They flee for 10 rounds (1 minute). If they cannot flee, they cower (giving any attack rolls against them a +2 bonus). If you approach within 10 feet of them, however, they overcome being turned and act normally. (You can stand within 10 feet without breaking the turning effect�you just can't approach them.) You can attack them with ranged attacks (from at least 10 feet away), and others can attack them in any fashion, without breaking the turning effect.}

\ability{Destroying Undead:}{If you have twice as many levels (or more) as the undead have Hit Dice, you destroy any that you would normally turn.}

\begin{table}
\caption{Turning Undead}
\centering
\begin{tabular}[h!]{c|c}
Turning & Most Powerful Undead Affected \\
Check Result & (Maximum Hit Dice) \\ \hline
0 or lower & Cleric's level \textendash 4	 \\   
1\textendash3 & Cleric's level \textendash 3 \\
4\textendash6 & Cleric's level \textendash 2 \\
7\textendash9 & Cleric's level \textendash 1 \\
10\textendash12 & Cleric's level \\
13\textendash15 & Cleric's level + 1 \\
16\textendash18 & Cleric's level + 2 \\
19\textendash21 & Cleric's level + 3 \\
22 or higher & Cleric's level + 4 \\
\end{tabular}
\end{table}

\ability{Effect and Duration of Turning:}{Turned undead flee from you by the best and fastest means available to them. They flee for 10 rounds (1 minute). If they cannot flee, they cower (giving any attack rolls against them a +2 bonus). If you approach within 10 feet of them, however, they overcome being turned and act normally. (You can stand within 10 feet without breaking the turning effect�you just can't approach them.) You can attack them with ranged attacks (from at least 10 feet away), and others can attack them in any fashion, without breaking the turning effect.}

\ability{Destroying Undead:}{If you have twice as many levels (or more) as the undead have Hit Dice, you destroy any that you would normally turn.}

\subsubsection{Evil Clerics and Undead}

Evil clerics channel negative energy to rebuke (awe) or command (control) undead rather than channeling positive energy to turn or destroy them. An evil cleric makes the equivalent of a turning check. Undead that would be turned are rebuked instead, and those that would be destroyed are commanded.

\ability{Rebuked:}{A rebuked undead creature cowers as if in awe (attack rolls against the creature get a +2 bonus). The effect lasts 10 rounds.}

\ability{Commanded:}{A commanded undead creature is under the mental control of the evil cleric. The cleric must take a standard action to give mental orders to a commanded undead. At any one time, the cleric may command any number of undead whose total Hit Dice do not exceed his level. He may voluntarily relinquish command on any commanded undead creature or creatures in order to command new ones.}

\ability{Dispelling Turning:}{An evil cleric may channel negative energy to dispel a good cleric's turning effect. The evil cleric makes a turning check as if attempting to rebuke the undead. If the turning check result is equal to or greater than the turning check result that the good cleric scored when turning the undead, then the undead are no longer turned. The evil cleric rolls turning damage of 2d6 + cleric level + Charisma modifier to see how many Hit Dice worth of undead he can affect in this way (as if he were rebuking them).}

\ability{Bolstering Undead:}{An evil cleric may also bolster undead creatures against turning in advance. He makes a turning check as if attempting to rebuke the undead, but the Hit Dice result on Table: Turning Undead becomes the undead creatures' effective Hit Dice as far as turning is concerned (provided the result is higher than the creatures' actual Hit Dice). The bolstering lasts 10 rounds. An evil undead cleric can bolster himself in this manner.}

\subsubsection{Neutral Clerics and Undead}

A cleric of neutral alignment can either turn undead but not rebuke them, or rebuke undead but not turn them. See Turn or Rebuke Undead for more information.

Even if a cleric is neutral, channeling positive energy is a good act and channeling negative energy is evil.

\subsubsection{Paladins and Undead}

Beginning at 4th level, paladins can turn undead as if they were clerics of three levels lower than they actually are.

\subsubsection{Turning Other Creatures}

Some clerics have the ability to turn creatures other than undead.

The turning check result is determined as normal.

\hypertarget{combat:twoweapon}{}
\normalsize\item\textbf{{Two\textendash Weapon Fighting}}\\\small

If you wield a second weapon in your off hand, you can get one extra attack per round with that weapon. You suffer a \textendash6 penalty with your regular attack or attacks with your primary hand and a \textendash10 penalty to the attack with your off hand when you fight this way. If your off\textendash hand weapon is light, the penalties are reduced by 2 each. (An unarmed strike is always considered light.) The Two\textendash Weapon Fighting feat eleminates the penalties entirely.
 
%Now that it's one feat to reduce penalties completely, a table is kinda silly
%Table: Two\textendash Weapon Fighting Penalties	   
%Circumstances	Primary Hand	Off Hand	   
%Normal penalties	\textendash6	\textendash10	   
%Off\textendash hand weapon is light	\textendash4	\textendash8	   
%Two\textendash Weapon Fighting feat	\textendash4	\textendash4	   
%Off\textendash hand weapon is light and Two\textendash Weapon Fighting feat	\textendash2	\textendash2	 

\ability{Double Weapons:}{You can use a double weapon to make an extra attack with the off\textendash hand end of the weapon as if you were fighting with two weapons. The penalties apply as if the off\textendash hand end of the weapon were a light weapon.}

\ability{Thrown Weapons:}{The same rules apply when you throw a weapon from each hand. Treat a dart or shuriken as a light weapon when used in this manner, and treat a bolas, javelin, net, or sling as a one\textendash handed weapon.}

\end{list}
\subsection{Special Initiative Actions}

Here are ways to change when you act during combat by altering your place in the initiative order.

\listone
\hypertarget{combat:delay}{}
\normalsize\item\textbf{{Delay}}\\\small
By choosing to delay, you take no action and then act normally on whatever initiative count you decide to act. When you delay, you voluntarily reduce your own initiative result for the rest of the combat. When your new, lower initiative count comes up later in the same round, you can act normally. You can specify this new initiative result or just wait until some time later in the round and act then, thus fixing your new initiative count at that point.

You never get back the time you spend waiting to see what's going to happen. You can't, however, interrupt anyone else's action (as you can with a readied action).

\ability{Initiative Consequences of Delaying:}{Your initiative result becomes the count on which you took the delayed action. If you come to your next action and have not yet performed an action, you don't get to take a delayed action (though you can delay again).
If you take a delayed action in the next round, before your regular turn comes up, your initiative count rises to that new point in the order of battle, and you do not get your regular action that round.}\\

\hypertarget{combat:ready}{}
\normalsize\item\textbf{{Ready}}\\\small
The ready action lets you prepare to take an action later, after your turn is over but before your next one has begun. Readying is a standard action. It does not provoke an attack of opportunity (though the action that you ready might do so).

\ability{Readying an Action:}{You can ready a standard action, a move action, or a free action. To do so, specify the action you will take and the conditions under which you will take it. Then, any time before your next action, you may take the readied action in response to that condition. The action occurs just before the action that triggers it. If the triggered action is part of another character's activities, you interrupt the other character. Assuming he is still capable of doing so, he continues his actions once you complete your readied action. Your initiative result changes. For the rest of the encounter, your initiative result is the count on which you took the readied action, and you act immediately ahead of the character whose action triggered your readied action.
You can take a 5\textendash foot step as part of your readied action, but only if you don't otherwise move any distance during the round.}

\ability{Initiative Consequences of Readying:}{Your initiative result becomes the count on which you took the readied action. If you come to your next action and have not yet performed your readied action, you don't get to take the readied action (though you can ready the same action again). If you take your readied action in the next round, before your regular turn comes up, your initiative count rises to that new point in the order of battle, and you do not get your regular action that round.}

\ability{Distracting Spellcasters:}{You can ready an attack against a spellcaster with the trigger ``if she starts casting a spell.'' If you damage the spellcaster, she may lose the spell she was trying to cast (as determined by her Concentration check result).}

\ability{Readying to Counterspell:}{You may ready a counterspell against a spellcaster (often with the trigger �if she starts casting a spell�). In this case, when the spellcaster starts a spell, you get a chance to identify it with a Spellcraft check (DC 15 + spell level). If you do, and if you can cast that same spell (are able to cast it and have it prepared, if you prepare spells), you can cast the spell as a counterspell and automatically ruin the other spellcaster's spell. Counterspelling works even if one spell is divine and the other arcane.

A spellcaster can use dispel magic to counterspell another spellcaster, but it doesn't always work.}

\ability{Readying a Weapon against a Charge:}{You can ready certain piercing weapons, setting them to receive charges. A readied weapon of this type deals double damage if you score a hit with it against a charging character.}

\end{list}