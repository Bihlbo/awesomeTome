
\section{Dungeons of Note}

Sure, you've been to the sewers under the town, and the maze in the wilderness, and the cave that opens up into the Underdark, but when was the last time you went into a dungeon that you cared about in any way? Which was the last one that had some traction, some \emph{pizzazz}? Here are some sample dungeons that will stick in the players' minds long after they leave them:

\subsection{The Hall of Records}

It's where information goes to die, except that it never dies. Located in a distant corner of Baator, the Hall of Records is a timeless library that contains a wealth of information dating back to when only the Aboleth had an empire in the mortal world. The filing system is intensely baroque, and it requires more than good searching skills to find the document that you need. The layout of the complex is inherently evil and unhelpful, designed to hamper and ruin those who need its services. The extradimensional floor plan is highly confusing and moreover the noneuclidian geometry is run with substantial changes on each layer. The index can tell you approximately where you need to be, and the only way in or out is teleportation.

Every visit to The Hall of Records is unique, and the players don't really need to map it all out. The really neat part about the place though is that it is strongly opposed to Divination magic and timeless. This means that creatures can (and do) hide out in here for hundreds of years when they make a lot of enemies. Many of these vagabonds make permanent camps in various parts of the Hall of Records. They live a limited, hermit-like existence and react strongly when other creatures enter the areas they have claimed as their own.

\subsection{The Tomb of Iuchiban}

The world's greatest blood mage made a quite credible attempt to gain godhood and nearly succeeded. Actually killing him permanently was essentially out of the question (and completely pointless for a being of such incredible and unethical power), so he was imprisoned into a block of jade. That block of jade is further suspended in a lake of mercury in the center of a lattice of tunnels filled with the most dangerous traps that the greatest architect magician of the time could create. The nature of the construction suppresses and confuses shadowlands creatures, as well as conjuration and divination magic, making them more and more unreliable as you get farther towards the center. The original architect set the last traps from the very center of the tomb and could not himself escape, so he committed suicide right there next to the final prison. If you can get to the middle, you'll see him there and get to read his last thoughts, still preserved after all these years.

And while crawling your way through metal lined tunnels (to stop burrowing creatures) filled with imaginative lethal traps might seem like a bad thing, remember that your progress through the Tomb is essentially timed. Guards patrol the outside of the Tomb constantly, and the Empire will send people into the tomb if and only if Blood Speakers have broken that perimeter in an attempt to revive their lord. So whether your party is composed of Blood Speakers or Imperial Agents, the other team is also making its way through the maze, and if you don't get to the center in time, things will go badly for you. Taking 20 on Search may not be possible.

\subsection{The Garden of the Gretel the Snowshaper}

Long ago there was a 15th level Illusionist with access to several of the effects that increase the reality of shadow spells, allowing her to make 90\% real simulacrums of herself with 13 levels and some spare XP, who were also able to make simulacrums of her, which were therefore also able to do so, and so on and so on. When she was finally slain, she had already amassed an army of approximately 100 13th level copies of herself in her workshop located in a valley blanketed in a constant layer of snow. And each of the simulacrums is unable to gain levels, so they have nothing better to do with XP than just make magic items, constructs, and wondrous architecture with it. Each simulacrum is completely aware that it must follow the orders of another simulacrum farther up the chain towards the original Snowshaper, so each takes great pains to avoid talking to other simulacrums lest they be forced to follow potentially self-destructive commands.

The Garden today is so wrapped in Illusions that it appears to be a garden in truth. Fountains, hedges, and colorful birds stand in stark contrast with the icy and forbidding mountains that surround the valley. Thermal Illusions make the region feel balmy and warm, but in truth the area is so cold that exposed skin will become frostbitten in a short period of time (noticing this is happening requires a successful DC 24 Willpower Save to disbelieve the affected temperature). Gretel's palace appears as a fancy pagoda made of paper and wood, but in truth it is an edifice of ice carved through with tunnels. About 80 Gretel Simulations persist to this day, and they are still under orders to remain in the valley and make things. Each of them has hidden herself in sections of the castle or the surrounding gardens, attempting to fend off other lower numbered Gretels who could command them. Reactions to non-Gretel characters entering the valley are highly mixed, often constrained by the last orders they received when the Garden was still functioning properly.

\subsection{The Closed Shafts}

Dwarves and Kobolds dig tunnels deep into the roots of the mountains in an attempt to get access to the veins of gold and mithril that run through the earth's rocky heart. Particularly deep shafts often yield the best results, so the different teams sometimes have been known to sacrifice a bit of safety to push down as far as possible. Rivalry between the dig teams of the different races is intense and when the mine shafts break through into one another, battles often erupt over mined out territory. Shafts compromised by enemy forces are sometimes boobytrapped by either (or both) of the races, and the maps of the shafts become confused for both sets of foremen. Such was the case in the section now simply called ``The Closed Section" by Dwarf and Kobold alike.

Both the Dwarves and Kobolds had been digging into what promised to be an exceptionally rich vein of mithril ore, and had been playing the territorial control game against each other, breaking shafts through into the other's territory and trapping it. The dangerous, yet not unusual game was upset when the Dwarves hit water, flooding the lower sections and threatening to terminate the entire project. Undaunted, the Dwarves began setting up machinery to pump the water out. Once that started coming online, the Aboleth attacked. Having massively more power than the Dwarven miners, they quickly overwhelmed the lowest worker teams and shut the machines off. The rest of the Dwarves, seeing their compatriots converted into Skum, quickly withdrew.

The Kobolds, seeing the Dwarven presence weaken in the mines (and not knowing about the Aboleth forces), quickly moved in to secure territory, moving throughout the mine and setting up make-shift traps all along the route in order to damage the Dwarves' ability to move back into position. When they encountered the Aboleth territory, they too were turned into Skum and slaves, and the Kobolds relinquished their claims on the shafts as well.

There's a lot of mithril down there, but even the partial maps of the shafts that were possessed by the foremen of the Dwarf and Kobold teams have been lost. And now, the Aboleth's Skum forces are moving up into the other territory. Both the Dwarves and the Kobolds want someone to go down there and overwhelm the hostile forces long enough to get those machines back up and pumping.
