\section{To Rule in Hell\ldots}

Some of our favorite bad guys are the Arch Devils and the Demon Lords. More people can name Jubilex than can name Erythnul, and that's no accident. Jubilex is just a little bit more awesome than any of the official Chaotic Evil Gods. Unfortunately, like many things related to the Lower Planes, there is substantial discrepancy available as to what exactly it is that these Dark Lords \textit{are}. We suggest that you make up your mind and distribute this decision to your player characters:

\subsection{Ascended to Godhood}

In this option, demon princes are gods, which means that they have all the rights, privileges, and pitfalls that status implies. Killing a god is, under this model, the same as trying to kill something as intrinsic to the universe as something like ``water'' or ``love''. This doesn't mean that you can't fight these guys: they have worshipers that can be killed, temples that can be burned, and avatars that can be trapped under mountains. Since avatars are like deities you can stab (exactly like that in fact), it doesn't really matter if Mephistopheles is an unkillable ideal that serves as a focus for the prayers and power of his worshippers since you can kick his avatar's can and loot his palace. The real question is this: are demon princes part of your pantheon, or are they merely aspects of existing gods?

If demon princes are gods in their own right, I can't imagine Nerull sitting comfortably next to Yeenoghu at the divine dining table, as those two have a certain amount of overlap in their portfolios (and no one likes a copy-cat). This could lead to interesting adventures as the more accepted evil deities compete for cosmic real estate with the demon princes leading to worshipper on worshipper violence and games of deception where good organizations are manipulated into fighting the followers of the other god.

The same kinds of conflicts could occur if the Demon Princes are just hardcore aspects of more accepted evil gods. Heresy inside a faith can be ripe opportunity for adventuring, as wars have destroyed entire continents over this kind of thing. If Orcus is just an aspect of Wee Jas, you could have the fuel for a truly righteous schism.

\subsubsection{Slaying the Gods}

Let's face it: in the original writeup, Lolth had \textbf{66 hit points}. Not an avatar or anything, the dark goddess of the Drow had less hit points than a 7th level Fighter who had just rolled well (everything had less hit points back then, and 7th level was more impressive, but you get the idea). D\&D has a long history of characters stabbing gods directly in the face. Nevertheless, the gods presented in Deities and Demigods are extremely unsatisfying for that purpose (for one thing, they aren't even epic characters, and for another thing they arbitrarily have infinite power in weird, poorly defined ways and that makes any rules adjudication into a game of Cops and Robbers). Orcus, on the other hand, can be knocked down and beaten like a King until he stops moving and a good time is had by all. By making the Dark Lords into ``real'' gods and keeping their essential nature as stabbable entities, you can achieve some classic D\&D moments that are largely missing otherwise from 3rd edition.

\subsection{President of the Corporation}
\vspace*{-8pt}
\quot{``Orcus is dead! Long live the new Orcus!''}

We all know Orcus, he's one of our favorite guys. He's been portrayed as a fat bat-winged dude with a skull-stick, and as a gaunt skeletal guy with a sword. But somehow he's always Orcus, the Demon Lord of the Undead. In the CEO model of Arch Fiends, Orcus is a \textit{title}. Once someone takes out the current Orcus, then that opens up the possibility for another person or monster to become Orcus. Being Orcus gets you Orcus' desk, the accumulated debts and assets of the Orcus estate, and the authority to molest members of Orcus' numerous cults across the planes. It also gets you a posse of paladins from distant planes who have already dedicated themselves to your destruction.

\subsection{Powerful Adventurers}

Alternately of course, the Demon Lords and Arch Devils can just be adventurers. Like the Player Characters, they've been the protagonists of their own stories and they aren't going to be permanently killed by anything. No matter what takes them down, they'll definitely come back. In this model, Yeenoghu is ``just'' an Epic Character, and that means that the only reason you know his name is that he's just a little bit cooler than Valishar Goldeneyes or any of the other epic characters that D\&D gets cluttered up with.

In this model, the archfiends can be defeated by powerful adventurers. But like the powerful adventurers that they face off against, they have the interest of other powerful party members and assistants who will faithfully cast powerful magic or go on dangerous quests to bring them back from death. D\&D characters of sufficient stature can be \textit{defeated}, but permanently killing them is generally not on the table. If enough people know your name, eventually someone is going to \spell{wish} you back.


\section{Rulership of the Lower Planes}

Player characters can gain Influence and Rule in the Lower Planes just as they can in the Prime Material plane. In fact, it's a little \textit{easier} to assume rule on the planes because the planes themselves are conditioned to accept themselves as the property of powerful beings. The Prime Material is just a bunch of rocks in space, but Carceri is \textit{divinely morphic}. This means that to a certain extent, belief shapes the planes in a way that the prime has no real equivalent to. With enough patience and sufficiently good propaganda you can change the weather, populate your region with unlikely creatures, and encourage the region to conform to your expectations in a myriad of subtle ways. Being the unquestioned ruler of a region of any lower plane increases your Influence by +1 per year, in addition to any other benefits you might accrue. Long established archfiends, therefore, have tremendous effective Leadership scores.

Rulership of an infinite plane is, of course, impossible. Even the gods themselves have boundaries to their personal dominions. The realms of the deities are tremendous in scope, but in the context of the plane they reside within they are grains of sand in the wind. The dominions that can be carved out by lesser beings are smaller still, though again they can extend across regions that in any more familiar world would be called vast.

Most of the evil planes are themselves Evilly aligned. That means that any region you happen to take control of is essentially like a huge Film Noir movie. Evil people are just smarter, more perceptive, and \textit{cooler} in these places than Good creatures are. Strongly evil regions are even worse: even neutral creatures are penalized (so badly in fact, that animals brought into Hades just curl up and die because the plane itself is dissing them so badly).

\subsection{Location, Location, Location}

So you're looking at some Infernal real estate, and you \textit{don't} want some Gray Waste that's 8 days hike to a planar oasis, and you don't want something across the street from a Nalfeshnee Spawning Pit. But what \textit{do} you want? The Lower Planes have a lot of locations that are financially, strategically, magically, or socially important -- in addition to those places that have a property value somewhere between ``no'' and ``\textit{Hell} no!''

\subsubsection{Finality}

Finality is a planar metropolis in the Infernal Battlefield of Acheron near a minor tributary of the River Styx. Its harsh laws are kept in rigid and uncompromising order by the will of powerful pit fiends, and the city serves as a marketplace for the lucrative trade in souls. Magic items can be bought or sold here, but the currency is always souls (as a planar metropolis, Finality has a gp limit of 600,000 gp). Souls are valued at their CR squared, multiplied by 100 gp. Many items purchased from this location radiate evil, buyer beware. Lodging may be purchased at flat rate of one soul per day per person. The section of Acheron that Finality rests in has the Timeless trait and is mildly Lawfully aligned. The population of The City is about 100,000 people (with uncounted millions of souls), most of whom are Baatezu.

The rules in Finality are uncompromising and bizarre, and the punishments for breaking them are vindictively carried out to the letter by powerful devils. But there is no warfare allowed in the city, and even Celestials and Tanar'ri come to participate in the great mercantile dance of soul collection. Characters must make a Knowledge (Local: Finality) check everyday with a DC of 10 + 2 per day they've been in the city or unknowingly break one of the city's many inscrutable laws (knowingly breaking the law by starting fights or stealing goods is a whole different thing). Punishments range from perplexing to fatal. Characters who stay away from Finality for more than a month are no longer subject to the baroque residency rules and their DCs are returned to 10 the next time they visit.

The city itself is a collection of gothic stone architecture twisted by infernal and magical power into fantastic and improbable spires and towers and nestled amid squat merchant houses, the very stones tainted by the trade that has given birth to the city. Gray stone is unknown, as the city is build from exotic dark-colored stones dragged from dozens of lower planes, and it is amusing to note that the city itself is not built of souls, unlike several other planar locales. Undying fires light the orderly and aesthetically similar city streets, and unsleeping outsiders conduct traffic and trade continuously throughout the day. Food and drink is only available at exorbitant rates, as few residents have a need for such unseemly mortal concerns.

As a city based on a trade practice that is undeniably evil, one would think that heroes and celestials assault the city on a daily basis; but nothing could be further from the truth. Two factors keep the city free from harassment: Finality serves as a way to recover powerful souls of heroes and angels, and the city itself has a glut of powerful spellcasters whose stock and trade requires the existence of the city. Even the most idealistic champion of good knows that to destroy the city is to spread the soul trade far and wide across the planes so that greater depravities would be necessary for these merchants to stay in business. As long as they are bound by the laws of the pit fiends ruling the city, the worst excesses of the soul trade are curbed and bargains can be made with the most untrustworthy fiend for the recovery of a valuable soul.

Though the city is based in a plane with more than a few gods, these powerful entities do not attempt to sack the city for its riches in souls and magic. It is rumored that far back in the city's past, a moderately powerful deity attempted such a feat (or perhaps was trying to recover some follower's souls), and that the spellcasters of the city banded together to capture this bold immortal's soul and bind it into Finality, and have been slowly burning this being's undying essence to fuel their own magics. Whether this story is true or not remains to be seen, but divine spellcasters who enter the city can feel a terrible sucking sensation emanating from the very ground beneath their feet.

\subsubsection{Soul Veins}

No one, not even in Baator, tries to keep an accurate census of the residents of the lower planes. That's because new residents are popping up all the time. People who were particularly awesome show up in death pretty much as they did in life. They get special attention from the gods who punish or reward them according to how closely their lives matched the expectations of the gods in charge of judging them (which means that most high-level adventurers end up living the sweet life in whatever outer plane they end up in). Furthermore, these guys are often whisked back to life by powerful magics that have no time cap at all. Famous heroes can be brought back to a living state hundreds of years after they die, and retain their sense of self continuously throughout. But people with small, unimportant lives get a much worse deal in the D\&D afterlife. The forgotten multitudes of the lower level mooks and farmers of the worlds get overlooked by the great judging. Their souls are used as building materials -- even in the ``Good'' planes (where, for example, Celestials take the souls of those not deemed important enough to warrant special attention to power street lights and provide illumination for cafes).

Souls of the forgotten are used to create outsiders. Every Hound Archon or Glabrezu you see was at one time one of the forgotten. But others are used to power magic, or to make equipment, furniture, or even walls. Souls are a valuable resource, and sight unseen sell for 100 gp (more interesting or valuable souls can sell for much more depending upon who's buying). The souls of the forgotten bubble up all over the place on the lower planes, and many of those places are pretty inhospitable or even \textit{under ground}. When the souls of the dead appear in a confined space they are crushed into a taffy-like substance called Soulstone. Veins of Soulstone are all over the lower planes, though ones near planar oases, fortresses, and other settlements are generally already tapped to one degree or another. A single soul's worth of Soulstone weighs 1/50th of a pound.

\subsubsection{Portals}

Portals connect every plane, and many of the portals on the Lower Planes are in areas sufficiently dangerous, that few creatures know of them. Still, so many of the planar denizens have the ability to use \spell{greater teleport} with a limit of 50 pounds of carried items that items of less than 50 pounds from all over Pandemonium can be found for sale in The Mad House at quite reasonable prices. The money in shipment isn't in moving small or fungible things within a plane (any Glabrezu can transport a tonne of rice to anywhere in 9 minutes), the money comes from transporting things \textit{between planes} or transporting objects that weigh \textit{over fifty pounds}. The mark-up there is \textit{intense}, and beings of the Lower Planes are willing to accept price gouging on interplanar and high-mass transport because they understand that the normal ``teleport tag'' model of goods transference doesn't work for those kinds of transports. Controlling a portal from anywhere in a Lower Plane to anywhere in any other Plane of existence can get a prospective merchant lord the benefits of being a Monopolist (+5 to Profit checks), but only if the portal is opened up to commercial use.

The easiest way to open up a portal to commercial use is to get the word out that you control a Portal and are willing to allow fiends to use it for a fee. That works, as there is enough rapid transport available on the serial teleportation circuit that goods will make their way to your portal as soon as the existence of your Portal becomes common knowledge amongst merchant fiends. There is, after all, a market for \textit{something} in every plane. You may not want to \textit{look} in the baskets that the Gelugons are hauling, but they'll pay in gold or souls, so the money is right. Unfortunately, having your Portal be well known is its own punishment -- fiends have a marked tendency towards greed, so if your Portal becomes profitable enough you may have to contend with hordes from the hells coming to take your stuff and not give it back.

With more difficulty, one could attempt to find and manipulate markets on both sides of the portal yourself. The rewards of doing so are even greater (no middle-men means more profits, a +10 bonus to Profit checks in fact). But you'll have to find a source of goods or services, a demand for those services, and transport those goods or services yourself. And while your operation is initially not under scrutiny, if enemies find out about it they'll be even more interested in knocking you over to take it away.

Portals that go between centers of economic activity can be valuable even if they are on the same plane of existence. Moving mid-sized and large objects around the planes is extremely difficult because there's generally no infrastructure for it. The very ease of moving bricks one at a time across the Wastes of Hades has led there to be almost nothing in the way of \textit{carts} or \textit{roads}. Transporting even a mid-sized stone is all but impossible across any meaningful distance. As a result, if a portal is capable of moving heavier objects and connects two places that host reasonable amounts of economic activity on both sides, the controller of that Portal is considered a monopolist (+5 to Profit checks). Portals connecting planar metropolises on different planes are even more valuable, and provide the bonus (and the potential hostile interest) of both a Portal between planes \textit{and} a Portal between centers of economic activity. And yes, you \textit{can} get the bonuses for being a monopolist twice.

\subsubsection{Planar Oases}

The planar traits on many of the planes (especially Hades) are\ldots not good. But there are places on any plane that lack those planar traits or have the planar traits of other planes. Those places are \textit{extremely valuable}, as they are pretty much the only place that most planar denizens can live, work, or play on many of the planes. Planar oases in places like Pandemonium or Acheron are fairly valuable because being Lawfully aligned in Pandemonium is unpleasant, but a Planar Oasis in the Gray Wastes is \textit{extremely} valuable because life in the Wastes is almost impossible for most extraplanar creatures. Owning a stronghold in a planar oasis draws planar denizens to your banner, causing your Influence modifier to increase (PoF).

\listone
	\item Every month that you hold a Planar Oasis on Pandemonium, Acheron, or Carceri increases your Influence by +1.
	\item Every month that you hold a Planar Oasis on Baator, Gehenna, or the Abyss increases your Influence by +2.
	\item Every month that you hold a Planar Oasis in the Gray Wastes of Hades increases your Influence by +5.
\end{list}

Note that holding a planar oasis isn't easy. Fiends and even Celestials from all over the planes will come to take your stuff for use as a military beachhead or planar resort. Although the bonuses to your influence are cumulative for holding a Planar Oasis for a long time, you lose them all if someone else takes your control away. When you take control of an oasis from another creature you can either allow the current tenants to stay (keeping the entire Influence onus of the previous owner for yourself and inheriting whatever problems the previous owner had allowed in), or attempting to clear out the old residents and start over (resetting the Influence bonus to zero as soon as you're done, but allowing you to do things ``right'').

\subsection{Wondrous Architecture of the Lower Planes}

The planes are well known for fantastic locations, and the fiendish constructions of the Lower Planes are no exception: demon cities of unusual construction compete with the infernal strongholds of powerful fiends in both grandeur and designed atrocities. Magic often goes into the construction of these locations so that these places become conduits for the energies of the Lower Planes. Such places must be built from scratch to create these effects; no existing city can be modified to gain this wondrous architecture.

Any Prerequisites for these kinds of cities are flavor considerations. It is a DM's option to allow such a city to be built, and he will determine any costs or prerequisites needed to create such a place. This is not because we are trying to keep these effects out of the hands of the player characters -- quite the opposite. In fact, it is because we \textit{want} players to use these effects now and again that we make them uncosted. Within the context of the D\&D metaeconomy (where one is specifically allowed to \textit{purchase} epic magic items with tonnes of gold), there is just no possible fair cost for an entire city covered with magical effects. That kind of thing is really awesome flavor-wise, but giving it a cost unfortunately leaves it transferable into magical equipment that can destabilize the game. See the Book of Gears in this document for information on getting around this issue.

\subsubsection{Necromantic City}

Built with materials associated with necromancy and populated by the undead, these cities have features like obsidian walls buildings, bone dust in the street instead of dirt, and images of death on every surface. Pale ghost light illuminates the streets and the living slowly die within its walls.

\ability{Prerequisites:}{Over 75\% of the population must be undead.}

\ability{Effects:}{All Necromancy effects are at a +4 caster level, and every day spent in the city inflicts a negative level (this heals undead of 5 HPs).}

\subsubsection{Serpentine Labyrinth}

The streets of this city are twisting mazes and the angles formed by buildings and walkways are designed to confuse and inspire disorientation.

\ability{Prerequisites:}{City must have been designed by someone with at least 30 ranks in Knowledge(architecture).}

\ability{Effects:}{Any non-native entering this city halves his movement rate while in the city.}

\subsubsection{Redstone City}

The stones of the city absorb blood, and the city itself inspires violence and hatred.

\ability{Prerequisites:}{The city must be build from stone pulled from sites of great violence.}

\ability{Effects:}{All starting reactions of NPCs are one level more hostile.}

\subsubsection{Spired}

Magic has been used in the construction of the city to enable tall spires of surpassing delicacy. Only the most agile fliers can enter such a city, but defenders can fire down upon invaders with surprising ease.

\ability{Prerequisites:}{City must have been designed by someone with at least 10 ranks in Knowledge(architecture) and the ability to cast 5th level effects.}

\ability{Effects:}{Only flyers with a Maneuverability Rating of Good or better can fly in the city. Attackers in the city suffer from archer attacks every 10 minutes as natives exploit the unique construction of their city.}

\subsubsection{Basalt}

Magically heated stone forms the basis of every building in this place, and the weather in this city is always equal to high summer in the dessert.

\ability{Prerequisites:}{These cities must be built by natives of Fire-aligned planes or with the ability to cast 5th level effects.}

\ability{Effects:}{The weather in the area is always at least as hot as ``Severe Heat''. Unearthly and burning heat happen wih surprising regularity.}

\subsubsection{Plague Town}

A Plague Town is known for its poor sanitation, lax attitude about corpse and garbage collection, and vile culinary practices. As a result, living within such a place means that disease is a constant companion, and few visitors travel to such a hellhole.

\ability{Prerequisites:}{Such cities are always in remote locations.}

\ability{Effects:}{Non-natives must make a DC 15 Fort save for every day spent in the city or else catch a random disease. Each native is a carrier of 1d4 diseases, but is immune to their effects.}

\subsubsection{Forgotten}

Some locales in mist covered mountains and secludeed valleys seem to slip from mortal memory, and travelers can seldom find these places after they have left them.

\ability{Prerequisites:}{Such cities are always in remote locations.}

\ability{Effects:}{This city is invisible to anyone not within 100' of it, and pathways leading to it are concealed by illusions to appear impassable. Any non-native leaving the city must make a DC 15 Will save or forget which pathways lead to its location and details of its interior (meaning that you cannot use teleportation or travel magic to return). Such a city can be found again if the general area is searched again (such as the entire plain, mountain range, swamp, or ocean).}

\subsubsection{Forsaken}

Forsaken cities are cursed and empty, the sites of great betrayals or massacres. Only the strong-willed can enter and remain in such a place.

\ability{Prerequisites:}{}

\ability{Effects:}{When entering the city, and every day afterwards, a DC 15 Will Save must be made or else the subject cannot bring themselves to willingly enter the city for a month.}

\subsubsection{Rune-built}

Streets and buildings in this city form runes when viewed from a great distance, granting the effects of a spell upon the city or its people. Some notable examples of this kind of city include:

\listone
	\bolditem{The Palace of the Maskers:}{A city known for secret meetings and negotiations where every person in this city is affected by \spell{alter self}.}
	\bolditem{Wide Sky:}{A floating city on the Elemental Plane of Air where everyone can \spell{fly}.}
	\bolditem{The Free Nation:}{A town in Limbo protected by a \spell{magic circle against law}.}
	\bolditem{The City of Secret Things:}{A major trading post in the Astral plane where \spell{obscure object} has been cast on every object.}
\end{list}

\ability{Prerequisites:}{}

\ability{Effects:}{A spell of up to 3rd level can affect every person, object, or area in the city, and this effect cannot be dispelled. This effect does not last beyond the borders of the city.}

\subsubsection{Hungry}

Some cities seem to have a life of their own, and they consume the weak and the foolish. People entering such a place vanish without a trace when they leave sight of their friends, and only the strong last long in such a place. Such a city might protect group of predators with magic effects, might have a high crime rate due to magically enforced disrespect for laws, or it might simply animate buildings or statues and devour the unwary; no two Hungry cities are the same. Such cities are noted for having small police forces as troublemakers are either protected or devoured by such places.

\ability{Prerequisites:}{Varies.}

\ability{Effects:}{For every day spent within this city, an individual must defeat a CR 5 encounter. Should an individual be defeated by this encounter, his body vanishes. Natives of this city are immune to its effects.}

\subsubsection{Magic Dead}

Some beings only trust the power of muscle and steel, and have carefully crafted their city in order to scour clean the flow of magic. Such a city is note for being well constructed and sturdy and its people hard-working, but unimaginative.

\ability{Prerequisites:}{Such a city is never ruled by a magicracy or race with racial spell-like or supernatural abilities and never sells magic items in its shops or markets.}

\ability{Effects:}{Every area of this city is inside an \spell{antimagic field}. \spell{Scrying} cannot pierce this place and travel magic cannot bring one closer than the gates of this city.}

\subsubsection{Wicked}

Terrible acts are performed in this place, and evidence to this fact is written in both the construction of the city and faces of the natives. Vile statuary and murals cover every available surface, and natives of this unholy place do little to hide their depraved desires or acts.

\ability{Prerequisites:}{City must be in a planar area that is evil aligned.}

\ability{Effects:}{When entering this city, and every day afterward, make a Will Save (DC 15). On a failed save, the victim is Shaken for the duration of his stay.}

\subsubsection{Infernal Fortress}

This city is designed to hinder the spell-like abilities of attacking fiends with features like shifting geography to foil \spell{greater teleport}, runes that block summonings, and mists that negate the advantage of being able to see in darkness.

\ability{Prerequisites:}{City must be designed by someone with knowledge in Architecture, History, and the Planes of at least 10 ranks.}

\ability{Effects:}{Other than the formidable defenses of the metropolis, there are no effects.}

\subsection{Business as Usual}

There is profit to be made on the lower planes for the unethical, and that means that almost everyone has a scheme to get rich quick or swindle the other man. In the lower planes, as everywhere else in the multiverse, everyone thinks that they're smarter than average, have a good sense of humor, and are good in bed. If someone in Gehenna tells you that they don't have a scheme, \textit{that's part of their scheme}.

\subsubsection{Orchards of Larvae}

Larvae appear whenever a particularly evil creature dies on the Prime Material, and none of the gods care enough to do anything specific with them. They appear all over the Lower Planes, but they appear in some places more than others. Those areas where larvae appear with more frequency are called \textit{orchards}, and those of them as have been discovered are generally heavily built out. An exception is those that are in the Gray Wastes of Hades, which are at best occasionally looted by Night Hags or Blood War Soldiers. Larvae burrow themselves into the ground and huddle in the dirt soaking up evil until they metamorphose into a Fiend or are eaten by infernal wildlife.

Gaining control of a Larvae Orchard is like gaining a business, save that its relative location isn't important. Even large Larvae are less than fifty pounds and quite portable by \spell{teleportation}. A Larvae Orchard, thus, has a +2 Profit Modifier regardless of whether it is in a Planar Metropolis or the middle of a hoary wilderness. Larvae sold on the open market are used for everything from a luxury food to a source of powerful servants.

\business{Larvae Orchard}{Search}{Handle Animal, Sense Motive}{Special (4,000 gp)}{Medium}{High}

Running a Larvae Orchard is a highly eventful proceeding. Every season, roll a d6: on a 1-3 subtract 10 from your Business Events roll each month, on a 4-6 add 5 to your Business Events roll each month. These modifiers are replaced each season.

\subsubsection{Pain Stills}

Liquid Pain can be harvested from any sentient creature tortured to near death over a long period of time. It is also a powerfully addictive drug and a source of intense magical power. The creation of Liquid Pain is quite Evil, but that in no way discourages anyone in the Lower Planes. Liquid Pain can be used to create magic items or empower spells, and no one even notices that it turns items created with it [Evil] because the environment is doing that anyway.

To make a Pain Still, one merely needs a relatively stable area to keep a tremendously expensive alchemical apparatus and a dungeon full of prisoners with a diverse assortment of torturing gear to agonize them. Some Pain Stills torture victims to near death and allow them to go, others simply kill victims who are no longer capable of being juiced.

\business{Pain Still}{Craft (Alchemy)}{Heal, Profession (Torturer)}{Low}{High}{High}

\subsubsection{Zombie Factories}

Fiends abound who are able to create undead with their spell-like abilities, and this allows them to create undead without using valuable Onyx. In a zombie factory, these powers are used as part of a service to create uncontrolled undead by the score. These services are often employed by wizards, clerics, and dread necromancers who have the ability to control uncontrolled unintelligent undead by any of a number of means.

Corpses and skeletons are brought to the zombie factory, where they are modified with the addition of metal plates and the like, and then animated with fiendish powers. It is then up to the customer to take command of their new toys and take them away. High priced zombie factories exist that procure specific requested bodies to be animated.

\business{Zombie Factory}{Craft (Armorer)}{Knowledge (Nature), Appraise}{Medium}{Low}{High}

\subsubsection{Fossil Storage}

Powers exist in the lower planes that can petrify creatures, leaving them as calcified statues sleeping away the eons in a blanket of stone. With the known relative ease of raising the dead, the ability to remove an opponent without literally killing them is in high demand.

Some creatures even hire these services, not as jailers, but as hiding places. Keeping themselves in storage out of the reach of vengeful arms long enough to be forgotten. The Petrification Guild guarantees that you'll be revived at the appropriate time specified in your contract.

A Petrifying Prison is different from ordinary businesses in that it is in all ways advantageous for it to be far away from any civilization. The profit check for a Petrifying Prison gains bonuses that are inverted for its location: the Wilderness grants a Profit Modifier of +10, Rural +4, Town +2, City +0, Metropolis -4, Planar Metropolis -10.

\business{Petrifying Storage}{Knowledge (Dungeoneering)}{Knowledge (Architecture), Listen}{Low}{High}{Medium}

\subsubsection{Arbitrage and Skullduggery}

Of course, lots of mortal businesses have counterparts in the lower planes. Mercenaries, Ferrymen, Merchants, and Farmers all exist aplenty in the dark planes of existence. A business run on the planes functions just like the normal businesses listed in the DMG2, save that the Risk is increased by one level (to a maximum of High).
