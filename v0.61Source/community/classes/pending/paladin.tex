\classname{Paladin}
\vspace*{-8pt}
\quot{``Good for the Good God!''}

\desc{A Paladin fights in melee and casts protective spells, enhancing their own defenses and those of others, and diverting attacks attacks onto themselves despite their fundamentally being a defensive character.  They also have a reasonable supply of information-gathering divination spells.}

\desc{As a Paladin, despite being a melee combatant, you can neglect your physical abilities some, and focus on Wisdom and Charisma, which power your abilities.  Your class features allow you to also rely on Wisdom and Charisma for physical combat to an extent, although Constitution and Strength are useful if you don't neglect them.}

\desc{Depending on the moral system of the setting, Paladins may or may not have a deific preference.  Under flag-based morality, Paladins often have patrons from their sponsoring plane.  On the other hand, if the planes of Good are Good because of what they do, rather than what they are, then Paladins are often not dedicated followers of any specific deity; their dedication to overall Good often interferes with their ability to maintain any other allegience, even a deific one.}

\ability{Alignment:}{Paladins must be of Good alignment.  Under Word is Bond ethics, Paladins must also be Lawful.  Under flag-based alignments, Paladins must be aligned with a plane that sponsors Paladins (Mount Celestia under the core cosmology).}

\ability{Hit Die:}{d8}

\ability{Base Attack Bonus:}{Good}

\ability{Good Saves:}{Fort, Reflex}

\ability{Skills:}{Climb (Str), Concentration (Con), Craft (Int), Diplomacy (Cha), Handle Animal (Cha), Heal (Wis), Knowledge (Nobility, Religion) (Int), Profession, Ride (Dex), Sense Motive (Wis), Spellcraft (Int), and Swim (Str).}

\ability{Skill Points/Level:}{2 + Int Mod}

\begin{table}[htb]
\begin{small}
\begin{tabular}[h]{lp{1.9cm}p{0.7cm}p{0.7cm}p{0.7cm}p{9cm}}
&   Base Attack Bonus&  Fort Save&  Ref Save&   Will Save&  Special\\
1&  +0& +2& +2& +0&  Aura of Good, Healing Touch, Exorcism, Insightful Strike, Personal Warding, Turn Undead\\
2&  +1& +3& +3& +0&  Detect Evil, Divine Grace, Divine Heart, Holy Shield, Smite Evil\\
3&  +1& +3& +3& +1&  Evasion, Expanded Prayers, Interrupting Spell 1/day\\
4&  +2& +4& +4& +1&  Snap Judgement\\
5&  +2& +4& +4& +1&  Divine Resilience, Guided Ward, Reactive Healing\\
6&  +3& +5& +5& +2&  Iron Will, Special Mount\\
7&  +3& +5& +5& +2&  Divine Reach, Expanded Prayers\\
8&  +4& +6& +6& +2&  Exorcising Smite, Interrupting Spell 2/day, Mortal Smite\\
9&  +4& +6& +6& +3&  Divine Chain, Hero's Heart, Prayers of the Faithful\\
10& +5& +7& +7& +3&  Timelessness  \\
11& +5& +7& +7& +3&  Expanded Prayers\\
12& +6& +8& +8& +4&  Radiant Healing\\
13& +6& +8& +8& +4&  Interrupting Spell 3/day\\
14& +7& +9& +9& +4&  Double Smite\\
15& +7& +9& +9& +5&  Expanded Prayers\\
16& +8& +10& +10& +5&  \\
17& +8& +10& +10& +5&  Divine Reaction\\
18& +9& +11& +11& +6&  Interrupting Spell 4/day\\
19& +9& +11& +11& +6&  Expanded Prayers, Guardian Beyond Death\\
20& +10& +12& +12& +6&  Ascendancy\\
\end{tabular}
\end{small}
\end{table}

\smallskip\noindent All of the following are Class Features of the Paladin class.

\ability{Weapon and Armor Proficiency:}{Paladins are proficient with all simple and martial weapons, all armor, and non-tower shields.}

\ability{Spellcasting:}{Paladins use the Sorcerer spells/day table.  Bonus spells and save DCs are based on Wisdom.  They cast divine spells off the Paladin spell list, and automatically know every spell on their spell list.  The somatic components of Paladin spells can be done even if both hands are occupied with a weapon and shield (but not two weapons).}

\ability{Aura of Good:}{A Paladin registers under Detect Good as a Cleric of a Good Deity.}

\ability{Healing Touch (Su):}{A Paladin has a pool of Healing Touch points equal to her class level * her Charisma modifier, which refreshes when her spell slots do.  As a swift action, the Paladin may heal herself or an ally within reach of her touch by any number of hit points up to the current value of their Healing Touch pool, removing that many points from the pool.  If the Paladin recieves healing in excess of her current damage taken, she may replenish her Healing Touch pool, but to no more than its usual maximum.}

\ability{Exorcism (Su):}{As an attack no more than once per round, a Paladin may deliver a melee touch attack to any undead creature, [evil] outsider, or any creature under the influence of a possessing spirit, such as a Ghost's Malevolence ability.  On a successful hit, deduct a number of points from the Paladin's Healing Touch pool equal to her class level, and inflict 1d6 damage per point taken.  This damage comes from pure divine power.  An exorcism on a possessed creature bypasses the possessed creature entirely and does damage directly to the possessor.  Exorcisms cannot make critical hits.  An exorcism, even one channeled through a weapon, does not do damage from any other source, such as weapon or unarmed strike damage.}

\ability{Insightful Strike:}{A 2nd-level Paladin gains Insightful Strike as a bonus feat.  If she already has Insightful Strike, she instead gains another [Combat] feat.}

\ability{Personal Warding:}{As long as she makes a melee attack on the same round, a Paladin may cast any Paladin spell with a casting time of one round or less on herself as a swift action.  She must attack before she can cast the spell.}

\ability{Turn Undead:}{As Cleric. A Paladin can never Rebuke Undead, and doing so is against their code of conduct should they somehow gain the ability from another class.}

\ability{Detect Evil (Sp):}{A Paladin of 2nd level or higher can cast \emph{Detect Evil} at-will as a spell-like ability.}

\ability{Divine Grace:}{A Paladin of 2nd level or higher adds her Charisma bonus to all saving throws.}

\ability{Divine Heart (Ex):}{A Paladin of at least 2nd level is immune to all Fear effects, including the conditions Shaken, Frightened, and Panicked, and all nonmagical diseases, including the Sickened condition, even when it has a magical source.}

\ability{Holy Shield (Ex):}{A Paladin of 2nd level or higher can substitute her Charisma modifier for her Dexterity modifier to armor class as long as she is using a shield in her hand (not an Animated Shield). She gains the full benefit of her shield against touch attacks. This bonus applies even when her Dexterity bonus would normally be denied, but she is still considered to have lost her dexterity bonus to AC for purposes of special abilities such as Sneak Attack under the normal circumstances.}

\ability{Smite Evil (Su):}{A Paladin of 2nd level or higher can attempt to smite a target as part of making a melee attack.  The attempt does not work (at no cost to her other than the attack) unless the target is evil.  If the target is evil, she adds her Charisma modifier to hit, and, if successful, she adds her Charisma modifier to hit and her level to damage.  She also gains supernatural strength, causing her to not take strength penalties to damage (bonuses still apply).

If, after the attack's damage is applied, the target has less than 4 hit points per class level of the Smiting paladin, it is instantly destroyed as though through hit point damage.  A single target may only be smote once per Paladin per day, and the Paladin cannot smite another target after successfully smiting one until either the first target is defeated or five rounds pass since the last Smite.  If the attack misses or is attempted on an invalid target, the Paladin may try again immediately against the same or a different target as soon as she can make another attack.}

\ability{Evasion (Ex):}{As a Rogue or Monk.  Gained at 3rd level.}

\ability{Expanded Prayers:}{At 3rd, 7th, 11th, 15th, and 19th levels, the Paladin adds one Abjuration, Divination, or Necromancy (Healing) spell of any level up to the highest level she can cast from any 9-level pre-Tome caster's spell list to her own.  The spell may not already be on her spell list at a different level.}

\ability{Interrupting Spell:}{A Paladin of 3rd level may, once per day, cast any Paladin spell with a casting time of one round or less as an immediate action.  This increases to twice per day at 8th level, three times at 13th level, and four times per day at 18th level.}

\ability{Snap Judgement (Su):}{As a swift action, a Paladin of fourth level or higher can focus on a target and gain information about them as if she had spent three rounds focusing on them with \emph{Detect Evil}.}

\ability{Divine Resilience (Su):}{A Paladin of 5th level or higher may expend a Turn Undead attempt as a free action on her turn to gain Energy Resistance equal to twice her class level to any two energy types, selected when she uses this ability.  This is shared with all allies within 20'.  This lasts for a number of rounds equal to her Charisma bonus + 1, minimum 1.  Allies that leave the area of this ability lose its effect, and allies that enter it gain the benefit.  If the Paladin activates this ability again, the first one automatically expires even if its duration had not ended.}

\ability{Guided Ward (Ex):}{A Paladin of 5th level or higher may focus on an enemy as a free action, impeding it from attacking her allies.  This ability, and its targeting, are obvious to any observer.  If the enemy attacks any of her allies before her next turn, she may cast any Paladin spell with a casting time of one round or less on any ally that creature attacked as a swift action.  The ally must still be within range and a valid target for the spell.}

\ability{Reactive Healing:}{A Paladin of 5th level or higher can use Healing Touch as an immediate action.}

\ability{Iron Will:}{At 6th level, a Paladin gains Iron Will as a bonus feat.  If she already has the Iron Will feat, she may pick any [Combat] feat.}

\ability{Special Mount:}{At 6th level, a Paladin gains a special mount.  This is a creature of either the same alignment as the Paladin or 4 or lower intelligence (including mindless) and non-evil alignment, and a CR three lower than the Paladin's, taking [Awesome] subtypes into account.  If she really wants to, she may instead use the SRD Paladin's Mount rules instead.  Regardless, the creature must be willing and able to serve as a mount (no Troll, Tendriculous, or Rat mounts, with an exception on the last for small values of Paladin and large values of Rat).  If necessary, the creature may be advanced to keep pace with the Paladin.}

\ability{Divine Reach:}{A Paladin of 7th level or higher may expend a Turn Undead attempt to change any of her Touch-range Paladin spells to Close range, as part of casting the spell. Such a modified spell can only affect willing targets.  She may also extend a Personal spell to Close range, but such a spell may only be cast on targets allowed by Guided Ward.}

\ability{Exorcising Smite (Su):}{A Paladin of 8th level or higher may make an Exorcising Smite.  As part of making the smite, she channels Exorcism through her weapon using the same attack roll; the Exorcism only lands on a successful hit, not just a touch hit unless the whole attack is a touch attack.  The Exorcism takes effect before weapon damage (including the Smite additions) are applied.  She may make no more than one attack in a round in which she makes an Exorcising Smite.  All costs of the Exorcism and the Smite are paid normally.}

\ability{Divine Chain:}{A Paladin of 9th level or higher may apply the Chain Spell metamagic feat to any of her Paladin spells when casting it without increasing the spell level, causing it to affect a number of additional targets equal to her caster level for half damage, or -4 to the save DC of nondamaging spells.  All secondary targets must be valid.  Using this ability costs two uses of her Turn Undead ability.}

\ability{Hero's Heart (Ex):}{A Hero's heart cannot be easily stopped by magic.  A Paladin of 9th level becomes immune to [Death] effects and petrification.}

\ability{Prayers of the Faithful (Su):}{A Paladin of 9th level or higher may pray for healing.  Praying is a full-round action that precludes taking a 5-foot step.  She is considered flat-footed while praying, but gains Fast Healing 10.  Up to one ally per level within 30' with line of sight to her may join her in prayer, taking the same action and gaining the same benefits.}

\ability{Radiant Healing (Su):}{A Paladin of 12th level or higher may use her Healing Touch ability at Close range instead of Touch.  By halving the healing done, she may have it affect all allies in range instead of just herself.}

\ability{Double Smite (Su):}{A Paladin of 14th level or higher gains a second Smite, which can be used just like the first and is recovered independently but in the same way.  So she can smite a target twice, or smite a second target before the first is neutralized, but not both at once.}

\ability{Divine Reaction (Su):}{A Paladin of 17th level or higher may expend a Turn Undead attempt as a free action to gain an extra swift action to use immediately.  She may do this when it is not her turn to gain an extra swift action on her next turn that must be spent immediately (i.e., an immediate action).  Each of her turns may only benefit from this ability once.}

\ability{Guardian Beyond Death (Su):}{A Paladin of 19th level is revived as if by a \emph{True Ressurection} spell where she fell at what would be the end of her next turn after she is slain.  She may use this ability once per day.  She also regains three uses of Turn Undead and her entire Healing Touch pool, as well as both Smites when this happens.  All of her equipment is teleported to its proper place when she returns.  If this ability is suppressed when she dies or when it would activate, it delays until it is not suppressed.}

\ability{Ascendancy (Ex):}{A Paladin of 20th level may be considered as her original type or a Native Outsider, whichever is most advantageous at the moment.  She gains Damage Reduction 10/evil, is no longer subject to aging penalties or death by old age, and may use \emph{Plane Shift} three times per day as a spell-like ability, but may only shift between a chosen Good-aligned plane (under flag morality, her sponsoring plane) or her home plane.  Once she chooses her upper plane, it cannot be changed except for through an alignment change.  Her powers also can no longer be withdrawn by her planar sponsors; she may change alignment freely.}

\ability{Code of Conduct:}{Something paladinish but not shit.  If she violates this code egregiously or habitually, she must recieve an Atonement spell from a character of her alignment and higher level before she can use her Supernatural and Spell-like class features and her spellcasting again.  She intuitively knows when an action pushes the boundaries of or violates the code.}

\spelllist{Paladin Spell List:}

\small\ability{0th level:}{\emph{Create Water, Cure Minor Wounds, Detect Magic, Detect Poison, Disrupt Undead, Guidance, Light, Mending, Read Magic, Resistance, Virtue}}

      \ability{1st level:}{\emph{Atonement, Bless, Bless Water, Bless Weapon, Cure Light Wounds, Deathwatch (no [Evil] descriptor), Delay Poison, Detect Chaos, Detect Good, Detect Law, Detect Undead, Greater Dispel Magic, Divine Favor, Endure Elements, Enlarge Person, Entropic Shield, Magic Vestment, Greater Magic Weapon, Nondetection, Protection From Chaos/Evil/Law (opposed alignments only), Remove Fear, Searing Light, Shield of Faith}}

      \ability{2nd level:}{\emph{Aid, Align Weapon, Bull's Strength, Consecrate, Cure Moderate Wounds, Desecrate (desanctifying use only, no [Evil] descriptor), Eagle's Splendor, Heroism, Owl's Wisdom, Protection from Arrows, Remove Paralysis, Resist Energy, Lesser Restoration, See Invisibility, Shield Other, Status, Zone of Truth}}

      \ability{3rd level:}{\emph{Arcane Sight, Continual Flame, Create Food and Water, Cure Serious Wounds, Daylight, Holy Smite, Magic Circle against Chaos/Evil/Law (Opposed alignments only), Neutralize Poison, Prayer, Protection from Energy, Remove Blindness/Deafness, Remove Curse, Remove Disease, True Strike}}

      \ability{4th level:}{\emph{Cure Critical Wounds, Death Ward, Detect Scrying, Dimensional Anchor, Discern Lies, Dismissal, Freedom of Movement, Haste, Lesser Planar Ally, Restoration, Spell Immunity}}

      \ability{5th level:}{\emph{Break Enchantment, Commune, Mass Cure Light Wounds, Dispel Chaos (Lawful only), Dispel Evil, Dispel Law (Chaotic only), Disrupting Weapon, Hallow, Heal Mount, Holy Sword, Interposing Hand, Mark of Justice (nonchaotic only), Raise Dead, Righteous Might, Spell Resistance, True Seeing}}

      \ability{6th level:}{\emph{Banishment, Mass Cure Moderate Wounds, Find the Path, Heal, Heroes' Feast, Greater Heroism, Planar Ally, Plane Shift, Quest, Undeath to Death}}

      \ability{7th level:}{\emph{Greater Arcane Sight, Mass Cure Serious Wounds, Holy Word, Limited Wish, Regenerate, Resurrection, Greater Restoration, Spell Turning}}

      \ability{8th level:}{\emph{Mass Cure Critical Wounds, Dimensional Lock, Holy Aura, Iron Body, Mind Blank, Moment of Prescience, Greater Planar Ally, Protection from Spells, Greater Spell Immunity}}

      \ability{9th level:}{\emph{Astral Projection, Foresight, Freedom, Gate, Mass Heal, Miracle, True Resurrection}}
\normalsize

\ability{Holy Avengers, Protectors, and Redeemers:}{Paladins use a special class of magic weapons known as Holy Avengers, Holy Protectors, and Holy Redeemers (collectively referred to as Paladin Weapons).  Such items are always magic melee weapons.  They always count as Good-aligned weapons.  They also gain additional powers depending on the class level of the Paladin, as shown on the table below.  A Paladin may only get the special benefits (beyond being a magic weapon) of one Paladin Weapon at a time, even if she fights with two weapons, or wears one and fights with the other.

All Types:
0: The weapon sheds light as a torch when drawn.
1: The Paladin continually benefits from Protection from Evil as long as the weapon is on her body, as a supernatural ability.  This may be suppressed or restored as a swift action while the weapon is stowed or sheathed; while it is in-hand, it is always active.
4: The Paladin continually generates a Magic Circle against Evil as long as the weapon is drawn.  The weapon sheds light as a Daylight spell when drawn.
8: The Paladin and all allies within 10' of her gain Spell Resistance 5 + the Paladin's character level as long as the weapon is drawn.  The weapon suppresses Darkness effects of lower spell level than the highest the Paladin can cast, and dispels those more than three levels lower.

Avenger:
Holy Avengers are carried by Paladins who serve as agents of vengeance.  Such Paladins have the grim duty of stopping those who have already done evil, pursuing justice for those wrongs which cannot be righted.  Paladins carrying Holy Avengers are the best known for their deeds, however, as the tale of the destruction of a powerful demonic temple will be told for far longer than that of the time your home wasn't burned by a dragon.
1: The Paladin may cast spells that target a weapon, such as Bless Weapon, on her Holy Avenger as a swift action as long as it is drawn.
3: When the Paladin successfully uses Smite Evil with her Holy Avenger, she ignores all material and alignment damage reduction, and does lethal damage to targets with Regeneration.  The Holy Avenger is also considered to be a Ghost Touch weapon when used in a Smite.
8: The Holy Avenger may be used to make a smiting attack separate from the Paladin's normal Smites.  She adds her Charisma to attack and her level to damage against an evil creature for one attack.  If the attack is successful, the creature must make a Fortitude save (DC 10 + 1/2 Paladin's level + Paladin's Wisdom modifier) or be stunned and dazed for 4 rounds.  If the attack misses or is made against a non-evil creature, it may be attempted again next round; if it hits, it may be attempted again either as soon as the first target leaves the fight (such as by retreat, surrender, or death) or after five rounds.
13: The Paladin may use Exorcism without spending points from her Healing Touch pool while weilding or wearing the Holy Avenger.  Exorcism damage dice are d8s.
14: The Paladin adds Destruction to her spell list as a 7th-level spell while wielding the Holy Avenger.
18: Any creature smote by the Paladin must make a Fortitude save or be destroyed as per the Destruction spell (normal weapon damage on a successful save).  Any creature killed by a Paladin's smite is destroyed as per the Destruction spell.  Undead are fully affected by this ability.

Protector:
Holy Protectors are used to defend others.  They are the signature weapon of Paladins of a more defensive bent, both those stopping rampaging dragons from burning villages and those stopping hordes of demons from shredding archmages.
1: The Paladin can add her level to her Healing Touch pool maximum, and gains an additional use of Turn Undead, essentially increasing her effective Charisma modifier for purposes of Healing Touch and Turn Undead by 1.  She gains these benefits as long as the Holy Protector is worn.
3: The Paladin may make an additional number of Attacks of Opportunity with her Holy Protector per round equal to her Charisma bonus.  She may also make attacks of opportunity as though it had Reach, although, if it does not, she does not gain an attack of opportunity against enemies closing with her.  When she hits with an attack of opportunity provoked by movement, the creature must make a Will save (DC 10 + Paladin's Charisma Modifier + 1/2 Paladin's level) or have its speed reduced to 0 until the beginning of its next turn.  Flying creatures instead must continue a straight-line course to maintain the minimum distance to avoid a fall, at a 45-degree downward angle.
8: The Holy Protector becomes a Defending weapon.  All allies of the Paladin within 30' gain the same AC bonus as she does from this or the Expertise attack option.
13: The Paladin gains Uncanny Dodge and Improved Uncanny Dodge.  She may also cast Shield Other on any creature within 60' as a free action spell-like ability usable as many times per round as she cares to use it, that lasts until the target goes out of range or it is dismissed as another free action (even when not her turn).  Finally, as long as her weapon is drawn and held, she gains Regeneration 3, with Evil-aligned weapons and spells doing full damage.
18: Three times per day, when she casts a spell with only willing targets, the Paladin may designate the spell to repeat again on the next round.

Redeemer:
Paladins who seek to show their enemies the benefits of switching to Team Good often carry Holy Redeemers for personal defense when higher-ups on Team Evil try to gank them for stealing their minions.  Occasionally a Paladin with this path makes a name for themselves with a retinue of redeemed devils or suchlike, but most work more quietly.
1: Attacks with a Holy Redeemer never take a penalty to inflict nonlethal damage.  Creatures destroyed by a Holy Redeemer's smite may instead merely be knocked unconcious or held for up to one hour per class level, at the Paladin's option.
4: If the target of her Snap Judgement ability knowingly lies before the Paladin's next turn, she knows (as the Discern Lies spell).
8: The Paladin may, as a swift action, allow any ally to repeat one Will save against a Charm, Compulsion, Possession, or [evil] affect, with her Charisma modifier as a bonus on the save.  Success causes the effect to end immediately.
13: The Paladin gains a bonus spell slot of her highest castable spell level.
18: Any creature smote with a Holy Redeemer may be forced to make a Will save (DC 10 + 1/2 Paladin level + her Charisma modifier), or be subject to an Imprisonment spell.  Imprisonment is added to the Paladin's spell list as long as she wears this weapon.}

\ability{Acquiring Paladin Weapons:}{Paladin weapons are intended to not be particularly imbalancing if a Paladin acquires them at any point in her career.  A Paladin that wants one should be allowed to acquire one at some point in the campaign, with the level range of the campaign determining which point.  If the campaign starts above level 12, Paladin characters should probably be allowed to start with Paladin weapons if they want one.  A Paladin that is supposed to find a Paladin Weapon may be given extraplanar help or advice in locating one (i.e., an angel shows up in one of her dreams and says that a Paladin with exactly the weapon they want died in the Dungeon of Dread some generations back, and his weapon is still in the dungeon).}

\ability{Distributing Paladin Weapons:}{Retaining more than one Paladin Weapon for longer than is necessary to give it to a deserving Paladin (including evaluation of deservingness), or deliberately destroying one, is a violation of the Paladin's Code; Paladin Weapons are in short supply, and hoarding them directly impedes the cause of good.}

\ability{Crafting Paladin Weapons:}{Paladin Weapons can only be crafted by Paladins.  They are considered Masterpieces at any level, have a level prerequisite of some kind, and each Paladin is only able to craft one ever.  There's probably some other cost too.  Thus, most Paladins find, rather than make, their weapons.}