\classname{Master of Snake Mountain}\label{class:masterofsnakemountaint}
\vspace*{-8pt}
\quot{``Now you muscle-bound boobs, prepare to meet your doom. Hahahahaha!"}

Dungeons are hot property. They are enormously expensive to build, and are by their nature defended from magical interventions that would otherwise render their occupants extremely vulnerable. Thus, when a dungeon is over run, it is generally not long before it gains a new occupant. The Master of Snake Mountain is in control of a dungeon, but there's no reason to believe he's the first occupant. He might not even be the second.

A Master of Snake Mountain is one who has taken control of a dungeon and used it as a military staging area to launch grand plans. Such men could politely be described as egomaniacs, and rarely have a kind word to say to anyone that isn't spoken in an extremely sarcastic fashion.

\ability{Prerequisites:}
\listprereq
\itemability{Skills:}{9 ranks in Knowledge Dungeoneering and Perform (oratory)\\
-or-\\
9 ranks in Knowledge Architecture and Engineering and Perform (oratory)}
\itemability{Feat:}{Leadership, Any Item Creation Feat.}
\itemability{Special:}{Must have control of a dungeon, whether by having it built yourself or by taking it from someone else by force.}
\end{list}\vspace*{8pt}

\ability{Hit Die:}{d8}

\ability{Class Skills:}{The Master of Snake Mountain's class skills (and the key ability for each skill) are Appraise (Int), Bluff (Cha), Climb (Str), Concentration (Con), Craft (Int), Decipher Script (Int), Diplomacy (Cha), Disguise (Cha), Escape Artist (Dex), Gather Information (Cha), Jump (Str), Knowledge (all skills, taken individually) (Int), Listen (Wis), Perform (Cha), Profession (Wis), Sense Motive (Wis), Sleight of Hand (Dex), Spellcraft (Int), and Swim (Str).}

\ability{Skills/Level:}{4 + Intelligence Bonus}


\begin{table}[tbh]
\begin{small}
\begin{tabular}{lp{1.9cm}p{0.7cm}p{0.7cm}p{0.7cm}p{6cm}l}
Level&Base Attack Bonus&Fort Save&Ref Save&Will Save&Special&Spellcasting\\
1&+0&+0&+2&+2&Stable of Henchmen, Bardic Music, Code of Conduct&+1 spellcasting level\\
2&+1&+0&+3&+3&Disposable Monstrous Cohort, Speak with Monsters&+1 spellcasting level\\
3&+2&+1&+3&+3&Belittling Tirade&+1 spellcasting level\\
4&+3&+1&+4&+4&Enhance Minions&+1 spellcasting level\\
5&+3&+1&+4&+4&Eyebeams, Wondrous Architect&+1 spellcasting level\\
\end{tabular}
\end{small}
\end{table}

\smallskip\noindent All of the following are Class Features of the Master of Snake Mountain class.

\ability{Weapon and Armor Proficiency:}{A Master of Snake Mountain gains no proficiency with any weapons or armor.}

\ability{Spellcasting:}{Every level, the Master of Snake Mountain casts spells (including gaining any new spell slots and spell knowledge) as if he had also gained a level in a spellcasting class he had previous to gaining that level.}

\ability{Stable of Henchmen:}{A Master of Snake Mountain is a landlord and his dungeon fills up will all manner of ne'er-do-wells and hooligans. Practically this means that a Master of Snake Mountain can swap out his cohort for another cohort of an appropriate level at the beginning of each adventure. This doesn't mean that the Master of Snake Mountain can simply loot a cohort's worth of equipment every adventure, because while the different available cohorts are interchangeable, they actually don't go anywhere special when they are traded out. A cohort that is traded out is not dismissed, he simply doesn't accompany the Master of Snake Mountain on a particular adventure. Such a cohort continues to be available in later adventures if the Master of Snake Mountain decides to swap back. All available cohorts gain levels when the Master of Snake Mountain does, whether they were accompanying his adventures or not.}

\ability{Bardic Music (Su):}{A Master of Snake Mountain can produce Bardic Music effects with his Perform (Oratory) as if he was a Bard with a level equal to his class level. If he actually has Bard levels, the abilities and uses per day stack if he is using Perform (Oratory).}

\ability{Code of Conduct:}{A Master of Snake Mountain must conduct his affairs with senseless, yet restrained villainy. He must abide by the following restrictions:

\listone
    \item A captured or surrendered foe may not be summarily executed, though they may be left in situations almost certain to kill them.
    \item A Master of Snake Mountain boasts constantly and gives believes himself far more accomplished and powerful than he is. He must explain his big plans to anyone who will listen.
    \item A Master of Snake Mountain must behave in a cowardly and villainous fashion. A Master of Snake Mountain may not accept a challenge he regards as fair or sacrifice himself for the good of others.
\end{list}

A Master of Snake Mountain who fails to abide by these restrictions loses his ability to use his Bardic Music abilities until he atones.}

\ability{Disposable Monstrous Cohort:}{At the beginning of every adventure, the Master of Snake Mountain is followed by a monster for no reason once he reaches 2nd level. This monster must be at least 2 CR less than his character level, and will be a Magical Beast, an Aberration, a Plant, a Dragon, or an Ooze. It will follow his orders to the best of its ability, but whether it survives or not it will be replaced just as mysteriously by another monster at the beginning of the next adventure.}

\ability{Speak with Monsters (Ex):}{The general gist of whatever a 2nd level Master of Snake Mountain happens to be ranting about gets across to any Magical Beasts, Plants, Aberrations, Dragons, or Oozes that can hear his tirades but do not have a language.}

\ability{Belittling Tirade (Su):}{At 3rd level, the Master of Snake Mountain can use his Bardic Orations to make people feel bad about themselves. Such creatures receive a -2 morale penalty to attack rolls and saving throws and a -4 morale penalty to their Level Check to oppose intimidate checks. A Master of Snake Mountain can belittle any number of creatures within medium range as a standard action. The feelings of inadequacy last for 1 hour.}

\ability{Enhance Minions (Su):}{At 4th level, the Master of Snake Mountain gains the ability to make grafts. He may supply grafts from any graft list, and may apply grafts from different lists to the same creature (though the maximum of 8 grafts still applies). The costs for applying these grafts are half normal, though he cannot implant grafts into himself.}

\ability{Eyebeams (Su):}{A Master of Snake Mountain of 5th level has the ability to fire painful or deadly rays from his eyes. The Eyebeams are a ray effect with short range, and a creature struck with them (a ranged touch attack) is affected as per a \spell{symbol of pain}. At his option, the Eyebeams may also inflict 4d6 of Force Damage. Once fired, the Eyebeams may not be used again for 1d4+1 rounds.}

\ability{Wondrous Architect:}{At 5th level, a Master of Snake Mountain becomes a master of improving his own pad. He may make Wondrous Architecture in half the normal time at half the normal expense.}
